\documentclass[nocopyrightspace,preprint,times,9pt]{sigplanconf}

% For pdflatex, replaced by fontspec:
\usepackage[T1]{fontenc}
\usepackage[utf8]{inputenc}


%% For general writing
\usepackage{fixltx2e}
\usepackage[usenames,dvipsnames,svgnames,table]{xcolor}
\usepackage{url}
\usepackage{fancyvrb}
\usepackage{mdwlist}  % Miscellaneous list-related commands
\usepackage{xspace}   % Smart spacing
\usepackage{ucs}

% https://www.nesono.com/?q=book/export/html/347
% Package for inserting TODO statements in nice colorful boxes - so that you
% won’t forget to fix/remove them. To add a todo statement, use something like
% \todo{Find better wording here}.
\usepackage{todonotes}


%% Math & theoretical computer science
\usepackage{amsmath}
\usepackage{amssymb}
\usepackage{amsthm}
\usepackage{bm}         % Bold symbols in maths mode
\usepackage{dsfont}
\usepackage{stmaryrd}
\usepackage{mathtools}  % For "::=" ( \Coloneqq )
% http://tex.stackexchange.com/questions/114151/how-do-i-reference-in-appendix-a-theorem-given-in-the-body
\usepackage{thmtools, thm-restate}

\theoremstyle{definition}
\newtheorem{definition}{Definition}

\theoremstyle{plain}
\newtheorem{theorem}{Theorem}
\newtheorem{lemma}{Lemma}


%% Code listings
\usepackage{listings}

\lstdefinestyle{f2j}{
    basicstyle=\ttfamily\small,
    keywordstyle=\sffamily\bfseries,
    tabsize=2,
    keepspaces=true,
    showstringspaces=false,
    escapeinside={(*}{*)},
    morekeywords={let,in}
}

\lstset{style=f2j}


%% Font
\usepackage[euler-digits,euler-hat-accent]{eulervm}

%% Typesetting inference rules
% \usepackage{styles/bcprules}    % by Benjamin C. Pierce
% \usepackage{styles/cmll}
\usepackage{styles/mathpartir}  % by Didier Rémy (http://gallium.inria.fr/~remy/latex/mathpartir.html


% Copied from the FCore paper:
\usepackage[colorlinks=true,allcolors=black,breaklinks,draft=false]{hyperref}   % hyperlinks, including DOIs and URLs in bibliography
% known bug: http://tex.stackexchange.com/questions/1522/pdfendlink-ended-up-in-different-nesting-level-than-pdfstartlink


\newcommand{\hast}{\!:\!}

% Relations
\newcommand{\subtype}   {<:}

\definecolor{facebook}{HTML}{3B5998}
\newcommand{\yields}[1]{\textcolor{facebook}{\; \hookrightarrow {#1}}}

% Helpers
\newcommand{\ftv}[1]{\textit{ftv}({#1})}

% Spacing
\newcommand{\binderSpacing}{\,}
\newcommand{\appSpacing}{\;}

% Types
% \newcommand{\top}{\{\}}
\newcommand{\andOp}{\with}

% Expressions
\newcommand{\lam}[3]{\lambda (#1 \hast #2).\binderSpacing #3}
\newcommand{\mergeOp}{,,}
\newcommand{\restrictOp}{\setminus}
\newcommand{\recordUpdate}[3]{#1 \; \mathbf{with} \; \{#2 = #3\}}


\newcommand{\Int}{\code{Int}}
\newcommand{\String}{\code{String}}
\newcommand{\Bool}{\code{Bool}}
\newcommand{\I}{\code{i}}
\newcommand{\J}{\code{j}}


% Rules

% Couleurs
\colorlet{subColor}{OliveGreen}
\colorlet{targetColor}{BrickRed}

% Subtyping labels
\newcommand{\ruleLabelSub}{\bm{\textcolor{subColor}{sub}}}
\newcommand{\ruleLabelSubVar}{\ruleLabelSub\text{var}}
\newcommand{\ruleLabelSubTop}{\ruleLabelSub\text{top}}
\newcommand{\ruleLabelSubFun}{\ruleLabelSub\text{fun}}
\newcommand{\ruleLabelSubForall}{\ruleLabelSub\text{forall}}
\newcommand{\ruleLabelSubAnd}{\ruleLabelSub\text{and}}
\newcommand{\ruleLabelSubAndLeft}{\ruleLabelSub{\text{and}_1}}
\newcommand{\ruleLabelSubAndRight}{\ruleLabelSub{\text{and}_2}}
\newcommand{\ruleLabelSubRec}{\ruleLabelSub\text{rec}}

% Source/elaboration and labels
\newcommand{\judgeSourceWF}[2]{#1 \; \textcolor{sourceColor}{\turns} \; #2}
\newcommand{\ruleLabelSourceRecUpd}{\ruleLabelSource\text{rec-upd}}


% Presentation
\definecolor{lightyellow}{HTML}{FFFFE0}


% To be retired
\newcommand{\turnsGet}{\turns_{\textrm{get}}}
\newcommand{\turnsPut}{\turns_{\textrm{put}}}
\newcommand{\turnsrec}{\turns_{\textrm{rec}}}
\newcommand{\rulename}[1]{(\textrm{#1})}



\newcommand{\formwf}{\framebox{$ \jwf \Gamma A $}}

\newcommand{\makelabelwf}[1]{WF$#1$}

\newcommand{\labelwfvar}{\makelabelwf \alpha}
\newcommand{\rulewfvar}{
  \inferrule* [right=\labelwfvar]
    {\alpha \in \Gamma}
    {\jwf \Gamma \alpha}
}

\newcommand{\rulewfvardis}{
  \inferrule* [right=\labelwfvar]
    {\alpha \disjoint A \in \Gamma}
    {\jwf \Gamma \alpha}
}

\newcommand{\labelwftop}{\makelabelwf \top}
\newcommand{\rulewftop}{
  \inferrule* [right=\labelwftop]
    { }
    {\jwf \Gamma \top}
}

\newcommand{\labelwfint}{\makelabelwf {\mathbb{Z}}}
\newcommand{\rulewfint}{
  \inferrule* [right=\labelwfint]
    { }
    {\jwf \Gamma {\code{Int}}}
}

\newcommand{\labelwffun}{\makelabelwf \rightarrow}
\newcommand{\rulewffun}{
  \mprset {sep=1em}
  \inferrule* [right=\labelwffun]
    {\jwf \Gamma A \\ \jwf \Gamma B}
    {\jwf \Gamma {A \to B}}
  \mprset {sep=2em}
}

\newcommand{\labelwfforall}{\makelabelwf \forall}
\newcommand{\rulewfforall}{
  \inferrule* [right=\labelwfforall]
    {\jwf {\Gamma, \alpha} A}
    {\jwf \Gamma {\for \alpha A}}
}

\newcommand{\rulewfforalldis}{
  \mprset {sep=1em}
  \inferrule* [right=\labelwfforall]
    {\jwf \Gamma A \\ \jwf {\Gamma, \alpha \disjoint A} B}
    {\jwf \Gamma {\fordis \alpha A B}}
  \mprset {sep=2em}
}

\newcommand{\labelwfinter}{\makelabelwf \inter}
\newcommand{\rulewfinter}{
  \inferrule* [right=\labelwfinter]
    {\jwf \Gamma A \\ \jwf \Gamma B}
    {\jwf \Gamma {A \inter B}}
}

\newcommand{\rulewfinterdis}{
  \mprset {sep=1em}
  \inferrule* [right=\labelwfinter]
    {\jwf \Gamma A \\
     \jwf \Gamma B \\ 
     \jdis \Gamma A B}
    {\jwf \Gamma {A \inter B}}
  \mprset {sep=2em}
}

\newcommand{\labelwfrec}{\makelabelwf R}
\newcommand{\rulewfrec}{
  \inferrule* [right=\labelwfrec]
    {\jwf \Gamma A}
    {\jwf \Gamma {\recordType l A}}
}

\newcommand{\disjointvar}{
  \inferrule* [right=DisjointVar]
    {\alpha * B \in \Gamma}
    {\isdisjoint \Gamma \alpha B}
}

\newcommand{\disjointinterleft}{
  \inferrule* [right=DisjointInter1]
    {\isdisjoint \Gamma A C \\ \isdisjoint \Gamma B C}
    {\isdisjoint \Gamma {A \& B} {C}}
}

\newcommand{\disjointinterright}{
  \inferrule* [right=DisjointInter2]
    {\isdisjoint \Gamma A B \\ \isdisjoint \Gamma A C}
    {\isdisjoint \Gamma {A} {B \& C}}
}

\newcommand{\disjointfun}{
  \inferrule* [right=DisjointFun]
    {\isdisjoint \Gamma B D}
    {\isdisjoint \Gamma {A \to B} {C \to D}}
}

\newcommand{\disjointforall}{
  \inferrule* [right=DisjointForall]
    {\isdisjoint \Gamma A C}
    {\isdisjoint \Gamma {\for {\alpha * B} A} {\for {\alpha * B} C}}
}

\newcommand{\disjointatomic}{
  \inferrule* [right=DisjointAtomic]
    {A \not\sim B}
    {\isdisjoint \Gamma {A} {B}}
}

\newcommand{\rulelabelSub}{\text{Sub\_}}
\newcommand{\rulelabelsubvar}{\rulelabelSub\text{Var}}
\newcommand{\rulelabelSubTop}{\rulelabelSub\text{Top}}
\newcommand{\rulelabelsubfun}{\rulelabelSub\text{Fun}}
\newcommand{\rulelabelsubforall}{\rulelabelSub\text{Forall}}
\newcommand{\rulelabelsubinter}{\rulelabelSub\text{And}}
\newcommand{\rulelabelsubinterl}{\rulelabelSub{\text{And}_1}}
\newcommand{\rulelabelsubinterr}{\rulelabelSub{\text{And}_2}}

\newcommand{\rulesubvar}{
\inferrule* [right=$\rulelabelsubvar$]
  { }
  {\alpha \subtype \alpha \yields {\lamty x {\im \alpha} x}}
}

\newcommand{\rulesubfun}{
\inferrule* [right=$\rulelabelsubfun$]
  {A_3 \subtype A_1 \yields {C_1} \\ A_2 \subtype A_4 \yields {C_2}}
  {A_1 \to A_2 \subtype A_3 \to A_4
  \yields
      {\lamty f {\im {A_1 \to A_2}}
      {\lamty x {\im {A_3}}
          {\app {C_2} {(\app f {(\app {C_1} x)})}}}}}
}

\newcommand{\rulesubforall}{
\inferrule* [right=$\rulelabelsubforall$]
  {A_1 \subtype \subst {\alpha_1} {\alpha_2} A_2 \yields C}
  {\for {\alpha_1} A_1 \subtype \for {\alpha_2} A_2
    \yields
      {\lamty f {\im {\for {\alpha_1} A_1}}
        {\blam \alpha {\app C {(\app f \alpha)}}}}}
}

\newcommand{\rulesubinter}{
\inferrule* [right=$\rulelabelsubinter$]
  {A_1 \subtype A_2 \yields {C_1} \\ A_1 \subtype A_3 \yields {C_2}}
  {A_1 \subtype A_2 \inter A_3
    \yields
      {\lamty x {\im {A_1}}
        {\pair {\app {C_1} x} {\app {C_2} x}}}}
}

\newcommand{\rulesubinterl}{
\inferrule* [right=$\rulelabelsubinterl$]
  {A_1 \subtype A_3 \yields C}
  {A_1 \inter A_2 \subtype A_3
    \yields
      {\lamty x {\im {A_1 \inter A_2}}
        {\app C {(\proj 1 x)}}}}
}

\newcommand{\rulesubinterr}{
\inferrule* [right=$\rulelabelsubinterr$]
  {A_2 \subtype A_3 \yields C}
  {A_1 \inter A_2 \subtype A_3
    \yields
      {\lamty x {\im {A_1 \inter A_2}}
        {\app C {(\proj 2 x)}}}}
}

% \newcommand{\rulelabel}{\text{Ty}}
\newcommand{\rulelabelSelect}{\text{Sel}}
\newcommand{\rulelabelRestrict}{\text{Res}}

% Var
% \newcommand{\rulelabelVar}{\rulelabel\text{Var}}
\newcommand{\tyvar} {
\inferrule* [right=TyVar]
  {x \hast A \in \Gamma}
  {\hastype \Gamma x A \yields x}
}

% Top
% \newcommand{\rulelabelTop}{\rulelabel\text{Top}}
\newcommand{\ruleTop} {
\inferrule* [right=TyTop]
  { }
  {\hastype \Gamma \top \top \yields {()}}
}

% Lam
% \newcommand{\rulelabelLam}{\rulelabel\text{Lam}}
\newcommand{\tylam} {
\inferrule* [right=TyLam]
  {\istype \Gamma A \\ \hastype {\Gamma, x \hast A} e B \yields E}
  {\hastype \Gamma {\lam x A e} {A \to B} \yields {\lam x {\im A} E}}
}

% App
% \newcommand{\rulelabelApp}{\rulelabel\text{App}}
\newcommand{\tyapp}{
\inferrule* [right=TyApp]
  {\hastype \Gamma {e_1} {A_1 \to A_2} \yields {E_1} \\
   \hastype \Gamma {e_2} {A_3} \yields {E_2} \\
   A_3 \subtype A_1 \yields C}
  {\hastype \Gamma {\app {e_1} {e_2}} {A_2} \yields {\app {E_1} {(\app C E_2)}}}
}

% BLam
% \newcommand{\rulelabelBLam}{\rulelabel\text{BLam}}
\newcommand{\tyblam}{
\inferrule* [right=TyBLam]
  {\hastype {\Gamma, \alpha * B} e A \yields E \\ \istype \Gamma B}
  {\hastype \Gamma {\blam {\alpha * B} e} {\for {\alpha * B} A} \yields {\blam \alpha E}}
}

% TApp
% \newcommand{\rulelabelTApp}{\rulelabel\text{TApp}}
\newcommand{\tytapp}{
\inferrule* [right=TyTApp]
  {\hastype \Gamma e {\for {\alpha * C} B} \yields E \\ \isdisjoint
  \Gamma A C \\\istype \Gamma A}
  {\hastype \Gamma {\tapp e A} {\subst A \alpha B} \yields {\tapp E {\im A}}}
}

% Merge
% \newcommand{\rulelabelMerge}{\rulelabel\text{Merge}}
\newcommand{\tymerge}{
\inferrule* [right=TyMerge]
  {\hastype \Gamma {e_1} A \yields {E_1} \\
   \hastype \Gamma {e_2} B \yields {E_2} \\
   % A \bot B}
   \isdisjoint \Gamma A B}
  {\hastype \Gamma {e_1 \mergeOp e_2} {A \intersect B} \yields {\pair {E_1} {E_2}}}
}

% ConstraintIntro
\newcommand{\rulelabelConstraintIntro}{\rulelabel\text{ConstraintIntro}}
\newcommand{\ruleConstraintIntro}{
  \inferrule* [right=$\rulelabelConstraintIntro$]
    {\istype \Gamma {A_1} \\ \istype \Gamma {A_2} \\
     \hastype {\Gamma, A_1 \disjoint A_2} e B \yields E}
    {\hastype \Gamma {\assume {(A_1 \disjoint A_1)} e} {\constraints {A_1   \disjoint A_2} B} \yields E}
}

% ConstraintElim
% \newcommand{\rulelabelConstraintElim}{\rulelabel\text{ConstraintElim}}
% \newcommand{\ruleConstraintElim}{
% \inferrule* [right=$\rulelabelConstraintElim$]
%   {\hastype \Gamma e {\constraints {A_1 \disjoint A_2} B} \yields E \\
%   \isdisjoint \Gamma {A_1} {A_2}}
%   {\hastype \Gamma {\app e {\_}} B \yields E}
% }

% rec-con
% \newcommand{\rulelabelRecConstruct}{\rulelabel\text{rec-construct}}
% \newcommand{\rulerecordConstruct}{
% \inferrule* [right=$\rulelabelRecConstruct$]
%   {\hastype \Gamma e A \yields E}
%   {\hastype \Gamma {\recordCon l e} {\recordType l A} \yields E}
% }

% rec-select
% \newcommand{\rulelabelRecSelect}{\rulelabel\text{rec-select}}
% \newcommand{\ruleRecSelect}{
% \inferrule* [right=$\rulelabelRecSelect$]
%   {\hastype \Gamma e A \yields E \\
%    \judgeSelect A l {A_1} \yields C}
%   {\hastype \Gamma {e.l} {A_1} \yields {\app C E}}
% }

% rec-restrict
% \newcommand{\rulelabelRecRestrict}{\rulelabel\text{rec-restrict}}
% \newcommand{\ruleRecRestrict}{
% \inferrule* [right=$\rulelabelRecRestrict$]
%   {\hastype \Gamma e A \yields E \\
%    \judgeRestrict A l {A_1} \yields C}
%   {\hastype \Gamma {e \restrictOp l} {A_1} \yields {\app C E}}
% }
%
% \newcommand{\judgeSelect}[3]{#1 \bullet #2 = #3}

% select
% \newcommand{\ruleGet}{
%   \inferrule* [right=$\rulelabelSelect$]
%   { }
%   {\judgeSelect {\recordType l A} l A \yields {\lam x {\im {\recordType l A}} x}}
% }

% select1
% \newcommand{\rulelabelSelectLeft}{{\rulelabelSelect}_1}
% \newcommand{\ruleGetLeft}{
%   \inferrule* [right=$\rulelabelSelectLeft$]
%   {\judgeSelect {A_1} l A \yields C}
%   {\judgeSelect {A_1 \intersect A_2} l A \yields {\lam x {\im {A_1
%           \intersect A_2}} {\app C {(\proj 1 x)}}}}
% }

% select2
% \newcommand{\rulelabelSelectRight}{{\rulelabelSelect}_2}
% \newcommand{\ruleGetRight}{
%   \inferrule* [right=$\rulelabelSelectRight$]
%   {\judgeSelect {A_2} l A \yields C}
%   {\judgeSelect {A_1 \intersect A_2} l A \yields {\lam x {\im {A_1
%           \intersect A_2}} {\app C {(\proj 2 x)}}}}
% }

% \newcommand{\judgeRestrict}[3]{#1 \bm{\restrictOp} #2 = #3}

% restrict
% \newcommand{\ruleRestrict}{
%   \inferrule* [right=$\rulelabelRestrict$]
%   { }
%   {\judgeRestrict {\recordType l A} l \top \yields {\lam x {\im {\recordType l A}} {()}}}
% }

% restrict1
% \newcommand{\rulelabelRestrictleft}{{\rulelabelRestrict}_1}
% \newcommand{\ruleRestrictLeft}{
%   \inferrule* [right=$\rulelabelRestrictleft$]
%   {\judgeRestrict {A_1} l A \yields C}
%   {\judgeRestrict {A_1 \intersect A_2} l {A \intersect A_2} \yields {\lam x {\im {A_1
%           \intersect A_2}} {\pair {\app C {(\proj 1 x)}} {\proj 2 x}}}}
% }

% restrict2
% \newcommand{\rulelabelRestrictRight}{{\rulelabelRestrict}_2}
% \newcommand{\ruleRestrictRight}{
%   \inferrule* [right=$\rulelabelRestrictRight$]
%   {\judgeRestrict {A_2} l A \yields C}
%   {\judgeRestrict {A_1 \intersect A_2} l {A_1 \intersect A} \yields {\lam x {\im {A_1
%           \intersect A_2}} {\pair {\proj 1 x} {\app C {(\proj 2 x)}}}}}
% }

\newcommand{\judgeTargetWF}[2]{#1 \; \textcolor{targetColor}{\turns} \; #2}
\newcommand{\judgeTarget}[3]{#1 \; \textcolor{targetColor}{\turns} \; #2 \; \textcolor{targetColor}{:} \; #3}
\newcommand{\ruleLabelTarget}{\bm{\textcolor{targetColor}{T}}}

\newcommand{\ruleLabelTargetvar}{\ruleLabelTarget\text{var}}
\newcommand{\ruleTargetVar} {
\inferrule* [right=$\ruleLabelTargetvar$]
  {(x,T) \in \Gamma}
  {\judgeTarget \Gamma x T}
}

\newcommand{\ruleLabelTargetUnit}{\ruleLabelTarget\text{unit}}
\newcommand{\ruleTargetUnit} {
\inferrule* [right=$\ruleLabelTargetUnit$]
  { }
  {\judgeTarget \Gamma {()} {()}}
}

\newcommand{\ruleLabelTargetlam}{\ruleLabelTarget\text{lam}}
\newcommand{\ruleTargetLam} {
\inferrule* [right=$\ruleLabelTargetlam$]
  {\judgeTarget {\Gamma, x \hast T} E {T_1} \andalso \judgeTargetWF \Gamma T}
  {\judgeTarget \Gamma {\lam x T E} {T \to T_1}}
}

\newcommand{\ruleLabelTargetApp}{\ruleLabelTarget\text{app}}
\newcommand{\ruleTargetApp}{
\inferrule* [right=$\ruleLabelTargetApp$]
  {\judgeTarget \Gamma {E_1} {T_1 \to T_2} \andalso \judgeTarget \Gamma {E_2} {T_1}}
  {\judgeTarget \Gamma {\app {E_1} {E_2}} {T_2}}
}

\newcommand{\ruleLabelTargetBLam}{\ruleLabelTarget\text{blam}}
\newcommand{\ruleTargetBLam}{
\inferrule* [right=$\ruleLabelTargetBLam$]
  {\judgeSource {\Gamma, \alpha} E T}
  {\judgeSource \Gamma {\blam \alpha E} {\for \alpha T}}
}

\newcommand{\ruleLabelTargetTApp}{\ruleLabelTarget\text{tapp}}
\newcommand{\ruleTargetTApp}{
\inferrule* [right=$\ruleLabelTargetTApp$]
  {\judgeTarget \Gamma E {\for \alpha {T_1}} \andalso \judgeTargetWF \Gamma T}
  {\judgeTarget \Gamma {\tapp E T} {\subst T \alpha T_1}}
}

\newcommand{\ruleLabelTargetPair}{\ruleLabelTarget\text{pair}}
\newcommand{\ruleTargetPair}{
\inferrule* [right=$\ruleLabelTargetPair$]
  {\judgeTarget \Gamma {E_1} {T_1} \andalso \judgeTarget \Gamma {E_2} {T_2}}
  {\judgeTarget \Gamma {\pair {E_1} {E_2}} {\pair {T_1} {T_2}}}
}

\newcommand{\ruleLabelTargetProjLeft}{\ruleLabelTarget\text{proj}_1}
\newcommand{\ruleTargetProjLeft}{
\inferrule* [right=$\ruleLabelTargetProjLeft$]
  {\judgeTarget \Gamma E {\pair {T_1} {T_2}}}
  {\judgeTarget \Gamma {\proj 1 E} {T_1}}
}

\newcommand{\ruleLabelTargetProjRight}{\ruleLabelTarget\text{proj}_2}
\newcommand{\ruleTargetProjRight}{
\inferrule* [right=$\ruleLabelTargetProjRight$]
  {\judgeTarget \Gamma E {\pair {T_1} {T_2}}}
  {\judgeTarget \Gamma {\proj 2 E} {T_2}}
}


\newcommand{\name}{{\bf $F_{\&}$}\xspace}

\newcommand{\target}{{\bf f}\xspace}
\newcommand{\Target}{{\bf f}\xspace}

\newcommand{\authornote}[3]{{\color{#2} {\sc #1}: #3}}
\newcommand\bruno[1]{\authornote{bruno}{red}{#1}}
\newcommand\george[1]{\authornote{george}{blue}{#1}}

\begin{document}

\special{papersize=8.5in,11in}
\setlength{\pdfpageheight}{\paperheight}
\setlength{\pdfpagewidth}{\paperwidth}

\conferenceinfo{CONF 'yy}{Month d--d, 20yy, City, ST, Country}
\copyrightyear{20yy}
\copyrightdata{978-1-nnnn-nnnn-n/yy/mm}
\doi{nnnnnnn.nnnnnnn}

\titlebanner{banner above paper title}        % These are ignored unless
\preprintfooter{\name}                        % 'preprint' option specified.

\title{Disjoint Intersection Types}
%%\subtitle{Subtitle Text, if any}

\authorinfo{Name1}
           {Affiliation1}
           {Email1}
\authorinfo{Name2\and Name3}
           {Affiliation2/3}
           {Email2/3}

\maketitle

\begin{abstract}

  Over the years there have been various proposals for \emph{design
    patterns} to improve \emph{extensibility} of programs.
  Examples include \emph{Object Algebras}, \emph{Modular Visitors} or
  Torgersen's design patterns using generics.
  Although those design patterns give practical
  benefits in terms of extensibility, they also expose limitations in
  existing mainstream OOP languages. Some pressing
  limitations are: 1) lack of good mechanisms for
  \emph{object-level} composition; 2) \emph{conflation of
    (type) inheritance with subtyping}; 3) \emph{heavy reliance on generics}.

  This paper presents System \name: an extension of System F with
  \emph{intersection types} and a \emph{merge operator}.  The goal of System \name
  is to study the minimal language constructs needed to support
  various extensible designs, while at the same time addressing the
  limitations of existing OOP languages. To address the lack of good
  object-level composition mechanisms, System \name uses the merge
  operator to do dynamic composition of values/objects. Moreover, in
  System \name type inheritance is independent of subtyping, and an
  extension can be a supertype of a base object type.  Finally, System
  \name replaces many uses of generics by intersection types or
  conventional subtyping. System \name is formalized and
  implemented. Moreover the paper shows how various extensible designs
  can be encoded in System \name.

\end{abstract}

%%\category{CR-number}{subcategory}{third-level}

% general terms are not compulsory anymore,
% you may leave them out
%%\terms
%%Design, Languages, Theory

%%\keywords
%%Intersecion Types, Polymorphism, Type System

\section{Introduction}

Dundfield's work showed how many language features can be encoded in terms
of intersection types with a merge operator. However two important
questions were left open by Dundfield:

\begin{enumerate}
\item How to allow coherent programs only?

\item If a restriction that allows coherent programs is in place, can
  all coherent programs conform to the restriction?
\end{enumerate}

In other words question 1) asks whether we can find sufficient
conditions to guarantee coherency; whereas question 2) asks
whether those conditions are also necessary. In terms of technical
lemmas that would correspond to:

\begin{enumerate}

\item Coherency theorem: $\Gamma \turns e : A \leadsto E_1 \wedge
  \Gamma \turns e : A \leadsto E_2~\to~E_1 = E_2$.

\item Completness of Coherency: ($\Gamma \turns_{old} e : A \leadsto E_1 \wedge
  \Gamma \turns_{old} e : A \leadsto E_2~\to~E_1 = E_2) \to \Gamma
  \turns e : A$.

\end{enumerate}

For these theorems we assume two type systems. On liberal type system
that ensures type-safety, but not coherence ($\Gamma \turns_{old} e :
A$); and another one that is both type-safe and coherent  ($\Gamma \turns e :
A$). What needs to be shown for completness is that if a coherent
program type-checks in the liberal type system, then it also
type-checks in the restricted system.


\special{papersize=8.5in,11in}
\setlength{\pdfpageheight}{\paperheight}
\setlength{\pdfpagewidth}{\paperwidth}

\title{\name}

% Coherence for well-typed terms.

% \begin{figure*}
%   \caption{Disjointness between types.}
% \end{figure*}

% \begin{figure*}
%   \colorlet{wfcolor}{BrickRed}


\newcommand{\rulelabelWF}{\bm{\textcolor{wfcolor}{wf}}}

\newcommand{\rulelabelWFVar}{\rulelabelWF\text{var}}
\newcommand{\rulelabelWFTop}{\rulelabelWF\text{top}}
\newcommand{\rulelabelWFFun}{\rulelabelWF\text{fun}}
\newcommand{\rulelabelWFForall}{\rulelabelWF\text{forall}}
\newcommand{\rulelabelWFAnd}{\rulelabelWF{\text{and}}}
\newcommand{\rulelabelWFRec}{\rulelabelWF\text{rec}}

\newcommand{\ruleWF}{
\inferrule* [right=$\rulelabelWF$]
  {\ftv \A \in \gamma}
  {\istype \gamma \A}
}

\newcommand{\ruleLabelTargetWF}{\ruleLabelTarget\text{wf}}
\newcommand{\ruleTargetWF}{
\inferrule* [right=$\ruleLabelTargetWF$]
  {\ftv T \in \Gamma}
  {\judgeTargetWF \Gamma T}
}

% Expanded form of well-formedness

\newcommand{\ruleWFVar}{
\inferrule* [right=$\rulelabelWFVar$]
  {\alpha \in \gamma}
  {\istype \gamma \alpha}
}

\newcommand{\ruleWFTop}{
\inferrule* [right=$\rulelabelWFTop$]
  { }
  {\istype \gamma \top}
}

\newcommand{\ruleWFFun}{
\inferrule* [right=$\rulelabelWFFun$]
  {\istype \gamma {\A_1} \\ \istype \Gamma {\A_2}}
  {\istype \gamma {\A_1 \to \A_2}}
}

\newcommand{\ruleWFForall}{
\inferrule* [right=$\rulelabelWFForall$]
  {\istype {\gamma, \alpha} \A}
  {\istype \gamma {\for \alpha \A}}
}

\newcommand{\ruleWFAnd}{
\inferrule* [right=$\rulelabelWFAnd$]
  {\istype \gamma {\A_1} \\ \istype \Gamma {\A_2}}
  {\istype \gamma {\A_1 \intersect \A_2}}
}

\newcommand{\ruleWFRec}{
\inferrule* [right=$\rulelabelWFRec$]
  {\istype \gamma \A}
  {\istype \gamma {\recordType l \A}}
}

%   \caption{Well-formedness of types.}
% \end{figure*}

% \begin{figure*}
% \begin{mathpar}
% \begin{array}{l}
%   \begin{array}{llrl}
%     \text{Values} & v & \Coloneqq & \top \mid \lam x \tau e \mid \blam \alpha e \mid v_1 \mergeOp v_2 \mid \recordCon l e
%   \end{array}
% \end{array}
% \end{mathpar}
%
%   \caption{Values.}
% \end{figure*}

% \begin{figure*}

%   \begin{mathpar}
%     \begin{array}{lcl}
%       \fields {v_1 \mergeOp v_2} &=& \fields {v_1} \concatOp \fields {v_2} \\
%       \fields {\recordCon l e}   &=& [(l, e)] \\
%       \fields v                  &=& []
%     \end{array}
%   \end{mathpar}
%   \caption{\code{fields}.}
% \end{figure*}

% \begin{figure*}
%   \begin{mathpar}
%     \begin{array}{lcl}
%       \remove {\recordCon l e} l &=& \top \\
%       \remove {\recordCon l e \mergeOp v_2} l &=& v_2 \\
%       \remove {\recordCon l e \mergeOp v_2} {l'} &=& \recordCon l e \mergeOp \remove {v_2} {l'} \quad \quad (l \neq l') \\
%       \remove {v_1 \mergeOp \recordCon l e} l &=& v_1 \\
%       \remove {v_1 \mergeOp \recordCon l e} {l'} &=& \remove {v_1} {l'} \mergeOp \recordCon l e \quad \quad (l \neq l') \\

%       \remove v l                  &=& v
%     \end{array}

%   \end{mathpar}

%   \caption{\code{remove}.}
% \end{figure*}

% \begin{figure*}
%   \begin{mathpar}
%     \inferrule* [right=Cast/UpCast]
%       {\tau_1 \subtype \tau}
%       {\cast \tau {\withType v {\tau_1}} \hookrightarrow v}
%
%     \inferrule* [right=Cast/TakeLeft]
%       {\cast \tau {\withType {v_1} {\tau_1}} \hookrightarrow v}
%       {\cast \tau {\withType {v_1 \mergeOp v_2} {\tau_1 \intersect \tau_2}} \hookrightarrow v}
%
%     \inferrule* [right=Cast/TakeRight]
%       {\cast \tau {\withType {v_2} {\tau_2}} \hookrightarrow v}
%       {\cast \tau {\withType {v_1 \mergeOp v_2} {\tau_1 \intersect \tau_2}} \hookrightarrow v}
%   \end{mathpar}
%
%   \caption{Casts.}
% \end{figure*}

% \begin{figure*}
%   \begin{mathpar}
%     \inferrule* [right=Dyn/Val]
%       { }
%       {v \Downarrow v}
%
%     \inferrule* [right=Dyn/App]
%       {e_1 \Downarrow \lam x \tau e \\
%        e_2 \Downarrow v_2 \\
%        \cast \tau {\withType {v_2} {\tau_2}} \hookrightarrow v_3 \\
%        \subst {v_3} x e \Downarrow v}
%       {\app {e_1} {\withType {e_2} {\tau_2}} \Downarrow v}
%
%     \inferrule* [right=Dyn/TApp]
%       {e_1 \Downarrow \for \alpha e \\
%        \subst \tau \alpha e \Downarrow v}
%       {\tapp {e_1} \tau \Downarrow v}
%
%     \inferrule* [right=Dyn/Merge]
%       {e_1 \Downarrow v_1 \\ e_2 \Downarrow v_2}
%       {e_1 \mergeOp e_2 \Downarrow v_1 \mergeOp v_2}
%
%     % \inferrule* [right=Dyn/RecSelect]
%     %   {e \Downarrow v \\
%     %    (l, e_1) \; \code{`uniqueElem`} \; \fields v \\
%     %    e_1 \Downarrow v_1}
%     %   {e.l \Downarrow v_1}
%
%     % \inferrule* [right=Dyn/RecRestrict]
%     %   {e \Downarrow v \\
%     %    (l, e_1) \; \code{`uniqueElem`} \; \fields v}
%     %   {e \restrictOp l \Downarrow v \; \code{`remove`} \; l}
%   \end{mathpar}
%
%   \caption{Dynamic semantics.}
% \end{figure*}
%
% \begin{figure*}
%   \framebox{$\im \tau = T$}

\begin{align*}
  \im \alpha                    &= \alpha \\
  \im \top                      &= () \\
  \im {\tau_1} \to \im {\tau_2} &= \im {\tau_1} \to \im {\tau_2} \\
  \im {\for \alpha \tau}        &= \for \alpha \im \tau \\
  \im {\tau_1 \intersect \tau_2} &= \pair {\im {\tau_1}} {\im {\tau_2}} \\
\end{align*}

%   \caption{Type translation.}
% \end{figure*}

\begin{figure}
  \begin{mathpar}
    \framebox{$\isatomic A$} \\

    \inferrule*
      {}
      {\isatomic \bot}

    \inferrule*
      {}
      {\isatomic {A \to B}}

    \inferrule*
      {}
      {\isatomic {\for {\alpha \disjoint B} A}}

  \end{mathpar}
  \caption{Atomic types.}
\end{figure}

\subsection{``Testsuite'' of examples}

\begin{enumerate}

\item $\lambda (x : Int * Int). (\lambda (z : Int) . z)~x$: This
  example should not type-check because it leads to an ambigous choice
  in the body of the lambda. In the current system the well-formedness
  checks forbid such example.

\item $\Lambda A.\Lambda B.\lambda (x:A).\lambda (y:B). (\lambda (z:A)
  . z) (x,,y)$: This example should not type-check because it is not
  guaranteed that the instantiation of A and B produces a well-formed
  type. The TyMerge rule forbids it with the disjointness check.

\item $\Lambda A.\Lambda B * A.\lambda (x:A).\lambda (y:B). (\lambda
  (z:A) . z) (x,,y)$: This example should type-check because B is
  guaranteed to be disjoint with A. Therefore instantiation should
  produce a well-formed type.

\item $(\lambda (z:Int) . z) ((1,,'c'),,(2,False))$: This example
  should not type-check, since it leads to an ambigous lookup of
  integers (can either be 1 or 2). The definition of disjointness is
  crutial to prevent this example from type-checking. When
  type-checking the large merge, the disjointness predicate will
  detect that more than one integer exists in the merge.

\item $(\lambda (f: Int \to Int \& Bool) . \lambda (g : Int \to Char \& Bool) . ((f,,g) : Int \to Bool)$:
  This example
  should not type-check, since it leads to an ambigous lookup of
  functions. It shows that in order to check disjointness
  of functions we must also check disjointness of the subcomponents.

\item $(\lambda (f: Int \to Int) . \lambda (g : Bool \to Int) . ((f,,g) : Bool \& Int \to Int)$:
  This example shows that whenever the return types overlap, so does the function type:
  we can always find a common subtype for the argument types.
\end{enumerate}

\subsection{Achieving coherence}

The crutial challenge lies in the generation of coercions, which can lead
to different results due to multiple possible choices in the rules that
can be used. In particular the rules SubAnd1 and SubAnd2 overlap and
can result in coercions that are not equivalent. A simple example is:

$(\lambda (x:Int) . x) (1,,2)$

The result of this program can be either 1 or 2 depending on whether
we chose SubAnd1 or SubAnd2.

Therefore the challenge of coherence lies in ensuring that, for any given
types A and B, the result of $A <: B$ always leads to the same (or semantically
equivalent) coercions.

It is clear that, in general, the following does not hold:

$if~A <: B \leadsto C1~and~A <: B \leadsto C2~then~C1 = C2$

We can see this with the example above. There are two possible coercions:\\

\noindent $(Int\&Int) <: Int \leadsto \lambda (x,y). x$\\
$(Int\&Int) <: Int \leadsto \lambda (x,y). y$\\

However $\lambda (x,y). x$ and $\lambda (x,y). y$ are not semantically equivalent.

One simple observation is that the use of the subtyping relation on the
example uses an ill-formed type ($Int\&Int$). Since the type system can prevent
such bad uses of ill-formed types, it could be that if we only allow well-formed
types then the uses of the subtyping relation do produce equivalent coercions.
Therefore the we postulate the following conjecture:

$if~A <: B \leadsto C1~and~A <: B \leadsto C2~and~A, B~well~formed~then~C1 = C2$

If the following conjecture does hold then it should be easy to prove that
the translation is coherent.

% \begin{mathpar}
%   \inferrule
%   {}
%   {\hastype \epsilon {1 \mergeOp 2} {\constraints {\integer \disjoint \integer} \integer \intersect \integer}}
% \end{mathpar}

% \begin{definition}{(Disjointness)}
% Two sets $S$ and $T$ are \emph{disjoint} if there does not exist an element $x$, such that $x \in S$ and $x \in T$.
% \end{definition}

% \begin{definition}{(Disjointness)}
% Two types $A$ and $B$ are \emph{disjoint} if there does not exist an expression $e$, which is not a merge, such that $\hastype \epsilon e A'$, $\hastype \epsilon e B'$, $A' \subtype A$, and $B' \subtype B$.
% \end{definition}

% \begin{figure}
%   % Typing
%   \begin{mathpar}
%     \framebox{$ \hastype \Gamma e A \yields E $} \\
%     \tyvar \and
%     \ruleTop \and
%     \tylam \and
%     \tyapp \and
%     \tyblam \and
%     \tytapp \and
%     \tymerge \and
%     \ruleDisjointAssume \and
%     \ruleDisjointCheck
%     % \rulerecordConstruct \and
%     % \ruleRecSelect \and \ruleRecRestrict
%   \end{mathpar}
%
%   % % Selection
%   % \begin{mathpar}
%   %   \framebox{$\judgeSelect {\tau_1} l \tau_2 \yields C$} \and
%   %   \ruleGet \and \ruleGetLeft \and \ruleGetRight
%   % \end{mathpar}
%   %
%   % % Restriction
%   % \begin{mathpar}
%   %   \framebox{$\judgeRestrict {\tau_1} l \tau_2 \yields C$} \and
%   %   \ruleRestrict \and \ruleRestrictLeft \and \ruleRestrictRight
%   % \end{mathpar}
%
%   \caption{Disjointness.}
% \end{figure}

\section{Introduction}

The benefit of a merge, compared to a pair, is that you don't need to explicitly extract an item out. For example, \lstinline@fst (1,'c')@

\begin{definition}{Determinism}
If $e : \tau_1 \hookrightarrow E_1$ and $e : \tau_2 \hookrightarrow E_2$,
then $\tau_1 = \tau_2$ and $E_1 = E_2$.
\end{definition}

\emph{Coherence} is a property about the relation between syntax and semantics. We say a semantics is \emph{coherent} if the syntax of a term uniquely determines its semantics.

\begin{definition}{Coherence}
If $e_1 : \tau_1 \hookrightarrow E_1$ and $e_2 : \tau_2 \hookrightarrow E_2$,
$E_1 \Downarrow v_1$ and $E_2 \Downarrow v_2$,
then $v_1 = v_2$.
\end{definition}

\begin{definition}{Disjointness}
Two types $A$ and $B$ are \emph{disjoint} (written as ``$\disjoint A B$'') if there does not exist a type $C$ such that $C \subtype A$ and $C \subtype B$ and $C \subtype A \intersect B$.
\end{definition}

\subsection{Equational reasoning}

We can define a \code{fst} function that extracts the first item of a merged value:
\begin{lstlisting}
let fst A B (x : A & B) = (\(y : A). y) x in ...
\end{lstlisting}
Then we have the following equational reasoning:
\begin{lstlisting}
fst Int Int (2,,3)
(\(y : Int). y) (2,,3)
\end{lstlisting}

\subsection{Discussion}

In our type-directed translation, some inference rules return conclusions having
\emph{the same constructor}. This phenomenon makes the translation
nondeterministic. As an example,

\begin{lstlisting}
({x=1},,{x=2}).x
\end{lstlisting}

can evaluate to either \lstinline@1@ or \lstinline@2@ (according their
translation in the target language). In this case, the constructor is the
intersection operator, for which both rules, (select1) and (select2), are
applicable.

One remedy, which you may have realised, is to enforce the order of applying
rules. Whenever the case as shown above happens, the right component of
\lstinline@&@ and \lstinline@,,@ will take precedence. In other words, the
(select2) rule is tried first. Only if (select2) fails, the (select1) rule is
tried. Therefore, \lstinline@({x=1},,{x=2}).x@ can only evaluate to 2. Likewise,
\lstinline@({x=1},,{x="hi"}).x@ will evaluate to \lstinline@"hi"@ and will be of
type \lstinline@String@. Generally, three pairs of rules in our system that
cause nondeterminism can all be implemented in the same fashion (sub-and2 is
favored over sub-and1), and (restrict2 is favored over restrict1).

This approach seem works fine until you think about how it interact with
parametric polymorphism.

\begin{lstlisting}
(/\A. \(x:A&Int). x) Int (1,,2) + 1
\end{lstlisting}

If we would like to have a deterministic elaboration result, another idea is to
tweak the rules a little bit so that given a term, it is no longer possible that
both of the twin rules described above can be used. For example, if
$\tau_1 \intersect \tau_2 \subtype
\tau_3$, we would like to be certain that either $\tau_1 \subtype
\tau_3$ holds or $\tau_2 \subtype \tau_3$ holds, but not both.

Formally, we can state this theorem as:

\begin{theorem}
  If $\tau_1$, $\tau_2$, $\tau_3$, and $\tau_1 \intersect \tau_2$ are well-formed
  types, and $\tau_1 \intersect \tau_2 \subtype \tau_3$, then $\tau_1 \intersect \tau_3$
  \emph{exclusive} or $\tau_2 \intersect \tau_3$.
\end{theorem}

Note that $A$ \emph{exclusive} or $B$ is true if and only if their truth value
differ. Next, we are going to investigate the minimal requirement (necessary and
sufficient conditions) such that the theorem holds.

If $\tau_1$ and $\tau_2$ in this setting are the same, for example,
$\code{Int} \intersect \code{Int} \subtype \code{Int}$, obviously the theorem will
not hold since both the left $\code{Int}$ and the right $\code{Int}$ are a
subtype of $\code{Int}$.

If our types include primitive subtyping such as
$\code{Nat} \subtype_\text{prim} \code{Int}$ (a natural number is also an
integer), which can be promoted to the normal subtyping with this rule:
\begin{mathpar}
  \inferrule
  {\tau_1 \subtype_\text{prim} \tau_2}
  {\tau_1 \subtype \tau_2}
\end{mathpar}
the theorem will also not hold because
$\code{Int} \intersect \code{Nat} \subtype \code{Int}$ and yet
$\code{Int} \subtype \code{Int}$ and $\code{Nat} \subtype \code{Int}$.

We can try to rule out such possibilities by making the requirement of
well-formedness stronger. This suggests that the two types on the sides of
$\intersect$ should not ``overlap''. In other words, they should be ``disjoint''. It
is easy to determine if two base types are disjoint. For example, $\code{Int}$
and $\code{Int}$ are not disjoint. Neither do $\code{Int}$ and $\code{Nat}$.
Also, types built with different constructors are disjoint. For example,
$\code{Int}$ and $\code{Int} \to \code{Int}$. For function types, disjointness
is harder to visualise. But bear in the mind that disjointness can defined by
the very requirement that the theorem holds.

We shall give two semantics and show the two are the same.

\begin{itemize}
\item an type-directed semantics
\item a direct operational semantics
\end{itemize}

say the example above:

without the cast, you could either get:
1,,'c'
or
1
depending on what rules you use

but I think with your change, you can only get the first

(which is what we want)

let me see how we can get `1` before the change

\begin{mathpar}

\end{mathpar}

% (Int & Char) (1 : Int) ~> 1
% ----------------------------------------------
% (Int & Char) ((1 ,, 'c') : Int & Char) ~> 1

With the change, we need $\code{Int} \subtype \code{Int} \intersect \code{Char}$ to
hold in order to get the premise, which does not. So it can be shown that
$(\code{Int} \intersect \code{Char}) ((1 \mergeOp 'c') : \code{Int} \intersect
\code{Char}) \hookrightarrow 1$ is not derivable.

\section{Introduction}

There has been a remarkable number of works aimed at improving support
for extensibility in programming languages. These works include:
visions of new programming models~\cite{}; new programming languages or
language extensions~\cite{}, and \emph{design patterns} that can be
used with existing mainstream languages~\cite{}.

%\cite{family polymorphism and virtual
%classes.}. Another line of work are proposals for precise formal models or new 
%programming languages. Yet another line are \emph{design patterns}
%that can be used  with existing mainstream languages. 
%%Part of the motivation behind 

Some of the more recent work on extensibility is focused on various
proposals for design patterns.  Examples include \emph{Object
  Algebras}~\cite{}, \emph{Modular Visitors}~\cite{} or
Torgersen's~\cite{} four design patterns using generics. In those
approaches the idea is to use some advanced (but already available)
features, such as \emph{generics}, in combination with conventional
OOP features to model more extensible designs.  Those designs work in
modern OOP languages such as Java, C\# or Scala.

Although such design patterns give practical benefits in terms of
extensibility, they also expose limitations in existing mainstream OOP
languages. In particular there are three pressing limitations: 
1) lack of good mechanisms for
  \emph{object-level} composition; 2) \emph{conflation of 
    (type) inheritance with subtyping}; 3) \emph{heavy reliance on generics}.

  The first limitation shows up, for example, in Oliveira et
  al.~\cite{} encodings of Feature-Oriented Programming using Object
  Algebras~\cite{}. These programs are best expressed using a form of
  \emph{type-safe}, \emph{dynamic}, \emph{delegation}-based
  composition. Although such form of composition can be encoded in
  languages like Scala, it requires the use of low-level reflection
  techniques, such as dynamic proxies, reflection or other forms of
  meta-programming~\cite{}. It is clear that better language support
  would be desirable.

  The second limitation shows up in designs for modelling
  modular or extensible visitors~\cite{}.  The vast majority of modern
  OOP languages combines type inheritance and subtyping. 
  That is a type extension induces a subtype. However
  as Cook et al.~\cite{} famously argued there are programs where
  ``\emph{subtyping is not inheritance}''. Interestingly previously
  not many practical programs have been reported in the literature
  where the distinction between subtyping and inheritance is
  relevant. However, as shown in this paper, it turns out that this
  difference does show up in practice when designing modular
  (extensible) visitors.  We believe that modular visitors provide a
  compeling practical example where inheritance and subtyping should
  not be conflated!

  Finally, the third limitation is prevalent in many extensible
  designs~\cite{}. Such designs rely on advanced features of generics,
  such as \emph{f-bounded polymorphism}~\cite{}, \emph{variance
    annotations}~\cite{}, \emph{wildcards}~\cite{} and/or \emph{higher-kinded
    types}~\cite{} to achieve type-safety. Sadly, the amount of
  type-annotations, combined with the lack of understanding of these
  features, usually deters programmers from using such designs.

\begin{comment}
Motivated by the insights gained in previous work, this paper presents 
a minimal core calculus that addresses current limitations and
provides a better foundational model for statically typed
delegation-based OOP? We show that Object Algebras fit nicely in this
model. 
\end{comment}

This paper presents System \name: an extension of System F~\cite{}
with intersection types and a merge operator~\cite{}.  The goal of
System \name is to study the \emph{minimal} foundational language
constructs that are needed to support various extensible designs,
while at the same time addressing the limitations of existing OOP
languages. To address the lack of good object-level composition
mechanisms, System \name uses the merge operator to allow dynamic
composition of values/objects. Moreover, in System \name (type-level)
extension is independent of subtyping, and it is possible for an
extension to be a supertype of a base object type. Furthermore,
intersection types and conventional subtyping can be used in many
cases instead of advanced features of generics. Indeed this paper 
shows how many previous designs in the literature can be encoded 
without such advanced features of generics.


Technically speaking System \name is mainly inspired by the work of
Dundfield~\cite{}.  Dundfield shows how to model a simply typed
calculus with intersection types and a merge operator. The presence of
a merge operator adds significant expressiveness to the language,
allowing encodings for many other language constructs as syntactic
sugar. System \name differs from Dundfield's work in a few
ways. Firstly it adds parametric polymorphism and formalizes a
extension for records to support a basic form of objects. Secondly,
the elaboration semantics into System F is done directly from the
source calculus with subtyping. In contrast Dunfield has an additional
step which eliminates subtyping.  Finally a non-technical difference
is that System \name is aimed at studying issues of OOP languages and
extensibility, whereas Dunfield's work was aimed at Functional
Programming and he did not consider applications to extensibility.
Like many other foundational formal models for OOP (for
example~\cite{}), System \name is purely functional and it uses
structural typing.

%%System \name is
%%formalized and implemented. Furthermore the paper illustrates how
%%various extensible designs can be encoded in System \name.

\begin{comment}
We present a polymorphic calculus containing intersection types and records, and show
how this language can be used to solve various common tasks in functional
programming in a nicer way.Intersection types provides a power mechanism for functional programming, in
particular for extensibility and allowing new forms of composition.

Prototype-based programming is one of the two major styles of object-oriented
programming, the other being class-based programming which is featured in
languages such as Java and C\#. It has gained increasing popularity recently
with the prominence of JavaScript in web applications. Prototype-based
programming supports highly dynamic behaviors at run time that are not possible
with traditional class-based programming. However, despite its flexibility,
prototype-based programming is often criticized over concerns of correctness and
safety. Furthermore, almost all prototype-based systems rely on the fact that
the language is dynamically typed and interpreted.
\end{comment}

In summary, the contributions of this paper are:

\begin{itemize}

\item {\bf A Minimal Core Language for Extensibility:} This paper
  identifies a minimal core language, System \name, capable of
  expressing various extensibility designs in the literature.
  System \name also addresses limitations of existing OOP
  languages that complicate extensible designs. 
  
\item {\bf Formalization of System \name:} An elaboration semantics of
  System \name into System F is given, and type-soundness is proved.

\item {\bf Encodings of Extensible Designs:} Various encodings of
  extensible designs into System \name, including \emph{Object
    Algebras} and \emph{Modular Visitors}. 

\item {\bf A Practical Example where ``Inheritance is not Subtyping''
    Matters:} This paper shows that in modular/extensible visitors
  suffer from the ``inheritance is not subtyping problem''. Moreover 
  with extensible visitors the extension should become a
  \emph{supertype}, not a subtype. \bruno{extension with accept method}

\item {\bf Implementation and Examples:} An implementation of an
  extension of System \name, as well as the examples presented in the
  paper, are publicly available. 

\begin{comment}

\item{elaboration typing rules which given a source expression with intersection
    types, typecheck and translate it into an ordinary F term. Prove a type
    preservation result: if a term $ e $ has type $ \tau $ in the source language,
    then the translated term $ \image e $ is well-typed and has type $ \image \tau $ in the
    target language.}

\item{present an algorithm for detecting incoherence which can be very important
    in practice.}

\item{explores the connection between intersection types and object algebra by
    showing various examples of encoding object algebra with intersection
    types.}

\end{comment}

\end{itemize}

\begin{comment}
\subsection{Other Notes}

finitary overloading: yes
but have other merits of intersection been explored?

-- Compare Scala:
-- merge[A,B] = new A with B

-- type IEval  = { eval :  Int }
-- type IPrint = { print : String }

-- F[\_]
\end{comment}
%*******************************************************************************
\section{Overview} \label{sec:overview}
%*******************************************************************************

\bruno{Be careful when using the word ``class'': we don't have classes in our system;
so talking about classes may simply confuse readers. You can talk about classes simply to say
that traits provide an alternative to classes. Often when you write ``class'' in the text, what
you mean is ``object``}

This section introduces \name and its support for intersection types and the
merge operator. It then discusses the issue of coherence and shows how the
notion of disjoint intersection types achieve a coherent semantics.

Note that this section uses some syntactic sugar, as well as standard
programming language features, to illustrate the various concepts in
\name. Although the minimal core language that we formalize in
Section~\ref{sec:fi} does not present all such features, our implementation
supports them.

\subsection{Intersection Types and the Merge Operator}
%%\subsection{Intersection Types, Merge and Polymorphism in \name}

Intersection types date back as early as Coppo et
al.'s work~\cite{coppo1981functional}. Since then various researchers have
studied intersection types, and some languages have adopted them in one
form or another.
%However, as we shall see in
%Section~\ref{subsec:incoherence}, it also introduces difficulties. In what follows
%intersection types and the merge operator are informally introduced.

\paragraph{Intersection types.}
The intersection of type $A$ and $B$ (denoted as \lstinline{A & B} in
\name) contains exactly those values
which can be used as either values of type $A$ or of type $B$. For instance,
consider the following program in \name:

\begin{lstlisting}
let x : Int & Char = (*$ \ldots $*) in -- definition omitted
let idInt (y : Int) : Int = y in
let idChar (y : Char) : Char = y in
(idInt x, idChar x)
\end{lstlisting}

\noindent If a value \lstinline{x} has type \lstinline{Int & Char} then
\lstinline{x} can be used as an integer or as a character. Therefore,
\lstinline{x} can be used as an argument to any function that takes
an integer as an argument, or any
function that take a character as an argument. In the program above
the functions \lstinline{idInt} and \lstinline{idChar} are the
identity functions on integers and characters, respectively.
Passing \lstinline{x} as an argument to either one (or both) of the
functions is valid.

\paragraph{Merge operator.}
In the previous program we deliberately did not show how to introduce values of
an intersection type. There are many variants of intersection types in the
literature. Our work follows a particular formulation, where intersection types
are introduced by a \emph{merge operator}. As
Dunfield~\cite{dunfield2014elaborating} has argued a merge operator adds
considerable expressiveness to a calculus. The merge operator allows two values
to be merged in a single intersection type. For example, an implementation of
\lstinline{x} is constructed in \name as follows:

\begin{lstlisting}
let x : Int & Char = 1,,'c' in (*$ \ldots $*)
\end{lstlisting}

\noindent In \name (following Dunfield's notation), the
merge of two values $v_1$ and $v_2$ is denoted as $v_1 ,, v_2$.

\paragraph{Merge operator and pairs.}
The merge operator is similar to the introduction construct on pairs.
An analogous implementation of \lstinline{x} with pairs would be:

\begin{lstlisting}
let xPair : (Int, Char) = (1, 'c') in (*$ \ldots $*)
\end{lstlisting}

\noindent The significant difference between intersection types with a
merge operator and pairs is in the elimination construct. With pairs
there are explicit eliminators (\lstinline{fst} and
\lstinline{snd}). These eliminators must be used to extract the
components of the right type. For example, in order to use
\lstinline{idInt} and \lstinline{idChar} with pairs, we would need to
write a program such as:

\begin{lstlisting}
(idInt (fst xPair), idChar (snd xPair))
\end{lstlisting}

\noindent In contrast the elimination of intersection types is done
implicitly, by following a type-directed process. For example,
when a value of type \lstinline{Int} is needed, but an intersection
of type \lstinline{Int & Char} is found, the compiler uses the
type system to extract the corresponding value.

\subsection{Incoherence}\label{subsec:incoherence}
Unfortunately the implicit nature of elimination for intersection
types built with a merge operator can lead to incoherence.
The merge operator combines two terms, of type $A$ and $B$
respectively, to form a term of type $A \inter B$. For example,
$1 \mergeop `c'$ is of type $\code{Int} \inter \code{Char}$. In this case, no
matter if $1 \mergeop `c'$ is used as $\code{Int}$ or $\code{Char}$, the result
of evaluation is always clear. However, with overlapping types, it is
not straightforward anymore to see the result. For example, what
should be the result of this program, which asks for an integer out of
a merge of two integers:
\begin{lstlisting}
(fun (x: Int) (*$ \to $*) x) (1,,2)
\end{lstlisting}
Should the result be \lstinline$1$ or \lstinline$2$?

If both results are accepted, we say that the semantics is \emph{incoherent}:
there are multiple possible meanings for the same valid program. Dunfield's
calculus~\cite{dunfield2014elaborating} is incoherent and accepts the program
above.

\paragraph{Getting around incoherence: biased choice.}
In a real implementation of Dunfield calculus a choice has to be made
on which value to compute. For example, one potential option is to
always take the left-most value matching the type in the
merge. Similarly, one could always take the right-most
value matching the type in the merge. Either way, the meaning
of a program will depend on a biased implementation choice,
which is clearly unsatisfying from the theoretical point of view
(although perhaps acceptable in practice).

\subsection{Restoring Coherence: Disjoint Intersection Types}\label{sec:restoring}
Coherence is a desirable property for a semantics. A semantics is said
to be coherent if any \emph{valid program} has exactly one
meaning~\cite{reynolds1991coherence} (that is, the semantics is not ambiguous).
One option to restore coherence is to reject programs which may have
multiple meanings.
%Of course, when rejecting programs it is important
%not to be too conservative, and reject too many programs which are
%actually coherent.
Analyzing the expression $1 \mergeop 2$, we can see that the reason
for incoherence is that there are multiple, overlapping, integers in the
merge. Generally speaking, if both terms can be assigned some type $C$,
both of them can be chosen as the meaning of the merge,
which leads to multiple meanings of a term.
Thus a natural option is to try to forbid such overlapping
values of the same type in a merge.

This is precisely the approach taken in \name. \name requires that the
two types of in intersection must be \emph{disjoint}.  However,
although disjointness seems a natural restriction to impose on
intersection types, it is not obvious to formalize it. Indeed Dunfield
has mentioned disjointness as an option to restore coherence, but he
left it for future work due to the non-triviality of the approach.

\paragraph{Searching for a definition of disjointness.}
The first step towards disjoint intersection types is to come up
with a definition of disjointness. A first attempt at such definition would
be to require that, given two types $A$ and $B$, both types are not
subtypes of each other. Thus, denoting disjointness as $A * B$, we would have:
\[A * B \equiv A \not<: B \wedge B \not<: A\]
At first sight this seems a reasonable definition and it does prevent
merges such as \lstinline{1,,2}. However some moments of thought are enough to realize that
such definition does not ensure disjointness. For example, consider
the following merge:

\begin{lstlisting}
(1,,'c') ,, (2,,True)
\end{lstlisting}

\noindent This merge has two components which are also intersection
types. The first component \lstinline{(1,,'c')} has type $\code{Int} \inter
\code{Char}$, whereas the second component \lstinline{(2 ,, True)} has type
$\code{Int} \inter \code{Bool}$. Clearly,
\[ \code{Int} \inter \code{Char} \not \subtype \code{Int} \inter \code{Bool} \wedge \code{Int} \inter \code{Bool} \not \subtype \code{Int} \inter \code{Char} \]
Nevertheless the following program still leads to
incoherence:
\begin{lstlisting}
(fun (x: Int) (*$ \to $*) x) ((1,,'c'),,(2,,True))
\end{lstlisting}
as both \lstinline{1} or \lstinline{2} are possible outcomes
of the program. Although this attempt to define disjointness failed,
it did bring us some additional insight: although the types of the two
components of the merge are not subtypes of each other, they share
some types in common.

\paragraph{A proper definition of disjointness.} In order for two types
to be truly disjoint, they must not have any subcomponents sharing
the same type. In a system with intersection types this can be ensured
by requiring the two types do not share a common supertype. The
following definition captures this idea more formally.

\begin{definition}[Disjointness]
  Given two types $A$ and $B$, two types are disjoint
  (written $A \disjoint B$) if there is no type $C$ such that both $A$ and $B$ are
  subtypes of $C$:
  \[A \disjoint B \equiv \not\exists C.~A \subtype C \wedge B \subtype C\]
\end{definition}

\noindent This definition of disjointness prevents the problematic
merge. Since $\code{Int}$ is a common supertype of both $\code{Int} \& \code{Char}$ and
$\code{Int} \& \code{Bool}$, those two types are not disjoint.

\name's type system only accepts programs that use disjoint intersection
types. As shown in Section~\ref{sec:disjoint} disjoint intersection types will
play a crucial rule in guaranteeing that the semantics is coherent.

% \subsection{Parametric Polymorphism and Intersection Types}\label{subsec:polymorphism}
% Before we show how \name extends the idea of disjointness to parametric
% polymorphism, we discuss some non-trivial issues that arise from
% the interaction between parametric polymorphism and intersection types.
%The interaction between parametric polymorphism and
%intersection types when coherence is a goal is non-trivial.
%In particular biased choice .
%The key challenge is to have a type
%system that still ensures coherence, but at the same time is not too
%restrictive in the programs that can be accepted.
% Dunfield~\cite{} provides a
% good illustrative example of the issues that arise when combining
% disjoint intersection types and parametric polymorphism:
% \[\lambda x. {\bf let}~y = 0 \mergeop x~{\bf in}~x\]
% Consider the attempt to write
% the following polymorphic function in \name (we use
% uppercase Latin letters to denote type variables):
% \begin{lstlisting}
% let fst A B (x: A & B) = (fun (z:A) (*$ \to $*) z) x in (*$ \ldots $*)
% \end{lstlisting}
% The
% \code{fst} function is supposed to extract a value of type
% (\lstinline{A}) from the merge value $x$ (of type \lstinline{A&B}). However
% this function is problematic.  The reason is that when
% \lstinline{A} and \lstinline{B} are instantiated to non-disjoint
% types, then uses of \lstinline{fst} may lead to incoherence.
% For example, consider the following use of \lstinline{fst}:
% \begin{lstlisting}
% fst Int Int (1,,2)
% \end{lstlisting}
% \noindent This program is clearly incoherent as both
% $1$ and $2$ can be extracted from the merge and still match the type
% of the first argument of \lstinline{fst}.

% \paragraph{Biased choice breaks equational reasoning.} At first sight, one option
% to workaround the issue incoherence would be to bias the type-based merge lookup
% to the left or to the right (as discussed in
% Section~\ref{subsec:incoherence}). Unfortunately, biased choice is
% very problematic when parametric polymorphism is present in the language.
% To see the issue, suppose we chose to always pick the
% rightmost value in a merge when multiple values of same type exist.
% Intuitively, it would appear that the result of the use of
% \lstinline{fst} above is $2$. Indeed simple equational reasoning
% seems to validate such result:
% \begin{lstlisting}
%    fst Int Int (1,,2)
% (*$ \rightsquigarrow $*) (fun (z: Int) (*$ \to $*) z) (1,,2) -- (* \textnormal{By the definition of \code{fst}} *)
% (*$ \rightsquigarrow $*) (fun (z: Int) (*$ \to $*) z) 2      -- (* \textnormal{Right-biased coercion} *)
% (*$ \rightsquigarrow $*) 2                          -- (* \textnormal{By $\beta$-reduction} *)
% \end{lstlisting}
%
% However (assumming a straightforward implementation of right-biased
% choice) the result of the program would be 1! The reason for this has
% todo with \emph{when} the type-based lookup on the merge happens. In
% the case of \lstinline{fst}, lookup is triggered by a coercion
% function inserted in the definition of \lstinline{fst} at
% compile-time.
% In the definition of \lstinline$fst$ all it is known is that a
% value of type $A$ should be returned from a merge with an intersection
% type $A\&B$.  Clearly the only type-safe choice to coerce the value of
% type $A\&B$ into $A$ is to
% take the left component of the merge. This works perfectly for merges
% such as \lstinline$(1,,'c')$, where the types of the first and second components
% of the merge are disjoint. For the merge \lstinline$(1,,'c')$, if a integer lookup
% is needed, then \lstinline$1$ is the rightmost integer, which is consistent with the
% biased choice. Unfortunately, when given the merge \lstinline$(1,,2)$ the
% left-component (\lstinline$1$) is also picked up, even though in this case \lstinline$2$
% is the rightmost integer in the merge. Clearly this is inconsistent
% with the biased choice!
%
% Unfortunately this subtle interaction of polymorphism and type-based lookup
%  means that equational reasoning is broken!
% In the equational reasoning steps above, doing apparently correct
% substitutions lead us to a wrong result. This is a major problem for
% biased choice and a reason to dismiss it as a possible implementation
% choice for \name.

\begin{comment}
\paragraph{Conservatively rejecting intersections.}
To avoid incoherence, and the issues of biased choice, another option
is simply to reject programs where the
instantiations of type variables may lead to incoherent programs.
In this case the definition of \lstinline$fst$ would be rejected, since there
are indeed some cases that may lead to incoherent programs.
Unfortunately this is too restrictive and prevents many useful
programs.

We have built a source language that is desugared into \name. The most
central feature is the trait declaration. Trait can take several parameters,
which is then in scope in the body of the trait definition. In fact, trait
creation is dynamic, which means it can be contained inside a function.
\end{comment}

\input{sections/examples.tex}
\section{The \name Calculus}\label{sec:fi}
This section presents the syntax, subtyping, and typing of \name: 
a calculus with intersection types, parametric polymorphism, records and a merge operator. 
This calculus is an extension of the \oldname calculus~\cite{oliveira16disjoint},
which is itself inspired by Dunfield's
calculus~\cite{dunfield2014elaborating}. \name extends \oldname with (disjoint) polymorphism.
%The novelty of \name is the addition of \emph{disjoint polymorphism}:
%a form of parametric polymorphism with disjointness contraints, which
%allows flexibility while at the same time retaining coherence. 
%As discussed in Section~\ref{overview} retaining
%coherence, while having an expressive form of polymorphism is non-trvial.
%Section~\ref{sec:disjoint} introduces \namedis, which shows the necessary changes
%for supporting disjoint intersection types and disjoint
%quantification and ensuring coherence.
Section~\ref{sec:alg-dis} introduces the necessary changes to the
definition of disjointness presented by Oliveira et al.~\cite{oliveira16disjoint} in
order to add disjoint polymorphism.

%\joao{we already say this in the introduction}
%All the meta-theory of \name has been mechanized in Coq, and is available in
%the supplementary materials submitted with the paper.

\subsection{Syntax}
The syntax of \name (with the differences to \oldname highlighted in gray) is: 
%are intersection types $A \inter B$ at the
%type-level and the ``merges'' $e_1 \mergeop e_2$ at the term level.

%TODO merge this figure with figure 5 (text too)
%\begin{figure}[!t]
\vspace{-15pt}
  \[
    \begin{array}{l}
      \begin{array}{llrll}
        \text{Types}
        & A, B & \!\!\Coloneqq & \!\top \mid \tyint \mid A \to B \mid A
                             \inter B \mid \highlight{$\alpha$} \mid \highlight{$\fordis \alpha A B$} \mid \highlight{$\recordType l A$} & \\ 

        \text{Terms}
        & e & \!\!\Coloneqq & \!\top \mid i \mid x \mid \lamty x A e \mid \app {e_1} {e_2} 
              \mid e_1 \mergeop e_2 \mid \!\highlight{$\blamdis \alpha A e$} \!\mid \!\highlight{$\tapp e A$} \!\mid 
              \!\highlight{$\recordCon l e$} \!\mid \!\highlight{$\recordProj e l$} & \\
        \text{Contexts}
        & \Gamma & \!\!\Coloneqq & \!\cdot
                   \mid \Gamma, \highlight{$\alpha \disjoint A$}
                   \mid \Gamma, x \oftype A  & \\
      \end{array}
    \end{array}
  \]

%  \caption{\name syntax.}
%  \label{fig:fi-syntax}
% \end{figure}

\paragraph{Types.} 
Metavariables $A$, $B$ range over types. 
Types include all constructs in \oldname (excluding product types): a top type $\top$; 
the type of integers $\tyint$;
function types $A \to B$; and intersection types $A \inter B$.
The main novelty are two standard constructs of System $F$ used to support
polymorphism: 
type variables $\alpha$ and disjoint (universal) quantification $\fordis \alpha A B$. 
Unlike traditional universal quantification, the disjoint
quantification includes a disjointness constraint associated to a type variable $\alpha$.
Finally, \name also includes singleton record types, which consist of a label $l$ and
an associated type $A$.
We will use $\subst {A} \alpha {B}$
to denote the capture-avoiding substitution of $A$ for $\alpha$ inside $B$ and
$\ftv \cdot$ for sets of free type variables. 

\paragraph{Terms.} 
Metavariables $e$ range over terms.  
Terms include all constructs in \oldname: a canonical top value $\top$; integer literals $i$;
variables $x$, lambda abstractions ($\lamty x A e$); applications 
($\app {e_1} {e_2}$); and the \emph{merge} of terms $e_1$ and $e_2 $ denoted as 
$e1 \mergeop e2$.
Terms are extended with two standard constructs in System $F$:
abstraction of type variables over terms $\blamdis \alpha A e$; and
application of terms to types $\tapp e A$. 
The former also includes an extra disjointness constraint tied to the type 
variable $\alpha$, due to disjoint quantification.
%If one regards $e_1$ and $e_2$ as objects, their merge will respond to
%every method that one or both of them have.
Singleton records consists of a label $l$ and an associated term $e$.
Finally, the accessor for a label $l$ in term $e$ is denoted as $\recordProj e l$.

\paragraph{Contexts.} Typing contexts $ \Gamma $ track bound type variables
$\alpha$ with disjointness constraints $A$; and variables $x$ with their type $A$. 
We will use $\subst {A} \alpha {\Gamma}$
to denote the capture-avoiding substitution of $A$ for $\alpha$ in the co-domain of
$\Gamma$ where the domain is a type variable (i.e all disjointness constraints).
Throughout this paper, we will assume that all contexts are
well-formed. Importantly, besides usual well-formedness conditions, in
well-formed contexts type variables must not appear free within its own disjointness constraint.
%All substitutions performed in environments must also lead to well-formed environments.
%In order to focus on the key features that make this language interesting, we do
%not include other forms such as type constants and fixpoints here. 
%However they can be included in the formalization in
%standard ways and we are using them in discussions and examples. %TODO are we?
\paragraph{Syntactic sugar}
In \name we may quantify a type variable and ommit its constraint. 
This means that its constraint is $\top$. 
For example, the function type $\forall \alpha. \alpha \to \alpha$ is syntactic sugar
for  $\fordis \alpha \top {\alpha \to \alpha}$.
This is discussed in more detail in Section~\ref{sec:disjoint}. 

% \paragraph{Discussion.} A natural question the reader might ask is that why we
% have excluded union types from the language. The answer is we found that
% intersection types alone are enough support extensible designs.

\subsection{Subtyping}
% In some calculi, the subtyping relation is external to the language: those
% calculi are indifferent to how the subtyping relation is defined. In \name, we
% take a syntactic approach, that is, subtyping is due to the syntax of types.
% However, this approach does not preclude integrating other forms of subtyping
% into our system. One is ``primitive'' subtyping relations such as natural
% numbers being a subtype of integers. The other is nominal subtyping relations
% that are explicitly declared by the programmer.


%\begin{figure}
%  \begin{mathpar}
%    \formsub \\
%    \rulesubvar \and \rulesubfun \and \rulesubforall \and \rulesubinter \and
%    \rulesubinterl \and \rulesubinterr
%  \end{mathpar}
%
%  \begin{mathpar}
%    \formwf \\
%    \rulewfvar \and \rulewffun \and \rulewfforall \and \rulewfinter
%  \end{mathpar}
%
%  \begin{mathpar}
%    \formt \\
%    \ruletvar \and \ruletlam \and \ruletapp \and \ruletblam \and \rulettapp \and
%    \ruletmerge
%  \end{mathpar}
%
%  \caption{The type system of \name.}
%  \label{fig:fi-type}
%\end{figure}

% Intersection types introduce natural subtyping relations among types. For
% example, $ \tyint \inter \tybool $ should be a subtype of $ \tyint $, since the former
% can be viewed as either $ \tyint $ or $ \tybool $. To summarize, the subtyping rules
% are standard except for three points listed below:
% \begin{enumerate}
% \item $ A_1 \inter A_2 $ is a subtype of $ A_3 $, if \emph{either} $ A_1 $ or
%   $ A_2 $ are subtypes of $ A_3 $,

% \item $ A_1 $ is a subtype of $ A_2 \inter A_3 $, if $ A_1 $ is a subtype of
%   both $ A_2 $ and $ A_3 $.

% \item $ \recordType {l_1} {A_1} $ is a subtype of $ \recordType {l_2} {A_2} $, if
%   $ l_1 $ and $ l_2 $ are identical and $ A_1 $ is a subtype of $ A_2 $.
% \end{enumerate}
% The first point is captured by two rules $ \reflabelsubinterl $ and
% $ \reflabelsubinterr $, whereas the second point by $ \reflabelsubinter $.
% Note that the last point means that record types are covariant in the type of
% the fields.

The subtyping rules of the form $A \subtype B$ are shown in 
Figure~\ref{fig:fi-subtype}. 
At the moment, the reader is advised to ignore the
gray-shaded parts, which will be explained later. 
Some rules are ported from \oldname: \reflabel{\labelsubtop}, 
\reflabel{\labelsubint},
\reflabel{\labelsubfun}, \reflabel{\labelsubinter}, \reflabel{\labelsubinterl} and
\reflabel{\labelsubinterr}.

\begin{figure}[t]
\begin{spacing}{0.5}
  \begin{mathpar}
    \framebox{$\jatomic A$} \\
    \inferrule*{}{\jatomic \tyint} \and 
    \inferrule*{}{\jatomic {A \to B}} \and
    \inferrule*{}{\jatomic \alpha} \and
    \inferrule*{}{\jatomic {\fordis \alpha B A}} \and
    \inferrule*{}{\jatomic {\recordType l A}}
  \end{mathpar}
  \begin{mathpar}
    \formsub \\ 
    \rulesubtop \and \rulesubinter \and 
    \rulesubint \and \rulesubinterlcoerce \and 
    \rulesubrec \and \rulesubinterrcoerce \and
    \rulesubvar  \and \rulesubfun \and 
    \rulesubforallext 
  \end{mathpar}
\end{spacing}
  \caption{Subtyping rules of \name.}
  \label{fig:fi-subtype}
\end{figure}



%There are three rules which rather straightforward: \reflabel{\labelsubtop}
%says that every type is a subtype of $\top$; \reflabel{\labelsubint} and 
%\reflabel{\labelsubvar} define subtyping as a reflexive relation on integers and
%type variables.
%The rule \reflabel{\labelsubfun} says that a function is contravariant in 
%its parameter type and covariant in its return type. 
%The three rules dealing with intersection types are just what one would expect 
%when interpreting types as sets. 
%Under this interpretation, for example, the rule \reflabel{\labelsubinter}
%says that if $A_1$ is both the subset of $A_2$ and the subset of $A_3$, then
%$A_1$ is also the subset of the intersection of $A_2$ and $A_3$.

\paragraph{Polymorphism and Records.}
The subtyping rules introduced by \name refer to polymorphic constructs and records. 
\reflabel{\labelsubvar} defines subtyping as a reflexive relation on type variables.
In \reflabel{\labelsubforall} a universal quantifier ($\forall$) 
is covariant in its body, and contravariant in its disjointness constraints.
The \reflabel{\labelsubrec} rule says that records are covariant
within their fields' types.
The subtyping relation uses an auxiliary unary $ordinary$ relation,
which identifies types that are not intersections. The $ordinary$ conditions on two of the intersection rules are necessary to 
produce unique coercions~\cite{oliveira16disjoint}. The $ordinary$
relation needed to be extended with respect to \oldname.
As shown at the top of Figure~\ref{fig:fi-subtype}, the new types it contains are 
type variables, universal quantifiers and record types.

\paragraph{Properties of Subtyping.} The subtyping relation is reflexive and transitive.
\thmprf{0cm}{
\begin{restatable}[Subtyping reflexivity]{lemma}{subrefl}
  \label{lemma:subrefl}
  For any type $A$, $A \subtype A$.
\end{restatable}}
{-0.1cm}
{By induction on $A$.}
{0cm}
%\noindent \emph{Proof.} By induction on $A$.
%\restatableproof{lemma}{Subtyping reflexivity}{subrefl}{lemma:subrefl}
%{For any type $A$, $A \subtype A$.}
%{By induction on $A$.}{-0.1cm}%
%\begin{prf}
%By induction on $A$.
%\end{prf}%
%\begin{restatable}[Subtyping transitivity]{lemma}{subtrans}
%  \label{lemma:subtrans}
%  If $A \subtype B$ and $B \subtype C$, then $A \subtype C$.
%\end{restatable}%
%\noindent \emph{Proof.} By double induction on both derivations.%
\restatableproof{lemma}{Subtyping transitivity}{subtrans}{lemma:subtrans}
{If $A \subtype B$ and $B \subtype C$, then $A \subtype C$.}
{By double induction on both derivations.}{-0.1cm}

%\begin{prf}
%By double induction on both derivations. 
%\end{prf}
%\bruno{Too much space waisted between Lemma and proof. reduce the
%  white space.}
%TODO example showing contravariance in disjointness constraints goes here or in the overview 
%section?
%\paragraph{Metatheory.} As other standard subtyping relations, we can show that
%subtyping defined by $\subtype$ is also reflexive and transitive.
%
%\begin{lemma}[Subtyping is reflexive] \label{lemma:sub-refl}
%  For all type $ A $, $ A \subtype A $.
%\end{lemma}
%
%\begin{lemma}[Subtyping is transitive] \label{lemma:sub-trans}
%  If $ A_1 \subtype A_2 $ and $ A_2 \subtype A_3 $,
%  then $ A_1 \subtype A_3 $.
%\end{lemma}
\subsection{Typing}

\begin{comment}
\begin{figure}[!t]
  \begin{mathpar}
    \formwf \\ \rulewfint \and \rulewfvardis \and \rulewffun \and \rulewfrec \and 
    \rulewftop \and \rulewfforalldis \and \rulewfinterdis 
  \end{mathpar}

  \caption{Well-formedness rules for types of \name.}
  \label{fig:wf}
\end{figure}
\end{comment}


%  \begin{mathpar}
%    \formt \\ \ruletvar \and \ruletlam \and \ruletapp \and
%    \ruletblam \and \rulettapp \and \ruletmergedis 
%  \end{mathpar}

\paragraph{Well-formedness.}
The well-formedness rules are shown in the top part of Figure~\ref{fig:fi-type}. 
The new rules over \oldname are \reflabel{\labelwfvar} and \reflabel{\labelwfforall}. 
Their definition is quite straightforward, but note that the constraint in the latter
must be well-formed.

\begin{figure}
  \begin{spacing}{1}
  \begin{mathpar}
    \formwf \\ \rulewfint \and \rulewfvardis \and \rulewfrec \and 
    \rulewffun \and \rulewftop \and \rulewfforalldis \and \rulewfinterdis 
  \end{mathpar}
  \begin{mathpar}
    \formbi \\ \brulettop \and \bruletint \and \bruletvar \and \bruletann \and 
    \bruletapp \and \brulettappdis \and \bruletmergedis \and \bruletrec \and 
    \bruletprojr \and \bruletblamdis 
  \end{mathpar}
  \begin{mathpar}
    \formbc \\ \bruletlam \and \bruletsub
  \end{mathpar}
  \end{spacing}
  \caption{Well-formedness and type system of \name.}
  \label{fig:fi-type}
\end{figure}

\paragraph{Typing rules.}
Our typing rules are formulated as a bi-directional type-system. 
Just as in \oldname, this ensures the type-system is not only syntax-directed, but
also that there is no type ambiguity: that is, inferred types are unique.
The typing rules are shown in the bottom part of Figure~\ref{fig:fi-type}. 
Again, the reader is advised to ignore the
gray-shaded parts, as these will be explained later. 
The typing judgements are of the form: $\jcheck \Gamma e A$ and  
$\jinfer \Gamma e A$.
They read: ``in the typing context $\Gamma$, the term $e$ can be
checked or inferred to
type $A$'', respectively. 
The rules ported from \oldname are the
check rules for $\top$ (\reflabel{\labelttop}), integers (\reflabel{\labeltint}), 
variables (\reflabel{\labeltvar}),  application (\reflabel{\labeltapp}), merge operator  
(\reflabel{\labeltmerge}), annotations (\reflabel{\labeltann}); and infer rules
for lambda abstractions (\reflabel{\labeltlam}), and the subsumption rule 
(\reflabel{\labeltsub}).

\paragraph{Disjoint quantification.}
The new rules, inspired by System $F$, are the infer rules for type
application \reflabel{\labelttapp}, and for type abstraction
\reflabel{\labeltblam}.  Type abstraction is introduced by the big
lambda $\blamdis \alpha A e$, eliminated by the usual type application
$\tapp e A$ (\reflabel{\labelttapp}).  The disjointness constraint is
added to the context in \reflabel{\labeltblam}. During a type application, the
type system makes sure that the type argument satisfies the
disjointness constraint.  Type application performs an extra check
ensuring that the type to be instantiated is compatible
(i.e. disjoint) with the constraint associated with the abstracted
variable.  This is important, as it will retain the desired coherence
of our type-system.  For ease of discussion, also in
\reflabel{\labeltblam}, we require the type variable introduced by the
quantifier to be fresh.  For programs with type variable shadowing,
this requirement can be met straighforwardly by variable renaming.

\paragraph{Records.}
Finally, $\reflabel{\labeltrec}$ and $\reflabel{\labeltprojr}$ deal with record types.
The former infers a type for a record with label $l$ if it can infer a type for the
inner expression; the latter says if one can infer a record type $\recordType l A$ 
from an expression $e$, then it is safe to access the field $l$, and infering type $A$.


\section{Type-directed Translation to System $ F $}

In this section we define the dynamic semantics of the call-by-value \name by
means of a type-directed translation to a variant of System $F$. This
translation turns merges into usual pairs, similar to Dunfield's elaboration
approach~\cite{dunfield2014elaborating}. But in addition, our translation
removes labels of records and rewrites record operations as function
applications. In the end the translated expressions can be typed and interpreted
within System $F$. We add the blue-color part to our rules presented in the
previous section. Besides that, they stay the same. We also tacitly assume the
variables introduced in the blue part are generated from a unique name supply and
are always fresh.

\subsection{Informal Discussion}

This subsection presents the translation informally by explaining the major
ideas.

\paragraph{Turning merges into pairs.}
The first idea is turning merges into pairs. For example,
\[
1 \mergeOp \code{"one"}
\]
becomes \pair 1 {\code{"one"}}.
In usage, the pair will be coerced according to type information. For example,
consider the function application:
\[
\app {(\lam x \tystring x)} {(1 \mergeOp \code{"one"})}
\]
It will be translated to
\[
\app {(\lam x \tystring x)} {(\app {(\lam x {\pair \tyint \tystring} {\proj 2 x})} {\pair 1 {\code{"one"}}})}
\]
The coercion in this case is $(\lam x {\pair \tyint \tystring} {\proj 2 x})$.

\noindent It extracts the second item from the pair since the function expects a $\tystring$
but the translated argument is of type $\pair \tyint \tystring$.

\paragraph{Erasing labels.}
The second idea is erasing record labels. For example,
\begin{lstlisting}
{name = "Barbara"}
\end{lstlisting}
becomes just \lstinline{"Barbara"}.
To see how the this and the previous idea are used together, consider the following program:
\begin{lstlisting}
{distance = {inKilometers = 8, inMiles = 5}}
\end{lstlisting}
Since multi-field records are just merges, the record is desugared as
\begin{lstlisting}
{distance = {inKilometers = 8} ,, {inMiles = 5}}
\end{lstlisting}
and then translated to \lstinline{(8,5)}.

\paragraph{Record operations as functions.}
The third idea is translating record operations into normal functions. For
example, the source program
\begin{lstlisting}
{distance = {inKilometers = 8, inMiles = 5}}.distance.inMiles
\end{lstlisting}
becomes an \name expression
\[
\app {(\lam x {\pair \tyint \tyint} {\proj 2 x})} {\pair 8 5}
\]
where $\lam x {\pair \tyint \tyint} {\proj 2 x}$
extracts the desired item $5$.

\subsection{Target Language}

Our target language is System $F$ extended with pair and unit types. The syntax
and typing is completely standard. The syntax of the target language is shown in
Figure~\ref{fig:f-syntax} and the typing rules in the appendix.
% \bruno{fill!}
\begin{figure}[h]
  \[
\begin{array}{lrrl}
  \text{Types} & T & \dcoloneq &
    \alpha
    \mid T \to T
    \mid \forall \alpha. T
    \mid \tupled {T, T} \\

  \text{Terms} & E, C & \dcoloneq &
    x
    \mid \abs {\rel x T} {E}
    \mid \Abs {\alpha} {E}
    \mid \app E E
    \mid \app E T \\ & & &
    \mid \tupled {E, E}
    \mid \fst E
    \mid \snd E
\end{array}
\]

  \caption{Target language syntax.}
  \label{fig:f-syntax}
\end{figure}

% \bruno{Why is this lemma placed here?}
% \bruno{Generaly Speaking this text seems out of place.Move to 5.4, maybe?}

% The main translation judgment is $ \hastype \gamma e \tau \yields E $ which
% states that with respect to the typing context $ \gamma $, the \name expression
% $e$ is of $\tau$ and its translation is a target expression $ E $.

\subsection{Type Translation}

\begin{figure}[h]
  \framebox{$\im \tau = T$}

\begin{align*}
  \im \alpha                    &= \alpha \\
  \im \top                      &= () \\
  \im {\tau_1} \to \im {\tau_2} &= \im {\tau_1} \to \im {\tau_2} \\
  \im {\for \alpha \tau}        &= \for \alpha \im \tau \\
  \im {\tau_1 \intersect \tau_2} &= \pair {\im {\tau_1}} {\im {\tau_2}} \\
\end{align*}

  \framebox{$\im \Gamma = G$}

\begin{align*}
  \im \epsilon                 &= \epsilon \\
  \im {\Gamma, \alpha}         &= \im \Gamma, \alpha \\
  \im {\Gamma, \alpha \oftype A} &= \im \Gamma, \alpha \oftype \im A
\end{align*}

  \caption{Type and context translation.}
  \label{fig:type-and-context-translation}
\end{figure}

Figure~\ref{fig:type-and-context-translation} defines the type translation
function $\im \cdot$ from \name types $\tau$ to target language types $T$. The
notation $\im \cdot$ is also overloaded for context translation from \name
contexts $\gamma$ to target language contexts $\Gamma$.

% The rules given in this section are identical with those in
% Section~\ref{sec:fi}, except for the light blue part. The translation consists
% of four sets of rules, which are explained below:

\subsection{Coercive Subtyping}

Figure~\todo{fig:elab-subtyping} shows subtyping with coercions. The judgment
\[
\tau_1 \subtype \tau_2 \yields C
\]
extends the subtyping judgment in Figure~\ref{fig:fi-subtyping} with a coercion
on the right hand side of $ \yields {} $. A coercion $ C $ is just an expression
in the target language and is ensured to have type
$ \im {\tau_1} \to \im {\tau_2} $ (Lemma~\ref{lemma:sub})\bruno{ref
  now showing}. For example,
\[
\tyint \intersect \tybool \subtype \tybool \yields {\lam x {\im {\tyint \intersect \tybool}} {\proj 2 x}}
\]

\noindent generates a coercion function from $\tyint \intersect \tybool$ to $\tybool$.

In rules \rulelabel{SubVar}, \rulelabel{SubTop}, \rulelabel{SubForall},
coercions are just identity functions. In \rulelabel{SubFun}, we elaborate the
subtyping of parameter and return types by $\eta$-expanding $f$ to
$\lam x {\im {\tau_3}} {\app f x}$, applying $C_1$ to the argument and $C_2$ to
the result. Rules \rulelabel{SubAnd1}, \rulelabel{SubAnd2}, and
\rulelabel{SubAnd} elaborate with intersection types. \rulelabel{SubAnd} uses
both coercions to form a pair. Rules \rulelabel{SubAnd1} and
\rulelabel{SubAnd2} reuse the coercion from the premises and create new ones
that cater to the changes of the argument type in the conclusions. Note that the
two rules are syntatically the same and hence a program can be elaborated
differently, depending on which rule is used. But in the implementation one
usually applies the rules sequentially with pattern matching, essentially
defining a deterministic order of lookup.
\begin{comment}
if we know $\tau_1$ is a subtype of $\tau_3$ and $C$ is a coercion from $\tau_1$
to $\tau_3$, then we can conclude that $\tau_1 \intersect \tau_2$ is also a subtype
of $\tau_3$ and the new coercion is a function that takes a value $ x $ of type
$\tau_1\intersect \tau_2$, project $x$ on the first item, and apply $ C $ to it.
\end{comment}

\begin{restatable}[Subtyping rules produce type-correct coercion]{lemma}{lemmasub}
  \label{lemma:sub}
  If $ \tau_1 \subtype \tau_2 \yields C $, then $ \judgeTarget \epsilon C {\im {\tau_1} \to \im {\tau_2}} $.
\end{restatable}

\begin{proof}
  By a straighforward induction on the derivation\footnote{The proofs of major lemmata and theorems can be found in the appendix.}.
\end{proof}

\subsection{Main Translation}

\begin{comment}
In this subsection we now present formally the translation rules that convert
\name expressions into System $ F $ ones. This set of rules essentially extends
those in the previous section with the light-blue part for the translation.
\end{comment}

% \bruno{Badly structured. Don't mention Coercion here, as it was already
% explained in the previous section.}
% \bruno{Don't use itemize and items. Use paragraphs instead!}

\paragraph{Main translation judgment.} The main translation judgment
$\hastype \gamma e \tau \yields E$ extends the typing judgment with an elaborated
expression on the right hand side of $\yields {}$. The translation ensures
that $E$ has type $\im \tau$. In \name, one may pass more information to a
function than what is required; but not in System $F$. To account for this
difference, in \rulelabel{App}, the coercion $C$ from the subtyping relation is
applied to the argument. \rulelabel{Merge} straighforwardly translates merges
into pairs.

% Consider the source program:
% \begin{lstlisting}
%   ({ name = "Isaac", age = 10 }).name
% \end{lstlisting}

%   Multi-field records are desugared into merge of single-field records:
%   \begin{lstlisting}
%     ({ name = "Isaac"} ,, { age = 10 }).name
%   \end{lstlisting}

%   By $ \ruleLabelSelect $,
%   \[ \turnsGet (\recordType {name} {String}; {name}) : String \]

%   we have the coercion
%   \[ \lam x {\im {\recordType {name} {String}}} x \]

%   which is just $ \lam x {String} x $ according to type translation.

%   By $ \ruleLabelSelectLeft $,
%   \[ \turnsGet (\recordType {name} {String} \intersect \recordType {age} {Int}; {name}) : String \]

%   % we have the coercion
%   % \[ \abs {\rel x {\im {\recordType {name} {String} \intersect \recordType
%   %         {age} {Int}}}} \app {(\abs {\rel x {\im {\recordType {name} {String}}}} x)} {(\fst ~ x)} \]
%   % which is just $ \abs {\rel x {(String, Int)}} {\app {(\abs {\rel x {String}} x)} {(\fst ~ x)}} $ by type translation.

%   By typing rules, the translation of the program is
%   \[ ("Isaac", 10) \]. If we apply the coercion to it, we get
%   \[ "Isaac" \]

\begin{restatable}[Translation preserves well-typing]{theorem}{theorempreservation}
  \label{theorem:preservation}
  If $ \hastype \gamma e \tau \yields E $,
  then $ \judgeTarget {\im \gamma} E {\im \tau} $.
\end{restatable}
\begin{proof}
(Sketch) By structural induction on the expression and the corresponding
inference rule.
\end{proof}

\begin{theorem}[Type safety]
  If $e$ is a well-typed \name expression, then $e$ evaluates to some System $F$
  value $v$.
\end{theorem}
\begin{proof}
  Since we define the dynamic semantics of \name in terms of the composition of
  the type-directed translation and the dynamic semantics of System $F$, type safety follows immediately.
\end{proof}

\section{Implementation}

\subsection{Type Synonyms}

We extend the implementation of the type system extended with type synonyms and
lazy arguments.

\begin{lstlisting}
type T A1 A2 = ... in
\end{lstlisting}

\subsection{Optimization}


\section{Disjoint Intersection Types}

This section shows how to restrict the system presented before
so that it supports coherence as well as type soundness.
The keys aspects are the notion of disjoint intersections,
and disjoint quantification for polymorphic types.

\subsection{Motivating design choices}\bruno{Maybe this belongs to Section 2?}

We need to motivate the 3 changes:

\paragraph{Well-formed types}: We need a new notion of well-formed types.

\paragraph{Disjoint quantification}: Needed when we have a
combination of polymorphism and intersection types.

With a subtyping relation in a type system, bounded polymorphism extends the universal quantifier by confining the polymorphic type to be a subtype of a given type. In our type system, the forall binder also extends the parametric polymorphism, but in a different vein: the polymorphic type can only be disjoint with a given type. Later during an instantiation, if the type provided overlaps with the constraint, such instantiation will be rejected by our type system.

\begin{itemize}
  \item \textbf{Bounded polymorphism}---the instantiation can only be the descendant of a given type
  \item \textbf{Polymorphism with disjoint constraint}---the instantiation cannot share a common ancestor with a given type
\end{itemize}

The intuition can be found in figure \ldots.

% http://tex.stackexchange.com/questions/158876/drawing-subgroup-lattices-in-tikz
\begin{figure}

% center everything in the figure
\centering
% horizontal node distance
\newcommand{\mydistance}{.6cm}
\begin{tikzpicture}[node distance=2cm]
\title{Untergruppenverband der $A_4$}
\node(A4)                           {$A_4$};
\node(V4)       [below right=2cm and 2cm of A4] {$V_4$};
\node(C31)      [below left=2cm and 0cm of A4]  {$C_3$};
\node(C32)      [left=\mydistance of C31]       {$C_3$};
\node(C33)      [left=\mydistance of C32]       {$C_3$};
\node(C34)      [left=\mydistance of C33]       {$C_3$};
\node(C22)      [below=2cm of V4]       {$C_2$};
\node(C21)      [left=\mydistance of C22]       {$C_2$};
\node(C23)      [right=\mydistance of C22]      {$C_2$};
\node(1)            [below=6cm of A4]     {$\left\{1\right\}$};
\draw(A4)       -- (V4);
\foreach \x\y in {1,2,3,4} {
    \draw (A4) -- (C3\x) node [midway, fill=white] {3};
    \draw (C3\x) -- (1);

}
\foreach \x\y in {1/2,2/3,3/4} {
    \draw(V4) -- (C2\x) node [midway, fill=white] {2};
\draw (C3\x) -- (C3\y);
\draw (C2\x) -- (1);
}
\draw(C21)      -- (C22);
\draw(C22)      -- (C23);
\end{tikzpicture}
\caption{Untergruppenverband}
\end{figure}

\paragraph{Restrictions on subtyping}:

The subtyping rules, without the atomic condition are overlapping. With the atomic constraint, one can guarantee that at any moment during the derivation of a subtyping relation, at most one rule can be used. Indeed, our restrictions on subtyping do not make the subtyping relation less expressive to one without such restrictions.\todo{Point to proofs}

\george{Add interpretation of the theorem}

\begin{theorem}
  If $A \subtype C$, then $A \intersect B \subtype C$.
  If $B \subtype C$, then $A \intersect B \subtype C$.
\end{theorem}

\begin{proof}
  By induction on $C$.
  If $C \neq C_1 \intersect C_2$, trivial.
  If $C = C_1 \intersect C_2$,
  Need to show $A \subtype C_1 \intersect C_2$ implies $A \intersect B \subtype C_1 \intersect C_2$.
  By inversion $A \subtype C_1$ and $A \subtype C_2$.
  By the i.h., $A \intersect B \subtype C_1$ and $A \intersect B \subtype C_2$.
  By \rulelabel{SubAnd}, $A \intersect B \subtype C_1 \intersect C_2$.
\end{proof}

\subsection{Disjointness}

Spec of disjointness/intuition ...

We say two types are \emph{disjoint} if they do not share a common supertype.

\begin{definition}[Disjointness]
$A \bot B = \not \exists C. A <: C \wedge B <: C$
\end{definition}

We require the types of two terms in a merge $e_1 \mergeOp e_2$ to be disjoint. Why do we require this? That is because if both terms can be assigned some type $C$, both of them can be chosen as the meaning of the merge, which leads to multiple meaning of a term, known as incoherence.

\subsection{Well-formed types}

A well-formed type is such that given any query type, it is always clear which subpart the query is referring to. The rules for well-formedness are standard except for intersection types we require the two components to be disjoint.

\subsection{Subtyping}

\subsection{Metatheory}

\begin{definition}{Type variable constraint}
We say the \emph{constraint} of a type variable $\alpha$ inside the context $\Gamma$ is $A$ if $\alpha \disjoint A \in \Gamma$.
\end{definition}

% \begin{lemma}
% If $A \subtype B$ where both $A$ and $B$ are well-formed, then $A$ and $B$ cannot be disjoint.
% \end{lemma}
%
% \begin{proof}
% $A \subtype B$ implies $B$ is a common supertype of $A$ and $B$. As a result, $A$ and $B$ are not disjoint by definition.
% \end{proof}

\begin{lemma}[Free type variables of disjoint bounds] \label{free-var-disjoint-bounds}
  If $\isdisjoint \Gamma \alpha A$, then $\alpha \not \in \ftv A$.
\end{lemma}

\begin{lemma}[Unique subtype contributor] \label{unique-subtype-contributor}
If $A \intersect B \subtype C$, where $A \intersect B$ and $C$ are well-formed types, then it is not possible that the following hold at the same time:
\begin{enumerate}
\item $A \subtype C$
\item $B \subtype C$
\end{enumerate}
\end{lemma}

If $A \intersect B \subtype C$, then either $A$ or $B$ contributes to that subtyping relation, but not both. The implication of this lemma is that during the derivation, it is not possible that two rules are applicable.

\newcommand{\wfinterlabel}{\textsc{WFInter}}

\begin{proof}
Since $A \intersect B$ is well-formed, $A \disjoint B$ by the formation rule of intersection types \wfinterlabel. Then by the definition of disjointness, there does not exist a type $C$ such that $A \subtype C$ and $B \subtype C$. It follows that $A \subtype C$ and $B \subtype C$ cannot hold simultaneously.
\end{proof}

The coercion of a subtyping relation $A \subtype B$ is uniquely determined.

\begin{lemma}[Unique coercion] \label{unique-coercion}
If $A \subtype B \yields {C_1}$ and $A \subtype B \yields {C_2}$, where $A$ and $B$ are well-formed types, then $C_1 \equiv C_2$
\end{lemma}

\begin{proof}
The set of rules for generating coercions is syntax-directed except for the three rules that involve intersection types in the conclusion. Therefore it suffices to show that if well-formed types $A$ and $B$ satisfy $A \subtype B$, where $A$ or $B$ is an intersection type, then at most one of the three rules applies. In the following, we do a case analysis on the shape of $A$ and $B$:

\begin{itemize}
  \item \textbf{Case} $A \neq A_1 \intersect A_2$ and $B = B_1 \intersect B_2$: Clearly only \textsc{SubAnd} can apply.
  \item \textbf{Case} $A = A_1 \intersect A_2$ and $B \neq B_1 \intersect B_2$: Only two rules can apply, \textsc{SubAnd1} and \textsc{SubAnd2}. Further, by the unique subtype contributor lemma, it is not possible that $A_1 \subtype B$ and that $A_2 \subtype B$. Thus we are certain that at most one rule of \textsc{SubAnd1} and \textsc{SubAnd2} will apply.
  \item \textbf{Case} $A = A_1 \intersect A_2$ and $B = B_1 \intersect B_2$\footnote{An example of this case is:
    \[ (\integer \intersect \bool) \intersect \character \subtype \bool \intersect \character \]}: Since $B$ is not atomic, only \rulelabel{SubAnd} apply.

  %   Suppose the contrary, that is, more than one of the three rules apply. Since it is not possible that both \textsc{SubAnd1} and \textsc{SubAnd2} apply by the unique subtype contributor lemma, only one of \textsc{SubAnd1} and \textsc{SubAnd2} apply. Therefore \textsc{SubAnd} has to hold. Without the loss of generality, assume \textsc{SubAnd1} apply. Then we have:
  % \[ A_1 \subtype B_1 \intersect B_2 \]
  % \[ A_1 \intersect A_2 \subtype B_1 \]
  % \[ A_1 \intersect A_2 \subtype B_2 \]
\end{itemize}
\end{proof}

In general, disjointness judgements are not invariant with respect to free-variable substitution. In other words, a careless substitution can violate the disjoint constraint in the context. For example, in the context $\alpha \disjoint \tyint$, $\alpha$ and $\tyint$ are disjoint:
\begin{mathpar}
\isdisjoint {\alpha \disjoint \tyint} \alpha \tyint
\end{mathpar}
But after the substitution of $\tyint$ for $\alpha$ on the two types, the sentence
\begin{mathpar}
\isdisjoint {\alpha \disjoint \tyint} \tyint \tyint
\end{mathpar}
is longer true since $\tyint$ is clearly not disjoint with itself.

\begin{lemma}{Invariance of disjointness} \label{invariance-of-disjointness}
If $\isdisjoint \Gamma A B$ and $R$ respects the constraints of $\beta$, then $\isdisjoint \Gamma {\subst R \beta A} {\subst R \beta B}$.
\end{lemma}

This lemma says that substitution for free type variables preserves disjointness of types if the combination of the replacement type and the type variable is proven disjoint.

\begin{proof}
By induction on the derivation of $\isdisjoint \Gamma A B$.
\begin{itemize}
  \item Case \[ \disjointvar \]
  We need to show \[ \isdisjoint \Gamma {\subst R \beta \alpha} {\subst R \beta B} \]
  If $\beta$ is not equivalent to $\alpha$ and is not free in $B$, then the above trivially holds by the def. of the substitution function. Otherwise, if $\beta$ is equivalent to $\alpha$, then we need to show
  \[ \isdisjoint \Gamma R {\subst R \beta B} \]

  % Note that $\beta \not \in \ftv B$. Thus $B$ is equivalent to $\subst R \beta B$.
  %
  % If $\beta$ is not equivalent to $\alpha$, $\subst R \beta \alpha$ is equivalent to $\alpha$. Therefore $\isdisjoint \Gamma {\subst R \beta \alpha} {\subst R \beta B}$ is true.
  % If $\beta$ is equivalent to $\alpha$, then $\subst R \beta \alpha$ is equivalent to $R$ by the def. of the substitution function. It now remains to show \[ \isdisjoint \Gamma R B \].

  \item Case \[ \disjointinterleft \]
  By applying the i.h. and the def. of the substitution function.

  \item Case \[ \disjointinterright \]
  Similar.

  \item Case \[ \disjointfun \]
  By applying the i.h. and the def. of the substitution function.

  \item Case \[ \disjointforall \]
  By applying the i.h. and the def. of the substitution function. Note that $\alpha$ is fresh.

  \item Case \[ \disjointatomic \]
  Substitution does not change the shape of types when the variable case is excluded. Therefore, the relation in the premise of the rule continue to hold and hence the conclusion.

\end{itemize}
\end{proof}

\begin{lemma}{Substitution} \label{substitution}
If $\istype \Gamma R$, $\istype \Gamma S$, and $R$ respects the constraints of $\beta$, then $\istype \Gamma {\subst R \beta S}$.
\end{lemma}

\begin{proof}
By induction on the derivation of $\istype \Gamma {\subst R \beta S}$.

\begin{itemize}
  \item Case \[ \wfvar \]
  If $\alpha$ happens to be the same as $\beta$, then by the def. of substitution $\subst R \beta \alpha = R$. Since $\istype \Gamma R$, we have $\istype \Gamma {\subst R \beta \alpha}$; On the other hand, if not, then by the def. of substitution $\subst R \beta S = S$. Since $\istype \Gamma S$, we also have $\istype \Gamma {\subst R \beta \alpha}$.

  \item Case
  \begin{mathpar}
    \wfbot
  \end{mathpar}
  Trivial.

  \item Case
  \begin{mathpar}
    \wffun
  \end{mathpar}
  By i.h., $\istype \Gamma {\subst R \beta A}$ and $\istype \Gamma {\subst R \beta B}$. By the def. of substitution, $\istype \Gamma {\subst R \beta {A \to B}}$.

  \item Case
  \begin{mathpar}
    \wfforall
  \end{mathpar}
  By the premise and the i.h.,
  \[ \istype {\Gamma} {\subst R \beta A} \]
  \[ \istype {\Gamma, \alpha \disjoint A} {\subst R \beta B} \]
  which by \rulelabel{WFForall} implies
  \[ \istype \Gamma {\for {\alpha \disjoint A} {\subst R \beta B}} \]
  By the def. of substitution, $\istype \Gamma {\subst R \beta {\for {\alpha \disjoint A} B}}$~\todo{Subst. of $A$}.

  \item Case
  \begin{mathpar}
    \wfinter
  \end{mathpar}
  By i.h., $\istype \Gamma {\subst R \beta A}$ and $\istype \Gamma {\subst R \beta B}$. By Lemma~\ref{invariance-of-disjointness}, we also have $\isdisjoint \Gamma {\subst R \beta A} {\subst R \beta B}$. Therefore by \rulelabel{WFInter}, $\istype \Gamma {\subst R \beta {A \intersect B}}$.
\end{itemize}
\end{proof}

\begin{lemma}{Instantiation} \label{instantiation}
If
  $\istype {\Gamma, \alpha \disjoint B} C$,
  $\istype \Gamma A$,
  $\isdisjoint \Gamma A B$
then
  $\istype \Gamma {\subst A \alpha C}$.
\end{lemma}

\begin{proof}
By induction.

\begin{itemize}
  \item Case \[ \wfvar \]
  If $C = \alpha$, then $\subst A \alpha \alpha = A$. Since $\istype \Gamma A$, it follows that $\istype \Gamma {\subst A \alpha \alpha}$; otherwise, let $C = \beta$, where $\beta$ is a type variable distinct from $\alpha$. Since $\istype {\Gamma, \alpha \disjoint B} \beta$ and $\alpha$ and $\beta$ are distinct, $\beta$ must be in $\Gamma$ and therefore $\istype {\Gamma} \beta$, which is equivalent to $\istype {\Gamma} {\subst A \alpha \beta}$.

  \item Case \[ \wffun \]
  By straightforwardly applying the i.h and the rule itself.

  \item Case \[ \wfbot \]
  Trivial.

  \item Case \[ \wfforall \]
  By straightforwardly applying the i.h and the rule itself.

  \item Case \[ \wfinter \]
  Let $C$ in the statement of this lemma be $C_1 \intersect C_2$.
  By the condition we know
  \[ \istype {\Gamma, \alpha \disjoint B} {C_1 \intersect C_2} \]
  Thus we must have,
  \[ \istype {\Gamma, \alpha \disjoint B} {C_1} \]
  By the i.h., $\istype \Gamma {\subst A \alpha {C_1}}$ and similarly $\istype \Gamma {\subst A \alpha {C_2}}$. By \rulelabel{WFInter}\todo{Show disjointness},
  \[ \istype \Gamma {\subst A \alpha {C_1} \intersect \subst A \alpha {C_2}} \]
  and hence
  \[ \istype \Gamma {\subst A \alpha {(C_1 \intersect C_2)}} \]

\end{itemize}

\end{proof}

\begin{lemma}{Well-formed typing} \label{wf-typing}
If $\hastype \Gamma e A$, then $\istype \Gamma e$.
\end{lemma}
Typing always produces a well-formed type.
\begin{proof}
By induction on the derivation of $\hastype \Gamma e A$. The case of \rulelabel{TyTApp} needs special attention
\begin{mathpar}
  \tytapp
\end{mathpar}
because we need to show that the result of substitution ($\subst A \alpha C$) is well-formed, which is evident by Lemma~\ref{instantiation}.
\end{proof}

\begin{theorem}[Unique elaboration] \label{unique-elaboration}
If $\hastype \Gamma e {A_1} \yields {E_1}$ and $\hastype \Gamma e {A_2} \yields {E_2}$, then $E_1 \equiv E_2$.
\end{theorem}
Given a source expression $e$, elaboration always produces the same target expression $E$.
\begin{proof}
The typing rules are syntax-directed. The case of \rulelabel{TyApp} needs special attention since we still need to show that the generated coercion $C$ is unique.
\begin{mathpar}
  \tyapp
\end{mathpar}
By Lemma~\ref{wf-typing}, we have $\istype \Gamma {A_1}$ and $\istype \Gamma {A_3}$. Therefore we are able to apply Lemma~\ref{unique-coercion} and conclude that $C$ is unique.
\end{proof}

\section{Algorithmic Disjointness}

Although the system in the previous section shows a formal system of
disjoint intersection types, it relies on a non-algorithmic
specification of disjointness. This section shows an algorithmic
specification of disjointness that is proved to be sound and complete.

The problem with the definition of disjointness is that it is a search problem. In this section, we are going to convert it that into an algorithm.

Let $\universe_0$ be the universe of $\tau$ types. Let $\universe$ be the quotient set of $\universe_0$ by $\approx$, where $\approx$ is defined by \ldots.

Let $\commonsuper$ be the ``common supertype'' function, and $\commonsub$ be the ``common subtype'' function. For example, assume $\integer$ and $\character$ share no common supertype. Then the fact can be expressed by $\commonsuper(\integer,\character)=\emptyset$. Formally,
\begin{align*}
  \commonsuper &: \universe \times \universe \to \powerset {\universe} \\
  \commonsub   &: \universe \times \universe \to \powerset {\universe}
\end{align*}
which, given two types, computes the set of their common supertypes. ($\powerset S$ denotes the power set of $S$, that is, the set of all subsets of $S$.)

\begin{align*}
  \commonsuper(\alpha,\alpha) &= \{ \alpha \} \\
  \commonsuper(\bot,\bot) &= \{ \bot \} \\
  \commonsuper(\tau_1 \to \tau_2, \tau_3 \to \tau_4) &= \commonsub(\tau_1,\tau_3) \to \commonsuper(\tau_2,\tau_4) \\
  % \commonsuper({\tau_1 \intersect \tau_2, \tau_3}) &= \commonsuper(\tau_1, \tau_3) \cup \commonsuper(\tau_1,\tau_3) \\
  % \commonsuper({\tau_1, \tau_2 \intersect \tau_3}) &= \commonsuper(\tau_1, \tau_2) \cup \commonsuper(\tau_1,\tau_3)
\end{align*}

Notation. We use $\commonsub(\tau_1,\tau_3) \to \commonsuper(\tau_2,\tau_4)$ as a shorthand for $\{ s \to t ~|~ s \in \commonsub(\tau_1 \to \tau_2), t \in \commonsuper(\tau_2,\tau_4) \}$. Therefore, the problem of determining if $\commonsub(\tau_1,\tau_3) \to \commonsuper(\tau_2,\tau_4)$ is empty reduces to the problem of determining if $\commonsuper(\tau_2,\tau_4)$ is empty.

Note that there always exists a common subtype of any two given types (case disjoint / case nondisjoint).

\subsection{Formal System}

Explain the rules and intuitions.

\section{Discussions}

\subsection{Systems without subtyping}

\subsection{Systems with a top type}

In type systems with a top type (such as \lstinline@Object@ in some OO languages), the definition of disjointness can be modified to:

We say two types are \emph{disjoint} if their only common supertype is the top type.

%%% Local Variables:
%%% mode: latex
%%% TeX-master: t
%%% End:

\documentclass[a4paper]{article}

% Remote packages

% Replaced by fontspec:
% \usepackage[T1]{fontenc}
% \usepackage[utf8]{inputenc}

% Works only with xelatex or lualatex
\usepackage{fontspec}
\setmainfont{Times New Roman}

\usepackage{amsmath}
\usepackage{amsthm}
\usepackage{mathtools} % For \Coloneqq
\usepackage{fixltx2e}
\usepackage{stmaryrd}
\usepackage{xcolor}
\usepackage{listings} % For code listings
\usepackage{minted}
\usemintedstyle{murphy}
\usepackage{fancyvrb}
\usepackage{url}

% Copied from the FCore paper:
\usepackage[colorlinks=true,allcolors=black,breaklinks,draft=false]{hyperref}   % hyperlinks, including DOIs and URLs in bibliography
% known bug: http://tex.stackexchange.com/questions/1522/pdfendlink-ended-up-in-different-nesting-level-than-pdfstartlink

% Figures with borders
% http://en.wikibooks.org/wiki/LaTeX/Floats,_Figures_and_Captions
\usepackage{float}
\floatstyle{boxed}
\restylefloat{figure}

% Local packages

\usepackage{bcprules}
\usepackage{cmll}
\usepackage{mathpartir}


% ! Always load mathastext last
% http://mirrors.ibiblio.org/CTAN/macros/latex/contrib/mathastext/mathastext.pdf
% \renewcommand\familydefault\ttdefault
% \usepackage{mathastext}
% \renewcommand\familydefault\rmdefault


\newtheorem{theorem}{Theorem}
\newtheorem{lemma}{Lemma}

\definecolor{facebook}{HTML}{3b5998}

% Define macros immediately before the \begin{document} command
\newcommand{\dom}[1]{\meta{dom}{#1}}
\newcommand{\ftv}[1]{\meta{ftv}(#1)}
\newcommand{\imageof}[1]{\llbracket #1 \rrbracket}
\newcommand{\meta}[2]{{\it{#1}} (#2)}
\newcommand{\proj}[2]{\app{proj_{#1}}{#2}}
\newcommand{\rel}[2]{#1 \!\!:\!\! #2}
\newcommand{\subst}[2]{\lbrack #1 \! \mapsto \! #2 \rbrack}
\newcommand{\subtype}{<\!:}
\newcommand{\yields}[1]{\textcolor{github}{\; \hookrightarrow #1}}
\newcommand{\yieldsnothing}[1]{}
\newcommand{\fst}{\texttt{fst}}
\newcommand{\snd}{\texttt{snd}}
\section{The \name Calculus}\label{sec:fi}
This section presents the syntax, subtyping, and typing of \name: 
a calculus with intersection types, parametric polymorphism, records and a merge operator. 
This calculus is an extension of the \oldname calculus~\cite{oliveira16disjoint},
which is itself inspired by Dunfield's
calculus~\cite{dunfield2014elaborating}. \name extends \oldname with (disjoint) polymorphism.
%The novelty of \name is the addition of \emph{disjoint polymorphism}:
%a form of parametric polymorphism with disjointness contraints, which
%allows flexibility while at the same time retaining coherence. 
%As discussed in Section~\ref{overview} retaining
%coherence, while having an expressive form of polymorphism is non-trvial.
%Section~\ref{sec:disjoint} introduces \namedis, which shows the necessary changes
%for supporting disjoint intersection types and disjoint
%quantification and ensuring coherence.
Section~\ref{sec:alg-dis} introduces the necessary changes to the
definition of disjointness presented by Oliveira et al.~\cite{oliveira16disjoint} in
order to add disjoint polymorphism.

%\joao{we already say this in the introduction}
%All the meta-theory of \name has been mechanized in Coq, and is available in
%the supplementary materials submitted with the paper.

\subsection{Syntax}
The syntax of \name (with the differences to \oldname highlighted in gray) is: 
%are intersection types $A \inter B$ at the
%type-level and the ``merges'' $e_1 \mergeop e_2$ at the term level.

%TODO merge this figure with figure 5 (text too)
%\begin{figure}[!t]
\vspace{-15pt}
  \[
    \begin{array}{l}
      \begin{array}{llrll}
        \text{Types}
        & A, B & \!\!\Coloneqq & \!\top \mid \tyint \mid A \to B \mid A
                             \inter B \mid \highlight{$\alpha$} \mid \highlight{$\fordis \alpha A B$} \mid \highlight{$\recordType l A$} & \\ 

        \text{Terms}
        & e & \!\!\Coloneqq & \!\top \mid i \mid x \mid \lamty x A e \mid \app {e_1} {e_2} 
              \mid e_1 \mergeop e_2 \mid \!\highlight{$\blamdis \alpha A e$} \!\mid \!\highlight{$\tapp e A$} \!\mid 
              \!\highlight{$\recordCon l e$} \!\mid \!\highlight{$\recordProj e l$} & \\
        \text{Contexts}
        & \Gamma & \!\!\Coloneqq & \!\cdot
                   \mid \Gamma, \highlight{$\alpha \disjoint A$}
                   \mid \Gamma, x \oftype A  & \\
      \end{array}
    \end{array}
  \]

%  \caption{\name syntax.}
%  \label{fig:fi-syntax}
% \end{figure}

\paragraph{Types.} 
Metavariables $A$, $B$ range over types. 
Types include all constructs in \oldname (excluding product types): a top type $\top$; 
the type of integers $\tyint$;
function types $A \to B$; and intersection types $A \inter B$.
The main novelty are two standard constructs of System $F$ used to support
polymorphism: 
type variables $\alpha$ and disjoint (universal) quantification $\fordis \alpha A B$. 
Unlike traditional universal quantification, the disjoint
quantification includes a disjointness constraint associated to a type variable $\alpha$.
Finally, \name also includes singleton record types, which consist of a label $l$ and
an associated type $A$.
We will use $\subst {A} \alpha {B}$
to denote the capture-avoiding substitution of $A$ for $\alpha$ inside $B$ and
$\ftv \cdot$ for sets of free type variables. 

\paragraph{Terms.} 
Metavariables $e$ range over terms.  
Terms include all constructs in \oldname: a canonical top value $\top$; integer literals $i$;
variables $x$, lambda abstractions ($\lamty x A e$); applications 
($\app {e_1} {e_2}$); and the \emph{merge} of terms $e_1$ and $e_2 $ denoted as 
$e1 \mergeop e2$.
Terms are extended with two standard constructs in System $F$:
abstraction of type variables over terms $\blamdis \alpha A e$; and
application of terms to types $\tapp e A$. 
The former also includes an extra disjointness constraint tied to the type 
variable $\alpha$, due to disjoint quantification.
%If one regards $e_1$ and $e_2$ as objects, their merge will respond to
%every method that one or both of them have.
Singleton records consists of a label $l$ and an associated term $e$.
Finally, the accessor for a label $l$ in term $e$ is denoted as $\recordProj e l$.

\paragraph{Contexts.} Typing contexts $ \Gamma $ track bound type variables
$\alpha$ with disjointness constraints $A$; and variables $x$ with their type $A$. 
We will use $\subst {A} \alpha {\Gamma}$
to denote the capture-avoiding substitution of $A$ for $\alpha$ in the co-domain of
$\Gamma$ where the domain is a type variable (i.e all disjointness constraints).
Throughout this paper, we will assume that all contexts are
well-formed. Importantly, besides usual well-formedness conditions, in
well-formed contexts type variables must not appear free within its own disjointness constraint.
%All substitutions performed in environments must also lead to well-formed environments.
%In order to focus on the key features that make this language interesting, we do
%not include other forms such as type constants and fixpoints here. 
%However they can be included in the formalization in
%standard ways and we are using them in discussions and examples. %TODO are we?
\paragraph{Syntactic sugar}
In \name we may quantify a type variable and ommit its constraint. 
This means that its constraint is $\top$. 
For example, the function type $\forall \alpha. \alpha \to \alpha$ is syntactic sugar
for  $\fordis \alpha \top {\alpha \to \alpha}$.
This is discussed in more detail in Section~\ref{sec:disjoint}. 

% \paragraph{Discussion.} A natural question the reader might ask is that why we
% have excluded union types from the language. The answer is we found that
% intersection types alone are enough support extensible designs.

\subsection{Subtyping}
% In some calculi, the subtyping relation is external to the language: those
% calculi are indifferent to how the subtyping relation is defined. In \name, we
% take a syntactic approach, that is, subtyping is due to the syntax of types.
% However, this approach does not preclude integrating other forms of subtyping
% into our system. One is ``primitive'' subtyping relations such as natural
% numbers being a subtype of integers. The other is nominal subtyping relations
% that are explicitly declared by the programmer.


%\begin{figure}
%  \begin{mathpar}
%    \formsub \\
%    \rulesubvar \and \rulesubfun \and \rulesubforall \and \rulesubinter \and
%    \rulesubinterl \and \rulesubinterr
%  \end{mathpar}
%
%  \begin{mathpar}
%    \formwf \\
%    \rulewfvar \and \rulewffun \and \rulewfforall \and \rulewfinter
%  \end{mathpar}
%
%  \begin{mathpar}
%    \formt \\
%    \ruletvar \and \ruletlam \and \ruletapp \and \ruletblam \and \rulettapp \and
%    \ruletmerge
%  \end{mathpar}
%
%  \caption{The type system of \name.}
%  \label{fig:fi-type}
%\end{figure}

% Intersection types introduce natural subtyping relations among types. For
% example, $ \tyint \inter \tybool $ should be a subtype of $ \tyint $, since the former
% can be viewed as either $ \tyint $ or $ \tybool $. To summarize, the subtyping rules
% are standard except for three points listed below:
% \begin{enumerate}
% \item $ A_1 \inter A_2 $ is a subtype of $ A_3 $, if \emph{either} $ A_1 $ or
%   $ A_2 $ are subtypes of $ A_3 $,

% \item $ A_1 $ is a subtype of $ A_2 \inter A_3 $, if $ A_1 $ is a subtype of
%   both $ A_2 $ and $ A_3 $.

% \item $ \recordType {l_1} {A_1} $ is a subtype of $ \recordType {l_2} {A_2} $, if
%   $ l_1 $ and $ l_2 $ are identical and $ A_1 $ is a subtype of $ A_2 $.
% \end{enumerate}
% The first point is captured by two rules $ \reflabelsubinterl $ and
% $ \reflabelsubinterr $, whereas the second point by $ \reflabelsubinter $.
% Note that the last point means that record types are covariant in the type of
% the fields.

The subtyping rules of the form $A \subtype B$ are shown in 
Figure~\ref{fig:fi-subtype}. 
At the moment, the reader is advised to ignore the
gray-shaded parts, which will be explained later. 
Some rules are ported from \oldname: \reflabel{\labelsubtop}, 
\reflabel{\labelsubint},
\reflabel{\labelsubfun}, \reflabel{\labelsubinter}, \reflabel{\labelsubinterl} and
\reflabel{\labelsubinterr}.

\begin{figure}[t]
\begin{spacing}{0.5}
  \begin{mathpar}
    \framebox{$\jatomic A$} \\
    \inferrule*{}{\jatomic \tyint} \and 
    \inferrule*{}{\jatomic {A \to B}} \and
    \inferrule*{}{\jatomic \alpha} \and
    \inferrule*{}{\jatomic {\fordis \alpha B A}} \and
    \inferrule*{}{\jatomic {\recordType l A}}
  \end{mathpar}
  \begin{mathpar}
    \formsub \\ 
    \rulesubtop \and \rulesubinter \and 
    \rulesubint \and \rulesubinterlcoerce \and 
    \rulesubrec \and \rulesubinterrcoerce \and
    \rulesubvar  \and \rulesubfun \and 
    \rulesubforallext 
  \end{mathpar}
\end{spacing}
  \caption{Subtyping rules of \name.}
  \label{fig:fi-subtype}
\end{figure}



%There are three rules which rather straightforward: \reflabel{\labelsubtop}
%says that every type is a subtype of $\top$; \reflabel{\labelsubint} and 
%\reflabel{\labelsubvar} define subtyping as a reflexive relation on integers and
%type variables.
%The rule \reflabel{\labelsubfun} says that a function is contravariant in 
%its parameter type and covariant in its return type. 
%The three rules dealing with intersection types are just what one would expect 
%when interpreting types as sets. 
%Under this interpretation, for example, the rule \reflabel{\labelsubinter}
%says that if $A_1$ is both the subset of $A_2$ and the subset of $A_3$, then
%$A_1$ is also the subset of the intersection of $A_2$ and $A_3$.

\paragraph{Polymorphism and Records.}
The subtyping rules introduced by \name refer to polymorphic constructs and records. 
\reflabel{\labelsubvar} defines subtyping as a reflexive relation on type variables.
In \reflabel{\labelsubforall} a universal quantifier ($\forall$) 
is covariant in its body, and contravariant in its disjointness constraints.
The \reflabel{\labelsubrec} rule says that records are covariant
within their fields' types.
The subtyping relation uses an auxiliary unary $ordinary$ relation,
which identifies types that are not intersections. The $ordinary$ conditions on two of the intersection rules are necessary to 
produce unique coercions~\cite{oliveira16disjoint}. The $ordinary$
relation needed to be extended with respect to \oldname.
As shown at the top of Figure~\ref{fig:fi-subtype}, the new types it contains are 
type variables, universal quantifiers and record types.

\paragraph{Properties of Subtyping.} The subtyping relation is reflexive and transitive.
\thmprf{0cm}{
\begin{restatable}[Subtyping reflexivity]{lemma}{subrefl}
  \label{lemma:subrefl}
  For any type $A$, $A \subtype A$.
\end{restatable}}
{-0.1cm}
{By induction on $A$.}
{0cm}
%\noindent \emph{Proof.} By induction on $A$.
%\restatableproof{lemma}{Subtyping reflexivity}{subrefl}{lemma:subrefl}
%{For any type $A$, $A \subtype A$.}
%{By induction on $A$.}{-0.1cm}%
%\begin{prf}
%By induction on $A$.
%\end{prf}%
%\begin{restatable}[Subtyping transitivity]{lemma}{subtrans}
%  \label{lemma:subtrans}
%  If $A \subtype B$ and $B \subtype C$, then $A \subtype C$.
%\end{restatable}%
%\noindent \emph{Proof.} By double induction on both derivations.%
\restatableproof{lemma}{Subtyping transitivity}{subtrans}{lemma:subtrans}
{If $A \subtype B$ and $B \subtype C$, then $A \subtype C$.}
{By double induction on both derivations.}{-0.1cm}

%\begin{prf}
%By double induction on both derivations. 
%\end{prf}
%\bruno{Too much space waisted between Lemma and proof. reduce the
%  white space.}
%TODO example showing contravariance in disjointness constraints goes here or in the overview 
%section?
%\paragraph{Metatheory.} As other standard subtyping relations, we can show that
%subtyping defined by $\subtype$ is also reflexive and transitive.
%
%\begin{lemma}[Subtyping is reflexive] \label{lemma:sub-refl}
%  For all type $ A $, $ A \subtype A $.
%\end{lemma}
%
%\begin{lemma}[Subtyping is transitive] \label{lemma:sub-trans}
%  If $ A_1 \subtype A_2 $ and $ A_2 \subtype A_3 $,
%  then $ A_1 \subtype A_3 $.
%\end{lemma}
\subsection{Typing}

\begin{comment}
\begin{figure}[!t]
  \begin{mathpar}
    \formwf \\ \rulewfint \and \rulewfvardis \and \rulewffun \and \rulewfrec \and 
    \rulewftop \and \rulewfforalldis \and \rulewfinterdis 
  \end{mathpar}

  \caption{Well-formedness rules for types of \name.}
  \label{fig:wf}
\end{figure}
\end{comment}


%  \begin{mathpar}
%    \formt \\ \ruletvar \and \ruletlam \and \ruletapp \and
%    \ruletblam \and \rulettapp \and \ruletmergedis 
%  \end{mathpar}

\paragraph{Well-formedness.}
The well-formedness rules are shown in the top part of Figure~\ref{fig:fi-type}. 
The new rules over \oldname are \reflabel{\labelwfvar} and \reflabel{\labelwfforall}. 
Their definition is quite straightforward, but note that the constraint in the latter
must be well-formed.

\begin{figure}
  \begin{spacing}{1}
  \begin{mathpar}
    \formwf \\ \rulewfint \and \rulewfvardis \and \rulewfrec \and 
    \rulewffun \and \rulewftop \and \rulewfforalldis \and \rulewfinterdis 
  \end{mathpar}
  \begin{mathpar}
    \formbi \\ \brulettop \and \bruletint \and \bruletvar \and \bruletann \and 
    \bruletapp \and \brulettappdis \and \bruletmergedis \and \bruletrec \and 
    \bruletprojr \and \bruletblamdis 
  \end{mathpar}
  \begin{mathpar}
    \formbc \\ \bruletlam \and \bruletsub
  \end{mathpar}
  \end{spacing}
  \caption{Well-formedness and type system of \name.}
  \label{fig:fi-type}
\end{figure}

\paragraph{Typing rules.}
Our typing rules are formulated as a bi-directional type-system. 
Just as in \oldname, this ensures the type-system is not only syntax-directed, but
also that there is no type ambiguity: that is, inferred types are unique.
The typing rules are shown in the bottom part of Figure~\ref{fig:fi-type}. 
Again, the reader is advised to ignore the
gray-shaded parts, as these will be explained later. 
The typing judgements are of the form: $\jcheck \Gamma e A$ and  
$\jinfer \Gamma e A$.
They read: ``in the typing context $\Gamma$, the term $e$ can be
checked or inferred to
type $A$'', respectively. 
The rules ported from \oldname are the
check rules for $\top$ (\reflabel{\labelttop}), integers (\reflabel{\labeltint}), 
variables (\reflabel{\labeltvar}),  application (\reflabel{\labeltapp}), merge operator  
(\reflabel{\labeltmerge}), annotations (\reflabel{\labeltann}); and infer rules
for lambda abstractions (\reflabel{\labeltlam}), and the subsumption rule 
(\reflabel{\labeltsub}).

\paragraph{Disjoint quantification.}
The new rules, inspired by System $F$, are the infer rules for type
application \reflabel{\labelttapp}, and for type abstraction
\reflabel{\labeltblam}.  Type abstraction is introduced by the big
lambda $\blamdis \alpha A e$, eliminated by the usual type application
$\tapp e A$ (\reflabel{\labelttapp}).  The disjointness constraint is
added to the context in \reflabel{\labeltblam}. During a type application, the
type system makes sure that the type argument satisfies the
disjointness constraint.  Type application performs an extra check
ensuring that the type to be instantiated is compatible
(i.e. disjoint) with the constraint associated with the abstracted
variable.  This is important, as it will retain the desired coherence
of our type-system.  For ease of discussion, also in
\reflabel{\labeltblam}, we require the type variable introduced by the
quantifier to be fresh.  For programs with type variable shadowing,
this requirement can be met straighforwardly by variable renaming.

\paragraph{Records.}
Finally, $\reflabel{\labeltrec}$ and $\reflabel{\labeltprojr}$ deal with record types.
The former infers a type for a record with label $l$ if it can infer a type for the
inner expression; the latter says if one can infer a record type $\recordType l A$ 
from an expression $e$, then it is safe to access the field $l$, and infering type $A$.



\newcommand{\name}{{\bf fi~}}
\newcommand{\Name}{{\bf fi}}

\newcommand{\target}{{\bf f~}}
\newcommand{\Target}{{\bf f}}

\lstdefinelanguage{F2J}{
  morekeywords={let,type,module,end,in},
  otherkeywords={->},
  sensitive=false, % whether keywords are case sensitive
  morecomment=[l]{--},
  morestring=[b]", % `b' means inside a string delimiters are escaped by a backslash.
  morestring=[b]'
}

\lstset{ %
  language=F2J,                % choose the language of the code
  columns=flexible,
  lineskip=-1pt,
  basicstyle=\ttfamily\small,       % the size of the fonts that are used for the code
  numbers=none,                   % where to put the line-numbers
  numberstyle=\ttfamily\tiny,      % the size of the fonts that are used for the line-numbers
  stepnumber=1,                   % the step between two line-numbers. If it's 1 each line will be numbered
  numbersep=5pt,                  % how far the line-numbers are from the code
  backgroundcolor=\color{white},  % choose the background color. You must add \usepackage{color}
  showspaces=false,               % show spaces adding particular underscores
  showstringspaces=false,         % underline spaces within strings
  showtabs=false,                 % show tabs within strings adding particular underscores
  morekeywords={var},
%  frame=single,                   % adds a frame around the code
  tabsize=2,                  % sets default tabsize to 2 spaces
  captionpos=none,                   % sets the caption-position to bottom
  breaklines=true,                % sets automatic line breaking
  breakatwhitespace=false,        % sets if automatic breaks should only happen at whitespace
  title=\lstname,                 % show the filename of files included with \lstinputlisting; also try caption instead of title
  escapeinside={(*}{*)},          % if you want to add a comment within your code
  keywordstyle=\ttfamily\bfseries,
% commentstyle=\color{Gray}
% stringstyle=\color{Green}
}

\newcommand{\fields}[1]{\code{fields} (#1)}
\newcommand{\remove}[2]{\code{remove} (#1, #2)}


\begin{document}

\begin{lemma}
  If $\tau_1 \andOp \tau_2 \subtype \tau_3$ and
  $\cdot \turns \tau_1 \andOp \tau_2$, then it is not possible that
  $\tau_1 \subtype \tau_3$ and $\tau_2 \subtype \tau_3$.
\end{lemma}

\begin{lemma}
  If $ \cdot \turns e : \tau \hookrightarrow E$, then $ e \Downarrow v $, and
  $E \Downarrow V$ where $\im v \Downarrow V$.
\end{lemma}

\end{document}


\section{Related Work}
\label{sec:related-work}

%*******************************************************************************
\paragraph{Coherence}
%*******************************************************************************

Reynolds invented Forsythe~\cite{reynolds1997design} in the 1980s. Our
merge operator is analogous to his operator $p_1, p_2 $. Forsythe has
a coherent semantics. The result was proved formally by
Reynolds~\cite{reynolds1991coherence} in a lambda calculus with
intersection types and a merge operator. However there are two key
differences to our work. Firstly the way coherence is ensured is
rather ad-hoc. He has four different typing rules for the merge
operator, each accounting for various possibilities of what the types
of the first and second components are. In some cases the meaning of
the second component takes precedence (that is, is biased) over the
first component. The set of rules is restrictive and it forbids, for
instance, the merge of two functions (even when they a provably
disjoint). In contrast, disjointness in \name has a well-defined
specification and it is quite flexible. Secondly, Reynolds calculus
does not support universal quantification. It is unclear to us whether
his set of rules would still ensure disjointness in the presence of
universal quantification. Since some biased choice is allowed in
Reynold's calculus the issues illustrated in Section~\ref{subsec:polymorphism} could be a problem.

Pierce~\cite{pierce1991programming2} made a comprehensive review
of coherence, especially on Curien and Ghelli~\cite{curienl1990coherence} and
Reynolds' methods of proving coherence; but he was not able to prove coherence
for his $F_\wedge$ calculus. He introduced a primitive $\code{glue}$ function as
a language extension which corresponds to our merge operator. However, in his
system users can ``glue'' two arbitrary values, which can lead to incoherence.

Our work is largely inspired by Dunfield~\cite{dunfield2014elaborating}. He
described a similar approach to ours: compiling a system with intersection types
and a merge operator into ordinary $ \lambda $-calculus terms with pairs. One
major difference is that our system does not include unions. However, as
acknowledged by Dunfield, his calculus lacks of coherence. He discusses the
issue of coherence throughout his paper, mentioning biased choice as an option
(albeit a rather unsatisfying one). He also mentioned that the notion of
disjoint intersection could be a good way to address the problem, but he did not
pursue this option in his work. In contrast to his work, we developed a type
system with disjoint intersection types and proposed disjoint quantification to
guarantee coherence in our calculus.

% \url{http://homepages.inf.ed.ac.uk/gdp/publications/Sub_Par.pdf}

% \cite{plotkin1994subtyping}

% Also discussed intersection types!~\cite{malayeri2008integrating}.

% Pierce Ph.D thesis: F<: + /|
%        technical report: F + /|, closer to ours

% \cite{barbanera1995intersection}
%
% \paragraph{Intersection types with polymorphism.}
% Our type system combines intersection types and parametric polymorphism. Closest
% to us is Pierce's work~\cite{pierce1991programming1} on a prototype
% compiler for a language with both intersection types, union types, and
% parametric polymorphism. Similarly to \name in his system universal
% quantifiers do not support bounded quantification. However Pierce did not try to prove any
% meta-theoretical results and his calculus does not have a merge
% operator.  Pierce also studied a system where both intersection
% types and bounded polymorphism are present in his Ph.D.
% dissertation~\cite{pierce1991programming2} and a 1997
% report~\cite{pierce1997intersection}.

Going in the direction of higher
kinds, Compagnoni and Pierce~\cite{compagnoni1996higher} added
intersection types to System $ F_{\omega} $ and used the new calculus,
$ F^{\omega}_{\wedge} $, to model multiple inheritance. In their
system, types include the construct of intersection of types of the
same kind $ K $. Davies and Pfenning
\cite{davies2000intersection} studied the interactions between
intersection types and effects in call-by-value languages. And they
proposed a ``value restriction'' for intersection types, similar to
value restriction on parametric polymorphism. Although they proposed a system with
parametric polymorphism, their subtyping rules are significantly different from ours,
since they consider parametric polymorphism
as the ``infinit analog'' of intersection polymorphism.

Recently,
Castagna et al.~\cite{Castagna:2014} studied an very expressive calculus that
has polymorphism and set-theoretic type connectives (such as intersections,
unions, and negations). As a result, in their calculus one is also able to
express a type variable that can be instantiated to any type other than
$\code{Int}$ as $\alpha \setminus \code{Int}$, which is syntactic sugar for
$\alpha \wedge \neg \code{Int}$. As a comparison, such a type will need a
disjoint quantifier, like $\fordis \alpha {\code{Int}} \alpha$, in our system.
Unfortunately their calculus does not include a merge operator like ours.

There have been attempts to provide a foundational calculus
for Scala that incorporates intersection
types~\cite{amin2014foundations,amin2012dependent}.
Although the minimal Scala-like calculus does not natively support
parametric polymorphism, it is possible to encode parametric
polymorphism with abstract type members. Thus it can be argued that
this calculus also supports intersection types and parametric
polymorphism. However, the type-soundness of a minimal Scala-like
calculus with intersection types and parametric polymorphism is not
yet proven. Recently, some form of intersection
types has been adopted in object-oriented languages such as Scala,
Ceylon, and Grace. Generally speaking,
the most significant difference to \name is that in all previous systems
there is no explicit introduction construct like our merge operator. As shown in
Section~\ref{subsec:OAs}, this feature is pivotal in supporting modularity
and extensibility because it allows dynamic composition of values.

\begin{comment}
only allow intersections of concrete types (classes),
whereas our language allows intersections of type variables, such as
\texttt{A \& B}. Without that vehicle, we would not be able to define
the generic \texttt{merge} function (below) for all interpretations of
a given algebra, and would incur boilerplate code:

\begin{lstlisting}
let merge [A, B] (f: ExpAlg A) (g: ExpAlg B) = {
  lit (x : Int) = f.lit x ,, g.lit x,
  add (x : A & B) (y : A & B) =
    f.add x y ,, g.add x y
}
\end{lstlisting}
\end{comment}

%*******************************************************************************
\paragraph{Other type systems with intersection types.}
%*******************************************************************************

% Although similar in spirit,
% our elaboration typing is simpler: we require subtyping in the case of
% applications, thus avoiding the subsumption rule. Besides, our treatment
% combines the merge rules ($ k $ ranges over $ \{1, 2\} $)
% \inferrule
% {\Gamma \turns e_k : A}
% {\Gamma \turns e_1 \mergeop e_2 : A}
% and the standard intersection-introduction rule
% \inferrule
% {\Gamma \turns e : A_1 \andalso \Gamma \turns e : A_2}
% {\Gamma \turns e : A_1 \inter A_2}
% into one rule:
% \inferrule [Merge]
% {\Gamma \turns e_1 : A_1 \andalso \Gamma \turns e_2 : A_2}
% {\Gamma \turns e_1 \mergeop e_2 : A_1 \inter A_2}
%Castagna, and Dunfield describe
%elaborating multi-fields records into merge of single-field records.
% Reynolds and Castagna do not consider elaboration and Dunfield do not
% formalize elaborating records.
% Both Pierce and Dunfield's system include a subsumption rule, which states that
% if an term has been inferred of type $ A $, then it is also of any
% supertype of $ A $. Our system does not have this rule.
Refinement
intersection~\cite{dunfield2007refined,davies2005practical,freeman1991refinement}
is the more conservative approach of adopting intersection types. It increases
only the expressiveness of types but not terms. But without a term-level
construct like ``merge'', it is not possible to encode various language
features. As an alternative to syntactic subtyping described in this paper,
Frisch et al.~\cite{frisch2008semantic} studied semantic subtyping. Semantic
subtyping seems to have important advantages over syntactic subtyping. One
worthy avenue for future work is to study languages with intersection types
and merge operator in a semantic subtyping setting.

%*******************************************************************************
\paragraph{Extensibility.}
%*******************************************************************************
One of our motivations to study systems
with intersections types is to better understand the
type system requirements needed to address extensibility problems.
A well-known problem in programming languages is the Expression
Problem~\cite{wadler1998expression}. In recent years there have been
various solutions to the Expression Problem in the literature. Mostly
the solutions are presented in a specific language, using the language
constructs of that language. For example, in Haskell, type classes~\cite{WadlerB89}
can be used to implement type-theoretic encodings of
datatypes~\cite{Hinze:2006}. It has been shown~\cite{finally-tagless}
that, when encodings of datatypes are modeled with type classes,
the subclassing mechanism of type classes can be used to achieve
extensibility and reuse of operations. Using such techniques provides
a solution to the Expression Problem. Similarly, in OO languages with
generics, it is possible to use generic interfaces and classes to
implement type-theoretic encodings of datatypes. Conventional
subtyping allows the interfaces and classes to be extended, which can
also be used to provide extensibility and reuse of operations. Using
such techniques, it is also possible to solve the Expression Problem
in OO languages~\cite{oliveira09modular,oliveira2012extensibility}.
It is even possible to solve the Expression Problem in theorem provers
like Coq, by exploiting Coq's type class mechanism~\cite{DelawareOS13}.
Nevertheless, although there is a clear connection between all those
techniques and type-theoretic encodings of datatypes, as far as we
know, no one has studied the expression problem from a more
type-theoretic point of view.

% As shown in Section~\ref{subsec:OAs}, a system
% with intersection types, parametric polymorphism, the merge operator
% and disjoint quantification can be used to explain type-theoretic
% encodings with subtyping and extensibility.

% Intersection types have been shown to be useful in designing languages that
% support modularity.~\cite{nystrom2006j}

% \paragraph{Extensible records.}

%\george{Record field deletion is also possible.}

% http://elm-lang.org/learn/Records.elm

% Encoding records using intersection types appeared in
% Reynolds~\cite{reynolds1997design} and Castagna et
% al.~\cite{castagna1995calculus}. Although Dunfield also discussed this idea in
% his paper \cite{dunfield2014elaborating}, he only provided an implementation but
% not a formalization. Very similar to our treatment of elaborating records is
% Cardelli's work~\cite{cardelli1992extensible} on translating a calculus, named
% $ F_{\subtype \rho}$, with extensible records to a simpler calculus that without
% records primitives (in which case is $ F_{\subtype} $). But he did not consider
% encoding multi-field records as intersections; hence his translation is more
% heavyweight. Crary~\cite{crary1998simple} used intersection types and
% existential types to address the problem that arises when interpreting method
% dispatch as self-application. But in his paper, intersection types are not used
% to encode multi-field records.

% Wand~\cite{wand1987complete} started the work on extensible records and proposed
% row types~\cite{wand1989type} for records. Cardelli and
% Mitchell~\cite{cardelli1990operations} defined three primitive operations on
% records that are similar to ours: \emph{selection}, \emph{restriction}, and
% \emph{extension}. The merge operator in \name plays the same role as extension.
% Following Cardelli and Mitchell's approach,
% of restriction and extension. Both Leijen's systems~\cite{leijen2004first,leijen2005extensible}
% and ours allow records that contain
% duplicate labels. Leijen's system is more sophisticated. For example, it supports
% passing record labels as arguments to functions. He also showed an encoding of
% intersection types using first-class labels.

% Chlipala's
% \texttt{Ur}~\cite{chlipala2010ur} explains record as type level
% constructs.\bruno{What is the point of citing Chlipala's paper?}

% Our system can be adapted to simulate systems that support extensible
% records but not intersection of ordinary types like \texttt{Int} and
% \texttt{Float} by allowing only intersection of record types.
%
% $ \turnsrec A $ states that $ A $ is a record type, or the intersection of
% record types, and so forth.
%
% \inferrule [RecBase] {} {\turnsrec \recordType l A}
%
% \inferrule [RecStep]
% {\turnsrec A_1 \andalso \turnsrec A_2}
% {\turnsrec A_1 \inter A_2}
%
% \inferrule [Merge']
% {\Gamma \turns e_1 : A_1 \yields {E_1} \andalso \turnsrec A_1 \\
%  \Gamma \turns e_2 : A_2 \yields {E_2} \andalso \turnsrec A_2}
% {\Gamma \turns e_1 \mergeop e_2 : A_1 \inter A_2 \yields {\pair {E_1} {E_2}}}
%
% R{\'e}my~\cite{remy1989type}

%*******************************************************************************
\paragraph{Trait calculi.}
%*******************************************************************************
Fisher and Reppy~\cite{fisher2004typed} provided a dedicated statically typed
calculus for modeling traits. \name is not dedicated to traits; but rather, it
supports a source language that models traits. Compared to Fisher and Reppy's
calculus, \name is more lightweight. For example, self reference is not in the
language of \name. One reason for the difference is that Fisher and Reppy's
calculus supports \emph{classes} in addition to traits, and considers the
interaction between them, whereas our object oriented source language is
\emph{prototype}-based---the mechanism for code reuse is purely trait.


\section{Conclusion and Future Work}
\label{sec:conclusion}

This paper described \name: a language that combines
intersection types and a merge operator.
The language is proved to be type-safe and coherent.
To ensure coherence the type system accepts only
disjoint intersections. We believe that disjoint intersection types are
intuitive, and at the same time expressive. We have shown the
applicability of disjoint intersection types to model a simple form of traits.

We implemented the core functionalities of the \name as part of a JVM-based
compiler. Based on the type system of \name, we have built an ML-like
source language compiler that offers interoperability with Java (such as object
creation and method calls). The source language is loosely based on the more
general System $F_{\omega}$ and supports a
number of other features, including records, polymorphism, mutually recursive
\code{let} bindings, type aliases, algebraic data types, pattern matching, and
first-class modules that are encoded using \code{letrec} and records.

For the future, we intend to improve our source language
and show the power of disjoint intersection types in large case
studies. One pressing challenge is to address the intersction between 
disjoint intersection types and polymorphism.
We are also interested in extending our work
to systems with a $\top$ type. This will also require an adjustment
to the notion of disjoint types. A suitable notion of
disjointness between two types $A$ and $B$ in the presence of $\top$
would be to require that the only common supertype of $A$ and $B$ is $\top$.
Finally we would like to study the
addition of union types. This will also require changes in our
notion of disjointness, since with union types there always exists
a type $A \union B$, which is the common supertype of two
types $A$ and $B$.

% Some immediate topics for
% further improvement of the results in this paper are discussed next.
%
% \paragraph{Union types.}
%
% If a type system ever contains union types (the counterpart of intersection
% types), with the following standard subtyping rules,
% \begin{mathpar}
%   \inferrule* [right=Sub\_Union\_1]
%     { }
%     {A \subtype A \union B}
%
%   \inferrule* [right=Sub\_Union\_2]
%     { }
%     {B \subtype A \union B}
% \end{mathpar}
% then no two types $A$ and $B$ can ever be disjoint, since there always exists
% the type $A \union B$, which is their common supertype. So it is reasonable to
% conjecture that such system cannot be coherent.
% \bruno{I wouldn't say this is a motivation: it sounds like we caould
%   not support union types, when I think this is not true. For example
% we could say something like: there does not exist an \emph{atomic} C ...}
%
%
% \paragraph{Implementation.}
%
% We implemented the core functionalities of the \name as part of a JVM-based
% compiler. Based on the type system of \name, we built an ML-like
% source language compiler that offers interoperability with Java (such as object
% creation and method calls). The source language is loosely based on the more
% general System $F_{\omega}$ and supports a
% number of other features, including records, mutually recursive
% \code{let} bindings, type aliases, algebraic data types, pattern matching, and
% first-class modules that are encoded using \code{letrec} and records.
%
% Relevant to this paper are the three phases in the compiler, which
% collectively turn source programs into System $F$:
%
% \begin{enumerate}
% \item A \emph{typechecking} phase that checks the usage of \name features and
%   other source language features against an abstract syntax tree that follows
%   the source syntax.
%
% \item A \emph{desugaring} phase that translates well-typed source terms into
%   \name terms. Source-level features such as multi-field records, type aliases
%   are removed at this phase. The resulting program is just an \name term
%   extended with some other constructs necessary for code generation.
%
% \item A \emph{translation} phase that turns well-typed \name terms into System
%   $F$ ones.
% \end{enumerate}
%
% Phase 3 is what we have formalized in this paper.
%
%
% \paragraph{Reduce the number of coercions.}
%
% Our translation inserts a coercion (many of them are identity functions)
% whenever subtyping occurs during a function application, which could mean
% notable run-time overhead. In the current implementation, we introduced a
% partial evaluator with three simple rewriting rules to eliminate the redundant
% identity functions as another compiler phase after the translation. In another
% version of our implementation, partial evaluation is weaved into the process of
% translation so that the unwanted identity functions are not introduced during
% the translation. Besides, since the order of the two types in a binary
% intersection does not matter, we may normalize them to avoid unnecessary
% coercions.


\input{sections/bibliography.tex}

% \acks
%
% Acknowledgments, if needed.

\clearpage
\onecolumn
\appendix
\section{Type well-formedness}

% $ \ftv \cdot $ reads: ``the free type variable of''.

% \begin{figure}[h]
%   \framebox{$ \jwf \Gamma A $}
%
%   \begin{mathpar}
%     \ruleWF
%   \end{mathpar}
%   \caption{Type well-formedness in \name.}
% \end{figure}
%
% \begin{figure}[h]
%   \framebox{$ \jwf \Gamma T $}
%
%   \begin{mathpar}
%     \ruletgtWF
%   \end{mathpar}
%   \caption{Type well-formedness in the target type system.}
% \end{figure}

\section{Target Type System}

\begin{figure}[h]
  \framebox{$ \hastype G E T $}
  \begin{mathpar}

    \ruletargetvar

    \ruletargetlam

    \ruletargetapp

    \ruletargetblam

    \ruletargettapp

    \ruletargetpair

    \ruletargetprojl

    \ruletargetprojr

  \end{mathpar}

  \caption{Target type system.}
\end{figure}

\section{Proofs}

\begin{proof}
By structural induction on the types and the corresponding inference rule. \\

\texttt{(SubVar)}

\texttt{(SubFun)}

\texttt{(SubForall)}

\texttt{(SubAnd1)}

\texttt{(SubAnd2)}

\texttt{(SubAnd3)}

\texttt{(SubRcd)}

\end{proof}

\begin{lemma}
  If $$ \Gamma \turnsget \ty ; l = C ; \ty_1 $$
  then $$ \image \Gamma \turns C : \image \ty \to \image {\ty_1} $$
\end{lemma}

\begin{proof}
By structural induction on the type and the corresponding inference rule. \\

\texttt{(Get-Base)} $ \Gamma \turnsget \recordtype l \ty ; l = \idmono {\image {\recordtype l \ty})} ; \ty $ \\

By the induction hypothesis
$$ \image \Gamma \turns \idmono {\image {\recordtype l \ty}} : \image {\recordtype l \ty} \to \image \ty $$

\texttt{(Get-Left)} \\
\texttt{(Get-Right)} \\

\end{proof}

\begin{lemma}
  If $$ \Gamma \turnsput \ty ; l ; E = C ; \ty_1 $$
  then $$ \image \Gamma \turns C : \image \ty \to \image \ty $$
\end{lemma}

\begin{proof}
By structural induction on the type and the corresponding inference rule. \\

\texttt{(Put-Base)} \\
\texttt{(Put-Left)} \\
\texttt{(Put-Right)} \\
\end{proof}

\begin{lemma} \label{preserve-wf}
  If   $$ \Gamma \turns \ty $$
  then $$ \image \Gamma \turns \image \ty $$
\end{lemma}

\begin{proof}
Since $$ \Gamma \turns \ty $$
It follows from \texttt{(FI-WF)} that
  $$ \ftv \ty  \subseteq \ftv {\Gamma} $$
And hence
  $$ \ftv {\image \ty} \subseteq \ftv {\image \Gamma} $$
By \texttt{(F-WF)} we have
  $$ \Gamma \turns \ty $$
\end{proof}

\begin{theorem}[Type preserving translation]
  If   $$ \Gamma \turns e : \ty \yields E  $$
  then $$ \image \Gamma \turns E : \image \ty $$
\end{theorem}

\begin{proof}
By structural induction on the expression and the corresponding inference rule. \\

\texttt{(TrVar)} $ \Gamma \turns x : \ty \yields x $ \\

It follows from \texttt{(TrVar)} that
  $$ (x : t) \in \Gamma $$
Based on the definition of $ \image \cdot $,
  $$ (x : \image t) \in \image \Gamma $$
Thus we have by \texttt{(F-Var)} that
  $$ \image \Gamma \turns x : \image \ty $$

\texttt{(TrAbs)} $ \Gamma \turns \lambda (x : \ty_1). e : \ty_1 \to \ty_2 \yields {\absty x {\image {\ty_1}} E} $ \\

It follows from \texttt{(TrAbs)} that
  $$ \Gamma, x : \ty_1 \turns e : \ty_2 \yields E $$
And by the induction hypothesis that
  $$ \image \Gamma, x : \image {\ty_1} \turns E : \image {\ty_2} $$
By \texttt{(TrAbs)} we also have
  $$ \Gamma \turns \ty_1 $$
It follows from Lemma \ref{preserve-wf} that
  $$ \image \Gamma \turns \image {\ty_1} $$
Hence by \texttt{(F-Abs)} and the definition of $ \image \cdot $ we have
  $$ \image \Gamma \turns \absty x {\image {\ty_1}} E : \image {\ty_1 \to \ty_2} $$

\texttt{(TrApp)} $ \Gamma \turns \app {e_1} {e_2} : \ty_2 \yields {E_1 (\app C {E_2})} $ \\

From \texttt{(TrApp)} we have
  $$ \Gamma \turns \ty_3 <: \ty_1 \yields C $$
Applying Lemma \ref{type-coerce} to the above we have
  $$ \image \Gamma \turns C : \image {\ty_3} \to \image {\ty_1} $$
Also from \texttt{(TrApp)} and the induction hypothesis
  $$ \image \Gamma \turns E_1 : \image {\ty_1} \to \image {\ty_2} $$
Also from \texttt{(TrApp)} and the induction hypothesis
  $$ \image \Gamma \turns E_2 : \image {\ty_3} $$
Assembling those parts using \texttt{(F-App)} we come to
  $$ \image \Gamma \turns E_1 (\app C {E_2}) : \image {\ty_2} $$
\end{proof}

\texttt{(TrTAbs)} $ \Gamma \turns \Lambda \alpha. e : \forall \alpha. \ty \yields {\forall \alpha. E} $ \\

From \texttt{(TrTAbs)} we have
  $$ \Gamma \turns e : \ty \yields E $$
By the induction hypothesis we have
  $$ \image \Gamma \turns E : \image \ty $$
Thus by \texttt{(F-TAbs)} and the definition of $ \image \cdot $
  $$ \Gamma \turns \Lambda \alpha. E : \image {\forall \alpha. \ty} $$


\texttt{(TrTApp)} $ \Gamma \turns e \; \ty  : [\alpha := t] \ty_1 \yields {E \; \image \ty} $ \\

From \texttt{(TrTApp)} we have
  $$ \Gamma \turns e : \forall \alpha. \ty_1 \yields E $$
And by the induction hypothesis that
  $$ \image \Gamma \turns E : \forall \alpha. \image {\ty_1} $$
Also from \texttt{(TrTApp)} and Lemma \ref{preserve-wf} we have
  $$ \image \Gamma \turns \image \ty $$
Then by \texttt{(F-TApp)} that
  $$ \image \Gamma \turns E \; \image \ty : [\alpha := \image \ty ] \image {\ty_1} $$
Therefore
  $$ \image \Gamma \turns E \; \image \ty : \image {[\alpha := \ty ] \image {\ty_1}} $$

% \texttt{(TrMerge)} $ \Gamma \turns e_1 \merge e_2 : \ty_1 \& \ty_2 % % \yields {\tupled {E1, E2} $ \\

From \texttt{(TrMerge)} and the induction hypothesis we have
  $$ \image \Gamma \turns E_1 : \image {\ty_1} $$
and
  $$ \image \Gamma \turns E_2 : \image {\ty_2} $$
Hence by \texttt{(F-Pair)}
  $$ \image \Gamma \turns \tupled {E_1, E_2} : \tupled {\image {\ty_1}, \image {\ty_2}} $$
Hence by the definition of $ \image \cdot $
  $$ \image \Gamma \turns \tupled {E_1, E_2} : \image {\ty_1 \& \ty_2} $$

\texttt{(TrRcdIntro)} $ \Gamma \turns \recordintro l e : \recordtype l \ty \yields E $ \\

From \texttt{(TrRcdIntro)} we have
  $$ \Gamma \turns e : \ty \yields E $$
And by the induction hypothesis that
  $$ \image \Gamma \turns E : \image \ty $$
Thus by the definition of $ \image \cdot $
  $$ \image \Gamma \turns E : \image {\recordtype l \ty} $$

\texttt{(TrRcdElim)} $ \Gamma \turns e.l : \ty_1 \yields {\app C E} $ \\

From \texttt{(TrRcdElim)}
  $$ \Gamma \turns e : \ty \yields E $$
And by the induction hypothesis that
  $$ \image \Gamma \turns E : \image \ty $$
Also from \texttt{(TrRcdEim)}
  $$ \Gamma \turnsget e ; l = C ; \ty_1 $$
Applying Lemma \ref{type-get} to the above we have
  $$ \image \Gamma \turns C : \image \ty \to \image {\ty_1}  $$
Hence by \texttt{(F-App)} we have
  $$ \image \Gamma \turns \app C E : \image {\ty_1} $$

% \texttt{(TrRcdUpd)} $ \Gamma \turns \rcdupd{e} {l} {e_1} : \ty \yields {\app C E} $ \\

From \texttt{(TrRcdUpd)}
  $$ \Gamma \turns e : \ty \yields E $$
And by the induction hypothesis that
  $$ \image \Gamma \turns E : \image \ty $$
Also from \texttt{(TrRcdUpd)}
  $$ \Gamma \turnsput \ty ; l; E = C ; \ty_1 $$
Applying Lemma \ref{type-put} to the above we have
  $$ \image \Gamma \turns C : \image \ty \to \image \ty  $$
Hence by \texttt{(F-App)} we have
  $$ \image \Gamma \turns \app C E : \image \ty $$


\end{document}
