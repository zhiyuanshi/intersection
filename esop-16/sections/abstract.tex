Dunfield has shown that a simply typed core calculus with
intersection types and a merge operator forms a powerful foundation
for various programming language features. While his calculus
is type-safe, it lacks \emph{coherence}:
different derivations for the same expression can lead to different
results. The lack of coherence is important disadvantage for adoption
of his core calculus in implementations of programming languages, as
the semantics of the programming language becomes implementation
dependent. 

This paper presents \name: a core calculus with a variant of
\emph{intersection types} and a
\emph{merge operator}. The semantics \name is both type-safe and
coherent. Coherence is achieved by ensuring that intersection types
are \emph{disjoint}. Formally two types are disjoint if they do not
share a common supertype. We present a type system that prevents
intersection types that are not disjoint, as well as an algorithmic
specification to determine whether two types are disjoint. Moreover 
we show the applicability of this calculus to express a powerful 
form of dynamically composable traits, paving the way for new 
designs for object-oriented programming languages.\bruno{polish better} 
