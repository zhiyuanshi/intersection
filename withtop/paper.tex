\documentclass[numbers]{sigplanconf}

% For pdflatex, replaced by fontspec:
\usepackage[T1]{fontenc}
\usepackage[utf8]{inputenc}


%% For general writing
\usepackage{fixltx2e}
\usepackage[usenames,dvipsnames,svgnames,table]{xcolor}
\usepackage{url}
\usepackage{fancyvrb}
\usepackage{mdwlist}  % Miscellaneous list-related commands
\usepackage{xspace}   % Smart spacing
\usepackage{ucs}

% https://www.nesono.com/?q=book/export/html/347
% Package for inserting TODO statements in nice colorful boxes - so that you
% won’t forget to fix/remove them. To add a todo statement, use something like
% \todo{Find better wording here}.
\usepackage{todonotes}


%% Math & theoretical computer science
\usepackage{amsmath}
\usepackage{amssymb}
\usepackage{amsthm}
\usepackage{bm}         % Bold symbols in maths mode
\usepackage{dsfont}
\usepackage{stmaryrd}
\usepackage{mathtools}  % For "::=" ( \Coloneqq )
% http://tex.stackexchange.com/questions/114151/how-do-i-reference-in-appendix-a-theorem-given-in-the-body
\usepackage{thmtools, thm-restate}

\theoremstyle{definition}
\newtheorem{definition}{Definition}

\theoremstyle{plain}
\newtheorem{theorem}{Theorem}
\newtheorem{lemma}{Lemma}


%% Code listings
\usepackage{listings}

\lstdefinestyle{f2j}{
    basicstyle=\ttfamily\small,
    keywordstyle=\sffamily\bfseries,
    tabsize=2,
    keepspaces=true,
    showstringspaces=false,
    escapeinside={(*}{*)},
    morekeywords={let,in}
}

\lstset{style=f2j}


%% Font
\usepackage[euler-digits,euler-hat-accent]{eulervm}

%% Typesetting inference rules
% \usepackage{styles/bcprules}    % by Benjamin C. Pierce
% \usepackage{styles/cmll}
\usepackage{styles/mathpartir}  % by Didier Rémy (http://gallium.inria.fr/~remy/latex/mathpartir.html


% Copied from the FCore paper:
\usepackage[colorlinks=true,allcolors=black,breaklinks,draft=false]{hyperref}   % hyperlinks, including DOIs and URLs in bibliography
% known bug: http://tex.stackexchange.com/questions/1522/pdfendlink-ended-up-in-different-nesting-level-than-pdfstartlink

\newcommand{\hast}{\!:\!}

% Relations
\newcommand{\subtype}   {<:}

\definecolor{facebook}{HTML}{3B5998}
\newcommand{\yields}[1]{\textcolor{facebook}{\; \hookrightarrow {#1}}}

% Helpers
\newcommand{\ftv}[1]{\textit{ftv}({#1})}

% Spacing
\newcommand{\binderSpacing}{\,}
\newcommand{\appSpacing}{\;}

% Types
% \newcommand{\top}{\{\}}
\newcommand{\andOp}{\with}

% Expressions
\newcommand{\lam}[3]{\lambda (#1 \hast #2).\binderSpacing #3}
\newcommand{\mergeOp}{,,}
\newcommand{\restrictOp}{\setminus}
\newcommand{\recordUpdate}[3]{#1 \; \mathbf{with} \; \{#2 = #3\}}


\newcommand{\Int}{\code{Int}}
\newcommand{\String}{\code{String}}
\newcommand{\Bool}{\code{Bool}}
\newcommand{\I}{\code{i}}
\newcommand{\J}{\code{j}}


% Rules

% Couleurs
\colorlet{subColor}{OliveGreen}
\colorlet{targetColor}{BrickRed}

% Subtyping labels
\newcommand{\ruleLabelSub}{\bm{\textcolor{subColor}{sub}}}
\newcommand{\ruleLabelSubVar}{\ruleLabelSub\text{var}}
\newcommand{\ruleLabelSubTop}{\ruleLabelSub\text{top}}
\newcommand{\ruleLabelSubFun}{\ruleLabelSub\text{fun}}
\newcommand{\ruleLabelSubForall}{\ruleLabelSub\text{forall}}
\newcommand{\ruleLabelSubAnd}{\ruleLabelSub\text{and}}
\newcommand{\ruleLabelSubAndLeft}{\ruleLabelSub{\text{and}_1}}
\newcommand{\ruleLabelSubAndRight}{\ruleLabelSub{\text{and}_2}}
\newcommand{\ruleLabelSubRec}{\ruleLabelSub\text{rec}}

% Source/elaboration and labels
\newcommand{\judgeSourceWF}[2]{#1 \; \textcolor{sourceColor}{\turns} \; #2}
\newcommand{\ruleLabelSourceRecUpd}{\ruleLabelSource\text{rec-upd}}


% Presentation
\definecolor{lightyellow}{HTML}{FFFFE0}


% To be retired
\newcommand{\turnsGet}{\turns_{\textrm{get}}}
\newcommand{\turnsPut}{\turns_{\textrm{put}}}
\newcommand{\turnsrec}{\turns_{\textrm{rec}}}
\newcommand{\rulename}[1]{(\textrm{#1})}



\newcommand{\formwf}{\framebox{$ \jwf \Gamma A $}}

\newcommand{\makelabelwf}[1]{WF$#1$}

\newcommand{\labelwfvar}{\makelabelwf \alpha}
\newcommand{\rulewfvar}{
  \inferrule* [right=\labelwfvar]
    {\alpha \in \Gamma}
    {\jwf \Gamma \alpha}
}

\newcommand{\rulewfvardis}{
  \inferrule* [right=\labelwfvar]
    {\alpha \disjoint A \in \Gamma}
    {\jwf \Gamma \alpha}
}

\newcommand{\labelwftop}{\makelabelwf \top}
\newcommand{\rulewftop}{
  \inferrule* [right=\labelwftop]
    { }
    {\jwf \Gamma \top}
}

\newcommand{\labelwfint}{\makelabelwf {\mathbb{Z}}}
\newcommand{\rulewfint}{
  \inferrule* [right=\labelwfint]
    { }
    {\jwf \Gamma {\code{Int}}}
}

\newcommand{\labelwffun}{\makelabelwf \rightarrow}
\newcommand{\rulewffun}{
  \mprset {sep=1em}
  \inferrule* [right=\labelwffun]
    {\jwf \Gamma A \\ \jwf \Gamma B}
    {\jwf \Gamma {A \to B}}
  \mprset {sep=2em}
}

\newcommand{\labelwfforall}{\makelabelwf \forall}
\newcommand{\rulewfforall}{
  \inferrule* [right=\labelwfforall]
    {\jwf {\Gamma, \alpha} A}
    {\jwf \Gamma {\for \alpha A}}
}

\newcommand{\rulewfforalldis}{
  \mprset {sep=1em}
  \inferrule* [right=\labelwfforall]
    {\jwf \Gamma A \\ \jwf {\Gamma, \alpha \disjoint A} B}
    {\jwf \Gamma {\fordis \alpha A B}}
  \mprset {sep=2em}
}

\newcommand{\labelwfinter}{\makelabelwf \inter}
\newcommand{\rulewfinter}{
  \inferrule* [right=\labelwfinter]
    {\jwf \Gamma A \\ \jwf \Gamma B}
    {\jwf \Gamma {A \inter B}}
}

\newcommand{\rulewfinterdis}{
  \mprset {sep=1em}
  \inferrule* [right=\labelwfinter]
    {\jwf \Gamma A \\
     \jwf \Gamma B \\ 
     \jdis \Gamma A B}
    {\jwf \Gamma {A \inter B}}
  \mprset {sep=2em}
}

\newcommand{\labelwfrec}{\makelabelwf R}
\newcommand{\rulewfrec}{
  \inferrule* [right=\labelwfrec]
    {\jwf \Gamma A}
    {\jwf \Gamma {\recordType l A}}
}

\newcommand{\disjointvar}{
  \inferrule* [right=DisjointVar]
    {\alpha * B \in \Gamma}
    {\isdisjoint \Gamma \alpha B}
}

\newcommand{\disjointinterleft}{
  \inferrule* [right=DisjointInter1]
    {\isdisjoint \Gamma A C \\ \isdisjoint \Gamma B C}
    {\isdisjoint \Gamma {A \& B} {C}}
}

\newcommand{\disjointinterright}{
  \inferrule* [right=DisjointInter2]
    {\isdisjoint \Gamma A B \\ \isdisjoint \Gamma A C}
    {\isdisjoint \Gamma {A} {B \& C}}
}

\newcommand{\disjointfun}{
  \inferrule* [right=DisjointFun]
    {\isdisjoint \Gamma B D}
    {\isdisjoint \Gamma {A \to B} {C \to D}}
}

\newcommand{\disjointforall}{
  \inferrule* [right=DisjointForall]
    {\isdisjoint \Gamma A C}
    {\isdisjoint \Gamma {\for {\alpha * B} A} {\for {\alpha * B} C}}
}

\newcommand{\disjointatomic}{
  \inferrule* [right=DisjointAtomic]
    {A \not\sim B}
    {\isdisjoint \Gamma {A} {B}}
}

\newcommand{\rulelabelSub}{\text{Sub\_}}
\newcommand{\rulelabelsubvar}{\rulelabelSub\text{Var}}
\newcommand{\rulelabelSubTop}{\rulelabelSub\text{Top}}
\newcommand{\rulelabelsubfun}{\rulelabelSub\text{Fun}}
\newcommand{\rulelabelsubforall}{\rulelabelSub\text{Forall}}
\newcommand{\rulelabelsubinter}{\rulelabelSub\text{And}}
\newcommand{\rulelabelsubinterl}{\rulelabelSub{\text{And}_1}}
\newcommand{\rulelabelsubinterr}{\rulelabelSub{\text{And}_2}}

\newcommand{\rulesubvar}{
\inferrule* [right=$\rulelabelsubvar$]
  { }
  {\alpha \subtype \alpha \yields {\lamty x {\im \alpha} x}}
}

\newcommand{\rulesubfun}{
\inferrule* [right=$\rulelabelsubfun$]
  {A_3 \subtype A_1 \yields {C_1} \\ A_2 \subtype A_4 \yields {C_2}}
  {A_1 \to A_2 \subtype A_3 \to A_4
  \yields
      {\lamty f {\im {A_1 \to A_2}}
      {\lamty x {\im {A_3}}
          {\app {C_2} {(\app f {(\app {C_1} x)})}}}}}
}

\newcommand{\rulesubforall}{
\inferrule* [right=$\rulelabelsubforall$]
  {A_1 \subtype \subst {\alpha_1} {\alpha_2} A_2 \yields C}
  {\for {\alpha_1} A_1 \subtype \for {\alpha_2} A_2
    \yields
      {\lamty f {\im {\for {\alpha_1} A_1}}
        {\blam \alpha {\app C {(\app f \alpha)}}}}}
}

\newcommand{\rulesubinter}{
\inferrule* [right=$\rulelabelsubinter$]
  {A_1 \subtype A_2 \yields {C_1} \\ A_1 \subtype A_3 \yields {C_2}}
  {A_1 \subtype A_2 \inter A_3
    \yields
      {\lamty x {\im {A_1}}
        {\pair {\app {C_1} x} {\app {C_2} x}}}}
}

\newcommand{\rulesubinterl}{
\inferrule* [right=$\rulelabelsubinterl$]
  {A_1 \subtype A_3 \yields C}
  {A_1 \inter A_2 \subtype A_3
    \yields
      {\lamty x {\im {A_1 \inter A_2}}
        {\app C {(\proj 1 x)}}}}
}

\newcommand{\rulesubinterr}{
\inferrule* [right=$\rulelabelsubinterr$]
  {A_2 \subtype A_3 \yields C}
  {A_1 \inter A_2 \subtype A_3
    \yields
      {\lamty x {\im {A_1 \inter A_2}}
        {\app C {(\proj 2 x)}}}}
}

\newcommand{\rulelabel}{\text{Ty}}
\newcommand{\rulelabelSelect}{\text{Sel}}
\newcommand{\rulelabelRestrict}{\text{Res}}

% Var
\newcommand{\rulelabelVar}{\rulelabel\text{Var}}
\newcommand{\ruleVar} {
\inferrule* [right=$\rulelabelVar$]
  {x \hast A \in \Gamma}
  {\hastype \Gamma x A \yields x}
}

% Top
\newcommand{\rulelabelTop}{\rulelabel\text{Top}}
\newcommand{\ruleTop} {
\inferrule* [right=$\rulelabelTop$]
  { }
  {\hastype \Gamma \top \top \yields {()}}
}

% Lam
\newcommand{\rulelabelLam}{\rulelabel\text{Lam}}
\newcommand{\ruleLam} {
\inferrule* [right=$\rulelabelLam$]
  {\istype \Gamma A \\ \hastype {\Gamma, x \hast A} e B \yields E}
  {\hastype \Gamma {\lam x A e} {A \to B} \yields {\lam x {\im A} E}}
}

% App
\newcommand{\rulelabelApp}{\rulelabel\text{App}}
\newcommand{\ruleApp}{
\inferrule* [right=$\rulelabelApp$]
  {\hastype \Gamma {e_1} {A_1 \to A_2} \yields {E_1} \\
   \hastype \Gamma {e_2} {A_3} \yields {E_2} \\
   A_3 \subtype A_1 \yields C}
  {\hastype \Gamma {\app {e_1} {e_2}} {A_2} \yields {\app {E_1} {(\app C E_2)}}}
}

% BLam
\newcommand{\rulelabelBLam}{\rulelabel\text{BLam}}
\newcommand{\ruleBLam}{
\inferrule* [right=$\rulelabelBLam$]
  {\hastype {\Gamma, \alpha} e A \yields E}
  {\hastype \Gamma {\blam \alpha e} {\for \alpha A} \yields {\blam \alpha E}}
}

% TApp
\newcommand{\rulelabelTApp}{\rulelabel\text{TApp}}
\newcommand{\ruleTApp}{
\inferrule* [right=$\rulelabelTApp$]
  {\hastype \Gamma e {\for \alpha B} \yields E \\ \istype \Gamma A}
  {\hastype \Gamma {\tapp e A} {\subst A \alpha B} \yields {\tapp E {\im A}}}
}

% Merge
\newcommand{\rulelabelMerge}{\rulelabel\text{Merge}}
\newcommand{\ruleMerge}{
\inferrule* [right=$\rulelabelMerge$]
  {\hastype \Gamma {e_1} A \yields {E_1} \\
   \hastype \Gamma {e_2} B \yields {E_2} \\
   % A \bot B}
   \isdisjoint \Gamma A B}
  {\hastype \Gamma {e_1 \mergeOp e_2} {A \intersect B} \yields {\pair {E_1} {E_2}}}
}

% ConstraintIntro
\newcommand{\rulelabelConstraintIntro}{\rulelabel\text{ConstraintIntro}}
\newcommand{\ruleConstraintIntro}{
  \inferrule* [right=$\rulelabelConstraintIntro$]
    {\istype \Gamma {A_1} \\ \istype \Gamma {A_2} \\
     \hastype {\Gamma, A_1 \disjoint A_2} e B \yields E}
    {\hastype \Gamma {\assume {(A_1 \disjoint A_1)} e} {\constraints {A_1   \disjoint A_2} B} \yields E}
}

% ConstraintElim
\newcommand{\rulelabelConstraintElim}{\rulelabel\text{ConstraintElim}}
\newcommand{\ruleConstraintElim}{
\inferrule* [right=$\rulelabelConstraintElim$]
  {\hastype \Gamma e {\constraints {A_1 \disjoint A_2} B} \yields E \\
  \isdisjoint \Gamma {A_1} {A_2}}
  {\hastype \Gamma {\app e {\_}} B \yields E}
}

% rec-con
\newcommand{\rulelabelRecConstruct}{\rulelabel\text{rec-construct}}
\newcommand{\rulerecordConstruct}{
\inferrule* [right=$\rulelabelRecConstruct$]
  {\hastype \Gamma e A \yields E}
  {\hastype \Gamma {\recordCon l e} {\recordType l A} \yields E}
}

% rec-select
\newcommand{\rulelabelRecSelect}{\rulelabel\text{rec-select}}
\newcommand{\ruleRecSelect}{
\inferrule* [right=$\rulelabelRecSelect$]
  {\hastype \Gamma e A \yields E \\
   \judgeSelect A l {A_1} \yields C}
  {\hastype \Gamma {e.l} {A_1} \yields {\app C E}}
}

% rec-restrict
\newcommand{\rulelabelRecRestrict}{\rulelabel\text{rec-restrict}}
\newcommand{\ruleRecRestrict}{
\inferrule* [right=$\rulelabelRecRestrict$]
  {\hastype \Gamma e A \yields E \\
   \judgeRestrict A l {A_1} \yields C}
  {\hastype \Gamma {e \restrictOp l} {A_1} \yields {\app C E}}
}

\newcommand{\judgeSelect}[3]{#1 \bullet #2 = #3}

% select
\newcommand{\ruleGet}{
  \inferrule* [right=$\rulelabelSelect$]
  { }
  {\judgeSelect {\recordType l A} l A \yields {\lam x {\im {\recordType l A}} x}}
}

% select1
\newcommand{\rulelabelSelectLeft}{{\rulelabelSelect}_1}
\newcommand{\ruleGetLeft}{
  \inferrule* [right=$\rulelabelSelectLeft$]
  {\judgeSelect {A_1} l A \yields C}
  {\judgeSelect {A_1 \intersect A_2} l A \yields {\lam x {\im {A_1
          \intersect A_2}} {\app C {(\proj 1 x)}}}}
}

% select2
\newcommand{\rulelabelSelectRight}{{\rulelabelSelect}_2}
\newcommand{\ruleGetRight}{
  \inferrule* [right=$\rulelabelSelectRight$]
  {\judgeSelect {A_2} l A \yields C}
  {\judgeSelect {A_1 \intersect A_2} l A \yields {\lam x {\im {A_1
          \intersect A_2}} {\app C {(\proj 2 x)}}}}
}

\newcommand{\judgeRestrict}[3]{#1 \bm{\restrictOp} #2 = #3}

% restrict
\newcommand{\ruleRestrict}{
  \inferrule* [right=$\rulelabelRestrict$]
  { }
  {\judgeRestrict {\recordType l A} l \top \yields {\lam x {\im {\recordType l A}} {()}}}
}

% restrict1
\newcommand{\rulelabelRestrictleft}{{\rulelabelRestrict}_1}
\newcommand{\ruleRestrictLeft}{
  \inferrule* [right=$\rulelabelRestrictleft$]
  {\judgeRestrict {A_1} l A \yields C}
  {\judgeRestrict {A_1 \intersect A_2} l {A \intersect A_2} \yields {\lam x {\im {A_1
          \intersect A_2}} {\pair {\app C {(\proj 1 x)}} {\proj 2 x}}}}
}

% restrict2
\newcommand{\rulelabelRestrictRight}{{\rulelabelRestrict}_2}
\newcommand{\ruleRestrictRight}{
  \inferrule* [right=$\rulelabelRestrictRight$]
  {\judgeRestrict {A_2} l A \yields C}
  {\judgeRestrict {A_1 \intersect A_2} l {A_1 \intersect A} \yields {\lam x {\im {A_1
          \intersect A_2}} {\pair {\proj 1 x} {\app C {(\proj 2 x)}}}}}
}

% \renewcommand{\yields}[1]{}

\newcommand{\formbi}{
\framebox{
$ \jinfer \Gamma e A \yields E\\e$ synthesizes type $A$}
}

\newcommand{\formbc}{
\framebox{
$ \jcheck \Gamma e A \yields E\\e$ checks against given type $A$
}
}

%$ \jtype \Gamma e A \Leftarrow E~~~~~
%e$ checks againts known type $A$}}

\newcommand{\bruletvar} {
\inferrule* [right=\labeltvar]
  {x \oftype A \in \Gamma}
  {\jinfer \Gamma x A \yields x}
}

\newcommand{\bruletint} {
\inferrule* [right=\labeltint]
  { }
  {\jinfer \Gamma i \tyint \yields i}
}

\newcommand{\bruletlam} {
\inferrule* [right=\labeltlam]
  {\jwf \Gamma A \\ \jcheck {\Gamma, x \oftype A} e B \yields E}
  {\jcheck \Gamma {\lam x e} {A \to B} \yields {\lam x E}}
}

\newcommand{\bruletapp}{
\inferrule* [right=\labeltapp]
  {\jinfer \Gamma {e_1} {A_{1} \to A_{2}} \yields {E_1} \\
   \jcheck \Gamma {e_2} {A_{1}} \yields {E_2}}
%%   {A_3} \subtype {A_1} \yields {E}}
  {\jinfer \Gamma {\app {e_1} {e_2}} {A_{2}} \yields {\app {E_1} E_2}}
}

\newcommand{\bruletblam}{
\inferrule* [right=\labeltblam]
  {\jinfer {\Gamma, \alpha} e A \yields E \\
   \alpha \not \in \ftv \Gamma}
  {\jinfer \Gamma {\blam {\alpha} e} {\for {\alpha} A} \yields {\blam \alpha E}}
}

\newcommand{\bruletblamdis}{
\inferrule* [right=\labeltblam]
  {\jwf \Gamma A \\
   \jinfer {\Gamma,\alpha \disjoint A} e B \yields E \\
   \alpha \not \in \ftv \Gamma}
  {\jinfer \Gamma {\blamdis \alpha A e} {\fordis \alpha A B}
    \yields {\blam \alpha E}}
}

\newcommand{\brulettapp}{
\inferrule* [right=\labelttapp]
  {\jinfer \Gamma e {\for \alpha B} \yields E \\
   \jwf \Gamma A}
  {\jinfer \Gamma {\tapp e A} {\subst A \alpha B} \yields {\tapp E {\im A}}}
}

\newcommand{\brulettappdis}{
\inferrule* [right=\labelttapp]
  {\jinfer \Gamma e {\fordis \alpha B C} \yields E \\
   \jwf \Gamma A \\
   \framebox{$\jdis \Gamma A B$}}
  {\jinfer \Gamma {\tapp e A} {\subst A \alpha C} \yields {\tapp E {\im A}}}
}

\newcommand{\brulettprod}{
\inferrule* [right=\labeltprod]
  {\jinfer \Gamma {e_1} A \yields {E_1} \\
   \jinfer \Gamma {e_2} B \yields {E_2}}
  {\jinfer \Gamma {\pair {e_1} {e_2}} {\tpair A B} \yields {\pair {E_1} {E_2}}}
}

\newcommand{\brulettproj}{
\inferrule* [right=\labeltproj]
  {\jinfer \Gamma {e} {\tpair {A_1} {A_2}} \yields {E}}
  {\jinfer \Gamma {\proj k e} {A_k} \yields {\proj k E}}
}

\newcommand{\brulettprojl}{
\inferrule* [right=\labeltprojl]
  {\jinfer \Gamma {e} {\tpair A B} \yields {E}}
  {\jinfer \Gamma {\proj 1 e} A \yields {\proj 1 E}}
}

\newcommand{\brulettprojr}{
\inferrule* [right=\labeltprojr]
  {\jinfer \Gamma {e} {\tpair A B} \yields {E}}
  {\jinfer \Gamma {\proj 2 e} B \yields {\proj 2 E}}
}

\newcommand{\bruletmerge}{
\inferrule* [right=\labeltmerge]
  {\jinfer \Gamma {e_1} A \yields {E_1} \\
   \jinfer \Gamma {e_2} B \yields {E_2}}
  {\jinfer \Gamma {e_1 \mergeop e_2} {A \inter B} \yields {\pair {E_1} {E_2}}}
}

\newcommand{\bruletsub}{
\inferrule* [right=\labeltsub]
  {\jinfer \Gamma {e} A \yields {E} \\ {A} \subtype {B} \yields {E_{sub}}}
  {\jcheck \Gamma {e} {B} \yields {\tapp {E_{sub}} E}}
}

\newcommand{\bruletmergedis}{
\inferrule* [right=\labeltmerge]
  {\jinfer \Gamma {e_1} A \yields {E_1} \\
   \jinfer \Gamma {e_2} B \yields {E_2} \\
   \jdis \Gamma A B}
  {\jinfer \Gamma {e_1 \mergeop e_2} {A \inter B} \yields {\pair {E_1} {E_2}}}
}

\newcommand{\brulettop}{
\inferrule* [right=\labelttop]
  { }
  {\jinfer \Gamma \top \top \yields \unit}
}

\newcommand{\labeltann}{\makelabelt Ann}
\newcommand{\bruletann}{
\inferrule* [right=\labeltann]
  {\jcheck \Gamma {e} A \yields {E}}
  {\jinfer \Gamma {e : A} {A} \yields {E}}
}

\newcommand{\labeltrec}{\makelabelt Rec}
\newcommand{\bruletrec}{
\inferrule* [right=\labeltrec]
  {\jinfer \Gamma e {A} \yields {E}}
  {\jinfer \Gamma {\recordCon l e} {\recordType l A} \yields {E}}
}

\newcommand{\labeltprojr}{\makelabelt RecAcc}
\newcommand{\bruletprojr}{
\inferrule* [right=\labeltprojr]
  {\jinfer \Gamma e {\recordType l A} \yields {E}}
  {\jinfer \Gamma {\recordProj e l} {A} \yields {E}}
}

\newcommand{\judgeTargetWF}[2]{#1 \; \textcolor{targetColor}{\turns} \; #2}
\newcommand{\judgeTarget}[3]{#1 \; \textcolor{targetColor}{\turns} \; #2 \; \textcolor{targetColor}{:} \; #3}
\newcommand{\ruleLabelTarget}{\bm{\textcolor{targetColor}{T}}}

\newcommand{\ruleLabelTargetvar}{\ruleLabelTarget\text{var}}
\newcommand{\ruleTargetVar} {
\inferrule* [right=$\ruleLabelTargetvar$]
  {(x,T) \in \Gamma}
  {\judgeTarget \Gamma x T}
}

\newcommand{\ruleLabelTargetUnit}{\ruleLabelTarget\text{unit}}
\newcommand{\ruleTargetUnit} {
\inferrule* [right=$\ruleLabelTargetUnit$]
  { }
  {\judgeTarget \Gamma {()} {()}}
}

\newcommand{\ruleLabelTargetlam}{\ruleLabelTarget\text{lam}}
\newcommand{\ruleTargetLam} {
\inferrule* [right=$\ruleLabelTargetlam$]
  {\judgeTarget {\Gamma, x \hast T} E {T_1} \andalso \judgeTargetWF \Gamma T}
  {\judgeTarget \Gamma {\lam x T E} {T \to T_1}}
}

\newcommand{\ruleLabelTargetApp}{\ruleLabelTarget\text{app}}
\newcommand{\ruleTargetApp}{
\inferrule* [right=$\ruleLabelTargetApp$]
  {\judgeTarget \Gamma {E_1} {T_1 \to T_2} \andalso \judgeTarget \Gamma {E_2} {T_1}}
  {\judgeTarget \Gamma {\app {E_1} {E_2}} {T_2}}
}

\newcommand{\ruleLabelTargetBLam}{\ruleLabelTarget\text{blam}}
\newcommand{\ruleTargetBLam}{
\inferrule* [right=$\ruleLabelTargetBLam$]
  {\judgeSource {\Gamma, \alpha} E T}
  {\judgeSource \Gamma {\blam \alpha E} {\for \alpha T}}
}

\newcommand{\ruleLabelTargetTApp}{\ruleLabelTarget\text{tapp}}
\newcommand{\ruleTargetTApp}{
\inferrule* [right=$\ruleLabelTargetTApp$]
  {\judgeTarget \Gamma E {\for \alpha {T_1}} \andalso \judgeTargetWF \Gamma T}
  {\judgeTarget \Gamma {\tapp E T} {\subst T \alpha T_1}}
}

\newcommand{\ruleLabelTargetPair}{\ruleLabelTarget\text{pair}}
\newcommand{\ruleTargetPair}{
\inferrule* [right=$\ruleLabelTargetPair$]
  {\judgeTarget \Gamma {E_1} {T_1} \andalso \judgeTarget \Gamma {E_2} {T_2}}
  {\judgeTarget \Gamma {\pair {E_1} {E_2}} {\pair {T_1} {T_2}}}
}

\newcommand{\ruleLabelTargetProjLeft}{\ruleLabelTarget\text{proj}_1}
\newcommand{\ruleTargetProjLeft}{
\inferrule* [right=$\ruleLabelTargetProjLeft$]
  {\judgeTarget \Gamma E {\pair {T_1} {T_2}}}
  {\judgeTarget \Gamma {\proj 1 E} {T_1}}
}

\newcommand{\ruleLabelTargetProjRight}{\ruleLabelTarget\text{proj}_2}
\newcommand{\ruleTargetProjRight}{
\inferrule* [right=$\ruleLabelTargetProjRight$]
  {\judgeTarget \Gamma E {\pair {T_1} {T_2}}}
  {\judgeTarget \Gamma {\proj 2 E} {T_2}}
}
\newcommand{\formtoplike}{\framebox{$ \toplike{A} $}}

\newcommand{\makelabeltopl}[1]{$TL#1$}

\newcommand{\labeltopltop}{\makelabeltopl \top}
\newcommand{\ruletopltop}{
  \inferrule* [right=\labeltopltop]
    { }
    { \toplike{\top} }
}

\newcommand{\labeltoplfun}{\makelabeltopl \rightarrow}
\newcommand{\ruletoplfun}{
  \inferrule* [right=\labeltoplfun]
    { \toplike{B} }
    { \toplike{A \to B} }
}

\newcommand{\labeltoplinterl}{\makelabeltopl Inter-1}
\newcommand{\ruletoplinterl}{
  \inferrule* [right=\labeltoplinterl]
    { \toplike{A} }
    { \toplike{A \inter B} }
}

\newcommand{\labeltoplinterr}{\makelabeltopl Inter-2}
\newcommand{\ruletoplinterr}{
  \inferrule* [right=\labeltoplinterr]
    { \toplike{B} }
    { \toplike{A \inter B} }
}

\newcommand{\labeltoplinter}{\makelabeltopl \&}
\newcommand{\ruletoplinter}{
  \inferrule* [right=\labeltoplinter]
    { \toplike{A} \\ \toplike{B}}
    { \toplike{A \inter B} }
}

\newcommand{\labeltoplforall}{\makelabeltopl \forall}
\newcommand{\ruletoplforall}{
  \inferrule* [right=\labeltoplforall]
    { \toplike{A} }
    { \toplike{\fordis \alpha B A} }
}


\begin{document}
\toappear{}
%\special{papersize=8.5in,11in}
%\setlength{\pdfpageheight}{\paperheight}
%\setlength{\pdfpagewidth}{\paperwidth}

\conferenceinfo{CONF 'yy}{Month d--d, 20yy, City, ST, Country}
\copyrightyear{20yy}
\copyrightdata{978-1-nnnn-nnnn-n/yy/mm}
\doi{nnnnnnn.nnnnnnn}

\titlebanner{banner above paper title}        % These are ignored unless
\preprintfooter{\name}                        % 'preprint' option specified.

\title{Disjoint Intersection Types} 
% \subtitle{Full Version with Appendix}

\authorinfo{Bruno C. d. S. Oliveira\and Zhiyuan Shi\and João Alpuim}
           {The University of Hong Kong}
           {\{bruno,zyshi,alpuim\}@cs.hku.hk}
%\authorinfo{Name2\and Name3}
%           {Affiliation2/3}
%           {Email2/3}

\maketitle

\begin{abstract}
  Dunfield has shown that a simply typed core calculus with
intersection types and a merge operator forms a powerful foundation
for various programming language features. While his calculus
is type-safe, it lacks \emph{coherence}:
different derivations for the same expression can lead to different
results. The lack of coherence is important disadvantage for adoption
of his core calculus in implementations of programming languages, as
the semantics of the programming language becomes implementation
dependent. Moreover his calculus did not account for parametric polymorphism.

This paper presents \namedis: a core calculus with a variant of
\emph{intersection types} and a
\emph{merge operator}. The semantics \namedis is both type-safe and
coherent. Coherence is achieved by ensuring that intersection types
are \emph{disjoint}. Formally two types are disjoint if they do not
share a common supertype. We present a type system that prevents
intersection types that are not disjoint, as well as an algorithmic
specification to determine whether two types are disjoint. ...

\end{abstract}

\category{D.3.2}{Language Classifications}{Applicative (functional) languages}
\category{F.3.3}{Studies of Program Constructs}{Functional constructs}

% general terms are not compulsory anymore,
% you may leave them out
\terms Design, Languages, Theory

\keywords Intersection Types, Type System

\section{Introduction}

There has been a remarkable number of works aimed at improving support
for extensibility in programming languages. These works include:
visions of new programming models~\cite{}; new programming languages or
language extensions~\cite{}, and \emph{design patterns} that can be
used with existing mainstream languages~\cite{}.

%\cite{family polymorphism and virtual
%classes.}. Another line of work are proposals for precise formal models or new 
%programming languages. Yet another line are \emph{design patterns}
%that can be used  with existing mainstream languages. 
%%Part of the motivation behind 

Some of the more recent work on extensibility is focused on various
proposals for design patterns.  Examples include \emph{Object
  Algebras}~\cite{}, \emph{Modular Visitors}~\cite{} or
Torgersen's~\cite{} four design patterns using generics. In those
approaches the idea is to use some advanced (but already available)
features, such as \emph{generics}, in combination with conventional
OOP features to model more extensible designs.  Those designs work in
modern OOP languages such as Java, C\# or Scala.

Although such design patterns give practical benefits in terms of
extensibility, they also expose limitations in existing mainstream OOP
languages. In particular there are three pressing limitations: 
1) lack of good mechanisms for
  \emph{object-level} composition; 2) \emph{conflation of 
    (type) inheritance with subtyping}; 3) \emph{heavy reliance on generics}.

  The first limitation shows up, for example, in Oliveira et
  al.~\cite{} encodings of Feature-Oriented Programming using Object
  Algebras~\cite{}. These programs are best expressed using a form of
  \emph{type-safe}, \emph{dynamic}, \emph{delegation}-based
  composition. Although such form of composition can be encoded in
  languages like Scala, it requires the use of low-level reflection
  techniques, such as dynamic proxies, reflection or other forms of
  meta-programming~\cite{}. It is clear that better language support
  would be desirable.

  The second limitation shows up in designs for modelling
  modular or extensible visitors~\cite{}.  The vast majority of modern
  OOP languages combines type inheritance and subtyping. 
  That is a type extension induces a subtype. However
  as Cook et al.~\cite{} famously argued there are programs where
  ``\emph{subtyping is not inheritance}''. Interestingly previously
  not many practical programs have been reported in the literature
  where the distinction between subtyping and inheritance is
  relevant. However, as shown in this paper, it turns out that this
  difference does show up in practice when designing modular
  (extensible) visitors.  We believe that modular visitors provide a
  compeling practical example where inheritance and subtyping should
  not be conflated!

  Finally, the third limitation is prevalent in many extensible
  designs~\cite{}. Such designs rely on advanced features of generics,
  such as \emph{f-bounded polymorphism}~\cite{}, \emph{variance
    annotations}~\cite{}, \emph{wildcards}~\cite{} and/or \emph{higher-kinded
    types}~\cite{} to achieve type-safety. Sadly, the amount of
  type-annotations, combined with the lack of understanding of these
  features, usually deters programmers from using such designs.

\begin{comment}
Motivated by the insights gained in previous work, this paper presents 
a minimal core calculus that addresses current limitations and
provides a better foundational model for statically typed
delegation-based OOP? We show that Object Algebras fit nicely in this
model. 
\end{comment}

This paper presents System \name: an extension of System F~\cite{}
with intersection types and a merge operator~\cite{}.  The goal of
System \name is to study the \emph{minimal} foundational language
constructs that are needed to support various extensible designs,
while at the same time addressing the limitations of existing OOP
languages. To address the lack of good object-level composition
mechanisms, System \name uses the merge operator to allow dynamic
composition of values/objects. Moreover, in System \name (type-level)
extension is independent of subtyping, and it is possible for an
extension to be a supertype of a base object type. Furthermore,
intersection types and conventional subtyping can be used in many
cases instead of advanced features of generics. Indeed this paper 
shows how many previous designs in the literature can be encoded 
without such advanced features of generics.


Technically speaking System \name is mainly inspired by the work of
Dundfield~\cite{}.  Dundfield shows how to model a simply typed
calculus with intersection types and a merge operator. The presence of
a merge operator adds significant expressiveness to the language,
allowing encodings for many other language constructs as syntactic
sugar. System \name differs from Dundfield's work in a few
ways. Firstly it adds parametric polymorphism and formalizes a
extension for records to support a basic form of objects. Secondly,
the elaboration semantics into System F is done directly from the
source calculus with subtyping. In contrast Dunfield has an additional
step which eliminates subtyping.  Finally a non-technical difference
is that System \name is aimed at studying issues of OOP languages and
extensibility, whereas Dunfield's work was aimed at Functional
Programming and he did not consider applications to extensibility.
Like many other foundational formal models for OOP (for
example~\cite{}), System \name is purely functional and it uses
structural typing.

%%System \name is
%%formalized and implemented. Furthermore the paper illustrates how
%%various extensible designs can be encoded in System \name.

\begin{comment}
We present a polymorphic calculus containing intersection types and records, and show
how this language can be used to solve various common tasks in functional
programming in a nicer way.Intersection types provides a power mechanism for functional programming, in
particular for extensibility and allowing new forms of composition.

Prototype-based programming is one of the two major styles of object-oriented
programming, the other being class-based programming which is featured in
languages such as Java and C\#. It has gained increasing popularity recently
with the prominence of JavaScript in web applications. Prototype-based
programming supports highly dynamic behaviors at run time that are not possible
with traditional class-based programming. However, despite its flexibility,
prototype-based programming is often criticized over concerns of correctness and
safety. Furthermore, almost all prototype-based systems rely on the fact that
the language is dynamically typed and interpreted.
\end{comment}

In summary, the contributions of this paper are:

\begin{itemize}

\item {\bf A Minimal Core Language for Extensibility:} This paper
  identifies a minimal core language, System \name, capable of
  expressing various extensibility designs in the literature.
  System \name also addresses limitations of existing OOP
  languages that complicate extensible designs. 
  
\item {\bf Formalization of System \name:} An elaboration semantics of
  System \name into System F is given, and type-soundness is proved.

\item {\bf Encodings of Extensible Designs:} Various encodings of
  extensible designs into System \name, including \emph{Object
    Algebras} and \emph{Modular Visitors}. 

\item {\bf A Practical Example where ``Inheritance is not Subtyping''
    Matters:} This paper shows that in modular/extensible visitors
  suffer from the ``inheritance is not subtyping problem''. Moreover 
  with extensible visitors the extension should become a
  \emph{supertype}, not a subtype. \bruno{extension with accept method}

\item {\bf Implementation and Examples:} An implementation of an
  extension of System \name, as well as the examples presented in the
  paper, are publicly available. 

\begin{comment}

\item{elaboration typing rules which given a source expression with intersection
    types, typecheck and translate it into an ordinary F term. Prove a type
    preservation result: if a term $ e $ has type $ \tau $ in the source language,
    then the translated term $ \image e $ is well-typed and has type $ \image \tau $ in the
    target language.}

\item{present an algorithm for detecting incoherence which can be very important
    in practice.}

\item{explores the connection between intersection types and object algebra by
    showing various examples of encoding object algebra with intersection
    types.}

\end{comment}

\end{itemize}

\begin{comment}
\subsection{Other Notes}

finitary overloading: yes
but have other merits of intersection been explored?

-- Compare Scala:
-- merge[A,B] = new A with B

-- type IEval  = { eval :  Int }
-- type IPrint = { print : String }

-- F[\_]
\end{comment}
%*******************************************************************************
\section{Overview} \label{sec:overview}
%*******************************************************************************

\bruno{Be careful when using the word ``class'': we don't have classes in our system;
so talking about classes may simply confuse readers. You can talk about classes simply to say
that traits provide an alternative to classes. Often when you write ``class'' in the text, what
you mean is ``object``}

This section introduces \name and its support for intersection types and the
merge operator. It then discusses the issue of coherence and shows how the
notion of disjoint intersection types achieve a coherent semantics.

Note that this section uses some syntactic sugar, as well as standard
programming language features, to illustrate the various concepts in
\name. Although the minimal core language that we formalize in
Section~\ref{sec:fi} does not present all such features, our implementation
supports them.

\subsection{Intersection Types and the Merge Operator}
%%\subsection{Intersection Types, Merge and Polymorphism in \name}

Intersection types date back as early as Coppo et
al.'s work~\cite{coppo1981functional}. Since then various researchers have
studied intersection types, and some languages have adopted them in one
form or another.
%However, as we shall see in
%Section~\ref{subsec:incoherence}, it also introduces difficulties. In what follows
%intersection types and the merge operator are informally introduced.

\paragraph{Intersection types.}
The intersection of type $A$ and $B$ (denoted as \lstinline{A & B} in
\name) contains exactly those values
which can be used as either values of type $A$ or of type $B$. For instance,
consider the following program in \name:

\begin{lstlisting}
let x : Int & Char = (*$ \ldots $*) in -- definition omitted
let idInt (y : Int) : Int = y in
let idChar (y : Char) : Char = y in
(idInt x, idChar x)
\end{lstlisting}

\noindent If a value \lstinline{x} has type \lstinline{Int & Char} then
\lstinline{x} can be used as an integer or as a character. Therefore,
\lstinline{x} can be used as an argument to any function that takes
an integer as an argument, or any
function that take a character as an argument. In the program above
the functions \lstinline{idInt} and \lstinline{idChar} are the
identity functions on integers and characters, respectively.
Passing \lstinline{x} as an argument to either one (or both) of the
functions is valid.

\paragraph{Merge operator.}
In the previous program we deliberately did not show how to introduce values of
an intersection type. There are many variants of intersection types in the
literature. Our work follows a particular formulation, where intersection types
are introduced by a \emph{merge operator}. As
Dunfield~\cite{dunfield2014elaborating} has argued a merge operator adds
considerable expressiveness to a calculus. The merge operator allows two values
to be merged in a single intersection type. For example, an implementation of
\lstinline{x} is constructed in \name as follows:

\begin{lstlisting}
let x : Int & Char = 1,,'c' in (*$ \ldots $*)
\end{lstlisting}

\noindent In \name (following Dunfield's notation), the
merge of two values $v_1$ and $v_2$ is denoted as $v_1 ,, v_2$.

\paragraph{Merge operator and pairs.}
The merge operator is similar to the introduction construct on pairs.
An analogous implementation of \lstinline{x} with pairs would be:

\begin{lstlisting}
let xPair : (Int, Char) = (1, 'c') in (*$ \ldots $*)
\end{lstlisting}

\noindent The significant difference between intersection types with a
merge operator and pairs is in the elimination construct. With pairs
there are explicit eliminators (\lstinline{fst} and
\lstinline{snd}). These eliminators must be used to extract the
components of the right type. For example, in order to use
\lstinline{idInt} and \lstinline{idChar} with pairs, we would need to
write a program such as:

\begin{lstlisting}
(idInt (fst xPair), idChar (snd xPair))
\end{lstlisting}

\noindent In contrast the elimination of intersection types is done
implicitly, by following a type-directed process. For example,
when a value of type \lstinline{Int} is needed, but an intersection
of type \lstinline{Int & Char} is found, the compiler uses the
type system to extract the corresponding value.

\subsection{Incoherence}\label{subsec:incoherence}
Unfortunately the implicit nature of elimination for intersection
types built with a merge operator can lead to incoherence.
The merge operator combines two terms, of type $A$ and $B$
respectively, to form a term of type $A \inter B$. For example,
$1 \mergeop `c'$ is of type $\code{Int} \inter \code{Char}$. In this case, no
matter if $1 \mergeop `c'$ is used as $\code{Int}$ or $\code{Char}$, the result
of evaluation is always clear. However, with overlapping types, it is
not straightforward anymore to see the result. For example, what
should be the result of this program, which asks for an integer out of
a merge of two integers:
\begin{lstlisting}
(fun (x: Int) (*$ \to $*) x) (1,,2)
\end{lstlisting}
Should the result be \lstinline$1$ or \lstinline$2$?

If both results are accepted, we say that the semantics is \emph{incoherent}:
there are multiple possible meanings for the same valid program. Dunfield's
calculus~\cite{dunfield2014elaborating} is incoherent and accepts the program
above.

\paragraph{Getting around incoherence: biased choice.}
In a real implementation of Dunfield calculus a choice has to be made
on which value to compute. For example, one potential option is to
always take the left-most value matching the type in the
merge. Similarly, one could always take the right-most
value matching the type in the merge. Either way, the meaning
of a program will depend on a biased implementation choice,
which is clearly unsatisfying from the theoretical point of view
(although perhaps acceptable in practice).

\subsection{Restoring Coherence: Disjoint Intersection Types}\label{sec:restoring}
Coherence is a desirable property for a semantics. A semantics is said
to be coherent if any \emph{valid program} has exactly one
meaning~\cite{reynolds1991coherence} (that is, the semantics is not ambiguous).
One option to restore coherence is to reject programs which may have
multiple meanings.
%Of course, when rejecting programs it is important
%not to be too conservative, and reject too many programs which are
%actually coherent.
Analyzing the expression $1 \mergeop 2$, we can see that the reason
for incoherence is that there are multiple, overlapping, integers in the
merge. Generally speaking, if both terms can be assigned some type $C$,
both of them can be chosen as the meaning of the merge,
which leads to multiple meanings of a term.
Thus a natural option is to try to forbid such overlapping
values of the same type in a merge.

This is precisely the approach taken in \name. \name requires that the
two types of in intersection must be \emph{disjoint}.  However,
although disjointness seems a natural restriction to impose on
intersection types, it is not obvious to formalize it. Indeed Dunfield
has mentioned disjointness as an option to restore coherence, but he
left it for future work due to the non-triviality of the approach.

\paragraph{Searching for a definition of disjointness.}
The first step towards disjoint intersection types is to come up
with a definition of disjointness. A first attempt at such definition would
be to require that, given two types $A$ and $B$, both types are not
subtypes of each other. Thus, denoting disjointness as $A * B$, we would have:
\[A * B \equiv A \not<: B \wedge B \not<: A\]
At first sight this seems a reasonable definition and it does prevent
merges such as \lstinline{1,,2}. However some moments of thought are enough to realize that
such definition does not ensure disjointness. For example, consider
the following merge:

\begin{lstlisting}
(1,,'c') ,, (2,,True)
\end{lstlisting}

\noindent This merge has two components which are also intersection
types. The first component \lstinline{(1,,'c')} has type $\code{Int} \inter
\code{Char}$, whereas the second component \lstinline{(2 ,, True)} has type
$\code{Int} \inter \code{Bool}$. Clearly,
\[ \code{Int} \inter \code{Char} \not \subtype \code{Int} \inter \code{Bool} \wedge \code{Int} \inter \code{Bool} \not \subtype \code{Int} \inter \code{Char} \]
Nevertheless the following program still leads to
incoherence:
\begin{lstlisting}
(fun (x: Int) (*$ \to $*) x) ((1,,'c'),,(2,,True))
\end{lstlisting}
as both \lstinline{1} or \lstinline{2} are possible outcomes
of the program. Although this attempt to define disjointness failed,
it did bring us some additional insight: although the types of the two
components of the merge are not subtypes of each other, they share
some types in common.

\paragraph{A proper definition of disjointness.} In order for two types
to be truly disjoint, they must not have any subcomponents sharing
the same type. In a system with intersection types this can be ensured
by requiring the two types do not share a common supertype. The
following definition captures this idea more formally.

\begin{definition}[Disjointness]
  Given two types $A$ and $B$, two types are disjoint
  (written $A \disjoint B$) if there is no type $C$ such that both $A$ and $B$ are
  subtypes of $C$:
  \[A \disjoint B \equiv \not\exists C.~A \subtype C \wedge B \subtype C\]
\end{definition}

\noindent This definition of disjointness prevents the problematic
merge. Since $\code{Int}$ is a common supertype of both $\code{Int} \& \code{Char}$ and
$\code{Int} \& \code{Bool}$, those two types are not disjoint.

\name's type system only accepts programs that use disjoint intersection
types. As shown in Section~\ref{sec:disjoint} disjoint intersection types will
play a crucial rule in guaranteeing that the semantics is coherent.

% \subsection{Parametric Polymorphism and Intersection Types}\label{subsec:polymorphism}
% Before we show how \name extends the idea of disjointness to parametric
% polymorphism, we discuss some non-trivial issues that arise from
% the interaction between parametric polymorphism and intersection types.
%The interaction between parametric polymorphism and
%intersection types when coherence is a goal is non-trivial.
%In particular biased choice .
%The key challenge is to have a type
%system that still ensures coherence, but at the same time is not too
%restrictive in the programs that can be accepted.
% Dunfield~\cite{} provides a
% good illustrative example of the issues that arise when combining
% disjoint intersection types and parametric polymorphism:
% \[\lambda x. {\bf let}~y = 0 \mergeop x~{\bf in}~x\]
% Consider the attempt to write
% the following polymorphic function in \name (we use
% uppercase Latin letters to denote type variables):
% \begin{lstlisting}
% let fst A B (x: A & B) = (fun (z:A) (*$ \to $*) z) x in (*$ \ldots $*)
% \end{lstlisting}
% The
% \code{fst} function is supposed to extract a value of type
% (\lstinline{A}) from the merge value $x$ (of type \lstinline{A&B}). However
% this function is problematic.  The reason is that when
% \lstinline{A} and \lstinline{B} are instantiated to non-disjoint
% types, then uses of \lstinline{fst} may lead to incoherence.
% For example, consider the following use of \lstinline{fst}:
% \begin{lstlisting}
% fst Int Int (1,,2)
% \end{lstlisting}
% \noindent This program is clearly incoherent as both
% $1$ and $2$ can be extracted from the merge and still match the type
% of the first argument of \lstinline{fst}.

% \paragraph{Biased choice breaks equational reasoning.} At first sight, one option
% to workaround the issue incoherence would be to bias the type-based merge lookup
% to the left or to the right (as discussed in
% Section~\ref{subsec:incoherence}). Unfortunately, biased choice is
% very problematic when parametric polymorphism is present in the language.
% To see the issue, suppose we chose to always pick the
% rightmost value in a merge when multiple values of same type exist.
% Intuitively, it would appear that the result of the use of
% \lstinline{fst} above is $2$. Indeed simple equational reasoning
% seems to validate such result:
% \begin{lstlisting}
%    fst Int Int (1,,2)
% (*$ \rightsquigarrow $*) (fun (z: Int) (*$ \to $*) z) (1,,2) -- (* \textnormal{By the definition of \code{fst}} *)
% (*$ \rightsquigarrow $*) (fun (z: Int) (*$ \to $*) z) 2      -- (* \textnormal{Right-biased coercion} *)
% (*$ \rightsquigarrow $*) 2                          -- (* \textnormal{By $\beta$-reduction} *)
% \end{lstlisting}
%
% However (assumming a straightforward implementation of right-biased
% choice) the result of the program would be 1! The reason for this has
% todo with \emph{when} the type-based lookup on the merge happens. In
% the case of \lstinline{fst}, lookup is triggered by a coercion
% function inserted in the definition of \lstinline{fst} at
% compile-time.
% In the definition of \lstinline$fst$ all it is known is that a
% value of type $A$ should be returned from a merge with an intersection
% type $A\&B$.  Clearly the only type-safe choice to coerce the value of
% type $A\&B$ into $A$ is to
% take the left component of the merge. This works perfectly for merges
% such as \lstinline$(1,,'c')$, where the types of the first and second components
% of the merge are disjoint. For the merge \lstinline$(1,,'c')$, if a integer lookup
% is needed, then \lstinline$1$ is the rightmost integer, which is consistent with the
% biased choice. Unfortunately, when given the merge \lstinline$(1,,2)$ the
% left-component (\lstinline$1$) is also picked up, even though in this case \lstinline$2$
% is the rightmost integer in the merge. Clearly this is inconsistent
% with the biased choice!
%
% Unfortunately this subtle interaction of polymorphism and type-based lookup
%  means that equational reasoning is broken!
% In the equational reasoning steps above, doing apparently correct
% substitutions lead us to a wrong result. This is a major problem for
% biased choice and a reason to dismiss it as a possible implementation
% choice for \name.

\begin{comment}
\paragraph{Conservatively rejecting intersections.}
To avoid incoherence, and the issues of biased choice, another option
is simply to reject programs where the
instantiations of type variables may lead to incoherent programs.
In this case the definition of \lstinline$fst$ would be rejected, since there
are indeed some cases that may lead to incoherent programs.
Unfortunately this is too restrictive and prevents many useful
programs.

We have built a source language that is desugared into \name. The most
central feature is the trait declaration. Trait can take several parameters,
which is then in scope in the body of the trait definition. In fact, trait
creation is dynamic, which means it can be contained inside a function.
\end{comment}

%%*******************************************************************************
\section{Dynamically Composable Traits} \label{sec:trait}
%*******************************************************************************

\george{after the new keyword, types should be specified.}

As an application of disjoint intersection types, we show how to model a
simple, yet expressive form of dynamically composable
traits~\cite{scharli2003traits} in \name. Traits provide a
mechanism of code reuse in object-oriented programming, that
can be used as an alternative to multiple inheritance.
The interesting aspect about traits is the way conflicts that
typically arise in multiple-inheritance~\cite{} are dealt with.
Instead of trying to automatically resolve conflicts, traits
detect those conflicts and require programmers to explicitly resolve
them. This is where the relation to disjoint intersection types comes
in: the mechanism to detect incoherence of disjoint intersection types
provides us with the mechanism to detect conflicts in traits.
We demonstrate various trait
features in a simple OO language, and then sketch a straightforward
translation from that language to \name as a basic form of
syntactic sugar. \george{Need to note the difference
between class-based and prototype-based somewhere.}

\subsection{Basic Traits}

A trait is a collection of related methods that characterizes only a specific
perspective of the features of an object. Therefore, compared with
programs using inheritance, programs using traits usually have a
larger number of small
traits rather than fewer but larger classes. Code reuse with traits is easier
with traits than with classes, since traits are usually shorter and traits can
be \emph{composed}. In fact, trait composition offers a variety of
possibilities: two traits can be ``added'' together (which is an symmetric
operation); methods can be removed from a trait; and trait systems provide
conflict detection, etc.

The first example shows basic trait composition. Many social networking sites
allow users to ``upvote'' a comment and the number of upvotes that comment has
received is also displayed. We would like to separate the logic for upvotes from
comments so that it can be reused in other entities such as posts and sharings.
The code below defines a trait, \lstinline$Comment$, which contains a single
method \lstinline$content$.\footnote{Our source language assumes regular record
type, record operations as well the unit type and unit literals.}

\begin{lstlisting}
type Comment = { content: () (*$ \to $*) String } in
trait Comment(content: String) { self: Comment (*$ \to $*)
  content() = content
} in
\end{lstlisting}

\noindent Next, we create another trait, \lstinline$Up$, for tracking the number
of upvotes.

\begin{lstlisting}
type Up = { upvotes: () (*$ \to $*) Int } in
trait Up(upvotes: Int) { self: Up (*$ \to $*)
  upvotes() = upvotes
} in
\end{lstlisting}

At this point the reader may wonder why there are duplicate declarations related
to \lstinline$Comment$ and \lstinline$Up$. In mainstream OO languages such as
Java, a class declaration such as \lstinline$class C { ... }$ does two things at
the same time:

\begin{itemize}
\item Declaring a \emph{template} for creating objects;
\item Declaring a new \emph{type}.
\end{itemize}

\noindent In contrast, trait declarations in this source language only does the
former. Back to our example, the purpose of declaring two types is just to use
them for type annotations of the self reference.  In traits literature, a trait
usually ``requires'' some methods and, based on that,  ``provides'' another set
of methods. In our examples, the type of \lstinline$self$ actually denotes what
methods are required.

A trait can expect parameters, which become in scope in the entire trait body.
For example, the \lstinline$Comment$ trait is parametrized by
\lstinline$content$, and the \lstinline$content$ method does nothing more than
returning the eponymous variable.
%In comparison, traits in Scala do not allow
%taking parameters.

The origin of self references is always explicit. The \lstinline$Comment$ trait
requires that \lstinline$self$ be of type \lstinline$Comment$, which is defined
as a type synonym for a record in the first line. But the name ``self'' is
nothing special. In fact, \lstinline$self$ is just another parameter after the
preceding parameter list, and becomes in scope after the arrow. We could have
named it \lstinline$this$ or even \lstinline$s$, but that is generally
discouraged.

Creating an object is via the \lstinline$new$ keyword, similar to many OO
languages, except for one crucial novelty: we can create an object from multiple
traits. More precisely, the object is created from the \emph{composition} of
those traits. Therefore, we are able to call methods from different traits on a
single object. For example, we can create a single object from
\lstinline$Comment$  and \lstinline$Up$ traits and test its
functionality as follows:

\begin{lstlisting}
let comment = new[Comment&Up] (Comment("Have fun!"), Up(4))
in println(comment.content(), comment.upvotes())
-- Output: "Have fun!" 4
\end{lstlisting}

\subsection{Traits with Dependencies}
The following example shows that a trait
can depend on another trait. First we define the type of a point and a trait for
a standard point.

\begin{lstlisting}
type Point = {x: () (*$ \to $*) Int, y: () (*$ \to $*) Int} in
trait Point(x: Int, y: Int) { self: Point (*$ \to $*)
  x() = x
  y() = y
} in
\end{lstlisting}

The norm of a point can be defined as its distance to the origin. We provide two
definitions of norm via two traits.

\begin{lstlisting}
type Norm = { norm: () (*$ \to $*) Double } in
trait EuclideanNorm() { self: Point (*$ \to $*)
  norm() = Math.sqrt(self.x() * self.x() + self.y() * self.y())
} in

trait ManhattanNorm() { self: Point (*$ \to $*)
  norm() = Math.abs(self.x()) + Math.abs(self.y())
} in
\end{lstlisting}

Note how in \lstinline$EuclideanNorm$ and \lstinline$ManhattanNorm$
the type of the self-reference is \lstinline$Point$! This is in
contrast to a typical object-oriented language, such as Java, where
the self-reference must always be of the same type as the class being
defined. It is this functionality that allow us to express
dependencies between traits. When the traits \lstinline$EuclideanNorm$
and \lstinline$ManhattanNorm$ are instantiated and composed with some
other traits, they must be composed with an implementation of
\lstinline$Point$.
%Thus, ater compose the definition of point and norm we can have
%different meanings. Composition of traits is anonymous in the sense that there
%is no need to explicitly give a name to the composition.

\begin{lstlisting}
println(new[Point&Norm] (Point(3,4) & EuclideanNorm()).norm()) --Prints 5
println(new[Point&Norm] (Point(3,4) & ManhattanNorm()).norm()) --Prints 7
\end{lstlisting}

\subsection{Mutual Dependencies}

The next example, although a little bit contrived, illustrates that when two
traits are composed, any two methods in those two traits can refer to each
other via the self reference, just as if they were inside the same class.

\begin{lstlisting}
type EvenOdd = {
  even: Int (*$ \to $*) Bool,
  odd:  Int (*$ \to $*) Bool
} in
trait Even() { self: EvenOdd (*$ \to $*)
  even(n: Int) = if n == 0 then True else self.odd(n - 1)
} in
trait Odd() { self: EvenOdd (*$ \to $*)
  odd(n: Int) = if n == 0 then False else self.even(n - 1)
} in
new (Even() & Odd()).odd(42)
\end{lstlisting}

When the two traits are composed, conceptually it is as if that a new object were
being created on the fly by copying all the definitions inside those two traits.
If there is any unresolved conflict, the program will be rejected by the type
system.

\subsection{Detecting and Resolving Conflicts in Trait Composition}

Traits usually supports explicit conflict detection and resolution.
In inheritance, one pattern is for the subclass to override methods defined in the parent.
The trait-based approach analog is excluding a method from a trait.
We show how the mechanism can be modeled in \name.
The following example shows a counter object and how we could extend its
behavior so that it supports reset. First we define a \lstinline$Counter$ as a
type synonym for a record that contains a \lstinline$val$ method, which returns
the current counter value. Next we define a trait \lstinline$Counter$ that
contains two methods. The \lstinline$val$ method just returns the value that is
bound at the parameter of the trait, and \lstinline$incr$ returns a new counter.

\begin{lstlisting}
type Counter = { val: () (*$ \to $*) Int } in
trait Counter(val: Int) { self: Counter (*$ \to $*)
  val() = val
  incr() = new[Counter] Counter(val + 1)
} in
type Reset = { reset: () (*$ \to $*) Counter } in
trait Reset() { self: Counter (*$ \to $*)
  reset() = new[Counter] Counter(0)
} in
let counter = new[Counter&Reset] (Counter(0) & Reset())
in counter.incr()
\end{lstlisting}

In the above code, even though \lstinline$counter$ has a reset method, after we
call the \lstinline$incr$ method, the resulting object no longer has that.
Therefore, naturally we would like to override the \lstinline$incr$ method
inside \lstinline$Reset$.

\begin{lstlisting}
type Counter = { val: () (*$ \to $*) Int } in
trait Counter(val: Int) { self: Counter (*$ \to $*)
  val() = val
  incr() = new Counter(val + 1)
} in
trait Reset() { self: Counter (*$ \to $*)
  incr() = new (Counter(val + 1) & Reset())
  reset() = new Counter(0)
} in
let counter = new (Counter(0) & Reset())
in counter.incr()
\end{lstlisting}

However the modified code should not typecheck according to the specification of
traits, since both \lstinline$Counter$ and \lstinline$Reset$ contains a
conflicting \lstinline$incr$ method. The code also does not typecheck since it
violates the disjoint intersection typing rules of \name. The programmer needs to resolve the conflict by
excluding the \lstinline$incr$ being overridden using the record exclusion
operator.

\begin{lstlisting}
(* \ldots *)
let counter = new (Counter(0) \ incr & Reset())
in counter.incr()
\end{lstlisting}

\subsection{Dynamic Instantiation}

One difference with traditional traits or classes is that in our
language we are able to compose traits \emph{dynamically} and then instantiate
them. This is impossible in traditional OO languages such as Java since classes
being instantiated must be known statically. Actually, since traits are just
terms, traits are first-class values and can be defined inside a function,
passed around or returned just as normal terms. The following function takes a
trait, \lstinline$log$, with unknown implementation and instantiate it.

\begin{lstlisting}
let f (log: Trait[Log]) = new [Log] log
in (* \ldots *)
\end{lstlisting}\bruno{example does not make sense (and uses no composition).}

\bruno{Explain the trait[Log]}

%*******************************************************************************
\subsection{Desugaring}
%*******************************************************************************

Of course, this whole section will lose its point if the source language cannot
be translated to \name and checked against the type system of \name. A more
formal description can be found in the appendix. The idea of trait translation
is inspired by the functional mixin
semantics~\cite{cook1989denotational} using open recursion, which was
proposed by Cook in an untyped setting. However, our translation is done context of a
statically-typed programming language, which is what provides the ability to \emph{statically}
detect conflicts in traits.

\paragraph{Trait Declarations.} A trait in the source language is translated into
nothing but a normal term in \name. For example,

\begin{lstlisting}
trait Point(x: Int, y: Int) { self: Point (*$ \to $*)
  x() = x
  y() = self.z()
}
(* \ldots *)
\end{lstlisting}

becomes

\begin{lstlisting}
let Point (x: Int, y: Int) (self: () (*$ \to $*) Point) = {
  x = (*$ \lambda $*)(_: ()) (*$ \to $*) x,
  y = (*$ \lambda $*)(_: ()) (*$ \to $*) (self ()).z()
} in
\end{lstlisting}\bruno{shouldn't the first two arguments be a pair
  rather than curried arguments? You don't allow partial application
  of constructors, do you?}

One difference is that the self reference becomes a thunk and all occurrences of
it have been replaced by \lstinline$self ()$ and the position of the self
reference in the parameter list is adjusted. In fact, \lstinline$self$ is not a
special keyword. It can have any name, but \lstinline$self$ is a
convention.

The body of the trait becomes a record whose labels are the method names.
\lstinline$Point$ has type:

\begin{lstlisting}
Int (*$ \to $*) Int (*$ \to $*) (() (*$ \to $*) Point) (*$ \to $*) Point
\end{lstlisting}

The syntax for construction such as \lstinline$Point(3,4)$ is just function
application in \name. And note that \lstinline$Point(3,4)$ is of type
\begin{lstlisting}
(() (*$ \to $*) Point) (*$ \to $*) Point
\end{lstlisting}

Therefore it is an open recursive term: the recursive call is passed as an argument.

\paragraph{The ``new'' Keyword.} \lstinline$new$ instantiates a trait by taking the
fixpoint of its corresponding open term. In fact, \lstinline$new$ is translated as
an inlined fixpoint. For example,

\begin{lstlisting}
new[Point] Point(3,4)
\end{lstlisting}

becomes

\begin{lstlisting}
let rec self : () (*$ \to $*) Point = (*$ \lambda $*)(_: ()) (*$ \to $*) Point (3, 4) self
in self ()
\end{lstlisting}

The composition of traits in the source language is desugared using the merge
operator. The reason that traits built on \name have conflict detection for free
is that the merge operator is enforcing that the two terms being merged are
disjoint. For example,

\begin{lstlisting}
new[Point3D] (Point(3,4) & Z(5))
\end{lstlisting}

is turned into

\begin{lstlisting}
let rec self : () (*$ \to $*) Point3D = (*$ \lambda $*)(_: ()) (*$ \to $*) (Point (3, 4) self) ,, (Z 5 self)
in self ()
\end{lstlisting}

\paragraph{The ``Trait'' Keyword.} The capitalized \lstinline$Trait$ keyword
expects a type and is translated into an open type. For example,

\begin{lstlisting}
Trait[Point]
\end{lstlisting}

\noindent becomes

\begin{lstlisting}
(() (*$ \to $*) Point) (*$ \to $*) Point
\end{lstlisting}

\bruno{There are two uses of the ``trait'' keyword in the source
  language: one is to denote the type of traits; the other is to
  declare traits. Perhaps you want to use ``Trait'' instead for the former?}

% \subsection{Intersection Types in Existing Languages}
%
% What is an intersection type? The intersection of types $A$ and $B$
% contains exactly those values which can be used as either of type $A$
% or of type $B$.  Just as not all intersection of sets are nonempty,
% not all intersections of types are inhabited.  For example, the
% intersection of a base type $\code{Int}$ and a function type
% $\code{Int} \to \code{Int}$ is not inhabited.\bruno{put this text somewhere?}
%
% Since then various researchers have
% studied intersection types, and some languages have adopted in one
% form or another. However, while intersection types are already used
% in various languages, the lack of a merge operator removes
% considerable expressiveness.
%
%
% A number of OO languages, such as
% Java, C\#, Scala, and Ceylon\footnote{\url{http://ceylon-lang.org/}},
% already support intersection types to different degrees. Intersection
% types are particularly relevant for OOP as they can be used to model
% multiple interface inheritance. In Java, for example,
%
% \begin{lstlisting}
% interface AwithB extends A, B {}
% \end{lstlisting}
%
% \noindent introduces a new interface \lstinline{AwithB} that satisfies the interfaces of
% both \lstinline{A} and \lstinline{B}. Arguably such type can be considered as a nominal
% intersection type. Scala takes one step further by eliminating the
% need of a nominal type. For example, given two concrete traits, it is possible to
% use \emph{mixin composition} to create an object that implements both
% traits. Such an object has a (structural) intersection type:
%
% \begin{lstlisting}
% trait A
% trait B
%
% val newAB : A with B = new A with B
% \end{lstlisting}
%
% \noindent Scala also allows intersection of type parameters. For example:
% \begin{lstlisting}
% def merge[A,B] (x: A) (y: B) : A with B = ...
% \end{lstlisting}
% uses the anonymous intersection of two type parameters \lstinline{A} and
% \lstinline{B}. However, in Scala it is not possible to dynamically
% compose two objects. For example, the following code:
%
% \begin{lstlisting}
% // Invalid Scala code:
% def merge[A,B] (x: A) (y: B) : A with B = x with y
% \end{lstlisting}
%
% \noindent is rejected by the Scala compiler. The problem is that the
% \lstinline{with} construct for Scala expressions can only be used to
% mixin traits or classes, and not arbitrary objects. Note that in the
% definition \lstinline{newAB} both \lstinline{A} and \lstinline{B} are
% \emph{traits}, whereas in the definition of \lstinline{merge} the variables
% \lstinline{x} and \lstinline{y} denote \emph{objects}.
%
% This limitation essentially put intersection types in Scala in a second-class
% status. Although \lstinline{merge} returns an intersection type, it is
% hard to actually build values with such types. In essence an
% object-level introduction construct for intersection types is missing.
% As it turns out using low-level type-unsafe programming features such
% as dynamic proxies, reflection or other meta-programming techniques,
% it is possible to implement such an introduction
% construct in Scala~\cite{oliveira2013feature,rendel14attributes}. However, this
% is clearly a hack and it would be better to provide proper language
% support for such a feature.
%
% To address the limitations of intersection types in languages like
% Scala, \name allows intersecting any two terms at run time using a
% \emph{merge} operator (denoted by $ \mergeop $)~\cite{dunfield2014elaborating}.  With the merge
% operator it is trivial to implement the \lstinline{merge} function in \name:
%
% \begin{lstlisting}
% let merge[A, B * A] (x : A) (y : B) : A & B = x ,, y in (*$ \ldots $*)
% \end{lstlisting}
%
% \noindent In contrast to Scala's term-level \lstinline{with}
% construct, the operator \lstinline{,,} allows two arbitrary values \lstinline{x}
% and \lstinline{y} to be merged. The resulting type is a \emph{disjoint}
% intersection of the types of  \lstinline{x}
% and \lstinline{y} (\lstinline{A & B} in this case).
%
% A well-formed type is such that given any query type,
% it is always clear which subpart the query is referring to.
% In terms of rules, this notion of well-formedness is almost the same as the one in System $F$
% except for intersection types we require the two components to be disjoint.
%

\section{The \name Calculus}\label{sec:fi}
This section presents the syntax, subtyping, and typing of \name: 
a calculus with intersection types, parametric polymorphism, records and a merge operator. 
This calculus is an extension of the \oldname calculus~\cite{oliveira16disjoint},
which is itself inspired by Dunfield's
calculus~\cite{dunfield2014elaborating}. \name extends \oldname with (disjoint) polymorphism.
%The novelty of \name is the addition of \emph{disjoint polymorphism}:
%a form of parametric polymorphism with disjointness contraints, which
%allows flexibility while at the same time retaining coherence. 
%As discussed in Section~\ref{overview} retaining
%coherence, while having an expressive form of polymorphism is non-trvial.
%Section~\ref{sec:disjoint} introduces \namedis, which shows the necessary changes
%for supporting disjoint intersection types and disjoint
%quantification and ensuring coherence.
Section~\ref{sec:alg-dis} introduces the necessary changes to the
definition of disjointness presented by Oliveira et al.~\cite{oliveira16disjoint} in
order to add disjoint polymorphism.

%\joao{we already say this in the introduction}
%All the meta-theory of \name has been mechanized in Coq, and is available in
%the supplementary materials submitted with the paper.

\subsection{Syntax}
The syntax of \name (with the differences to \oldname highlighted in gray) is: 
%are intersection types $A \inter B$ at the
%type-level and the ``merges'' $e_1 \mergeop e_2$ at the term level.

%TODO merge this figure with figure 5 (text too)
%\begin{figure}[!t]
\vspace{-15pt}
  \[
    \begin{array}{l}
      \begin{array}{llrll}
        \text{Types}
        & A, B & \!\!\Coloneqq & \!\top \mid \tyint \mid A \to B \mid A
                             \inter B \mid \highlight{$\alpha$} \mid \highlight{$\fordis \alpha A B$} \mid \highlight{$\recordType l A$} & \\ 

        \text{Terms}
        & e & \!\!\Coloneqq & \!\top \mid i \mid x \mid \lamty x A e \mid \app {e_1} {e_2} 
              \mid e_1 \mergeop e_2 \mid \!\highlight{$\blamdis \alpha A e$} \!\mid \!\highlight{$\tapp e A$} \!\mid 
              \!\highlight{$\recordCon l e$} \!\mid \!\highlight{$\recordProj e l$} & \\
        \text{Contexts}
        & \Gamma & \!\!\Coloneqq & \!\cdot
                   \mid \Gamma, \highlight{$\alpha \disjoint A$}
                   \mid \Gamma, x \oftype A  & \\
      \end{array}
    \end{array}
  \]

%  \caption{\name syntax.}
%  \label{fig:fi-syntax}
% \end{figure}

\paragraph{Types.} 
Metavariables $A$, $B$ range over types. 
Types include all constructs in \oldname (excluding product types): a top type $\top$; 
the type of integers $\tyint$;
function types $A \to B$; and intersection types $A \inter B$.
The main novelty are two standard constructs of System $F$ used to support
polymorphism: 
type variables $\alpha$ and disjoint (universal) quantification $\fordis \alpha A B$. 
Unlike traditional universal quantification, the disjoint
quantification includes a disjointness constraint associated to a type variable $\alpha$.
Finally, \name also includes singleton record types, which consist of a label $l$ and
an associated type $A$.
We will use $\subst {A} \alpha {B}$
to denote the capture-avoiding substitution of $A$ for $\alpha$ inside $B$ and
$\ftv \cdot$ for sets of free type variables. 

\paragraph{Terms.} 
Metavariables $e$ range over terms.  
Terms include all constructs in \oldname: a canonical top value $\top$; integer literals $i$;
variables $x$, lambda abstractions ($\lamty x A e$); applications 
($\app {e_1} {e_2}$); and the \emph{merge} of terms $e_1$ and $e_2 $ denoted as 
$e1 \mergeop e2$.
Terms are extended with two standard constructs in System $F$:
abstraction of type variables over terms $\blamdis \alpha A e$; and
application of terms to types $\tapp e A$. 
The former also includes an extra disjointness constraint tied to the type 
variable $\alpha$, due to disjoint quantification.
%If one regards $e_1$ and $e_2$ as objects, their merge will respond to
%every method that one or both of them have.
Singleton records consists of a label $l$ and an associated term $e$.
Finally, the accessor for a label $l$ in term $e$ is denoted as $\recordProj e l$.

\paragraph{Contexts.} Typing contexts $ \Gamma $ track bound type variables
$\alpha$ with disjointness constraints $A$; and variables $x$ with their type $A$. 
We will use $\subst {A} \alpha {\Gamma}$
to denote the capture-avoiding substitution of $A$ for $\alpha$ in the co-domain of
$\Gamma$ where the domain is a type variable (i.e all disjointness constraints).
Throughout this paper, we will assume that all contexts are
well-formed. Importantly, besides usual well-formedness conditions, in
well-formed contexts type variables must not appear free within its own disjointness constraint.
%All substitutions performed in environments must also lead to well-formed environments.
%In order to focus on the key features that make this language interesting, we do
%not include other forms such as type constants and fixpoints here. 
%However they can be included in the formalization in
%standard ways and we are using them in discussions and examples. %TODO are we?
\paragraph{Syntactic sugar}
In \name we may quantify a type variable and ommit its constraint. 
This means that its constraint is $\top$. 
For example, the function type $\forall \alpha. \alpha \to \alpha$ is syntactic sugar
for  $\fordis \alpha \top {\alpha \to \alpha}$.
This is discussed in more detail in Section~\ref{sec:disjoint}. 

% \paragraph{Discussion.} A natural question the reader might ask is that why we
% have excluded union types from the language. The answer is we found that
% intersection types alone are enough support extensible designs.

\subsection{Subtyping}
% In some calculi, the subtyping relation is external to the language: those
% calculi are indifferent to how the subtyping relation is defined. In \name, we
% take a syntactic approach, that is, subtyping is due to the syntax of types.
% However, this approach does not preclude integrating other forms of subtyping
% into our system. One is ``primitive'' subtyping relations such as natural
% numbers being a subtype of integers. The other is nominal subtyping relations
% that are explicitly declared by the programmer.


%\begin{figure}
%  \begin{mathpar}
%    \formsub \\
%    \rulesubvar \and \rulesubfun \and \rulesubforall \and \rulesubinter \and
%    \rulesubinterl \and \rulesubinterr
%  \end{mathpar}
%
%  \begin{mathpar}
%    \formwf \\
%    \rulewfvar \and \rulewffun \and \rulewfforall \and \rulewfinter
%  \end{mathpar}
%
%  \begin{mathpar}
%    \formt \\
%    \ruletvar \and \ruletlam \and \ruletapp \and \ruletblam \and \rulettapp \and
%    \ruletmerge
%  \end{mathpar}
%
%  \caption{The type system of \name.}
%  \label{fig:fi-type}
%\end{figure}

% Intersection types introduce natural subtyping relations among types. For
% example, $ \tyint \inter \tybool $ should be a subtype of $ \tyint $, since the former
% can be viewed as either $ \tyint $ or $ \tybool $. To summarize, the subtyping rules
% are standard except for three points listed below:
% \begin{enumerate}
% \item $ A_1 \inter A_2 $ is a subtype of $ A_3 $, if \emph{either} $ A_1 $ or
%   $ A_2 $ are subtypes of $ A_3 $,

% \item $ A_1 $ is a subtype of $ A_2 \inter A_3 $, if $ A_1 $ is a subtype of
%   both $ A_2 $ and $ A_3 $.

% \item $ \recordType {l_1} {A_1} $ is a subtype of $ \recordType {l_2} {A_2} $, if
%   $ l_1 $ and $ l_2 $ are identical and $ A_1 $ is a subtype of $ A_2 $.
% \end{enumerate}
% The first point is captured by two rules $ \reflabelsubinterl $ and
% $ \reflabelsubinterr $, whereas the second point by $ \reflabelsubinter $.
% Note that the last point means that record types are covariant in the type of
% the fields.

The subtyping rules of the form $A \subtype B$ are shown in 
Figure~\ref{fig:fi-subtype}. 
At the moment, the reader is advised to ignore the
gray-shaded parts, which will be explained later. 
Some rules are ported from \oldname: \reflabel{\labelsubtop}, 
\reflabel{\labelsubint},
\reflabel{\labelsubfun}, \reflabel{\labelsubinter}, \reflabel{\labelsubinterl} and
\reflabel{\labelsubinterr}.

\begin{figure}[t]
\begin{spacing}{0.5}
  \begin{mathpar}
    \framebox{$\jatomic A$} \\
    \inferrule*{}{\jatomic \tyint} \and 
    \inferrule*{}{\jatomic {A \to B}} \and
    \inferrule*{}{\jatomic \alpha} \and
    \inferrule*{}{\jatomic {\fordis \alpha B A}} \and
    \inferrule*{}{\jatomic {\recordType l A}}
  \end{mathpar}
  \begin{mathpar}
    \formsub \\ 
    \rulesubtop \and \rulesubinter \and 
    \rulesubint \and \rulesubinterlcoerce \and 
    \rulesubrec \and \rulesubinterrcoerce \and
    \rulesubvar  \and \rulesubfun \and 
    \rulesubforallext 
  \end{mathpar}
\end{spacing}
  \caption{Subtyping rules of \name.}
  \label{fig:fi-subtype}
\end{figure}



%There are three rules which rather straightforward: \reflabel{\labelsubtop}
%says that every type is a subtype of $\top$; \reflabel{\labelsubint} and 
%\reflabel{\labelsubvar} define subtyping as a reflexive relation on integers and
%type variables.
%The rule \reflabel{\labelsubfun} says that a function is contravariant in 
%its parameter type and covariant in its return type. 
%The three rules dealing with intersection types are just what one would expect 
%when interpreting types as sets. 
%Under this interpretation, for example, the rule \reflabel{\labelsubinter}
%says that if $A_1$ is both the subset of $A_2$ and the subset of $A_3$, then
%$A_1$ is also the subset of the intersection of $A_2$ and $A_3$.

\paragraph{Polymorphism and Records.}
The subtyping rules introduced by \name refer to polymorphic constructs and records. 
\reflabel{\labelsubvar} defines subtyping as a reflexive relation on type variables.
In \reflabel{\labelsubforall} a universal quantifier ($\forall$) 
is covariant in its body, and contravariant in its disjointness constraints.
The \reflabel{\labelsubrec} rule says that records are covariant
within their fields' types.
The subtyping relation uses an auxiliary unary $ordinary$ relation,
which identifies types that are not intersections. The $ordinary$ conditions on two of the intersection rules are necessary to 
produce unique coercions~\cite{oliveira16disjoint}. The $ordinary$
relation needed to be extended with respect to \oldname.
As shown at the top of Figure~\ref{fig:fi-subtype}, the new types it contains are 
type variables, universal quantifiers and record types.

\paragraph{Properties of Subtyping.} The subtyping relation is reflexive and transitive.
\thmprf{0cm}{
\begin{restatable}[Subtyping reflexivity]{lemma}{subrefl}
  \label{lemma:subrefl}
  For any type $A$, $A \subtype A$.
\end{restatable}}
{-0.1cm}
{By induction on $A$.}
{0cm}
%\noindent \emph{Proof.} By induction on $A$.
%\restatableproof{lemma}{Subtyping reflexivity}{subrefl}{lemma:subrefl}
%{For any type $A$, $A \subtype A$.}
%{By induction on $A$.}{-0.1cm}%
%\begin{prf}
%By induction on $A$.
%\end{prf}%
%\begin{restatable}[Subtyping transitivity]{lemma}{subtrans}
%  \label{lemma:subtrans}
%  If $A \subtype B$ and $B \subtype C$, then $A \subtype C$.
%\end{restatable}%
%\noindent \emph{Proof.} By double induction on both derivations.%
\restatableproof{lemma}{Subtyping transitivity}{subtrans}{lemma:subtrans}
{If $A \subtype B$ and $B \subtype C$, then $A \subtype C$.}
{By double induction on both derivations.}{-0.1cm}

%\begin{prf}
%By double induction on both derivations. 
%\end{prf}
%\bruno{Too much space waisted between Lemma and proof. reduce the
%  white space.}
%TODO example showing contravariance in disjointness constraints goes here or in the overview 
%section?
%\paragraph{Metatheory.} As other standard subtyping relations, we can show that
%subtyping defined by $\subtype$ is also reflexive and transitive.
%
%\begin{lemma}[Subtyping is reflexive] \label{lemma:sub-refl}
%  For all type $ A $, $ A \subtype A $.
%\end{lemma}
%
%\begin{lemma}[Subtyping is transitive] \label{lemma:sub-trans}
%  If $ A_1 \subtype A_2 $ and $ A_2 \subtype A_3 $,
%  then $ A_1 \subtype A_3 $.
%\end{lemma}
\subsection{Typing}

\begin{comment}
\begin{figure}[!t]
  \begin{mathpar}
    \formwf \\ \rulewfint \and \rulewfvardis \and \rulewffun \and \rulewfrec \and 
    \rulewftop \and \rulewfforalldis \and \rulewfinterdis 
  \end{mathpar}

  \caption{Well-formedness rules for types of \name.}
  \label{fig:wf}
\end{figure}
\end{comment}


%  \begin{mathpar}
%    \formt \\ \ruletvar \and \ruletlam \and \ruletapp \and
%    \ruletblam \and \rulettapp \and \ruletmergedis 
%  \end{mathpar}

\paragraph{Well-formedness.}
The well-formedness rules are shown in the top part of Figure~\ref{fig:fi-type}. 
The new rules over \oldname are \reflabel{\labelwfvar} and \reflabel{\labelwfforall}. 
Their definition is quite straightforward, but note that the constraint in the latter
must be well-formed.

\begin{figure}
  \begin{spacing}{1}
  \begin{mathpar}
    \formwf \\ \rulewfint \and \rulewfvardis \and \rulewfrec \and 
    \rulewffun \and \rulewftop \and \rulewfforalldis \and \rulewfinterdis 
  \end{mathpar}
  \begin{mathpar}
    \formbi \\ \brulettop \and \bruletint \and \bruletvar \and \bruletann \and 
    \bruletapp \and \brulettappdis \and \bruletmergedis \and \bruletrec \and 
    \bruletprojr \and \bruletblamdis 
  \end{mathpar}
  \begin{mathpar}
    \formbc \\ \bruletlam \and \bruletsub
  \end{mathpar}
  \end{spacing}
  \caption{Well-formedness and type system of \name.}
  \label{fig:fi-type}
\end{figure}

\paragraph{Typing rules.}
Our typing rules are formulated as a bi-directional type-system. 
Just as in \oldname, this ensures the type-system is not only syntax-directed, but
also that there is no type ambiguity: that is, inferred types are unique.
The typing rules are shown in the bottom part of Figure~\ref{fig:fi-type}. 
Again, the reader is advised to ignore the
gray-shaded parts, as these will be explained later. 
The typing judgements are of the form: $\jcheck \Gamma e A$ and  
$\jinfer \Gamma e A$.
They read: ``in the typing context $\Gamma$, the term $e$ can be
checked or inferred to
type $A$'', respectively. 
The rules ported from \oldname are the
check rules for $\top$ (\reflabel{\labelttop}), integers (\reflabel{\labeltint}), 
variables (\reflabel{\labeltvar}),  application (\reflabel{\labeltapp}), merge operator  
(\reflabel{\labeltmerge}), annotations (\reflabel{\labeltann}); and infer rules
for lambda abstractions (\reflabel{\labeltlam}), and the subsumption rule 
(\reflabel{\labeltsub}).

\paragraph{Disjoint quantification.}
The new rules, inspired by System $F$, are the infer rules for type
application \reflabel{\labelttapp}, and for type abstraction
\reflabel{\labeltblam}.  Type abstraction is introduced by the big
lambda $\blamdis \alpha A e$, eliminated by the usual type application
$\tapp e A$ (\reflabel{\labelttapp}).  The disjointness constraint is
added to the context in \reflabel{\labeltblam}. During a type application, the
type system makes sure that the type argument satisfies the
disjointness constraint.  Type application performs an extra check
ensuring that the type to be instantiated is compatible
(i.e. disjoint) with the constraint associated with the abstracted
variable.  This is important, as it will retain the desired coherence
of our type-system.  For ease of discussion, also in
\reflabel{\labeltblam}, we require the type variable introduced by the
quantifier to be fresh.  For programs with type variable shadowing,
this requirement can be met straighforwardly by variable renaming.

\paragraph{Records.}
Finally, $\reflabel{\labeltrec}$ and $\reflabel{\labeltprojr}$ deal with record types.
The former infers a type for a record with label $l$ if it can infer a type for the
inner expression; the latter says if one can infer a record type $\recordType l A$ 
from an expression $e$, then it is safe to access the field $l$, and infering type $A$.


\section{Semantics and Coherence}
\label{sec:disjoint}

\subsection{Semantics}

We define the dynamic semantics of the call-by-value \name by means of
a type-directed translation to an extension of System $F$ with pairs~\footnote{
  For simplicity, we will just refer to this system as ``System $F$''
  from now on.}.
%The type-directed translation is also shown in
%Figure~\ref{}, where the resulting System F terms are highlighted in
%gray.


\paragraph{Target language.}
The syntax and typing of our target language is unsurprising. The syntax of the
target language is shown in Figure~\ref{fig:f-syntax}. The highlighted part
shows its difference with the standard System $F$. The typing rules can be found
in the appendix.


\begin{figure}[!t]
  \[
    \begin{array}{llrl}
      \text{Types}    & T & \Coloneqq & \alpha \\
                      &   & \mid      & \highlight {$ () $} \\
                      &   & \mid      & \tyint \\
                      &   & \mid      & {T_1} \to {T_2} \\
                      &   & \mid      & \for \alpha T \\
                      &   & \mid      & \highlight {$ \pair {T_1} {T_2} $} \\
      \text{Terms}    & E & \Coloneqq & x \\
                      &   & \mid      & \highlight {$ () $} \\
                      &   & \mid      & i \\
                      &   & \mid      & \lamty x T E \\
                      &   & \mid      & \app {E_1} {E_2} \\
                      &   & \mid      & \blam \alpha E \\
                      &   & \mid      & \tapp E T \\
                      &   & \mid      & \highlight {$ \pair {E_1} {E_2} $} \\
                      &   & \mid      & \highlight {$ \proj k E $} \quad k \in \{ 1, 2 \} \\
      \text{Contexts} & G & \Coloneqq & \cdot \mid G, \alpha \mid G, x \oftype T \\
    \end{array}
  \]
  \caption{Target language syntax.}
  \label{fig:f-syntax}
\end{figure}

\paragraph{Type and context translation.}

Figure~\ref{fig:type-and-context-translation} defines the type translation
function $\im \cdot$ from \name types $A$ to target language types $T$. The
notation $\im \cdot$ is also overloaded for context translation from \name
contexts $\Gamma$ to target language contexts $G$.

\begin{figure}[!t]
  \framebox{$\im A = T$}

  \begin{align*}
    \im \alpha                &= \alpha \\
    \im \top                  &= () \\
    \im {A_1 \to A_2}         &= \im {A_1} \to \im {A_2} \\
    \im {\fordis \alpha A B}  &= \for \alpha \im B \\
    \im {A_1 \inter A_2}      &= \pair {\im {A_1}} {\im {A_2}} \\
  \end{align*}

  \framebox{$\im \Gamma = G$}

  \begin{align*}
    \im \cdot                      &= \cdot \\
    \im {\Gamma, \alpha}           &= \im \Gamma, \alpha \\
    \im {\Gamma, \alpha \oftype A} &= \im \Gamma, \alpha \oftype \im A
  \end{align*}

  \caption{Type and context translation.}
  \label{fig:type-and-context-translation}
\end{figure}

% The rules given in this section are identical with those in
% Section~\ref{sec:fi}, except for the light blue part. The translation consists
% of four sets of rules, which are explained below:

\subsection{Well-formedness and substitution}
As mentioned previously, the disjointness of a type variable is deferred until its sanity 
check occurs at type instantation.
Because of this, our system needs to ensure that well-formed types are stable under substitution,
as this will play a crucial role in retaining coherence.

Typically, in polymorphic systems with explicit instantiation it is necessary to show that
types are stable under substitution, in order to avoid ill-formed types.
In the presence of disjoint quantification, we cannot prove such property.
However, a weaker version of that property -- but strong enough for our type-system's metatheory -- 
can be proven, namely:

\joao{missing talking about wellformedness of contexts}
\begin{lemma}[Types are stable under substitution]
  \label{lemma:wfsubst}

  If $\jwf \Gamma A$ and $\jwf \Gamma B$ and $(x \disjoint C) \in \Gamma$ 
  and $\jdisimpl \Gamma B C$, then $\jwf {\Gamma \subst x B} {A \subst x B}$.
\end{lemma}

%\begin{restatable}[Instantiation]{lemma}{instantiation}
%  \label{lemma:instantiation}
%
%  If $\jwf {\Gamma, \alpha \disjoint B} C$, $\jwf \Gamma A$, $\jdis \Gamma A B$
%  then $\jwf \Gamma {\subst A \alpha C}$.
%\end{restatable}
\joao{add proof}

This lemma enables us to show that all types produced by the type-system are well-typed.
More formally, we have that:

\begin{restatable}[Well-formed typing]{lemma}{wellformedtyping}
  \label{lemma:wellformed-typing}

  If $\jcheck \Gamma e A$, then $\jwf \Gamma A$. 

  If $\jinfer \Gamma e A$, then $\jwf \Gamma A$.
\end{restatable}

\begin{proof}
  By induction on the derivation and applying
  Lemma~\ref{lemma:wfsubst} in the case of \reflabel{\labelttapp}.
\end{proof}

\subsection{Top-like types and their coercions}

\begin{figure}[!t]
  \begin{mathpar}
    \formtoplike \\ %\framebox{$\jatomic A$} \\

    \ruletopltop \and \ruletoplinter \and \ruletoplfun \and \ruletoplforall
  \end{mathpar}


  %\begin{center}
    \framebox{$\andcoerce{A}_{C} = T$}
  %\end{center}

  \[
  \andcoerce{A}_{C} = 
  \begin{cases} 
        \toplike{A} & \andcoerce{A} \\ 
        %A = \top & () \\
        %A = A_1 \to A_2 \; \wedge \; \toplike{A_2} & \lam x \andcoerce{A_2}_{C} \\
        \text{otherwise} & {C} 
  \end{cases}
  \]

  %\begin{center}
    \framebox{$\andcoerce{A} = T$}
  %\end{center}

  \[
  \andcoerce{A} = 
  \begin{cases} 
        A = \top & () \\
        A = A_1 \to A_2 & \lam x \andcoerce{A_2} \\
        A = A_1 \inter A_2 & \pair {\andcoerce{A_1}} {\andcoerce{A_2}} \\
        A = \fordis \alpha B A & \blam \alpha {\andcoerce{A}}
  \end{cases}
  \]
  \caption{Top-like types and their coercions.}
  \label{fig:andcoercion}
\end{figure}

Our definition of top-like types is naturally extended from \oldname. 
The rules that compose this unary relation, denoted as $\toplike{ . }$, are presented at the top of 
Figure~\ref{fig:andcoercion}. 
The only new rule is \reflabel{\labeltoplforall}, which extends the notion of top-like types for
the (disjoint) quantifier case.

It is important pointing out that, despite the similarity of these rules with the simply-typed system, 
our notion of disjointness has changed.
Consequently, the set of well-formed top-like types has changed.
Therefore, we also extend the meta-function $\andcoerce{A}$, as shown in the bottom of 
Figure~\ref{fig:andcoercion}.
Note that, in contrast to \oldname, not only the $\forall$ case is defined, but also the intersection 
case.
This is extremely important as it plays a fundamental role in ensuring the coherence of subtyping, as
we will describe next.

\subsection{Coercive Subtyping and Coherence}

\paragraph{Coercive subtyping.}

The judgement
\[
A_1 \subtype A_2 \yields E
\]
extends the subtyping judgement in Figure~\ref{fig:fi-subtype} with a coercion
on the right hand side of $ \yields {} $. A coercion $ E $ is just an term
in the target language and is ensured to have type
$ \im {A_1} \to \im {A_2} $ (by Lemma~\ref{lemma:sub}). For example,
\[
\tyint \inter \tybool \subtype \tybool \yields {\lamty x {\im {\tyint \inter \tybool}} {\proj 2 x}}
\]
generates a coercion function with type: $\tyint \inter \tybool \to \tybool$.

In rules \reflabel{\labelsubvar}, \reflabel{\labelsubforall}, coercions are just
identity functions. In \reflabel{\labelsubfun}, we elaborate the subtyping of
parameter and return types by $\eta$-expanding $f$ to $\lamty x {\im {A_3}}
{\app f x}$, applying $E_1$ to the argument and $E_2$ to the result. Rules
\reflabel{\labelsubinterl}, \reflabel{\labelsubinterr}, and
\reflabel{\labelsubinter} elaborate intersection types.
\reflabel{\labelsubinter} uses both coercions to form a pair. Rules
\reflabel{\labelsubinterl} and \reflabel{\labelsubinterr} reuse the coercion
from the premises and create new ones that cater to the changes of the argument
type in the conclusions. Note that the two rules are overlapping and
hence a program can be elaborated differently, depending on which rule
is used. Finally, all rules produce type-correct coercions:

%But in the implementation one usually applies the rules sequentially with
%pattern matching, essentially defining a deterministic order of lookup.

% if we know $A_1$ is a subtype of $A_3$ and $C$ is a coercion from $A_1$
% to $A_3$, then we can conclude that $A_1 \inter A_2$ is also a subtype
% of $A_3$ and the new coercion is a function that takes a value $ x $ of type
% $A_1\inter A_2$, project $x$ on the first item, and apply $ C $ to it.

\begin{restatable}[Subtyping rules produce type-correct coercions]{lemma}{lemmasub}
  \label{lemma:sub}
  If $ A_1 \subtype A_2 \yields E $, then $ \jtype \cdot E {\im {A_1} \to \im {A_2}} $.
\end{restatable}

\begin{proof}
  By a straightforward induction on the derivation.
\end{proof}

% \george{Explain \reflabel{\labelsubforall} and distinction of kernel and all version.}

%\paragraph{Ordinary types}
%Atomic types are just those which are not intersection types, and are asserted by the judgement \[ \jatomic A \]

%In the left decomposition rules for intersections we introduce a requirement that
%$A_3$ is atomic. 
%The consequence of this requirement is that when $A_3$ is an intersection type, then
%the only rule that can be applied is \reflabel{\labelsubinter}.
%With the atomic constraint, one can guarantee that at any moment during the
%derivation of a subtyping relation, at most one rule can be used.

\paragraph{Unique coercions}

%The subtyping rules need some adjustment.
%Note that the $\bot$ type does not participate in subtyping since it holds no value.
%An important
%problem with the subtyping rules in Figure~\ref{fig:fi-type} is that the all three rules
%dealing with intersection types
%(\reflabel{\labelsubinterl} and \reflabel{\labelsubinterr} and \reflabel{\labelsubinter})
%overlap. Unfortunatelly,
%this means that different coercions may be given when checking the subtyping
%between two types, depending on which derivation is chosen. This is ultimately the reason
%for incoherence.
%There are two important types of overlap:
%
%\begin{enumerate}
%
%\item The left decomposition rules for intersections (\reflabel{\labelsubinterl} and \reflabel{\labelsubinterr}) overlap with each other.
%
%\item The left decomposition rules for intersections (\reflabel{\labelsubinterl} and \reflabel{\labelsubinterr})
%overlap with the right decomposition rules for intersections (\reflabel{\labelsubinter}).
%
%\end{enumerate}
%
%\noindent Fortunately, disjoint intersections (which are enforced by well-formedness)
%deal with problem 1): only one of the two left decomposition rules
%can be chosen for a disjoint intersection type. Since the two types in the intersection
%are disjoint it is impossible that both of the preconditions of the left decompositions are satisfied
%at the same time. 

In order to ensure a coherent type-system, we should prove that our subtyping relation is also coherent.
More formally, as also shown in ..., with disjoint intersections the following theorem holds:

\begin{lemma}[Unique subtype contributor]
  \label{lemma:unique-subtype-contributor}

  If $A_1 \inter A_2 \subtype B$, 
  where $A_1 \inter A_2$ and $B$ are well-formed types, and
  $B$ is not top-like,
  then it is not possible that the following holds at the same time:
  \begin{enumerate}
    \item $A_1 \subtype B$
    \item $A_2 \subtype B$
  \end{enumerate}
\end{lemma}

%Unfortunately, disjoint intersections alone are insufficient to deal with problem 2).
%In order to deal with problem 2), we introduce a distinction between types, and atomic types.

Finally, we can show that the coercion of a subtyping relation $A \subtype B$ 
is uniquely determined.
This fact is captured by the following lemma:

\begin{restatable}[Unique coercion]{lemma}{uniquecoercion}
  \label{lemma:unique-coercion}

  If $A \subtype B \yields {E_1}$ and $A \subtype B \yields {E_2}$, where $A$
  and $B$ are well-formed types, then $E_1 \equiv E_2$.
\end{restatable}

\begin{comment}
\paragraph{Expressiveness.}
Remarkably, our restrictions on subtyping do not sacrifice the expressiveness of
subtyping since we have the following two theorems:
\begin{theorem}
  If $A_1 \subtype A_3$, then $A_1 \inter A_2 \subtype A_3$.
\end{theorem}
\begin{theorem}
If $A_2 \subtype A_3$, then $A_1 \inter A_2 \subtype A_3$.
\end{theorem}

The interpretation of the theorem is that: even though the premise is made more
strict by the atomic condition, we can still derive the every subtyping relation
in the unrestricted system.
% \george{Explain why the proof shows this.}

Note that $A$ \emph{exclusive} or $B$ is true if and only if their truth value
differ. Next, we are going to investigate the minimal requirement (necessary and
sufficient conditions) such that the theorem holds.

If $A_1$ and $A_2$ in this setting are the same, for example,
$\tyint \inter \tyint \subtype \tyint$, obviously the theorem will
not hold since both the left $\tyint$ and the right $\tyint$ are a
subtype of $\tyint$.

We can try to rule out such possibilities by making the requirement of
well-formedness stronger. This suggests that the two types on the sides of
$\inter$ should not ``overlap''. In other words, they should be ``disjoint''. It
is easy to determine if two base types are disjoint. For example, $\tyint$
and $\tyint$ are not disjoint. Neither do $\tyint$ and $\code{Nat}$.
Also, types built with different constructors are disjoint. For example,
$\tyint$ and $\tyint \to \tyint$. For function types, disjointness
is harder to visualize. But bear in the mind that disjointness can defined by
the very requirement that the theorem holds.


This result is captured more formally by the following lemma:
\end{comment}

% \george{Note that $\bot$ does not participate in subtyping and why (because the
% empty set intersecting the empty set is still empty).}

% \george{What's the variance of the disjoint constraint? C.f. bounded
% polymorphism.}

% \george{Two points are being made here: 1) nondisjoint intersections, 2) atomic
% types. Show an offending example for each?}


%\subsection{Disjointness} Throughout the paper we already presented an intuitive
%definition for disjoiness. Here such definition is made a bit more precise, and
%well-suited to \namedis.
%
%\begin{definition}[Disjoint types]
%
%  Given a context $\Gamma$, two types $A$ and $B$ are said to be disjoint
%  (written $\jdis \Gamma A B$) if they do not share a common supertype. That is,
%  there does not exist a type $C$ such that $A \subtype C$ and that $B \subtype
%  C$. Note that we assume that all free type variables in $A$,
%    $B$ and $C$ are bound in $\Gamma$.
%
%  \[\jdis \Gamma A B \equiv \not \exists C.~A \subtype C \wedge B \subtype C\]
%
%\end{definition}
%
%To see this definition in action, $\tyint$ and $\tychar$ are disjoint,
%because there is
%no type that is a supertype of the both. On the other hand, $\tyint$ is not
%disjoint with itself, because $\tyint \subtype \tyint$. This implies that
%disjointness is not reflexive as subtyping is. Two types with different shapes
%are always disjoint, unless one of them is an intersection type. For example, a function
%type and a universally quantified type must be disjoint. But a function type and an intersection
%type may not be. Consider:
%\[ \tyint \to \tyint \quad \text{and} \quad (\tyint \to \tyint) \inter (\tystring \to \tystring) \]
%Those two types are not disjoint since $\tyint \to \tyint$ is their common supertype.

%\subsection{Syntax}
%
%\begin{figure}
%  \[
%    \begin{array}{l}
%      \begin{array}{llrll}
%        \text{Types}
%        & A, B & \Coloneqq & \alpha                  & \\
%        &      & \mid & \highlight {$\bot$}          & \\
%        &      & \mid & A \to B                      & \\
%        &      & \mid & \for {(\alpha \highlight {$\disjoint A$})} B  & \\
%        &      & \mid & A \inter B                   & \\
%
%        \\
%        \text{Terms}
%        & e & \Coloneqq & x                        & \\
%        &   & \mid & \lam {(x \oftype A)} e          & \\
%        &   & \mid & \app {e_1} {e_2}              & \\
%        &   & \mid & \blam {(\alpha \highlight {$\disjoint A$})} e  & \\
%        &   & \mid & \tapp e A                     & \\
%        &   & \mid & e_1 \mergeop e_2              & \\
%
%        \\
%        \text{Contexts}
%        & \Gamma & \Coloneqq & \cdot
%                   \mid \Gamma, \alpha \highlight {$\disjoint A$}
%                   \mid \Gamma, x \oftype A  & \\
%
%        \text{Syntactic sugar} & \blam \alpha e & \equiv & \blamdis \alpha \bot e & \\
%                               & \forall \alpha.~A & \equiv & \forall (\alpha * \bot)~.~A & \\
%      \end{array}
%    \end{array}
%  \]
%
%  \caption{Amendments of the rules.}
%  \label{fig:fi-syntax-dis}
%\end{figure}
%
%\begin{figure}
%  \framebox{$\im A = T$}
%
%  \begin{align*}
%    \im \bot                  &= () \\
%    \im {\fordis \alpha A B}  &= \for \alpha \im B \\
%  \end{align*}
%
%  \framebox{$\im \Gamma = G$}
%
%  \begin{align*}
%    \im {\Gamma, \alpha \disjoint A} &= \im \Gamma, \alpha \\
%  \end{align*}
%
%  \caption{Additional type and context translation.}
%  \label{fig:additional-type-and-context-translation}
%\end{figure}
%
%% \george{May also note on the scoping of type variables inside contexts.}
%
%Figure~\ref{fig:fi-syntax-dis} shows the updated syntax with the
%changes highlighted and Figure~\ref{fig:additional-type-and-context-translation}
%shows type and context translations for the new constructs.
%Note how similar the changes are to those needed
%to extend System $F$ with bounded quantification. First, type
%variables are now always associated with their disjointness
%constraints (like $\alpha \disjoint A$) in types, terms, and
%contexts. Second, the bottom type ($\bot$) is introduced so that
%universal quantification becomes a special case of disjoint
%quantification: $\blam \alpha e$ is really a syntactic sugar for
%$\blamdis \alpha \bot e$. The underlying idea is that any type is
%disjoint with the bottom type.  Note the analogy with bounded
%quantification, where the top type is the trivial upper bound in
%bounded quantification, while the bottom type is the trivial
%disjointness constraint in disjoint quantification.

%Indeed, \bruno{unfinidhed sentence}\george{Mabe show a diagram here to contrast
%with bounded polymorphism.}


\subsection{Elaboration of type-system and coherence} 
In order to prove the coherence result, we refer to the previously introduced bidirectional 
type-system.
The bidirectional type-system is elaborating, producing a term in the target language while
performing the typing derivation.

%TODO move this to overview and show here example with polymorphism?
\paragraph{Key idea of the translation.}
This translation turns merges into usual pairs, similar to Dunfield's
elaboration approach~\cite{dunfield2014elaborating}.
For example, \[ 1 \mergeop \code{"one"} \] becomes \pair 1
{\code{"one"}}. In usage, the pair will be coerced according to type
information. For example, consider the function application: \[ \app {(\lamty x
\tystring x)} {(1 \mergeop \code{"one"})} \] This expression will be translated to \[ \app
{(\lamty x \tystring x)} {(\app {(\lamty x {\pair \tyint \tystring} {\proj 2 x})}
{\pair 1 {\code{"one"}}})} \] The coercion in this case is $(\lamty x {\pair
\tyint \tystring} {\proj 2 x})$. It extracts the second item from the pair, since
the function expects a $\tystring$ but the translated argument is of type $\pair
\tyint \tystring$.

\paragraph{The translation judgement.} The translation judgement $\jtype \Gamma e
A \yields E$ extends the typing judgement with an elaborated term on the right
hand side of $\yields {}$. The translation ensures that $E$ has type $\im A$. In
\name, one may pass more information to a function than what is required; but
not in System $F$. To account for this difference, in \reflabel{\labeltapp}, the
coercion $E$ from the subtyping relation is applied to the argument.
\reflabel{\labeltmerge} straighforwardly translates merges into pairs.

\paragraph{Type-safety}

The type-directed translation is type-safe. This property is captured
by the following two theorems.

\begin{restatable}[Type preservation]{theorem}{typepreservation}
  \label{theorem:type-preservation}
  We have that:
  \begin{itemize}
  \item If $ \jinfer \Gamma e A \yields E $, 
        then $ \jtype {\im \Gamma} E {\im A} \,$.
  \item If $ \jcheck \Gamma e A \yields E $, 
        then $ \jtype {\im \Gamma} E {\im A} $.
  \end{itemize}

\end{restatable}

\begin{proof}
  (Sketch) By structural induction on the term and the corresponding
  inference rule.
\end{proof}

\begin{theorem}[Type safety]
  If $e$ is a well-typed \name term, then $e$ evaluates to some System $F$
  value $v$.
\end{theorem}

\begin{proof}
  Since we define the dynamic semantics of \name in terms of the composition of
  the type-directed translation and the dynamic semantics of System $F$, type safety follows immediately.
\end{proof}

%From the theoretical point-of-view, the end goal of this section is to show that the resulting system has
%a coherent (or unique) elaboration semantics:
%\begin{restatable}[Unique elaboration]{theorem}{uniqueelaboration}
%  \label{theorem:unique-elaboration}
%
%  If $\jtype \Gamma e {A_1} \yields {E_1}$ and $\jtype \Gamma e {A_2} \yields
%  {E_2}$, then $E_1 \equiv E_2$. (``$\equiv$'' means syntactical equality, up to
%  $\alpha$-equality.)
%
%\end{restatable}
%
%\noindent In other words, given a source term $e$, elaboration always produces
%the same target term $E$. The most important hurdle we need to overcome is that
%if $A \inter B \subtype C$, then either $A$ or $B$ contributes to that subtyping
%relation, resulting in two possible coercions.



%Figure~\ref{fig:fi-type-patch} shows modifications to Figure~\ref{fig:fi-type} in
%order to support disjoint intersection types and disjoint
%quantification. Only new rules or rules that different are shown.
%Importantly, the disjointness judgement appears in the well-formedness rule for intersection
%types and the typing rule for merges.

%\begin{figure}
%  \begin{mathpar}
%    \framebox{$\jatomic A$} \\
%
%    \inferrule*
%    {}
%    {\jatomic \bot}
%
%    \inferrule*
%    {}
%    {\jatomic {A \to B}}
%
%    \inferrule*
%    {}
%    {\jatomic {\fordis \alpha B A}}
%  \end{mathpar}
%
%  \begin{mathpar}
%    \formsub \\ \rulesubinterldis \and \rulesubinterrdis \and \rulesubforalldis
%  \end{mathpar}
%
%  \begin{mathpar}
%    \formwf \\ \rulewfforalldis \and \rulewfinterdis
%  \end{mathpar}
%
%  \begin{mathpar}
%    \formt \\ \ruletblamdis \and \rulettappdis \and \ruletmergedis
%  \end{mathpar}
%
%  \caption{Affected rules.}
%  \label{fig:fi-type-patch}
%\end{figure}

%\begin{figure}
%  \begin{mathpar}
%    \framebox{$\jatomic A$} \\
%
%    \inferrule*{}{\jatomic {A \to B}}
%
%    \inferrule*{}{\jatomic \alpha}
%
%    \inferrule*{}{\jatomic {\for \alpha A}}
%  \end{mathpar}
%
%  \begin{mathpar}
%    \formsub \\ \rulesubint \and \rulesubvar \and \rulesubfun \and \rulesubforall
%    \and \rulesubinter  \and \rulesubinterldis \and \rulesubinterrdis 
%  \end{mathpar}
%
%  \begin{mathpar}
%    \formwf \\ \rulewfint \and \rulewfvar \and \rulewffun \and \rulewfforall \and \rulewfinter
%  \end{mathpar}
%
%%  \begin{mathpar}
%%    \formt \\ \ruletvar \and \ruletlam \and \ruletapp \and
%%    \ruletblam \and \rulettapp \and \ruletmergedis 
%%  \end{mathpar}
%
%  \begin{mathpar}
%    \formbi \\ \bruletint \and \bruletvar \and \bruletapp \and
%    \brulettapp \and \bruletmergedis \and \bruletann 
%  \end{mathpar}
%
%  \begin{mathpar}
%    \formbc \\ \bruletlam \and  \bruletblam \and \bruletsub
%  \end{mathpar}
%
%  \caption{Rules for Naive system.}
%  \label{fig:fi-type-naive}
%\end{figure}

%\begin{figure}
%  \begin{mathpar}
%    \framebox{$\jatomic A$} \\
%
%    \inferrule*{}{\jatomic {\fordis \alpha B A}}
%  \end{mathpar}
%
%  \begin{mathpar}
%    \formsub \\ \rulesubtop \and \rulesubforalldis \and 
%    \rulesubinterlcoerce \and \rulesubinterrcoerce
%  \end{mathpar}
%
%  \begin{mathpar}
%    \formwf \\ \rulewftop \and \rulewfvardis \and \rulewfforalldis 
%  \end{mathpar}
%
%  \begin{mathpar}
%    \formbi \\ \brulettop \and \brulettappdis 
%  \end{mathpar}
%
%  \begin{mathpar}
%    \formbc \\ \bruletblamdis 
%  \end{mathpar}
%
%
%  \caption{Changes for Extended systems.}
%  \label{fig:fi-type-extended}
%\end{figure}

%\begin{figure}
%  \begin{mathpar}
%    \formsub \\ \rulesubforallext
%  \end{mathpar}
%
%  \caption{Rules for Extended system with improved ForAll.}
%  \label{fig:fi-type-extended_forall}
%\end{figure}

%\begin{figure}[t]
%  \begin{mathpar}
%    \formtoplike \\ %\framebox{$\jatomic A$} \\
%
%    \ruletopltop \and \ruletoplinter \and \ruletoplfun \and \ruletoplforall
%  \end{mathpar}
%
%
%  \begin{center}
%    \framebox{$\andcoerce{A}_{C} = T$}
%  \end{center}
%
%  \[
%  \andcoerce{A}_{C} = 
%  \begin{cases} 
%        \toplike{A} & \andcoerce{A} \\ 
%        %A = \top & () \\
%        %A = A_1 \to A_2 \; \wedge \; \toplike{A_2} & \lam x \andcoerce{A_2}_{C} \\
%        \text{otherwise} & {C} 
%  \end{cases}
%  \]
%
%  \begin{center}
%    \framebox{$\andcoerce{A} = T$}
%  \end{center}
%
%  \[
%  \andcoerce{A} = 
%  \begin{cases} 
%        A = \top & () \\
%        A = A_1 \to A_2 & \lam x \andcoerce{A_2} \\
%        A = A_1 \inter A_2 & \pair {\andcoerce{A_1}} {\andcoerce{A_2}} \\
%        A = \fordis \alpha B A & \blam \alpha {\andcoerce{A}}
%  \end{cases}
%  \]
%  \caption{Coercion generation considering Top-like types.}
%  \label{fig:andcoercion}
%\end{figure}



%\paragraph{Atomic Types.} The new system introduces atomic types. Essentially a type
%is atomic if it is any type, which is not an intersection type.
%The notion of atomic
%type will be helpful

\paragraph{Coherency of Elaboration}
Combining the previous results, we are finally able to show the central theorem:

\begin{restatable}[Unique elaboration]{theorem}{uniqueelaboration}
  \label{theorem:unique-elaboration}
  We have that:
  \begin{itemize*}
    \item If $\jinfer \Gamma e {A_1} \yields {E_1}$ and $\jinfer \Gamma e {A_2} \yields
          {E_2}$, then $E_1 \equiv E_2$. 
    \item If $\jcheck \Gamma e {A_1} \yields {E_1}$ and $\jcheck \Gamma e {A_2} \yields
          {E_2}$, then $E_1 \equiv E_2$.
  \end{itemize*}(``$\equiv$'' means syntactical equality, up to
  $\alpha$-equality.)
  

\end{restatable}

\joao{review this proof, i.e. should we also mention (and show) uniqueness of type inference?}
\begin{proof}
  Note that the typing rules are already syntax-directed but the case of
  \reflabel{\labeltapp} (copied below) still needs special attention since we
  need to show that the generated coercion $E$ is unique.
  \begin{mathpar}
    \ruletapp
  \end{mathpar}
  Luckily, by Lemma~\ref{lemma:wellformed-typing}, we know that typing
  judgements give well-formed types, and thus $\jwf \Gamma {A_1}$ and $\jwf
  \Gamma {A_3}$. Therefore we are able to apply
  Lemma~\ref{lemma:unique-coercion} and conclude that $E$ is unique.

\end{proof}

\section{Disjointness} \label{sec:alg-dis}
Section~\ref{sec:fi} presented a type system with disjoint
intersection types and disjoint quantification. In order to prove 
both type-safety and coherence (in Section~\ref{sec:disjoint}), it is necessary to first introduce a
notion of disjointness, considering polymorphism and disjointness quantification.
%In constrast with \oldname, we no longer use a specification for disjointness, as
%that would require variable unification.
This section presents an algorithmic set of rules for determining whether two types are disjoint. 
After, it will show a few important properties regarding substitution, which will turn out
to be crucial to ensure both type-safety and coherence.
Finally, it will discuss the bounds of disjoint quantification and what implications
they have on \name. %with a special focus on the $\top$ type. 

\begin{comment}
\subsection{Derivation of the Algorithmic Rules}

In this subsection, we illustrate how to derive the algorithmic disjointness
rules by showing a detailed example for functions. For the ease of discussion,
first we introduce some notation.

\begin{definition}[Set of common supertypes]
  For any two types $A$ and $B$, we can denote the \emph{set of their common
  supertypes} by \[ \commonsuper(A,B) \] In other words, a type $C \in \;
  \commonsuper(A,B)$ if and only  if $A \subtype C$ and $B \subtype C$.
\end{definition}

\begin{example}
  $\commonsuper(\tyint,\tychar)$ is empty, since $\tyint$ and $\tychar$
  share no common supertype.
\end{example}

Parallel to the notion of the set of common supertypes is the notion of the set
of common subtypes.

\begin{definition}[Set of common subtypes]
  For any two types $A$ and $B$, we can denote the \emph{set of their common
  subtypes} by \[ \commonsub(A,B) \] In other words, a type $C \in \; \commonsub(A,B)$
  if and only  if $C \subtype A$ and $C \subtype B$.
\end{definition}

\begin{example}
  $\commonsub(\tyint,\tychar)$ is an infinite set which contains $\tyint \inter
  \tychar$, $\tychar \inter \tyint$, $(\tyint \inter \tybool) \inter \tychar$
  and so on. But the type $\tybool$ is not inside, since it is not a subtype of
  $\tyint$ or $\tychar$.
\end{example}


\paragraph{Shorthand notation.} For brevity, we will use \[ \mathcal{A} \to
\mathcal{B} \] as a shorthand for the \emph{set} of types of the form $A \to B$,
where $A \in \mathcal{A}$ and $B \in \mathcal{B}$. The same shorthand applies to
all other constructors of types, in addition to functions, as well. As a simple
example,  \[ \{ \tyint, \tystring \} \to \{ \tyint, \tychar \} \] is a shorthand for \[ \{
\tyint \to \tyint, \tyint \to \tychar, \tystring \to \tyint, \tystring \to \tychar \} \]


Recall that we say two types $A$ and $B$ are disjoint if they do not share a
common supertype. Therefore, determining if two types $A$ and $B$ are disjoint
is the same as determining if $\commonsuper(A,B)$ is empty.

\paragraph{Determining disjointness of functions.} Now let's dive into the case
where both $A$ and $B$ are functions and consider how to compute
$\commonsuper(A_1 \to A_2, B_1 \to B_2)$. By the subtyping rules, the supertype
of a function must also be a function.\george{Nah... only after normalization.
If not, it can also be $\inter$} Let $C_1 \to C_2$ be a common supertype
of $A_1 \to A_2$ and $B_1 \to B_2$. Then it must satisfy the following:
\begin{mathpar}
  \inferrule
    {C_1 \subtype A_1 \\ A_2 \subtype C_2}
    {A_1 \to A_2 \subtype C_1 \to C_2}

  \inferrule
    {C_1 \subtype B_1 \\ B_2 \subtype C_2}
    {B_1 \to B_2 \subtype C_1 \to C_2}
\end{mathpar}
From which we see that $C_1$ is a common subtype of $A_1$ and $B_1$ and that
$C_2$ is a common supertype of $A_2$ and $B_2$. Therefore, we can write:
\[ \commonsuper(A_1 \to A_2, B_1 \to B_2) \ = \ \commonsub(A_1,B_1) \to \commonsuper(A_2,B_2) \]
By definition, $\commonsub(A_1,B_1) \to \commonsuper(A_2,B_2)$ is not empty if and only if both
$\commonsub(A_1,B_1)$ and $\commonsuper(A_2,B_2)$ is nonempty. However, note
that with intersection types, $\commonsub(A_1,B_1)$ is always nonempty because
$A_1 \inter B_1$ belongs to $\commonsub(A_1,B_1)$. Therefore, the problem of
determining if $\commonsuper(A_1 \to A_2, B_1 \to B_2)$ is empty reduces to the
problem of determining if $\commonsuper(B_1 \to B_2)$ is empty, which is, by
definition, if $B_1$ and $B_2$ are disjoint. Finally, we have derived a rule for
functions:
\begin{mathpar}
  \ruledisfun
\end{mathpar}

The analysis needed for determining if types with other constructors are
disjoint is similar. Below are the major results of the recursive definitions for
$\commonsuper$:
\begin{align*}
  \commonsuper(A_1 \to A_2, B_1 \to B_2) &= \commonsub(A_1,B_1) \to \commonsuper(A_2,B_2) \\
  \commonsuper({A_1 \inter A_2, B})      &= \commonsuper(A_1, B) \cup \commonsuper(A_1,B) \\
  \commonsuper({A, B_1 \inter B_2})      &= \commonsuper(A, B_1) \cup \commonsuper(A,B_2)
\end{align*}\george{Missing the forall case. But we're just going to
  drop the formulae.}
\end{comment}

\subsection{Algorithmic Rules for Disjointness}

\begin{figure}[!t]
  \begin{spacing}{0.5}
  \begin{mathpar}
    \formdis \\
    \ruledistop \and \ruledistopsym \and 
    \ruledisvar \and \ruledissym \and \ruledisforallext \and 
    \ruledisreceq \and \ruledisrecneq \and
    \ruledisfun \and \ruledisinterl \and \ruledisinterr \and 
    \ruledisatomic
  \end{mathpar}
  \begin{mathpar}
    \formax \\
    \ruleaxsym \and \ruleaxintfun \and \ruleaxintrec \and \ruleaxintforalldis \and  
    \ruleaxfunforalldis \and \ruleaxfunrec \and \ruleaxforalldisrec
  \end{mathpar}
  \end{spacing}
  \caption{Algorithmic disjointness.}
  \label{fig:disjointness}
\end{figure}

The rules for the disjointness judgement are shown in
Figure~\ref{fig:disjointness}, which consists of two judgements.
 
\paragraph{Main judgement.} The judgement $\jdis \Gamma A B$ says
two types $A$ and $B$ are disjoint in a context $\Gamma$.
The rules are inspired in the disjointness algorithm described by \oldname.
\reflabel{\labeldistop} and \reflabel{\labeldistopsym} say that any type is disjoint to 
$\top$.
This is a major difference to \oldname, where the notion of disjointness explicitly forbids
the presence of $\top$ types in intersections.
%It turns out that even though $\top$ overlaps with every other type,
%it does not affect coherence in any way.
We will further discuss this difference in Section~\ref{sec:disjoint}.
 
Type variables are dealt with two rules:
\reflabel{\labeldisvar} is the base rule; and \reflabel{\labeldissym}
is its twin symmetrical rule. 
Both rules state that a type variable is disjoint to some type $B$, if $\Gamma$ contains any
subtype of the corresponding disjointness constraint. 
This rule is a specialization of the more general lemma:
\defaultthmprf{
\begin{restatable}[Covariance of disjointness]{lemma}{disjointnesscovariance}
  \label{lemma:disjointness-covariance}
  If $\jwf \Gamma {A \disjoint B}$ and $B \subtype C$, then $\jwf \Gamma {A \disjoint C}$.
\end{restatable}}
{By double induction, first on the disjointness derivation and then on the subtyping derivation.
The first induction case for~\reflabel{\labeldisvar} does not need the second induction as it is 
a straightforward application of subtyping transitivity.}
%\begin{restatable}[Covariance of disjointness]{lemma}{disjointnesscovariance}
%  \label{lemma:disjointness-covariance}
%
%  If $\jwf \Gamma {A \disjoint B}$ and $B \subtype C$, then $\jwf \Gamma {A \disjoint C}$.
%\end{restatable}
%\begin{proof}
%By double induction, first on the disjointness derivation and then on the subtyping derivation.
%The first induction case for~\reflabel{\labeldisvar} does not need the second induction as it is 
%a straightforward application of subtyping transitivity. 
%\end{proof}

The lemma states that if a type $A$ is disjoint to $B$ under $\Gamma$, then it is also disjoint
to any supertype of $B$. 
Note how these two variable rules would allow one to prove $\alpha \disjoint \alpha$, for any 
variable $\alpha$.
However, under the assumption that contexts are well-formed, such derivation is not possible 
as $\alpha$ cannot occur free in $A$. 

The rule for disjoint quantification \reflabel{\labeldisforall} is the most interesting. 
To illustrate this rule, consider the following two types:

\[ (\forall (\alpha * \tyint).~\tyint \& \alpha) \qquad 
(\forall (\alpha * \tychar). ~\tychar \& \alpha) \]
When are these two types disjoint?
In the first type $\alpha$ cannot be instantiated with $\tyint$ and in
the second case $\alpha$ cannot be instantiated with $\tychar$.
Therefore for both bodies to be disjoint, $\alpha$ cannot be instantiated with either $\tyint$ 
or $\tychar$. 
The rule for disjoint quantification adds a constraint composed of the intersection of both constraints into $\Gamma$ and checks for 
disjointness in the bodies under that environment.
The reader might notice how this intersection does not necessarily need to be well-formed,
in the sense that the types that compose it might not be disjoint.
This is not problematic because the intersections present as constraints in the environment
do not contribute directly to the (coherent) coercions generated by the type-system.
In other words, intersections play two different roles in \name, as:
\begin{enumerate}
\item \textbf{Types}: restricted (i.e. disjoint) intersections are required to ensure coherence.
\item \textbf{Constraints}: arbitrary intersections are sufficient to serve as constraints under 
polymorphic instantiation. 
\end{enumerate}

The rules \reflabel{\labeldisreceq} and \reflabel{\labeldisrecneq} define disjointness between
two single label records.
If the labels coincide, then the records are disjoint whenever their fields' types are also disjoint;
otherwise they are always disjoint.
Finally, the remaining rules are identical to the original rules. 
%The rule for functions \reflabel{\labeldisfun} says that two function
%types are disjoint if and only if their return types are disjoint. 
%The rules dealing with intersection types (\reflabel{\labeldisinterl}
%and \reflabel{\labeldisinterr}) say that an intersection is disjoint to some type $B$, whenever
%both of their components are also disjoint to $B$.
%We emphasize that the rule \reflabel{\labeldisax} says two different type constructs are disjoint 
%if the axiom rules (explained below) apply. 

\paragraph{Axioms.} Axiom rules take care of two types with different language constructs.
These rules capture the set of rules is that $A \disjointax B$ holds for all 
two types of different constructs unless any of them is an intersection type, a type variable,
or $\top$.
Note that disjointness with type variables is already captured by \reflabel{\labeldisvar} and 
\reflabel{\labeldissym}, and disjointness with the $\top$ type is captured by 
{\reflabel{\labeldistop}} and {\reflabel{\labeldistopsym}}.

%\subsection{Stability under Substitution}
%The combination of polymorphism and disjoint intersection types
%invalidates various conventional substitution lemmas related to
%well-formedness and typing.  
%For example, as shown in Section~\ref{sec:overview}, in the type 
%$\fordis A \tyint {(\tyint \inter A) \to \tyint}$, the type $A$ cannot be substituted by any type.
%However, under certain conditions, weaker versions of substitution lemmas do hold. 
%The conditions are guaranteed by the type-system by only
%allowing instantiation of a type variable with types disjoint to the
%variable's disjointness constraints.
%
%\paragraph{Problematic substitutions.}
%One rule of thumb in disjoint intersection types is that, if a type
%$A$ is disjoint to a type $B$, then the intersection $A \inter B$ is
%well-typed.  However, during type instantiation (i.e. when type
%substitution should be stable), both types $A$ and $B$ can change.  It
%should follow naturally that this instantiation will not produce an
%ill-formed type $A \inter B$, or, more generally, disjointness should
%be stable under substitution.  
%Let us illustrate this with an example.
%Consider the following judgement, where in the context $\alpha
%\disjoint \tyint$, $\alpha$ and $\tyint$ are disjoint:
%\[ \jdis {\alpha \disjoint \tyint} \alpha \tyint \]
%After the substitution of $\tyint$ for $\alpha$ on the two types, the sentence
%\[ \jdis {\alpha \disjoint \tyint} \tyint \tyint \]
%is no longer true since $\tyint$ is clearly not disjoint with itself.
%Generally speaking, a careless substitution can violate the disjoint constraint in the context.
%This explains the need to ensure that during type-instantiation the target of the substitution  
%is compatible with such disjointness constraint. 

\subsection{Well-formed Types}
In \name it is important to show that the type-system only produces well-formed types.
This is crucial to ensure coherence, as shown in Section~\ref{sec:disjoint}.
However, in the presence of both polymorphism and disjoint intersection types, extra
effort is needed to show that all types in \name are well-formed.
To achieve this, not only we need to show that a weaker version of the general substitution lemma holds, 
but also that disjointness between two types is preserved after substitution.
To motivate the former (i.e. why general substitution does not hold in \name), 
consider the type $\fordis \alpha \tyint {(\alpha \inter \tyint)}$. 
The type variable $\alpha$ cannot be substituted by any type: substituting with $\tyint$ will lead to the
ill-formed type $\tyint \inter \tyint$.
To motivate the latter, consider the judgement $ \jdis {\alpha \disjoint \tyint} \alpha \tyint$.
After the substitution of $\tyint$ for $\alpha$ on the two types, the sentence
$\jdis {\alpha \disjoint \tyint} \tyint \tyint$ is no longer true, since $\tyint$ is
clearly not disjoint with itself.
Generally speaking, a careless substitution can violate the constraints in the context.
%\paragraph{Disjoint substitutions.}
%While disjointness cannot be preserved for general substitutions,
%if appropriate disjointness pre-conditions are met then disjointness can
%be preserved. 
However, if appropriate disjointness pre-conditions are met, then disjointness can
be preserved.
More formally, the following lemma holds: 

\defaultthmprf{
\begin{lemma}[Disjointness is stable under substitution]
  \label{lemma:orthosubst}
  If $(\alpha \disjoint D) \in \Gamma$ and $\jdis \Gamma C D$ and $\jdis \Gamma A B$ and well-formed
  context $\subst C \alpha \Gamma$, 
  then $\jdis {\subst C \alpha \Gamma} {\subst C \alpha A} {\subst C \alpha B}$.
\end{lemma}}
{By induction on the disjointness derivation of $C$ and $D$.
  Special attention is needed for the variable case, where it is necessary to prove stability
  of substitution for the subtyping relation.
  It is also needed to show that, if $C$ and $D$ do not contain any variable $x$, then it is
  safe to make a substitution in the co-domain of the environment.}

%\paragraph{Well-formedness substitution stability.}
%Typically polymorphic systems with explicit instantiation are required to show that their
%types are stable under substitution, in order to avoid ill-formed types.
%In the presence of disjoint quantification, we cannot prove such property.
%However, a weaker version of that property -- but strong enough for our type-system's metatheory 
%-- can be proven, namely:

We can now prove a weaker version of the general substitution lemma:

\defaultthmprf{
\begin{lemma}[Types are stable under substitution]
  \label{lemma:wfsubst}
  If $\jwf \Gamma A$ and $\jwf \Gamma B$ and $(\alpha \disjoint C) \in \Gamma$ 
  and $\jdis \Gamma B C$ and well-formed context $\subst B \alpha \Gamma$, 
  then $\jwf {\subst B \alpha \Gamma} {\subst B \alpha A}$.
\end{lemma}}
%\begin{restatable}[Instantiation]{lemma}{instantiation}
%  \label{lemma:instantiation}
%
%  If $\jwf {\Gamma, \alpha \disjoint B} C$, $\jwf \Gamma A$, $\jdis \Gamma A B$
%  then $\jwf \Gamma {\subst A \alpha C}$.
%\end{restatable}
{By induction on the well-formedness derivation of $A$.
The intersection case requires the use of Lemma~\ref{lemma:orthosubst}.
Also, the variable case required proving that if $\alpha$ does not occur free in $A$, and it is safe
to substitute it in the co-domain of $\Gamma$, then it is safe to perform the substitution.}

Now we can finally show that all types produced by the type-system are well-formed and,
more specifically, justify that the disjointness premise on \reflabel{\labelttapp} is sufficient for that 
purpose.
More formally, we have that:

\defaultthmprf{
\begin{restatable}[Well-formed typing]{lemma}{wellformedtyping}
  \label{lemma:wellformed-typing} 
  We have that:
  \begin{itemize}
  \item If $\jcheck \Gamma e A$, then $\jwf \Gamma A$. 
  \item If $\jinfer \Gamma e A$, then $\jwf \Gamma A$.
  \end{itemize}
\end{restatable}}
{By induction on the derivation and applying
  Lemma~\ref{lemma:wfsubst} in the case of \reflabel{\labelttapp}.}

Even though the meta-theory is consistent, we can still ask: 
what are the bounds of disjoint quantification?
In other words, which type(s) can be used to allow unrestricted instantiation, and which
one(s) will completely restrict instantiation?
As the reader might expect, the answer is tightly related to subtyping. 

\subsection{Bounds of Disjoint Quantification}
Substitution raises the question of what range of types can be instantiated for a given variable
$\alpha$, under a given context $\Gamma$.
%To get a feeling about this, we ask the reader to recall 
%Lemma~\ref{lemma:disjointness-covariance}, used to justify the rule for
%disjointness of variables.
%If one takes $A$ as some variable $\alpha$, then the lemma should read as:   
%\[ \inferrule {\jwf \Gamma {\alpha \disjoint B} \\ B \subtype C }
%              {\jwf \Gamma {\alpha \disjoint C}} \]
To answer this question, we ask the reader to recall the rule \reflabel{\labeldisvar}, copied below:
\[ \ruledisvar \]
Given that the cardinality of \name's types is infinite, for the sake of this example we will 
restrict the type universe to a finite number of 
primitive types (i.e. $\tyint$ and $\tystring$), disjoint intersections of these types,
$\top$ and $\bot$.
Now we may ask: how many suitable types are there to instantiate $\alpha$ with, depending on $A$?
The rule above tells us that the more super-types $A$ has, the more types $\alpha$ has to be disjoint
with.
In other words, the choices for instantiating $\alpha$ are inversely proportional to the number
of super-types of $A$.
It is easy to see that the number of super-types of $A$ is directly proportional to the number of
intersections in $A$.
For example, taking $A$ as $\tyint$ leads $B$ to be either $\top$ or $\tyint$; whereas $A$ as 
$\tyint \inter \tystring$ leaves $B$ as either $\top$, $\tyint$ or $\tystring$.
Thus, the choices of $\alpha$ are inversely proportional to the number of intersections in $A$.
%Now we may ask: how many suitable types are there to instantiate $\alpha$ with, depending on $B$?
%Before we answer this, let us ask first how many options are there for $C$, depending on the 
%shape of $B$?
%Given that the cardinality of \name's types is infinite, for the sake of this example we will 
%restrict the type universe to a finite number of 
%primitive types (i.e. $\tyint$, $\tystring$, etc), disjoint intersections of these types,
%$\top$ and $\bot$.
%Having this in mind, we can answer the second question: the number of choices for $C$ is directly 
%proportional to the number of intersections present in $B$.
%For example, taking $B$ as $\tyint$ leads $C$ to be either $\top$ or $\tyint$; whereas $B$ as 
%$\tyint \inter \tystring$ leaves $C$ as either $\top$, $\tyint$ or $\tystring$.
%However, as the choices for $C$ grows, the less choices we are left to instantiate 
%the variable $\alpha$, since $\alpha$ must be disjoint to all possible $C$'s. 
%Thus, to answer the first question, the options for instantiating $\alpha$ are inversely proportional 
%to the number of intersections present in $B$.
By analogy, one may think of a disjointness constraint as a set of (forbidden) types, where 
primitive types are the singleton set and each $\inter$ is the set union.
%We can now turn our attention to the two extreme cases, namely $\top$ 
%(i.e. the 0-ary intersection) and $\bot$ (i.e. the infinite intersection) 
%\footnote{$\bot$ would not add anything to the hypothetical finite type 
%system, however it can be seen as the infinite intersection in \name.}.
Following the same logic, choosing $\top$ (i.e. the 0-ary intersection)
as constraint leaves $\alpha$ with the most options for instantiation; 
whereas $\bot$ (i.e. the infinite intersection) will deliver the least options.
Consequently, we may conclude that $\top$ is the empty constraint: 
a variable associated to it can be instantiated to any well-formed type.
It is a subtle but very important property, since \name is a generalization of System $F$. 
Any type variable quantified in System $F$, can be quantified equivalently in \name
by assigning it a $\top$ disjointness constraint
(as seen in Section~\ref{subsec:disjoint-quantification}).

\section{Disjoint Intersection Types with $\top$}

Discuss the two variants of \name with $\top$. 

\begin{figure}[t]
  \[
    \begin{array}{l}
      \begin{array}{llrll}
        \text{Types}
        & A, B, C & \Coloneqq & \ldots \mid \highlight{$\top$}  & \\

        \\
        \text{Terms}
        & e & \Coloneqq & \ldots \mid \highlight{$\top$} & \\
      \end{array}
    \end{array}
  \]

  \begin{mathpar}
    \formsub \\
    \rulesubtop 
  \end{mathpar}

  \begin{mathpar}
    \formwf \\
    \rulewftop
  \end{mathpar}

  \begin{mathpar}
    \formt \\
    \brulettop
  \end{mathpar}

  \caption{Extending \name with $\top$.}
  \label{fig:fi-syntax-top}
\end{figure}\bruno{rule form needs to be fixed in typing.}

\subsection{Disjointness} Show the new definition of disjointness

\subsection{A Naive Calculus with $\top$}

\paragraph{Top-Like Types}
Before we discuss that definition, let us introduce first the notion of a top-like type 

%Here such definition is made a bit more precise, and
%well-suited to \name.
\begin{figure}[t]
  \begin{mathpar}
    \formtoplike \\ %\framebox{$\jatomic A$} \\

    \ruletopltop \and \ruletoplfun \and \ruletoplinterl \and \ruletoplinterr

  \end{mathpar}
  \caption{Top-like types.}
  \label{fig:fi-toplike}
\end{figure}

\begin{definition}[Top-like types]
  
  One type $A$ is a top-like type, denoted as $\toplike{A}$, if it has the form $A_k \to \top$, where $k \in {0,1,..}$.
  That is, any type with arity $k$ can be a top-like type, as long as $\top$ is the result type. 

\end{definition}
Thus a top-like type is a unary relation at the type level, which can be formalized according to \ref{fig:fi-toplike}.

Also, $\code{Int} \to \code{Char}$ and $\code{Int} \to \code{String}$ are disjoint, 
since their supertypes are all types with the form $A \to \top$ 
(where A \emph{contains} \code{Int}, i.e. $\code{Int}$, $\code{Int} \inter \code{Char}$) and $\top$.


At last, take as an example ($\top \inter \top$).
In a pure language (i.e. no side-effects), this type can be safely allowed since both components of the merge
will evaluate to the same value. 
However, in an effectful language, the evaluation of either the right or left component might lead to distinct results. 
Following the same reasoning, we might want to allow or restrict, for instance $String \to Int \inter String \to \top$,  

\paragraph{Algorithmic disjointness rules}

\subsection{An Improved Calculus with $\top$}

motivation ...

\paragraph{Top-Like Types}
\joao{do we want to allow this?
 f: String $\to$ Int \\ 
 g: String $\to$ T \\
 ($\lam$ h : String $\to$ T. h "") (f ,, g) }

\paragraph{Coercive Subtyping} Discuss the changes in 
coercive subtyping rules. Namely the more flexible way to 
generate coercions depending on whether a type is top-like 
or not. 

\paragraph{Algorithmic disjointness rules}
%\section{Design Space}

This section discusses some alternatives in the design-space.

\subsection{Disjointness of Functions}

Talk about the option of not allowing subtyping of function arguments. 
This should allow for a more flexible rule for disjointness of functions.
Maybe a good option for OO type systems, where methods are invariant 
with respect to subtyping of arguments. It would allow for static overloading, 
similar to what is present in conventional OO languages.

\begin{mathpar}
  \inferrule* [right=\labelsubfun]
    {{A_2} \subtype {B_2} }
    {{A_1 \to A_2} \subtype {A_1 \to B_2}}
\end{mathpar}

\begin{mathpar}
  \inferrule* [right=\labeldisfun]
    {\jdisimpl \Gamma {A_1} {B_1}}
    {\jdisimpl \Gamma {A_1 \to A_2} {B_1 \to B_2}}
\end{mathpar}

not exists E . A -> B <: E /\ C -> D <: E ->  

$Int -> Int \& Char -> Int$ (disjoint according to the spec and algorithmic rules)



Are those 2 functions disjoint? 

$ f,,g : Int -> Int \& Char -> Int$

Well, two things to consider:

1) what happens if they are applied:  

$(f,,g) (3,'c')$ 

well, a type-annotation will then select one of the functions. So this seems to be ok.

2) what happens if the functions are selected. I have two choices:

f,,g : Int -> Int

f,,g : Char -> Int 

$f,,g : Char\&Int -> Int$ (fails because subtyping of functions is invariant).



\subsection{Union Types}

\begin{lstlisting}
case 3,,'c' of
   Int -> 1
   Char -> 2 : Int
\end{lstlisting}

Here we have $Int \& Char <: Int | Char$, but this leads to ambiguity. The program can either 
be $1$ or $2$. 

Possible solution: require atomic constraints in or-rules, similar to the and-rules. 
Big Problem: subtyping is no longer transitive. Minor problem, type-system is incomplete.

\subsection{Parametric Polymorphism?}

In principle it should be easy to extend disjointness to parametric polymorphism. 
bruno{Show rules for parametric polymorphism}

However, such rules would be quite restrictive. Future work includes how to integrate 
parametric polymorphism is a more flexible way.  

\section{Related Work}
\label{sec:related-work}

%*******************************************************************************
\paragraph{Coherence}
%*******************************************************************************

Reynolds invented Forsythe~\cite{reynolds1997design} in the 1980s. Our
merge operator is analogous to his operator $p_1, p_2 $. Forsythe has
a coherent semantics. The result was proved formally by
Reynolds~\cite{reynolds1991coherence} in a lambda calculus with
intersection types and a merge operator. However there are two key
differences to our work. Firstly the way coherence is ensured is
rather ad-hoc. He has four different typing rules for the merge
operator, each accounting for various possibilities of what the types
of the first and second components are. In some cases the meaning of
the second component takes precedence (that is, is biased) over the
first component. The set of rules is restrictive and it forbids, for
instance, the merge of two functions (even when they a provably
disjoint). In contrast, disjointness in \name has a well-defined
specification and it is quite flexible. Secondly, Reynolds calculus
does not support universal quantification. It is unclear to us whether
his set of rules would still ensure disjointness in the presence of
universal quantification. Since some biased choice is allowed in
Reynold's calculus the issues illustrated in Section~\ref{subsec:polymorphism} could be a problem.

Pierce~\cite{pierce1991programming2} made a comprehensive review
of coherence, especially on Curien and Ghelli~\cite{curienl1990coherence} and
Reynolds' methods of proving coherence; but he was not able to prove coherence
for his $F_\wedge$ calculus. He introduced a primitive $\code{glue}$ function as
a language extension which corresponds to our merge operator. However, in his
system users can ``glue'' two arbitrary values, which can lead to incoherence.

Our work is largely inspired by Dunfield~\cite{dunfield2014elaborating}. He
described a similar approach to ours: compiling a system with intersection types
and a merge operator into ordinary $ \lambda $-calculus terms with pairs. One
major difference is that our system does not include unions. However, as
acknowledged by Dunfield, his calculus lacks of coherence. He discusses the
issue of coherence throughout his paper, mentioning biased choice as an option
(albeit a rather unsatisfying one). He also mentioned that the notion of
disjoint intersection could be a good way to address the problem, but he did not
pursue this option in his work. In contrast to his work, we developed a type
system with disjoint intersection types and proposed disjoint quantification to
guarantee coherence in our calculus.

% \url{http://homepages.inf.ed.ac.uk/gdp/publications/Sub_Par.pdf}

% \cite{plotkin1994subtyping}

% Also discussed intersection types!~\cite{malayeri2008integrating}.

% Pierce Ph.D thesis: F<: + /|
%        technical report: F + /|, closer to ours

% \cite{barbanera1995intersection}
%
% \paragraph{Intersection types with polymorphism.}
% Our type system combines intersection types and parametric polymorphism. Closest
% to us is Pierce's work~\cite{pierce1991programming1} on a prototype
% compiler for a language with both intersection types, union types, and
% parametric polymorphism. Similarly to \name in his system universal
% quantifiers do not support bounded quantification. However Pierce did not try to prove any
% meta-theoretical results and his calculus does not have a merge
% operator.  Pierce also studied a system where both intersection
% types and bounded polymorphism are present in his Ph.D.
% dissertation~\cite{pierce1991programming2} and a 1997
% report~\cite{pierce1997intersection}.

Going in the direction of higher
kinds, Compagnoni and Pierce~\cite{compagnoni1996higher} added
intersection types to System $ F_{\omega} $ and used the new calculus,
$ F^{\omega}_{\wedge} $, to model multiple inheritance. In their
system, types include the construct of intersection of types of the
same kind $ K $. Davies and Pfenning
\cite{davies2000intersection} studied the interactions between
intersection types and effects in call-by-value languages. And they
proposed a ``value restriction'' for intersection types, similar to
value restriction on parametric polymorphism. Although they proposed a system with
parametric polymorphism, their subtyping rules are significantly different from ours,
since they consider parametric polymorphism
as the ``infinit analog'' of intersection polymorphism.

Recently,
Castagna et al.~\cite{Castagna:2014} studied an very expressive calculus that
has polymorphism and set-theoretic type connectives (such as intersections,
unions, and negations). As a result, in their calculus one is also able to
express a type variable that can be instantiated to any type other than
$\code{Int}$ as $\alpha \setminus \code{Int}$, which is syntactic sugar for
$\alpha \wedge \neg \code{Int}$. As a comparison, such a type will need a
disjoint quantifier, like $\fordis \alpha {\code{Int}} \alpha$, in our system.
Unfortunately their calculus does not include a merge operator like ours.

There have been attempts to provide a foundational calculus
for Scala that incorporates intersection
types~\cite{amin2014foundations,amin2012dependent}.
Although the minimal Scala-like calculus does not natively support
parametric polymorphism, it is possible to encode parametric
polymorphism with abstract type members. Thus it can be argued that
this calculus also supports intersection types and parametric
polymorphism. However, the type-soundness of a minimal Scala-like
calculus with intersection types and parametric polymorphism is not
yet proven. Recently, some form of intersection
types has been adopted in object-oriented languages such as Scala,
Ceylon, and Grace. Generally speaking,
the most significant difference to \name is that in all previous systems
there is no explicit introduction construct like our merge operator. As shown in
Section~\ref{subsec:OAs}, this feature is pivotal in supporting modularity
and extensibility because it allows dynamic composition of values.

\begin{comment}
only allow intersections of concrete types (classes),
whereas our language allows intersections of type variables, such as
\texttt{A \& B}. Without that vehicle, we would not be able to define
the generic \texttt{merge} function (below) for all interpretations of
a given algebra, and would incur boilerplate code:

\begin{lstlisting}
let merge [A, B] (f: ExpAlg A) (g: ExpAlg B) = {
  lit (x : Int) = f.lit x ,, g.lit x,
  add (x : A & B) (y : A & B) =
    f.add x y ,, g.add x y
}
\end{lstlisting}
\end{comment}

%*******************************************************************************
\paragraph{Other type systems with intersection types.}
%*******************************************************************************

% Although similar in spirit,
% our elaboration typing is simpler: we require subtyping in the case of
% applications, thus avoiding the subsumption rule. Besides, our treatment
% combines the merge rules ($ k $ ranges over $ \{1, 2\} $)
% \inferrule
% {\Gamma \turns e_k : A}
% {\Gamma \turns e_1 \mergeop e_2 : A}
% and the standard intersection-introduction rule
% \inferrule
% {\Gamma \turns e : A_1 \andalso \Gamma \turns e : A_2}
% {\Gamma \turns e : A_1 \inter A_2}
% into one rule:
% \inferrule [Merge]
% {\Gamma \turns e_1 : A_1 \andalso \Gamma \turns e_2 : A_2}
% {\Gamma \turns e_1 \mergeop e_2 : A_1 \inter A_2}
%Castagna, and Dunfield describe
%elaborating multi-fields records into merge of single-field records.
% Reynolds and Castagna do not consider elaboration and Dunfield do not
% formalize elaborating records.
% Both Pierce and Dunfield's system include a subsumption rule, which states that
% if an term has been inferred of type $ A $, then it is also of any
% supertype of $ A $. Our system does not have this rule.
Refinement
intersection~\cite{dunfield2007refined,davies2005practical,freeman1991refinement}
is the more conservative approach of adopting intersection types. It increases
only the expressiveness of types but not terms. But without a term-level
construct like ``merge'', it is not possible to encode various language
features. As an alternative to syntactic subtyping described in this paper,
Frisch et al.~\cite{frisch2008semantic} studied semantic subtyping. Semantic
subtyping seems to have important advantages over syntactic subtyping. One
worthy avenue for future work is to study languages with intersection types
and merge operator in a semantic subtyping setting.

%*******************************************************************************
\paragraph{Extensibility.}
%*******************************************************************************
One of our motivations to study systems
with intersections types is to better understand the
type system requirements needed to address extensibility problems.
A well-known problem in programming languages is the Expression
Problem~\cite{wadler1998expression}. In recent years there have been
various solutions to the Expression Problem in the literature. Mostly
the solutions are presented in a specific language, using the language
constructs of that language. For example, in Haskell, type classes~\cite{WadlerB89}
can be used to implement type-theoretic encodings of
datatypes~\cite{Hinze:2006}. It has been shown~\cite{finally-tagless}
that, when encodings of datatypes are modeled with type classes,
the subclassing mechanism of type classes can be used to achieve
extensibility and reuse of operations. Using such techniques provides
a solution to the Expression Problem. Similarly, in OO languages with
generics, it is possible to use generic interfaces and classes to
implement type-theoretic encodings of datatypes. Conventional
subtyping allows the interfaces and classes to be extended, which can
also be used to provide extensibility and reuse of operations. Using
such techniques, it is also possible to solve the Expression Problem
in OO languages~\cite{oliveira09modular,oliveira2012extensibility}.
It is even possible to solve the Expression Problem in theorem provers
like Coq, by exploiting Coq's type class mechanism~\cite{DelawareOS13}.
Nevertheless, although there is a clear connection between all those
techniques and type-theoretic encodings of datatypes, as far as we
know, no one has studied the expression problem from a more
type-theoretic point of view.

% As shown in Section~\ref{subsec:OAs}, a system
% with intersection types, parametric polymorphism, the merge operator
% and disjoint quantification can be used to explain type-theoretic
% encodings with subtyping and extensibility.

% Intersection types have been shown to be useful in designing languages that
% support modularity.~\cite{nystrom2006j}

% \paragraph{Extensible records.}

%\george{Record field deletion is also possible.}

% http://elm-lang.org/learn/Records.elm

% Encoding records using intersection types appeared in
% Reynolds~\cite{reynolds1997design} and Castagna et
% al.~\cite{castagna1995calculus}. Although Dunfield also discussed this idea in
% his paper \cite{dunfield2014elaborating}, he only provided an implementation but
% not a formalization. Very similar to our treatment of elaborating records is
% Cardelli's work~\cite{cardelli1992extensible} on translating a calculus, named
% $ F_{\subtype \rho}$, with extensible records to a simpler calculus that without
% records primitives (in which case is $ F_{\subtype} $). But he did not consider
% encoding multi-field records as intersections; hence his translation is more
% heavyweight. Crary~\cite{crary1998simple} used intersection types and
% existential types to address the problem that arises when interpreting method
% dispatch as self-application. But in his paper, intersection types are not used
% to encode multi-field records.

% Wand~\cite{wand1987complete} started the work on extensible records and proposed
% row types~\cite{wand1989type} for records. Cardelli and
% Mitchell~\cite{cardelli1990operations} defined three primitive operations on
% records that are similar to ours: \emph{selection}, \emph{restriction}, and
% \emph{extension}. The merge operator in \name plays the same role as extension.
% Following Cardelli and Mitchell's approach,
% of restriction and extension. Both Leijen's systems~\cite{leijen2004first,leijen2005extensible}
% and ours allow records that contain
% duplicate labels. Leijen's system is more sophisticated. For example, it supports
% passing record labels as arguments to functions. He also showed an encoding of
% intersection types using first-class labels.

% Chlipala's
% \texttt{Ur}~\cite{chlipala2010ur} explains record as type level
% constructs.\bruno{What is the point of citing Chlipala's paper?}

% Our system can be adapted to simulate systems that support extensible
% records but not intersection of ordinary types like \texttt{Int} and
% \texttt{Float} by allowing only intersection of record types.
%
% $ \turnsrec A $ states that $ A $ is a record type, or the intersection of
% record types, and so forth.
%
% \inferrule [RecBase] {} {\turnsrec \recordType l A}
%
% \inferrule [RecStep]
% {\turnsrec A_1 \andalso \turnsrec A_2}
% {\turnsrec A_1 \inter A_2}
%
% \inferrule [Merge']
% {\Gamma \turns e_1 : A_1 \yields {E_1} \andalso \turnsrec A_1 \\
%  \Gamma \turns e_2 : A_2 \yields {E_2} \andalso \turnsrec A_2}
% {\Gamma \turns e_1 \mergeop e_2 : A_1 \inter A_2 \yields {\pair {E_1} {E_2}}}
%
% R{\'e}my~\cite{remy1989type}

%*******************************************************************************
\paragraph{Trait calculi.}
%*******************************************************************************
Fisher and Reppy~\cite{fisher2004typed} provided a dedicated statically typed
calculus for modeling traits. \name is not dedicated to traits; but rather, it
supports a source language that models traits. Compared to Fisher and Reppy's
calculus, \name is more lightweight. For example, self reference is not in the
language of \name. One reason for the difference is that Fisher and Reppy's
calculus supports \emph{classes} in addition to traits, and considers the
interaction between them, whereas our object oriented source language is
\emph{prototype}-based---the mechanism for code reuse is purely trait.

\section{Conclusion and Future Work}
\label{sec:conclusion}

This paper described \name: a language that combines
intersection types and a merge operator.
The language is proved to be type-safe and coherent.
To ensure coherence the type system accepts only
disjoint intersections. We believe that disjoint intersection types are
intuitive, and at the same time expressive. We have shown the
applicability of disjoint intersection types to model a simple form of traits.

We implemented the core functionalities of the \name as part of a JVM-based
compiler. Based on the type system of \name, we have built an ML-like
source language compiler that offers interoperability with Java (such as object
creation and method calls). The source language is loosely based on the more
general System $F_{\omega}$ and supports a
number of other features, including records, polymorphism, mutually recursive
\code{let} bindings, type aliases, algebraic data types, pattern matching, and
first-class modules that are encoded using \code{letrec} and records.

For the future, we intend to improve our source language
and show the power of disjoint intersection types in large case
studies. One pressing challenge is to address the intersction between 
disjoint intersection types and polymorphism.
We are also interested in extending our work
to systems with a $\top$ type. This will also require an adjustment
to the notion of disjoint types. A suitable notion of
disjointness between two types $A$ and $B$ in the presence of $\top$
would be to require that the only common supertype of $A$ and $B$ is $\top$.
Finally we would like to study the
addition of union types. This will also require changes in our
notion of disjointness, since with union types there always exists
a type $A \union B$, which is the common supertype of two
types $A$ and $B$.

% Some immediate topics for
% further improvement of the results in this paper are discussed next.
%
% \paragraph{Union types.}
%
% If a type system ever contains union types (the counterpart of intersection
% types), with the following standard subtyping rules,
% \begin{mathpar}
%   \inferrule* [right=Sub\_Union\_1]
%     { }
%     {A \subtype A \union B}
%
%   \inferrule* [right=Sub\_Union\_2]
%     { }
%     {B \subtype A \union B}
% \end{mathpar}
% then no two types $A$ and $B$ can ever be disjoint, since there always exists
% the type $A \union B$, which is their common supertype. So it is reasonable to
% conjecture that such system cannot be coherent.
% \bruno{I wouldn't say this is a motivation: it sounds like we caould
%   not support union types, when I think this is not true. For example
% we could say something like: there does not exist an \emph{atomic} C ...}
%
%
% \paragraph{Implementation.}
%
% We implemented the core functionalities of the \name as part of a JVM-based
% compiler. Based on the type system of \name, we built an ML-like
% source language compiler that offers interoperability with Java (such as object
% creation and method calls). The source language is loosely based on the more
% general System $F_{\omega}$ and supports a
% number of other features, including records, mutually recursive
% \code{let} bindings, type aliases, algebraic data types, pattern matching, and
% first-class modules that are encoded using \code{letrec} and records.
%
% Relevant to this paper are the three phases in the compiler, which
% collectively turn source programs into System $F$:
%
% \begin{enumerate}
% \item A \emph{typechecking} phase that checks the usage of \name features and
%   other source language features against an abstract syntax tree that follows
%   the source syntax.
%
% \item A \emph{desugaring} phase that translates well-typed source terms into
%   \name terms. Source-level features such as multi-field records, type aliases
%   are removed at this phase. The resulting program is just an \name term
%   extended with some other constructs necessary for code generation.
%
% \item A \emph{translation} phase that turns well-typed \name terms into System
%   $F$ ones.
% \end{enumerate}
%
% Phase 3 is what we have formalized in this paper.
%
%
% \paragraph{Reduce the number of coercions.}
%
% Our translation inserts a coercion (many of them are identity functions)
% whenever subtyping occurs during a function application, which could mean
% notable run-time overhead. In the current implementation, we introduced a
% partial evaluator with three simple rewriting rules to eliminate the redundant
% identity functions as another compiler phase after the translation. In another
% version of our implementation, partial evaluation is weaved into the process of
% translation so that the unwanted identity functions are not introduced during
% the translation. Besides, since the order of the two types in a binary
% intersection does not matter, we may normalize them to avoid unnecessary
% coercions.


\section*{Acknowledgments}
We would like to thank the ICFP reviewers for their helpful comments.
This work has been sponsored by the Hong Kong Research Grant Council Early Career Scheme project number 27200514.

%\newpage
\bibliographystyle{abbrvnat}
\bibliography{references}

%\clearpage
%\onecolumn
%
%\appendix
%\section{Target Type System}

\begin{figure}[h]
  \framebox{$ \hastype G E T $}
  \begin{mathpar}

    \ruletargetvar

    \ruletargetlam

    \ruletargetapp

    \ruletargetblam

    \ruletargettapp

    \ruletargetpair

    \ruletargetprojl

    \ruletargetprojr

  \end{mathpar}

  \caption{Target type system.}
\end{figure}

%\section{Proofs}

\begin{proof}
By structural induction on the types and the corresponding inference rule. \\

\texttt{(SubVar)}

\texttt{(SubFun)}

\texttt{(SubForall)}

\texttt{(SubAnd1)}

\texttt{(SubAnd2)}

\texttt{(SubAnd3)}

\texttt{(SubRcd)}

\end{proof}

\begin{lemma}
  If $$ \Gamma \turnsget \ty ; l = C ; \ty_1 $$
  then $$ \image \Gamma \turns C : \image \ty \to \image {\ty_1} $$
\end{lemma}

\begin{proof}
By structural induction on the type and the corresponding inference rule. \\

\texttt{(Get-Base)} $ \Gamma \turnsget \recordtype l \ty ; l = \idmono {\image {\recordtype l \ty})} ; \ty $ \\

By the induction hypothesis
$$ \image \Gamma \turns \idmono {\image {\recordtype l \ty}} : \image {\recordtype l \ty} \to \image \ty $$

\texttt{(Get-Left)} \\
\texttt{(Get-Right)} \\

\end{proof}

\begin{lemma}
  If $$ \Gamma \turnsput \ty ; l ; E = C ; \ty_1 $$
  then $$ \image \Gamma \turns C : \image \ty \to \image \ty $$
\end{lemma}

\begin{proof}
By structural induction on the type and the corresponding inference rule. \\

\texttt{(Put-Base)} \\
\texttt{(Put-Left)} \\
\texttt{(Put-Right)} \\
\end{proof}

\begin{lemma} \label{preserve-wf}
  If   $$ \Gamma \turns \ty $$
  then $$ \image \Gamma \turns \image \ty $$
\end{lemma}

\begin{proof}
Since $$ \Gamma \turns \ty $$
It follows from \texttt{(FI-WF)} that
  $$ \ftv \ty  \subseteq \ftv {\Gamma} $$
And hence
  $$ \ftv {\image \ty} \subseteq \ftv {\image \Gamma} $$
By \texttt{(F-WF)} we have
  $$ \Gamma \turns \ty $$
\end{proof}

\begin{theorem}[Type preserving translation]
  If   $$ \Gamma \turns e : \ty \yields E  $$
  then $$ \image \Gamma \turns E : \image \ty $$
\end{theorem}

\begin{proof}
By structural induction on the expression and the corresponding inference rule. \\

\texttt{(TrVar)} $ \Gamma \turns x : \ty \yields x $ \\

It follows from \texttt{(TrVar)} that
  $$ (x : t) \in \Gamma $$
Based on the definition of $ \image \cdot $,
  $$ (x : \image t) \in \image \Gamma $$
Thus we have by \texttt{(F-Var)} that
  $$ \image \Gamma \turns x : \image \ty $$

\texttt{(TrAbs)} $ \Gamma \turns \lambda (x : \ty_1). e : \ty_1 \to \ty_2 \yields {\absty x {\image {\ty_1}} E} $ \\

It follows from \texttt{(TrAbs)} that
  $$ \Gamma, x : \ty_1 \turns e : \ty_2 \yields E $$
And by the induction hypothesis that
  $$ \image \Gamma, x : \image {\ty_1} \turns E : \image {\ty_2} $$
By \texttt{(TrAbs)} we also have
  $$ \Gamma \turns \ty_1 $$
It follows from Lemma \ref{preserve-wf} that
  $$ \image \Gamma \turns \image {\ty_1} $$
Hence by \texttt{(F-Abs)} and the definition of $ \image \cdot $ we have
  $$ \image \Gamma \turns \absty x {\image {\ty_1}} E : \image {\ty_1 \to \ty_2} $$

\texttt{(TrApp)} $ \Gamma \turns \app {e_1} {e_2} : \ty_2 \yields {E_1 (\app C {E_2})} $ \\

From \texttt{(TrApp)} we have
  $$ \Gamma \turns \ty_3 <: \ty_1 \yields C $$
Applying Lemma \ref{type-coerce} to the above we have
  $$ \image \Gamma \turns C : \image {\ty_3} \to \image {\ty_1} $$
Also from \texttt{(TrApp)} and the induction hypothesis
  $$ \image \Gamma \turns E_1 : \image {\ty_1} \to \image {\ty_2} $$
Also from \texttt{(TrApp)} and the induction hypothesis
  $$ \image \Gamma \turns E_2 : \image {\ty_3} $$
Assembling those parts using \texttt{(F-App)} we come to
  $$ \image \Gamma \turns E_1 (\app C {E_2}) : \image {\ty_2} $$
\end{proof}

\texttt{(TrTAbs)} $ \Gamma \turns \Lambda \alpha. e : \forall \alpha. \ty \yields {\forall \alpha. E} $ \\

From \texttt{(TrTAbs)} we have
  $$ \Gamma \turns e : \ty \yields E $$
By the induction hypothesis we have
  $$ \image \Gamma \turns E : \image \ty $$
Thus by \texttt{(F-TAbs)} and the definition of $ \image \cdot $
  $$ \Gamma \turns \Lambda \alpha. E : \image {\forall \alpha. \ty} $$


\texttt{(TrTApp)} $ \Gamma \turns e \; \ty  : [\alpha := t] \ty_1 \yields {E \; \image \ty} $ \\

From \texttt{(TrTApp)} we have
  $$ \Gamma \turns e : \forall \alpha. \ty_1 \yields E $$
And by the induction hypothesis that
  $$ \image \Gamma \turns E : \forall \alpha. \image {\ty_1} $$
Also from \texttt{(TrTApp)} and Lemma \ref{preserve-wf} we have
  $$ \image \Gamma \turns \image \ty $$
Then by \texttt{(F-TApp)} that
  $$ \image \Gamma \turns E \; \image \ty : [\alpha := \image \ty ] \image {\ty_1} $$
Therefore
  $$ \image \Gamma \turns E \; \image \ty : \image {[\alpha := \ty ] \image {\ty_1}} $$

% \texttt{(TrMerge)} $ \Gamma \turns e_1 \merge e_2 : \ty_1 \& \ty_2 % % \yields {\tupled {E1, E2} $ \\

From \texttt{(TrMerge)} and the induction hypothesis we have
  $$ \image \Gamma \turns E_1 : \image {\ty_1} $$
and
  $$ \image \Gamma \turns E_2 : \image {\ty_2} $$
Hence by \texttt{(F-Pair)}
  $$ \image \Gamma \turns \tupled {E_1, E_2} : \tupled {\image {\ty_1}, \image {\ty_2}} $$
Hence by the definition of $ \image \cdot $
  $$ \image \Gamma \turns \tupled {E_1, E_2} : \image {\ty_1 \& \ty_2} $$

\texttt{(TrRcdIntro)} $ \Gamma \turns \recordintro l e : \recordtype l \ty \yields E $ \\

From \texttt{(TrRcdIntro)} we have
  $$ \Gamma \turns e : \ty \yields E $$
And by the induction hypothesis that
  $$ \image \Gamma \turns E : \image \ty $$
Thus by the definition of $ \image \cdot $
  $$ \image \Gamma \turns E : \image {\recordtype l \ty} $$

\texttt{(TrRcdElim)} $ \Gamma \turns e.l : \ty_1 \yields {\app C E} $ \\

From \texttt{(TrRcdElim)}
  $$ \Gamma \turns e : \ty \yields E $$
And by the induction hypothesis that
  $$ \image \Gamma \turns E : \image \ty $$
Also from \texttt{(TrRcdEim)}
  $$ \Gamma \turnsget e ; l = C ; \ty_1 $$
Applying Lemma \ref{type-get} to the above we have
  $$ \image \Gamma \turns C : \image \ty \to \image {\ty_1}  $$
Hence by \texttt{(F-App)} we have
  $$ \image \Gamma \turns \app C E : \image {\ty_1} $$

% \texttt{(TrRcdUpd)} $ \Gamma \turns \rcdupd{e} {l} {e_1} : \ty \yields {\app C E} $ \\

From \texttt{(TrRcdUpd)}
  $$ \Gamma \turns e : \ty \yields E $$
And by the induction hypothesis that
  $$ \image \Gamma \turns E : \image \ty $$
Also from \texttt{(TrRcdUpd)}
  $$ \Gamma \turnsput \ty ; l; E = C ; \ty_1 $$
Applying Lemma \ref{type-put} to the above we have
  $$ \image \Gamma \turns C : \image \ty \to \image \ty  $$
Hence by \texttt{(F-App)} we have
  $$ \image \Gamma \turns \app C E : \image \ty $$



\end{document}
