\subsection{Coherence of \name}

\instantiation*

\begin{proof}
By induction.

\begin{itemize}
  \item Case
  \begin{flalign*}
    & \rulewfvar &
  \end{flalign*}

  If $C = \alpha$, then $\subst A \alpha \alpha = A$. Since $\jwf \Gamma A$, it follows that $\jwf \Gamma {\subst A \alpha \alpha}$; otherwise, let $C = \beta$, where $\beta$ is a type variable distinct from $\alpha$. Since $\jwf {\Gamma, \alpha \disjoint B} \beta$ and $\alpha$ and $\beta$ are distinct, $\beta$ must be in $\Gamma$ and therefore $\jwf {\Gamma} \beta$, which is equivalent to $\jwf {\Gamma} {\subst A \alpha \beta}$. \\

  \item Case
  \begin{flalign*}
    & \rulewffun &
  \end{flalign*}

  By straightforwardly applying the i.h and the rule itself. \\

  \item Case
  \begin{flalign*}
    & \rulewfbot &
  \end{flalign*}

  Trivial. \\

  \item Case
  \begin{flalign*}
    & \rulewfforall &
  \end{flalign*}

  By straightforwardly applying the i.h and the rule itself. \\

  \item Case
  \begin{flalign*}
    & \rulewfinterdis &
  \end{flalign*}

  Let $C$ in the statement of this lemma be $C_1 \inter C_2$. By the condition
  we know \[ \jwf {\Gamma, \alpha \disjoint B} {C_1 \inter C_2} \] Thus we must
  have, \[ \jwf {\Gamma, \alpha \disjoint B} {C_1} \] By the induction hypothesis, $\jwf \Gamma
  {\subst A \alpha {C_1}}$ and similarly $\jwf \Gamma {\subst A \alpha {C_2}}$.
  Note that $\ftv {C_1} \cap \ftv {C_2} = \varnothing $. Otherwise $C_1$ and
  $C_2$ cannot be disjoint. By \reflabel{\labelwfinter}, \[ \jwf \Gamma {\subst
  A \alpha {C_1} \inter \subst A \alpha {C_2}} \] and hence \[ \jwf \Gamma
  {\subst A \alpha {(C_1 \inter C_2)}} \]

\end{itemize}
\end{proof}


\wellformedtyping*

\begin{proof}
  By induction on the derivation of $\jtype \Gamma e A$. The case of \reflabel{\labelttapp} needs special attention
  \begin{mathpar}
    \rulettappdis
  \end{mathpar}
  because we need to show that the result of substitution ($\subst A \alpha C$) is well-formed, which is evident by Lemma~\ref{lemma:instantiation}.
\end{proof}

% \begin{definition}[Type variable constraint]
%   We say the \emph{constraint} of a type variable $\alpha$ inside the context
%   $\Gamma$ is $A$ if $\alpha \disjoint A \in \Gamma$.
% \end{definition}

% \begin{lemma}
%   If $A \subtype B$ where both $A$ and $B$ are well-formed, then $A$ and $B$ cannot be disjoint.
% \end{lemma}
%
% \begin{proof}
%   $A \subtype B$ implies $B$ is a common supertype of $A$ and $B$. As a result, $A$ and $B$ are not disjoint by definition.
% \end{proof}

% \begin{lemma}[Free type variables of disjoint bounds] \label{lemma:free-var-disjoint-bounds}
%   If $\jdis \Gamma \alpha A$, then $\alpha \not \in \ftv A$.
% \end{lemma}
%
% \begin{proof}
% Since $A \inter B$ is well-formed, $A \disjoint B$ by the formation rule of intersection types \reflabel{WF\_Inter}. Then by the definition of disjointness, there does not exist a type $C$ such that $A \subtype C$ and $B \subtype C$. It follows that $A \subtype C$ and $B \subtype C$ cannot hold simultaneously.
% \end{proof}

\uniquecoercion*

\begin{proof}
The set of rules for generating coercions is syntax-directed except for the three rules that involve intersection types in the conclusion. Therefore it suffices to show that if well-formed types $A$ and $B$ satisfy $A \subtype B$, where $A$ or $B$ is an intersection type, then at most one of the three rules applies. In the following, we do a case analysis on the shape of $A$ and $B$:

\begin{itemize}
  \item \textbf{Case} $A \neq A_1 \inter A_2$ and $B = B_1 \inter B_2$: Clearly only \reflabel{\labelsubinter} can apply.

  \item \textbf{Case} $A = A_1 \inter A_2$ and $B \neq B_1 \inter B_2$: Only two rules can apply, \reflabel{\labelsubinterl} and \reflabel{\labelsubinterr}. Further, by Lemma~\ref{lemma:unique-subtype-contributor}, it is not possible that $A_1 \subtype B$ and that $A_2 \subtype B$. Thus we are certain that at most one rule of \reflabel{\labelsubinterl} and \reflabel{\labelsubinterr} will apply.

  \item \textbf{Case} $A = A_1 \inter A_2$ and $B = B_1 \inter B_2$\footnote{An example of this case is:
    \[ (\code{Int} \inter \code{Bool}) \inter \code{Char} \subtype \code{Bool} \inter \code{Char} \]}: Since $B$ is not atomic, only \reflabel{\labelsubinter} apply.

  %   Suppose the contrary, that is, more than one of the three rules apply. Since it is not possible that both \reflabel{subinter1} and \reflabel{subinter2} apply by the unique subtype contributor lemma, only one of \reflabel{subinter1} and \reflabel{subinter2} apply. Therefore \reflabel{subinter} has to hold. Without the loss of generality, assume \reflabel{subinter1} apply. Then we have:
  % \[ A_1 \subtype B_1 \inter B_2 \]
  % \[ A_1 \inter A_2 \subtype B_1 \]
  % \[ A_1 \inter A_2 \subtype B_2 \]
\end{itemize}
\end{proof}

%
% \begin{lemma}{Invariance of disjointness}
%   \label{lemma:invariance-of-disjointness}
%
%   If $\jdis \Gamma A B$ and $R$ respects the constraints of $\beta$, then
%   $\jdis \Gamma {\subst R \beta A} {\subst R \beta B}$.
%
% \end{lemma}
%
% This lemma says that substitution for free type variables preserves disjointness
% of types if the combination of the replacement type and the type variable is
% proven disjoint.
%
% \begin{proof}
% By induction on the derivation of $\jdis \Gamma A B$.
% \begin{itemize}
%   \item Case
%   \begin{flalign*}
%     & \ruledisvar &
%   \end{flalign*}
%
%   We need to show \[ \jdis \Gamma {\subst R \beta \alpha} {\subst R \beta B} \]
%   If $\beta$ is not equivalent to $\alpha$ and is not free in $B$, then the above trivially holds by the def. of the substitution function. Otherwise, if $\beta$ is equivalent to $\alpha$, then we need to show
%   \[ \jdis \Gamma R {\subst R \beta B} \].  \\
%
%   % Note that $\beta \not \in \ftv B$. Thus $B$ is equivalent to $\subst R \beta B$.
%   %
%   % If $\beta$ is not equivalent to $\alpha$, $\subst R \beta \alpha$ is equivalent to $\alpha$. Therefore $\jdis \Gamma {\subst R \beta \alpha} {\subst R \beta B}$ is true.
%   % If $\beta$ is equivalent to $\alpha$, then $\subst R \beta \alpha$ is equivalent to $R$ by the def. of the substitution function. It now remains to show \[ \jdis \Gamma R B \].
%
%   \item Case
%   \begin{flalign*}
%     & \ruledisinterl &
%   \end{flalign*}
%
%   By applying the induction hypothesis and the def. of the substitution function. \\
%
%   \item Case
%   \begin{flalign*}
%     & \ruledisinterr &
%   \end{flalign*}
%
%   Similar. \\
%
%   \item Case
%   \begin{flalign*}
%     & \ruledisfun &
%   \end{flalign*}
%
%   By applying the induction hypothesis and the def. of the substitution function. \\
%
%   \item Case
%   \begin{flalign*}
%     & \ruledisforall &
%   \end{flalign*}
%
%   By applying the induction hypothesis and the def. of the substitution function. Note that $\alpha$ is fresh. \\
%
%   \item Case
%   \begin{flalign*}
%     & \ruledisatomic &
%   \end{flalign*}
%
%   Substitution does not change the shape of types when the variable case is excluded. Therefore, the relation in the premise of the rule continue to hold and hence the conclusion.
%
% \end{itemize}
% \end{proof}

%
% \begin{lemma}{Substitution}
%   \label{lemma:substitution}
%
%   If $\jwf \Gamma R$, $\jwf \Gamma S$, and $R$ respects the constraints of
%   $\beta$, then $\jwf \Gamma {\subst R \beta S}$.
%
% \end{lemma}
%
% \begin{proof}
% By induction on the derivation of $\jwf \Gamma {\subst R \beta S}$.
%
% \begin{itemize}
%   \item Case
%   \begin{flalign*}
%     & \rulewfvar &
%   \end{flalign*}
%
%   If $\alpha$ happens to be the same as $\beta$, then by the def. of substitution $\subst R \beta \alpha = R$. Since $\jwf \Gamma R$, we have $\jwf \Gamma {\subst R \beta \alpha}$; On the other hand, if not, then by the def. of substitution $\subst R \beta S = S$. Since $\jwf \Gamma S$, we also have $\jwf \Gamma {\subst R \beta \alpha}$. \\
%
%   \item Case
%   \begin{flalign*}
%     & \rulewfbot &
%   \end{flalign*}
%
%   Trivial. \\
%
%   \item Case
%   \begin{flalign*}
%     & \rulewffun &
%   \end{flalign*}
%
%   By induction hypothesis, $\jwf \Gamma {\subst R \beta A}$ and $\jwf \Gamma {\subst R \beta B}$. By the def. of substitution, $\jwf \Gamma {\subst R \beta {A \to B}}$. \\
%
%   \item Case
%   \begin{flalign*}
%     & \rulewfforall &
%   \end{flalign*}
%
%   By the premise and the induction hypothesis,
%   \[ \jwf {\Gamma} {\subst R \beta A} \]
%   \[ \jwf {\Gamma, \alpha \disjoint A} {\subst R \beta B} \]
%   which by \reflabel{WFForall} implies
%   \[ \jwf \Gamma {\for {\alpha \disjoint A} {\subst R \beta B}} \]
%   By the def. of substitution, $\jwf \Gamma {\subst R \beta {\for {\alpha \disjoint A} B}}$. \\
%
%   \item Case
%   \begin{flalign*}
%     & \rulewfinter &
%   \end{flalign*}
%
%   By induction hypothesis, $\jwf \Gamma {\subst R \beta A}$ and $\jwf \Gamma {\subst R \beta B}$. By Lemma~\ref{lemma:invariance-of-disjointness}, we also have $\jdis \Gamma {\subst R \beta A} {\subst R \beta B}$. Therefore by \reflabel{WFInter}, $\jwf \Gamma {\subst R \beta {A \inter B}}$.
% \end{itemize}
% \end{proof}
