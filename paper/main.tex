\documentclass[preprint]{sigplanconf}

% Use packages immediately following the \documentclass command
\usepackage{amsmath}
\usepackage{amsthm}
\usepackage{fixltx2e}
\usepackage{listings}
\usepackage{mdframed}
\usepackage{xcolor}

\newmdtheoremenv{theorem}{Theorem}
\newmdtheoremenv{lemma}{Lemma}

\definecolor{github}{HTML}{4183C4}

% Define macros immediately before the \begin{document} command
\newcommand{\yields}[1] {\textcolor{github}{\; \hookrightarrow #1}}
\newcommand{\meta}[1]   {{\rm #1 }}
\newcommand{\ftv}[1]    {\meta{ftv}(#1)}
\newcommand{\merge}{,\!,}
\newcommand{\rcdupd}[3]{#1 \; \textrm{with} \; \{#2 = #3\}}


\newcommand{\FI}{{\bf FI} }
\newcommand{\F}{{\bf F} }

\newcommand{\FIend}{{\bf FI}}
\newcommand{\Fend}{{\bf F}}

% "define" code highlights for Java and Scala
\lstdefinelanguage{JavaScala}{
  morekeywords={public,int,interface,implements,default,
    abstract,case,catch,class,def,%
    do,else,extends,false,final,finally,%
    for,if,implicit,import,match,mixin,%
    new,null,object,override,package,%
    private,protected,requires,return,sealed,%
    super,this,throw,trait,true,try,%
    type,val,var,while,with,yield},
  otherkeywords={=>,<-,<\%,<:,>:,\#,@},
  sensitive=true,
  morecomment=[l]{//},
  morecomment=[n]{/*}{*/},
  morestring=[b]",
  morestring=[b]',
  morestring=[b]"""
}

\lstset{ %
language=Haskell,                % choose the language of the code
columns=flexible,
lineskip=-1pt,
basicstyle=\ttfamily\small,       % the size of the fonts that are used for the code
numbers=none,                   % where to put the line-numbers
numberstyle=\ttfamily\tiny,      % the size of the fonts that are used for the line-numbers
stepnumber=1,                   % the step between two line-numbers. If it's 1 each line will be numbered
numbersep=5pt,                  % how far the line-numbers are from the code
backgroundcolor=\color{white},  % choose the background color. You must add \usepackage{color}
showspaces=false,               % show spaces adding particular underscores
showstringspaces=false,         % underline spaces within strings
showtabs=false,                 % show tabs within strings adding particular underscores
morekeywords={var},
%  frame=single,                   % adds a frame around the code
tabsize=2,                  % sets default tabsize to 2 spaces
captionpos=none,                   % sets the caption-position to bottom
breaklines=true,                % sets automatic line breaking
breakatwhitespace=false,        % sets if automatic breaks should only happen at whitespace
title=\lstname,                 % show the filename of files included with \lstinputlisting; also try caption instead of title
escapeinside={(*}{*)},          % if you want to add a comment within your code
keywordstyle=\ttfamily\bfseries,
% commentstyle=\color{Gray},
% stringstyle=\color{Green}
}

\begin{document}

\special{papersize=8.5in,11in}
\setlength{\pdfpageheight}{\paperheight}
\setlength{\pdfpagewidth}{\paperwidth}

\title{\FI}

\authorinfo{Name1}
           {Affiliation1}
           {Email1}
\authorinfo{Name2\and Name3}
           {Affiliation2/3}
           {Email2/3}

\maketitle

\begin{abstract}
\end{abstract}

\keywords
intersecion types, inheritance

\section{Introduction}

There has been a remarkable number of works aimed at improving support
for extensibility in programming languages. These works include:
visions of new programming models~\cite{}; new programming languages or
language extensions~\cite{}, and \emph{design patterns} that can be
used with existing mainstream languages~\cite{}.

%\cite{family polymorphism and virtual
%classes.}. Another line of work are proposals for precise formal models or new 
%programming languages. Yet another line are \emph{design patterns}
%that can be used  with existing mainstream languages. 
%%Part of the motivation behind 

Some of the more recent work on extensibility is focused on various
proposals for design patterns.  Examples include \emph{Object
  Algebras}~\cite{}, \emph{Modular Visitors}~\cite{} or
Torgersen's~\cite{} four design patterns using generics. In those
approaches the idea is to use some advanced (but already available)
features, such as \emph{generics}, in combination with conventional
OOP features to model more extensible designs.  Those designs work in
modern OOP languages such as Java, C\# or Scala.

Although such design patterns give practical benefits in terms of
extensibility, they also expose limitations in existing mainstream OOP
languages. In particular there are three pressing limitations: 
1) lack of good mechanisms for
  \emph{object-level} composition; 2) \emph{conflation of 
    (type) inheritance with subtyping}; 3) \emph{heavy reliance on generics}.

  The first limitation shows up, for example, in Oliveira et
  al.~\cite{} encodings of Feature-Oriented Programming using Object
  Algebras~\cite{}. These programs are best expressed using a form of
  \emph{type-safe}, \emph{dynamic}, \emph{delegation}-based
  composition. Although such form of composition can be encoded in
  languages like Scala, it requires the use of low-level reflection
  techniques, such as dynamic proxies, reflection or other forms of
  meta-programming~\cite{}. It is clear that better language support
  would be desirable.

  The second limitation shows up in designs for modelling
  modular or extensible visitors~\cite{}.  The vast majority of modern
  OOP languages combines type inheritance and subtyping. 
  That is a type extension induces a subtype. However
  as Cook et al.~\cite{} famously argued there are programs where
  ``\emph{subtyping is not inheritance}''. Interestingly previously
  not many practical programs have been reported in the literature
  where the distinction between subtyping and inheritance is
  relevant. However, as shown in this paper, it turns out that this
  difference does show up in practice when designing modular
  (extensible) visitors.  We believe that modular visitors provide a
  compeling practical example where inheritance and subtyping should
  not be conflated!

  Finally, the third limitation is prevalent in many extensible
  designs~\cite{}. Such designs rely on advanced features of generics,
  such as \emph{f-bounded polymorphism}~\cite{}, \emph{variance
    annotations}~\cite{}, \emph{wildcards}~\cite{} and/or \emph{higher-kinded
    types}~\cite{} to achieve type-safety. Sadly, the amount of
  type-annotations, combined with the lack of understanding of these
  features, usually deters programmers from using such designs.

\begin{comment}
Motivated by the insights gained in previous work, this paper presents 
a minimal core calculus that addresses current limitations and
provides a better foundational model for statically typed
delegation-based OOP? We show that Object Algebras fit nicely in this
model. 
\end{comment}

This paper presents System \name: an extension of System F~\cite{}
with intersection types and a merge operator~\cite{}.  The goal of
System \name is to study the \emph{minimal} foundational language
constructs that are needed to support various extensible designs,
while at the same time addressing the limitations of existing OOP
languages. To address the lack of good object-level composition
mechanisms, System \name uses the merge operator to allow dynamic
composition of values/objects. Moreover, in System \name (type-level)
extension is independent of subtyping, and it is possible for an
extension to be a supertype of a base object type. Furthermore,
intersection types and conventional subtyping can be used in many
cases instead of advanced features of generics. Indeed this paper 
shows how many previous designs in the literature can be encoded 
without such advanced features of generics.


Technically speaking System \name is mainly inspired by the work of
Dundfield~\cite{}.  Dundfield shows how to model a simply typed
calculus with intersection types and a merge operator. The presence of
a merge operator adds significant expressiveness to the language,
allowing encodings for many other language constructs as syntactic
sugar. System \name differs from Dundfield's work in a few
ways. Firstly it adds parametric polymorphism and formalizes a
extension for records to support a basic form of objects. Secondly,
the elaboration semantics into System F is done directly from the
source calculus with subtyping. In contrast Dunfield has an additional
step which eliminates subtyping.  Finally a non-technical difference
is that System \name is aimed at studying issues of OOP languages and
extensibility, whereas Dunfield's work was aimed at Functional
Programming and he did not consider applications to extensibility.
Like many other foundational formal models for OOP (for
example~\cite{}), System \name is purely functional and it uses
structural typing.

%%System \name is
%%formalized and implemented. Furthermore the paper illustrates how
%%various extensible designs can be encoded in System \name.

\begin{comment}
We present a polymorphic calculus containing intersection types and records, and show
how this language can be used to solve various common tasks in functional
programming in a nicer way.Intersection types provides a power mechanism for functional programming, in
particular for extensibility and allowing new forms of composition.

Prototype-based programming is one of the two major styles of object-oriented
programming, the other being class-based programming which is featured in
languages such as Java and C\#. It has gained increasing popularity recently
with the prominence of JavaScript in web applications. Prototype-based
programming supports highly dynamic behaviors at run time that are not possible
with traditional class-based programming. However, despite its flexibility,
prototype-based programming is often criticized over concerns of correctness and
safety. Furthermore, almost all prototype-based systems rely on the fact that
the language is dynamically typed and interpreted.
\end{comment}

In summary, the contributions of this paper are:

\begin{itemize}

\item {\bf A Minimal Core Language for Extensibility:} This paper
  identifies a minimal core language, System \name, capable of
  expressing various extensibility designs in the literature.
  System \name also addresses limitations of existing OOP
  languages that complicate extensible designs. 
  
\item {\bf Formalization of System \name:} An elaboration semantics of
  System \name into System F is given, and type-soundness is proved.

\item {\bf Encodings of Extensible Designs:} Various encodings of
  extensible designs into System \name, including \emph{Object
    Algebras} and \emph{Modular Visitors}. 

\item {\bf A Practical Example where ``Inheritance is not Subtyping''
    Matters:} This paper shows that in modular/extensible visitors
  suffer from the ``inheritance is not subtyping problem''. Moreover 
  with extensible visitors the extension should become a
  \emph{supertype}, not a subtype. \bruno{extension with accept method}

\item {\bf Implementation and Examples:} An implementation of an
  extension of System \name, as well as the examples presented in the
  paper, are publicly available. 

\begin{comment}

\item{elaboration typing rules which given a source expression with intersection
    types, typecheck and translate it into an ordinary F term. Prove a type
    preservation result: if a term $ e $ has type $ \tau $ in the source language,
    then the translated term $ \image e $ is well-typed and has type $ \image \tau $ in the
    target language.}

\item{present an algorithm for detecting incoherence which can be very important
    in practice.}

\item{explores the connection between intersection types and object algebra by
    showing various examples of encoding object algebra with intersection
    types.}

\end{comment}

\end{itemize}

\begin{comment}
\subsection{Other Notes}

finitary overloading: yes
but have other merits of intersection been explored?

-- Compare Scala:
-- merge[A,B] = new A with B

-- type IEval  = { eval :  Int }
-- type IPrint = { print : String }

-- F[\_]
\end{comment}
\section{A Taste of FI}

\begin{footnote}
  Change the examples later to something very simple.
\end{footnote}

This section provides the reader with the intuition of \FIend, while we postpone
the presentation of the details in later sections.

In short, \FI generalizes \Fend by adding intersection polymorphism. \FI terms
are elaborated into \Fend, a variant of System F. System F, or polymorphic
lambda calculus lays the foundation of functional programming languages such as
Haskell.

The type system of \FI permits a subtyping relation naturally and enables
prototype-based inheritance. We will explore the usefulness of such a type
system in practice by showing various examples.

\subsection{Intersection Types}

The central addition to the type system of \F in \FI is intersection types. What
is an intersection type? One classic view is from set-theoretic interpretation
of types: $ A \& B $ stands for the intersection of the set of values of $ A $
and $ B $. The other view, adopted in this paper, regards types as a kind of
interface: a value of type $ A \& B $ satisfies both of the interfaces of $ A $
and $ B $. For example, $ { eval : Int } $ is the interface that supports
evaluation to integers, while $ { eval : Int } \& { print : String } $ supports
both evaluation and pretty printing. Those interfaces are akin to interfaces in
Java or traits in Scala. But one key difference is that they are unnamed in
\FIend.

Intersection types provide a simple mechanism for ad-hoc polymorphism, similar
to what type classes in Haskell achieve. The key constructs are the ``merge''
operator, denoted by ``$ ,, $'', at the value level and the corresponding type
intersection operator, denoted by, ``$ \& $'' at the type level.

For example, we can define an (ad-hoc)-polymorphic $ \texttt{show} $ function
that is able to convert integers and booleans to strings. In \FI such function
can be given the type
\begin{lstlisting}
  (Int -> String) & (Bool -> String)
\end{lstlisting}
and be defined using the merge operator $ ,, $ as
\begin{lstlisting}
  let show = showInt ,, showBool
\end{lstlisting}
where \texttt{showString} and \texttt{showBool} are ordinary monomorphic
functions. Later suppose the integer $ 1 $ is applied to the $ show $ function,
the first component $ showInt $ will be picked because the type of $ showInt $
is compatible with $ 1 $ while $ showBool $ is not.

% The merge construct in the original function is elaborated into a pair in the
% target language:

% \begin{verbatim}
% show = (showInt, showBool)
% \end{verbatim}

% In the target language where there is no intersection types, the application
% of the integer \texttt{1} to this function does not typecheck. However, we may
% rescue this situtation by inserting a coercion that extracts the first item
% out of this pair.

% Thus \texttt{show 1} in FI corresponds to \texttt{(fst show) 1} in F.

% While elaborating intersection types, this paper is the first that presents a
% type system that incorporates both parametric polymorphism and intersection
% polymorphism.

% Describe intersection types, encoding records with Intersecion types

% \lstinputlisting[linerange=-]{} % APPLY:linerange=MIXIN_LIB

\subsection{Encoding Records}

In addition to introduction of record literals using the usual notation, \FI
support two more operations on records: record elimination and record update.

A record type of the form $ \{ l : t \} $ can be thought as a normal type $ t $
tagged by the label $ l $.

% A basic example

% \lstinputlisting[linerange=2-6]{../src/Basics.sf} % APPLY:linerange=BASICS_ADD

$ e1 $ and $ e2 $ are two expressions that support both evaluation and pretty
printing and each has type $ \{ eval : Int, print : String \} $. $ add $ takes
two expressions and computes their sum. Note that in order to compute a sum,
$ add $ only requires that the two expressions support evaluation and hence the
type of the parameter $ \{ eval : Int \} $. As a result, the type of $ e1 $ and
$ e2 $ are not exactly the same with that of the parameters of $ add $. However,
under a structural type system, this program should typecheck anyway because the
arguments being passed has more information than required. In other words,
$ \{ eval : Int, print : String \} $ is a subtype of $ \{ eval : Int \} $.

How is this subtyping relation derived? In \FIend, multi-field record types are
excluded from the type system because $ \{ eval : Int, print : String \} $ can
be encoded as $ \{ eval : Int \} \& \{ print : String \} $. And by one of
subtyping rules derives that $ \{ eval : Int \} \& \{ print : String \} $ is a
subtype of $ \{ eval : Int \} $.

% This example is elaborated into the following in \Fend.

% \lstinputlisting[linerange=2-6]{../src/BasicsElab.sf} % APPLY:linerange=BASICSELAB_ADD

\subsection{Parametric Polymorphism}

The presence of both parametric polymorphism and intersection is critical, as we
shall see in the next section, in solving modularity problems. Here is a code
snippet from the next section (The reader is not required to understand the
purpose of this code at this stage; just recognizing the two types of
polymorphism is enough.)
\begin{lstlisting}
type SubExpAlg E = (ExpAlg E) \& { sub : E -> E -> E };
let e2 E (f : SubExpAlg E) = f.sub (exp1 E f) (f.lit 2);
\end{lstlisting}
$ SubExpAlg $ is a type synonym (a la Haskell) defined as the intersection of
$ ExpAlg E $ and $ \{ sub : E -> E -> E \} $, parametrized by a type parameter
$ E $. $ e2 $ exhibits parametric polymorphism as it takes a type argument
$ E $.

\section{Application: Extensible records}
\label{subsec:records}

\joao{change syntax to the one used in overview}
Our system can be used to encode records, similarly to way as discussed in 
\cite{dunfield2014elaborating}. 
However, describing and implementing records within programming languages is certainly not novel 
and has been extensively studied in the past.
Most of the systems are entirely focused on concrete aspects of records 
(i.e. expressiveness, compilation, etc), while ours will specialize the more general notion of  
intersection types.
In this section we aim at comparing our approach with such systems. 

Systems with records usually rely on 3 basic operations: selection, restriction and 
extension/concatenation. 
We will first introduce these basic operations in the context of \name.

\subsection{Basic operations}
\paragraph{Selection}

The select operator is directly embedded in our language. 
It follows the usual syntax of $e.l$, where $e$ is an expression of type $\recordType l \alpha$ and 
$l$ is a label.
A polymorphic function which extracts any record that include the label $l$ of type
$\alpha$ could be written as:
\begin{align*}
&\text{select} :: \fordis \alpha \top {\recordType l \alpha \rightarrow \alpha} \\
&\text{select} \, = \blamdis \alpha \top {\lam x {x.l}}
\end{align*}

Note how, through the use of subtyping, this function will accept any intersection type that contains
the single recorld $\recordType l \alpha$.
This resembles other systems ..., although it is slightly more general, as any it is not 
restricted only to record types. \joao{references}

\paragraph{Restriction}

In constrast with most systems, restriction is not directly embedded on our language.
Instead, we can make use of subtyping to define such operator: 
\begin{align*}
&\text{remove} :: \fordis \alpha \top {\fordis r {\recordType l \alpha} 
{({\recordType l \alpha \inter r) \rightarrow r}}} \\ 
&\text{remove} \, = \blamdis \alpha \top {\blamdis r {\recordType l \alpha} {\lam x {x}}} 
\end{align*}

\paragraph{Extension/Concatenation}

The most usual operators for combining records are extension and concatenation.
Even though that in some systems, the latter is defined in terms of the former, languages that
opt to include concatenation usually rely on specific semantics for it. \joao{add references}
Our system is suitable for encoding both of these operations, but we argue that concatenation is
the natural primitive operator, due to the resemblance with our merge operator.
Indeed, (Harper \& Pierce) also define a \emph{merge} operator, which is quite similar to
our \emph{merge} for intersection types, except it envoles only record types.
For instance, a function which concatenates a single record with field $l$ of type $\tyint$
with another record that lacks this field, is the following (slightly modified in terms of 
notation):
\begin{align*}
&\text{addL}_1 :: \forall \alpha\#l. \alpha \rightarrow (\alpha || \recordType l \tyint) \\ 
&\text{addL}_1 \, = ...
\end{align*}
The reader might notice the resemblance with our system:
\begin{align*}
&\text{addL}_2 :: \fordis \alpha {\recordType l \tyint} {
\alpha \rightarrow (\alpha \inter \recordType l \tyint)} \\
&\text{addL}_2 \, = ...
\end{align*}
This shows that one can use disjoint quantification to express negative field information.
It is very close to what (Harper \& Pierce) describe in their system.  
Note how we have to explicitly state the type of the constraint in $\text{addL}_2$, whereas 
$\text{addL}_1$ does not require this.
The same generality of disjoint intersection types that allows one to encode record types is the
one that forces us to add this extra type in the constraint.
However, there is a slight gain with this approach: $\text{addL}_2$ accepts more types than
$\text{addL}_1$.
Namely, all (intersection) types that contain label $l$, with a field type \emph{disjoint} to
$\tyint$.

Had one meant to forbid records with \emph{any} $l$ fields, then one could write:

\joao{how about this? fresh beta vs bottom?}
\begin{align*}
&\text{addL}_3 :: \fordis \beta \top {\fordis \alpha {\recordType l \beta} {
\alpha \rightarrow (\alpha \inter \recordType l \beta)}} \\
&\text{addL}_3 \, = ...
\end{align*}

Other systems with record concatenation usually define predicates, in terms of field absence or 
presence (with a type $\alpha$). 
This rises the question: how would one classify our system in terms of extension? 
As noted in~\cite{leijen2004first}, systems typically can be categorized into two distinct groups
in what concerns extension: the strict and the free.
The former does not allow field overriding when extending a record (i.e. one can only extend a 
record with a field that is not present in it); while the latter does account for field overriding.
Our system can be seen as hybrid of these two kinds of systems.
Next we will show a comparison in terms of expressability between \name and other systems 
with records that hopefully will enlighten the reader on this matter.

\subsection{Expressibility}
In ... (SPJ \& MJ) -- a strict system with extension -- an example of a function that uses 
record types is the following:
\begin{align*}
&\text{average}_1 :: (r \backslash y, r \backslash x) \Rightarrow \{r | x :: \tyint, y :: \tyint\} \rightarrow \tyint \\
&\text{average}_1 \, r = (r.x + r.y) / 2
\end{align*}

The type signature says that for any record with type $r$, that lacks both $x$ $y$, can be accepted
as parameter extended with $x$ $y$, returning an integer.
Note how the bounded polymorphism is essential to ensure that $r$ does not contain $x$ nor $y$.
On the other hand, in a system with free extension as in~\cite{leijen2005extensible}, 
the more general program would be accepted:
\begin{align*}
&\text{average}_2 :: \forall x \; y, \{ x :: \tyint, y :: \tyint | r \} \rightarrow \tyint \\
&\text{average}_2 \, r = (r.x + r.y) / 2
\end{align*}

In this case, if $r$ contains either field $x$ or field $y$, they would be shadowed by the labels 
present in the type signature.
In other words, if a record with multiple $x$ fields, the most recent (i.e. left-most) would be used 
in any function which accesses $x$.
\joao{add example of a system using subtyping?}

In \name, such function could be re-written as
\footnote{We do not support exactly this function definition style; 
however the type signature and expression (module infix operators) are exactly as one would write them
in \name}:
\begin{align*}
&\text{average}_3 :: \fordis r 
{\recordType x \tyint \inter \recordType y \tyint} 
{\recordType x \tyint \inter \recordType y \tyint \inter r \rightarrow \tyint} \\
&\text{average}_3 = 
{\blamdis t {\recordType x \tyint \inter \recordType y \tyint} {\lam r {(r.x + r.y) / 2}}}
\end{align*}

Thus more types are accepted this function than in the first system, but less than the second. 
Another major difference between \name and the two other mentioned systems, is the ability to 
combine records with arbitrary types.
Our system does not account for well-formedness of record types as the other two systems do 
(i.e. using a special \emph{row} kind), since our encoding of records piggybacks on the more
general notion of disjoint intersection types. 

Finally, it is also worth noting that systems using subtyping may suffer from the so-called 
\emph{update} problem.
\joao{show example (for both update problems?)}
\name does not suffer from this problem. \joao{since we have no refinement types?}
We may illustrate by defining a suitable update function, in a similar fashion 
to~\cite{leijen2005extensible}:
\begin{align*}
&\text{update} :: \fordis \alpha \top {\fordis r {\recordType l \alpha} 
{{\recordType l \alpha \inter r \rightarrow \beta \rightarrow \recordType l \beta \inter r}}} \\ 
&\text{update} \, = \blamdis \alpha \top {\blamdis r {\recordType l \alpha} {\lam x {\lam v {
\recordCon l v \mergeop (\text{remove } \alpha \; r \; x)}}}} 
\end{align*}



\begin{comment}
\section{Application: Extensibility}
\label{subsec:OAs}

Various solutions to the Expression
Problem~\cite{wadler1998expression} in the
literature~\cite{Swierstra:2008,finally-tagless,oliveira09modular,oliveira2012extensibility,DelawareOS13}
are closely related to type-theoretic encodings of datatypes. Indeed, variants
of the same idea keep appearing in different programming languages,
because the encoding of the idea needs to exploit the particular
features of the programming language (or theorem prover).
Unfortunately language-specific constructs obscure the key ideas
behind those solutions.

In this section we show a solution to the
Expression Problem that intends to capture the key ideas of various
solutions in the literature. Moreover, it is shown how \emph{all the
  features} of \namedis (intersection types, the merge operator,
parametric polymorphism and disjoint quantification) are needed to
properly encode one important combinator~\cite{oliveira2013feature} used to compose
multiple operations over datatypes.

%The combination of parametric polymorphism, intersection types and the
%merge operator is useful to encode datatypes with subtyping and
%support extensibility in \namedis. The combination of
%polymorphism and the merge operator is essential to allow the
%composition of multiple operations over datatypes.
%In the following
%we present a step-by-step solution to the Expression Problem in \namedis,
%and illustrate how to combine multiple operations.

%Oliveira and Cook~\cite{oliveira2012extensibility} proposed a design pattern that can solve the
%Expression Problem in languages like Java. An advantage of the pattern
%over previous solutions is that it is relatively lightweight in terms
%of type system features. In a latter paper, Oliveira et al.~\cite{oliveira2013feature}
%noted some limitations of the original design pattern and proposed
%some new techniques that generalized the original pattern, allowing it
%to express programs in a Feature-Oriented Programming~\cite{Prehofer97} style.
%Key to these techniques was the ability to dynamically compose object
%algebras.
%
%Unfortunatelly, dynamic composition of object algebras is
%non-trivial. At the type-level it is possible to express the resulting
%type of the composition using intersection types. Thus, it is still
%possible to solve that part problem nicely in a language like Scala (which
%has basic support for intersection types). However, the dynamic
%composition itself cannot be easily encoded in Scala. The fundamental
%issue is that Scala lacks a \lstinline{merge} operator (see the
%discussion in Section~\ref{subsec:interScala}). Although both Oliveira et al.~\cite{oliveira2013feature} and
%Rendell et al.~\cite{rendel14attributes} have shown that such a \lstinline{merge} operator can
%be encoded in Scala, the encoding fundamentally relies in low-level
%programming techniques such as dynamic proxies, reflection or
%meta-programming.
%
%Because \namedis supports a \lstinline{merge} operator natively, dynamic
%object algebra composition becomes easy to encode. The remainder of
%this section shows how object algebras and object algebra composition
%can be encoded in \namedis. We will illustrate this point
%step-by-step by solving the Expression Problem.
%%Prior knowledge of object algebras is not assumed.

% can be cumbersome and
% language support for intersection types would solve that problem.
% Our type system is just a simple extension of System $ F $; yet surprisingly, it
% is able to solve the limitations of using object algebras in languages such as
% Java and Scala.

\subsection{Church Encoded Arithmetic Expressions}
In the Expression Problem, the idea is to start with a very simple
system modeling arithmetic expressions and evaluation.
The standard typed Church encoding ~\cite{BoehmBerarducci} for arithmetic expressions,
denoted as the type \lstinline{CExp}, is:

\begin{lstlisting}
type CExp = (*$ \forall $*)E. (Int (*$ \to $*) E) (*$ \to $*) (E (*$ \to $*) E (*$ \to $*) E) (*$ \to $*) E
\end{lstlisting}

\noindent However, as done in various solutions to extensibility, it is better to break
down the type of the Church encoding into two parts:

\begin{lstlisting}
type ExpAlg[E] = {
  lit: Int (*$ \to $*) E,
  add: E (*$ \to $*) E (*$ \to $*) E
} in (*$ \ldots $*)
\end{lstlisting}

\noindent The first part, captured by the type \lstinline{ExpAlg[E]}
is constitutes the so-called algebra of the datatype. For additional
clarity of presentation, records (supported in the implementation of \namedis)
are used to capture the two components of the algebra.
The first component abstracts over the type of the
constructor for literal expressions ($\tyint \to E$). The second
component abstracts over the type of addition expressions
($E \to E \to E$).

The second part, which is the actual type of the Church encoding, is:

\begin{lstlisting}
type Exp = { accept: (*$ \forall $*)E. ExpAlg[E] (*$ \to $*) E } in (*$ \ldots $*)
\end{lstlisting}

\noindent It should be clear that, modulo some refactoring, and the
use of records, the type \lstinline{Exp} and \lstinline{CExp}
are equivalent.

\paragraph{Data constructors.}
Using \lstinline{Exp} the two data constructors are defined as follows:

\begin{lstlisting}
let lit (n: Int): Exp = {
  accept = (*$ \Lambda $*)E (*$ \to $*) fun (f: ExpAlg[E]) (*$ \to $*) f.lit n
} in
let add (e1: Exp) (e2: Exp): Exp = {
  accept = (*$ \Lambda $*)E (*$ \to $*) fun (f: ExpAlg[E]) (*$ \to $*)
    f.add (e1.accept[E] f) (e2.accept[E] f)
} in
(*$ \ldots $*)
\end{lstlisting}

Note that the notation $\Lambda E$ in the definition of the
\lstinline{accept} fields is a type abstraction: it
introduces a type variable in the environment. The definition of the
constructors themselves follows the usual Church encodings.

Simple expressions, can be built using the data constructors:

\begin{lstlisting}
let five : Exp = add (lit 3) (lit 2) in (*$ \ldots $*)
\end{lstlisting}

\paragraph{Operations.} Defining operations over expressions requires
implementing \lstinline{ExpAlg[E]}. For example, an interesting
operation over expressions is evaluation. The first step is to define
the evaluation operation is to chose how to instantiate the type
parameter \lstinline{E} in \lstinline{ExpAlg[E]} with a suitable
concrete type for evaluation. One such suitable type is:

\begin{lstlisting}
type IEval = { eval: Int } in (*$ \ldots $*)
\end{lstlisting}

\noindent Using \lstinline{IEval}, a record \lstinline{evalAlg}
implementing \lstinline{ExpAlg} is defined as follows:

\begin{lstlisting}
let evalAlg: ExpAlg[IEval] = {
  lit = fun (x: Int) (*$ \to $*) { eval = x },
  add = fun (x: IEval) (y: IEval) (*$ \to $*) {
    eval = x.eval + y.eval
  }
} in (*$ \ldots $*)
\end{lstlisting}

In this record, the two operations
\lstinline{lit} and \lstinline{add} return a record with type
\lstinline{IEval}. The definition of \lstinline{eval} for
\lstinline{lit} and \lstinline{add} is straightforward.


%The type \lstinline$ExpAlg[IEval]$ is the type of object algebras
%supporting evaluation.
%However, the one interesting point
%of object algebras is that other operations can be supported as
%well.

Using \lstinline{evalAlg}, the expression \lstinline{five} can
be evaluated as follows:

\begin{lstlisting}
(five.accept[IEval] evalAlg).eval
\end{lstlisting}

\subsection{Extensibility and Subtyping} Of course, in the Expression
Problem the goal is to achieve extensibility in two dimensions: constructors and operations.
Moreover, in the presence of subtyping it is also interesting to see how the extended datatypes
relate to the original datatypes. We discuss the two topics next.

\paragraph{New constructors.} Here is the code needed to add a new subtraction constructor:
%Arithmetic expressions with subtyping are defined using the type \lstinline{SubExp}:

\begin{lstlisting}
type SubExpAlg[E] = ExpAlg[E] & { sub: E (*$ \to $*) E (*$ \to $*) E } in
type SubExp = { accept: (*$ \forall $*)A. SubExpAlg[A] (*$ \to $*) A } in
let sub (e1: SubExp) (e2: SubExp): SubExp = {
  accept = (*$ \Lambda $*)E (*$ \to $*) fun (f : SubExpAlg[E]) (*$ \to $*)
    f.sub (e1.accept[E] f) (e2.accept[E] f)
} in
(*$ \ldots $*)
\end{lstlisting}

\noindent Firstly \lstinline{SubExpAlg} defines an extended algebra
that contains the constructors of \lstinline{ExpAlg} plus the new
subtraction constructor. Intersection types are used to do the type
composition. Secondly, a new type of expressions with subtraction
(\lstinline{SubExp}) is needed. For \lstinline{SubExp} it is important
that the \lstinline{accept} field now takes an algebra of type
\lstinline{SubExpAlg} as argument. This is necessary to define the
constructor for subtraction (\lstinline{sub}), which requires the
algebra to have the field \lstinline{sub}.


\paragraph{Extending existing operations.} In order to use evaluation
with the new type of expressions, it is necessary to also extend
evaluation. Importantly, extension is achieved using the merge operator:

\begin{lstlisting}
let subEvalAlg = evalAlg ,, {
  sub = fun (x: IEval) (y: IEval) (*$ \to $*) {
    eval = x.eval - y.eval
  }
} in (*$ \ldots $*)
\end{lstlisting}

\noindent In the code, the merge operator takes \lstinline{evalAlg}
and a new record with the implementation of evaluation for
subtraction, to define the implementation for arithmetic expressions
with subtraction.

\paragraph{Subtyping.} In the presence of subtyping, there are
interesting subtyping relations between datatypes and their
extensions~\cite{oliveira09modular}.  Such subtyping relations are usually not
discussed in theoretical treatments of Church encodings. This is
probably partly due to most work on typed Church encodings being
done in calculi without subtyping.

The interesting aspect about subtyping in typed Church encodings is
that subtyping follows the opposite direction of the extension.  In
other words subtyping is contravariant with respect to the
extension. Such contravariance is explained by the type of the
\lstinline{accept} field, which is a function where the argument type
is refined in the extensions. Thus, due to the contravariance of
subtyping on functions, the extension becomes a supertype of the
original datatype.

In the particular case of expressions \lstinline{Exp} (the original and smaller
datatype) is a subtype of \lstinline{SubExp} (the larger and extended
datatype). Because of this subtyping relation, writing the following
expression is valid in \namedis:

\begin{lstlisting}
let three : SubExp = sub five (lit 2)
\end{lstlisting}

\noindent Note the \lstinline{three} is of type \lstinline{SubExp},
but the first argument (\lstinline{five}) to the constructor
\lstinline{sub} is of type \lstinline{Exp}. This can only type-check
if \lstinline{Exp} is indeed a subtype of \lstinline{SubExp}.





\paragraph{New operations.}
The second type of extension is adding a new operation, such as pretty printing.
Similarly to evaluation, the interface of the pretty printing feature
is modeled as:
%\begin{comment}
%  \begin{lstlisting}
%    type IPrint = { print : String } in (*$ \ldots $*)
%  \end{lstlisting}
%\end{comment}
\begin{lstlisting}
type IPrint = { print: String } in (*$ \ldots $*)
\end{lstlisting}
The implementation of pretty printing for expressions that support literals,
addition, and subtraction is:

\begin{lstlisting}
let printAlg: SubExpAlg[IPrint] = {
  lit = fun (x: Int) (*$ \to $*) {
    print = x.toString()
  },
  add = fun (x: IPrint) (y: IPrint) (*$ \to $*) {
    print = x.print ++ " + " ++ y.print
  },
  sub = fun (x: IPrint) (y: IPrint) (*$ \to $*) {
    print = x.print ++ " - " ++ y.print
  }
} in (*$ \ldots $*)
\end{lstlisting}

\noindent The definition of \lstinline{printAlg} is unremarkable.
With \lstinline{printAlg} we can pretty print the expression represented
by \lstinline{three}:

\begin{lstlisting}
(three.accept[IPrint] printAlg).print
\end{lstlisting}

%\begin{comment}
%The result is \lstinline$"7 - 2"$. Note that the programmer is able to pass \lstinline{lit 2}, which is of type \lstinline{Exp},
%to \lstinline{sub}, which expects a \lstinline{SubExp}. The types are compatible
%because because \lstinline$Exp$ is a \emph{subtype} of \lstinline$SubExp$. Code
%reuse is achieved since we can use the constructors from \lstinline$Exp$ as the
%constructor for \lstinline$SubExp$. In Scala, we would have to define two
%literal constructors, one for \lstinline$Exp$ and another for
%\lstinline$SubExp$.
%\end{comment}
%
\subsection{Composition of Algebras}
The final example shows a non-trivial combinator for algebras that
allows multiple algebras to be combined into one. A version of this
combinator has been encoded in Scala before using intersection types
(which Scala supports) and an encoding of the merge
operator~\cite{oliveira2013feature,rendel14attributes}.
Unfortunatelly, the Scala encoding of the merge operator is quite complex as
it relies on low-level type-unsafe programming features such as
dynamic proxies, reflection or other meta-programming techniques.
In \namedis there is no need for such hacky encoding, as the
merge operator is natively supported. Therefore the combinator for
composing algebras is implemented much more elegantly.
The combinator is defined by the \lstinline$combine$ function, which takes two object algebras to create
a combined algebra. It does so by constructing a new algebra
where each field is a function that delegates the input to the two
algebra parameters.

\begin{lstlisting}
let combine A (B * A) (f: ExpAlg[A]) (g: ExpAlg[B]) :
  ExpAlg[A & B] = {
    lit = fun (x: Int) (*$ \to $*) f.lit x ,, g.lit x,
    add = fun (x: A & B) (y: A & B) (*$ \to $*)
      f.add x y ,, g.add x y
}
\end{lstlisting}

Note how \lstinline{combine} requires all the interesting
features of \namedis. Parametric polymorphism is needed because
\lstinline{combine} must compose algebras with arbitrary type
parameters. Intersection types are needed because the resulting
algebra will create values with an intersection type composing
the two type parameters of the two input algebras. The merge operator
is needed to compose the results of each algebra together. Finally,
a disjointness constraint is needed to ensure that the two input
algebras build values of disjoint types (otherwise ambiguity could
arize).

With \lstinline{combine} printing and evaluation of expressions with
subtraction is done as follows:

\begin{lstlisting}
let newAlg: ExpAlg[IEval&IPrint] =
    combine[IEval,IPrint] evalAlg printAlg in
let o = five.accept[IEval&IPrint] newAlg in
o.print ++ " = " ++ o.eval.toString()
\end{lstlisting}

Note that \lstinline$o$ is a value that supports both
evaluation and printing. The final expression uses \lstinline{o}
for doing both printing and evaluation.
\end{comment}

\newcommand{\judgeTargetWF}[2]{#1 \; \textcolor{targetColor}{\turns} \; #2}
\newcommand{\judgeTarget}[3]{#1 \; \textcolor{targetColor}{\turns} \; #2 \; \textcolor{targetColor}{:} \; #3}
\newcommand{\ruleLabelTarget}{\bm{\textcolor{targetColor}{T}}}

\newcommand{\ruleLabelTargetvar}{\ruleLabelTarget\text{var}}
\newcommand{\ruleTargetVar} {
\inferrule* [right=$\ruleLabelTargetvar$]
  {(x,T) \in \Gamma}
  {\judgeTarget \Gamma x T}
}

\newcommand{\ruleLabelTargetUnit}{\ruleLabelTarget\text{unit}}
\newcommand{\ruleTargetUnit} {
\inferrule* [right=$\ruleLabelTargetUnit$]
  { }
  {\judgeTarget \Gamma {()} {()}}
}

\newcommand{\ruleLabelTargetlam}{\ruleLabelTarget\text{lam}}
\newcommand{\ruleTargetLam} {
\inferrule* [right=$\ruleLabelTargetlam$]
  {\judgeTarget {\Gamma, x \hast T} E {T_1} \andalso \judgeTargetWF \Gamma T}
  {\judgeTarget \Gamma {\lam x T E} {T \to T_1}}
}

\newcommand{\ruleLabelTargetApp}{\ruleLabelTarget\text{app}}
\newcommand{\ruleTargetApp}{
\inferrule* [right=$\ruleLabelTargetApp$]
  {\judgeTarget \Gamma {E_1} {T_1 \to T_2} \andalso \judgeTarget \Gamma {E_2} {T_1}}
  {\judgeTarget \Gamma {\app {E_1} {E_2}} {T_2}}
}

\newcommand{\ruleLabelTargetBLam}{\ruleLabelTarget\text{blam}}
\newcommand{\ruleTargetBLam}{
\inferrule* [right=$\ruleLabelTargetBLam$]
  {\judgeSource {\Gamma, \alpha} E T}
  {\judgeSource \Gamma {\blam \alpha E} {\for \alpha T}}
}

\newcommand{\ruleLabelTargetTApp}{\ruleLabelTarget\text{tapp}}
\newcommand{\ruleTargetTApp}{
\inferrule* [right=$\ruleLabelTargetTApp$]
  {\judgeTarget \Gamma E {\for \alpha {T_1}} \andalso \judgeTargetWF \Gamma T}
  {\judgeTarget \Gamma {\tapp E T} {\subst T \alpha T_1}}
}

\newcommand{\ruleLabelTargetPair}{\ruleLabelTarget\text{pair}}
\newcommand{\ruleTargetPair}{
\inferrule* [right=$\ruleLabelTargetPair$]
  {\judgeTarget \Gamma {E_1} {T_1} \andalso \judgeTarget \Gamma {E_2} {T_2}}
  {\judgeTarget \Gamma {\pair {E_1} {E_2}} {\pair {T_1} {T_2}}}
}

\newcommand{\ruleLabelTargetProjLeft}{\ruleLabelTarget\text{proj}_1}
\newcommand{\ruleTargetProjLeft}{
\inferrule* [right=$\ruleLabelTargetProjLeft$]
  {\judgeTarget \Gamma E {\pair {T_1} {T_2}}}
  {\judgeTarget \Gamma {\proj 1 E} {T_1}}
}

\newcommand{\ruleLabelTargetProjRight}{\ruleLabelTarget\text{proj}_2}
\newcommand{\ruleTargetProjRight}{
\inferrule* [right=$\ruleLabelTargetProjRight$]
  {\judgeTarget \Gamma E {\pair {T_1} {T_2}}}
  {\judgeTarget \Gamma {\proj 2 E} {T_2}}
}
\section{Source Language}

The source language, System FI, is identical to the source language described in
the previous section, except for the two additions: intersection types and
records. The formalization includes only single records and single record types as the multi-records can be desugared into the merge of multiple single records.

Dunfield has described a language that includes a ``top'' type but it does not appear in our language. Our work differs from Dunfield in that ...

Remark. The operational semantics of FI is not presented in this paper. However,

\subsection{Source Syntax}

\subsection{Source Subtyping}

\subsection{Source Typing}

\section{Elaboration Typing}

In order to give the reader an intuitive idea of how the elaboration works,
let's first imagine a manual translation.

First, multi-field record literals are desugared into merges of single-field
record literals. Therefore $ \{ eval = 4, print = ``4'' \} $ becomes
$ \{ eval = 4 \} ,, \{ print = ``4'' \} $. Merges of two values are elaborated
into just a pair of them and single-field record literals lose their field
labels during the elaboration. Hence $ \{ eval = 4 \} ,, \{ print = ``4'' \} $
becomes $ (4, ``4'') $.

Finally, $ e1 $ and $ e2 $ are both coerced by a projection function
$ \\(x:(Int,String)). x.\_1 $ before being applied to $ add $. We adopt a
Scala-like syntax where $ .\_1 $ denotes the projection of a tuple on the first
element, and so on.

\framebox{$|\tau| = T$}

\[
\begin{array}{rcl}
  |\alpha|               & = & \alpha \\
  |\tau_1 \to \tau_2|    & = & |\tau_1| \to |\tau_2| \\
  |\forall \alpha. \tau| & = & \forall \alpha. |\tau| \\
  |t_1 \& t_2|           & = & \langle |\tau_1|, |\tau_2| \rangle \\
  |\{ l : \tau \}|       & = & |\tau|
\end{array}
\]

\begin{lemma} \label{type-coerce}
  If $$ \Gamma \vdash \tau_1 <: \tau_2 \yields{C} $$
  then $$ |\Gamma| \vdash C : |\tau_1| \to |\tau_2| $$
\end{lemma}

In this section, we present a relatively lightweight type-directed elaboration
from FI to F. The elaboration consists of four sets of rules, which are
explained below:

\begin{itemize}

\item{\bf Coercion}

  The coercion judgment $ \Gamma \vdash \tau_1 <: \tau_2 \yields{C} $ extends
  the subtyping judgment with a coercion on the right hand side of
  $ \hookrightarrow $. A coercion, which is just an expression in the target
  language, is guaranteed to have type $ \tau_1 \to \tau_2 $, as proved by Lemma
  \ref{type-coerce}. It is read ``In the environment $ \Gamma $, $ \tau_1 $ is a
  subtype of $ \tau_2 $; and if any expression $ e $ has a type $ t_1 $ that is
  a subtype of the type of $ t_2 $, the elaborated $ e $, when applied to the
  corresponding coercion $ C $, has exactly type $ |t_2| $''. For example,
  $\Gamma \vdash Int \& Bool <: Bool \yields{fst} $, where $ fst $ is the
  projection of a tuple on the first element. The coercion judgment is only used
  in the \texttt{TrApp} case.

\item{\bf Elaboration}

  The elaboration judgment $ \Gamma \vdash e : \tau \yields{E} $ extends the
  typing judgment with an elaborated expression on the right hand side of
  $ \hookrightarrow $. It is also standard, except for the case of
  \texttt{TrApp}, in which a coercion from the inferred type of the argument to
  the expected type of the parameter is inserted before the argument; and the
  case of \texttt{TrRcdEim} and \texttt{TrRcdUpd}, where the ``get'' and ``put''
  rules will be used. The two set of rules are explained below.

\item{\bf ``get'' rules}

  The ``get'' judgment can be thought as producing a field accessor.

\item{\bf ``put'' rules}

  The ``put'' judgment can be thought as producing a field updater.

\end{itemize}

Type-Directed Translation to System F.
Main results: type-preservation + coherence.
\section{Implementation}

\subsection{Type Synonyms}

We extend the implementation of the type system extended with type synonyms and
lazy arguments.

\begin{lstlisting}
type T A1 A2 = ... in
\end{lstlisting}

\subsection{Optimization}

\section{Related Work}
\label{sec:related-work}

%*******************************************************************************
\paragraph{Coherence}
%*******************************************************************************

Reynolds invented Forsythe~\cite{reynolds1997design} in the 1980s. Our
merge operator is analogous to his operator $p_1, p_2 $. Forsythe has
a coherent semantics. The result was proved formally by
Reynolds~\cite{reynolds1991coherence} in a lambda calculus with
intersection types and a merge operator. However there are two key
differences to our work. Firstly the way coherence is ensured is
rather ad-hoc. He has four different typing rules for the merge
operator, each accounting for various possibilities of what the types
of the first and second components are. In some cases the meaning of
the second component takes precedence (that is, is biased) over the
first component. The set of rules is restrictive and it forbids, for
instance, the merge of two functions (even when they a provably
disjoint). In contrast, disjointness in \name has a well-defined
specification and it is quite flexible. Secondly, Reynolds calculus
does not support universal quantification. It is unclear to us whether
his set of rules would still ensure disjointness in the presence of
universal quantification. Since some biased choice is allowed in
Reynold's calculus the issues illustrated in Section~\ref{subsec:polymorphism} could be a problem.

Pierce~\cite{pierce1991programming2} made a comprehensive review
of coherence, especially on Curien and Ghelli~\cite{curienl1990coherence} and
Reynolds' methods of proving coherence; but he was not able to prove coherence
for his $F_\wedge$ calculus. He introduced a primitive $\code{glue}$ function as
a language extension which corresponds to our merge operator. However, in his
system users can ``glue'' two arbitrary values, which can lead to incoherence.

Our work is largely inspired by Dunfield~\cite{dunfield2014elaborating}. He
described a similar approach to ours: compiling a system with intersection types
and a merge operator into ordinary $ \lambda $-calculus terms with pairs. One
major difference is that our system does not include unions. However, as
acknowledged by Dunfield, his calculus lacks of coherence. He discusses the
issue of coherence throughout his paper, mentioning biased choice as an option
(albeit a rather unsatisfying one). He also mentioned that the notion of
disjoint intersection could be a good way to address the problem, but he did not
pursue this option in his work. In contrast to his work, we developed a type
system with disjoint intersection types and proposed disjoint quantification to
guarantee coherence in our calculus.

% \url{http://homepages.inf.ed.ac.uk/gdp/publications/Sub_Par.pdf}

% \cite{plotkin1994subtyping}

% Also discussed intersection types!~\cite{malayeri2008integrating}.

% Pierce Ph.D thesis: F<: + /|
%        technical report: F + /|, closer to ours

% \cite{barbanera1995intersection}
%
% \paragraph{Intersection types with polymorphism.}
% Our type system combines intersection types and parametric polymorphism. Closest
% to us is Pierce's work~\cite{pierce1991programming1} on a prototype
% compiler for a language with both intersection types, union types, and
% parametric polymorphism. Similarly to \name in his system universal
% quantifiers do not support bounded quantification. However Pierce did not try to prove any
% meta-theoretical results and his calculus does not have a merge
% operator.  Pierce also studied a system where both intersection
% types and bounded polymorphism are present in his Ph.D.
% dissertation~\cite{pierce1991programming2} and a 1997
% report~\cite{pierce1997intersection}.

Going in the direction of higher
kinds, Compagnoni and Pierce~\cite{compagnoni1996higher} added
intersection types to System $ F_{\omega} $ and used the new calculus,
$ F^{\omega}_{\wedge} $, to model multiple inheritance. In their
system, types include the construct of intersection of types of the
same kind $ K $. Davies and Pfenning
\cite{davies2000intersection} studied the interactions between
intersection types and effects in call-by-value languages. And they
proposed a ``value restriction'' for intersection types, similar to
value restriction on parametric polymorphism. Although they proposed a system with
parametric polymorphism, their subtyping rules are significantly different from ours,
since they consider parametric polymorphism
as the ``infinit analog'' of intersection polymorphism.

Recently,
Castagna et al.~\cite{Castagna:2014} studied an very expressive calculus that
has polymorphism and set-theoretic type connectives (such as intersections,
unions, and negations). As a result, in their calculus one is also able to
express a type variable that can be instantiated to any type other than
$\code{Int}$ as $\alpha \setminus \code{Int}$, which is syntactic sugar for
$\alpha \wedge \neg \code{Int}$. As a comparison, such a type will need a
disjoint quantifier, like $\fordis \alpha {\code{Int}} \alpha$, in our system.
Unfortunately their calculus does not include a merge operator like ours.

There have been attempts to provide a foundational calculus
for Scala that incorporates intersection
types~\cite{amin2014foundations,amin2012dependent}.
Although the minimal Scala-like calculus does not natively support
parametric polymorphism, it is possible to encode parametric
polymorphism with abstract type members. Thus it can be argued that
this calculus also supports intersection types and parametric
polymorphism. However, the type-soundness of a minimal Scala-like
calculus with intersection types and parametric polymorphism is not
yet proven. Recently, some form of intersection
types has been adopted in object-oriented languages such as Scala,
Ceylon, and Grace. Generally speaking,
the most significant difference to \name is that in all previous systems
there is no explicit introduction construct like our merge operator. As shown in
Section~\ref{subsec:OAs}, this feature is pivotal in supporting modularity
and extensibility because it allows dynamic composition of values.

\begin{comment}
only allow intersections of concrete types (classes),
whereas our language allows intersections of type variables, such as
\texttt{A \& B}. Without that vehicle, we would not be able to define
the generic \texttt{merge} function (below) for all interpretations of
a given algebra, and would incur boilerplate code:

\begin{lstlisting}
let merge [A, B] (f: ExpAlg A) (g: ExpAlg B) = {
  lit (x : Int) = f.lit x ,, g.lit x,
  add (x : A & B) (y : A & B) =
    f.add x y ,, g.add x y
}
\end{lstlisting}
\end{comment}

%*******************************************************************************
\paragraph{Other type systems with intersection types.}
%*******************************************************************************

% Although similar in spirit,
% our elaboration typing is simpler: we require subtyping in the case of
% applications, thus avoiding the subsumption rule. Besides, our treatment
% combines the merge rules ($ k $ ranges over $ \{1, 2\} $)
% \inferrule
% {\Gamma \turns e_k : A}
% {\Gamma \turns e_1 \mergeop e_2 : A}
% and the standard intersection-introduction rule
% \inferrule
% {\Gamma \turns e : A_1 \andalso \Gamma \turns e : A_2}
% {\Gamma \turns e : A_1 \inter A_2}
% into one rule:
% \inferrule [Merge]
% {\Gamma \turns e_1 : A_1 \andalso \Gamma \turns e_2 : A_2}
% {\Gamma \turns e_1 \mergeop e_2 : A_1 \inter A_2}
%Castagna, and Dunfield describe
%elaborating multi-fields records into merge of single-field records.
% Reynolds and Castagna do not consider elaboration and Dunfield do not
% formalize elaborating records.
% Both Pierce and Dunfield's system include a subsumption rule, which states that
% if an term has been inferred of type $ A $, then it is also of any
% supertype of $ A $. Our system does not have this rule.
Refinement
intersection~\cite{dunfield2007refined,davies2005practical,freeman1991refinement}
is the more conservative approach of adopting intersection types. It increases
only the expressiveness of types but not terms. But without a term-level
construct like ``merge'', it is not possible to encode various language
features. As an alternative to syntactic subtyping described in this paper,
Frisch et al.~\cite{frisch2008semantic} studied semantic subtyping. Semantic
subtyping seems to have important advantages over syntactic subtyping. One
worthy avenue for future work is to study languages with intersection types
and merge operator in a semantic subtyping setting.

%*******************************************************************************
\paragraph{Extensibility.}
%*******************************************************************************
One of our motivations to study systems
with intersections types is to better understand the
type system requirements needed to address extensibility problems.
A well-known problem in programming languages is the Expression
Problem~\cite{wadler1998expression}. In recent years there have been
various solutions to the Expression Problem in the literature. Mostly
the solutions are presented in a specific language, using the language
constructs of that language. For example, in Haskell, type classes~\cite{WadlerB89}
can be used to implement type-theoretic encodings of
datatypes~\cite{Hinze:2006}. It has been shown~\cite{finally-tagless}
that, when encodings of datatypes are modeled with type classes,
the subclassing mechanism of type classes can be used to achieve
extensibility and reuse of operations. Using such techniques provides
a solution to the Expression Problem. Similarly, in OO languages with
generics, it is possible to use generic interfaces and classes to
implement type-theoretic encodings of datatypes. Conventional
subtyping allows the interfaces and classes to be extended, which can
also be used to provide extensibility and reuse of operations. Using
such techniques, it is also possible to solve the Expression Problem
in OO languages~\cite{oliveira09modular,oliveira2012extensibility}.
It is even possible to solve the Expression Problem in theorem provers
like Coq, by exploiting Coq's type class mechanism~\cite{DelawareOS13}.
Nevertheless, although there is a clear connection between all those
techniques and type-theoretic encodings of datatypes, as far as we
know, no one has studied the expression problem from a more
type-theoretic point of view.

% As shown in Section~\ref{subsec:OAs}, a system
% with intersection types, parametric polymorphism, the merge operator
% and disjoint quantification can be used to explain type-theoretic
% encodings with subtyping and extensibility.

% Intersection types have been shown to be useful in designing languages that
% support modularity.~\cite{nystrom2006j}

% \paragraph{Extensible records.}

%\george{Record field deletion is also possible.}

% http://elm-lang.org/learn/Records.elm

% Encoding records using intersection types appeared in
% Reynolds~\cite{reynolds1997design} and Castagna et
% al.~\cite{castagna1995calculus}. Although Dunfield also discussed this idea in
% his paper \cite{dunfield2014elaborating}, he only provided an implementation but
% not a formalization. Very similar to our treatment of elaborating records is
% Cardelli's work~\cite{cardelli1992extensible} on translating a calculus, named
% $ F_{\subtype \rho}$, with extensible records to a simpler calculus that without
% records primitives (in which case is $ F_{\subtype} $). But he did not consider
% encoding multi-field records as intersections; hence his translation is more
% heavyweight. Crary~\cite{crary1998simple} used intersection types and
% existential types to address the problem that arises when interpreting method
% dispatch as self-application. But in his paper, intersection types are not used
% to encode multi-field records.

% Wand~\cite{wand1987complete} started the work on extensible records and proposed
% row types~\cite{wand1989type} for records. Cardelli and
% Mitchell~\cite{cardelli1990operations} defined three primitive operations on
% records that are similar to ours: \emph{selection}, \emph{restriction}, and
% \emph{extension}. The merge operator in \name plays the same role as extension.
% Following Cardelli and Mitchell's approach,
% of restriction and extension. Both Leijen's systems~\cite{leijen2004first,leijen2005extensible}
% and ours allow records that contain
% duplicate labels. Leijen's system is more sophisticated. For example, it supports
% passing record labels as arguments to functions. He also showed an encoding of
% intersection types using first-class labels.

% Chlipala's
% \texttt{Ur}~\cite{chlipala2010ur} explains record as type level
% constructs.\bruno{What is the point of citing Chlipala's paper?}

% Our system can be adapted to simulate systems that support extensible
% records but not intersection of ordinary types like \texttt{Int} and
% \texttt{Float} by allowing only intersection of record types.
%
% $ \turnsrec A $ states that $ A $ is a record type, or the intersection of
% record types, and so forth.
%
% \inferrule [RecBase] {} {\turnsrec \recordType l A}
%
% \inferrule [RecStep]
% {\turnsrec A_1 \andalso \turnsrec A_2}
% {\turnsrec A_1 \inter A_2}
%
% \inferrule [Merge']
% {\Gamma \turns e_1 : A_1 \yields {E_1} \andalso \turnsrec A_1 \\
%  \Gamma \turns e_2 : A_2 \yields {E_2} \andalso \turnsrec A_2}
% {\Gamma \turns e_1 \mergeop e_2 : A_1 \inter A_2 \yields {\pair {E_1} {E_2}}}
%
% R{\'e}my~\cite{remy1989type}

%*******************************************************************************
\paragraph{Trait calculi.}
%*******************************************************************************
Fisher and Reppy~\cite{fisher2004typed} provided a dedicated statically typed
calculus for modeling traits. \name is not dedicated to traits; but rather, it
supports a source language that models traits. Compared to Fisher and Reppy's
calculus, \name is more lightweight. For example, self reference is not in the
language of \name. One reason for the difference is that Fisher and Reppy's
calculus supports \emph{classes} in addition to traits, and considers the
interaction between them, whereas our object oriented source language is
\emph{prototype}-based---the mechanism for code reuse is purely trait.

\section{Conclusion and Future Work}
\label{sec:conclusion}

This paper described \name: a language that combines
intersection types and a merge operator.
The language is proved to be type-safe and coherent.
To ensure coherence the type system accepts only
disjoint intersections. We believe that disjoint intersection types are
intuitive, and at the same time expressive. We have shown the
applicability of disjoint intersection types to model a simple form of traits.

We implemented the core functionalities of the \name as part of a JVM-based
compiler. Based on the type system of \name, we have built an ML-like
source language compiler that offers interoperability with Java (such as object
creation and method calls). The source language is loosely based on the more
general System $F_{\omega}$ and supports a
number of other features, including records, polymorphism, mutually recursive
\code{let} bindings, type aliases, algebraic data types, pattern matching, and
first-class modules that are encoded using \code{letrec} and records.

For the future, we intend to improve our source language
and show the power of disjoint intersection types in large case
studies. One pressing challenge is to address the intersction between 
disjoint intersection types and polymorphism.
We are also interested in extending our work
to systems with a $\top$ type. This will also require an adjustment
to the notion of disjoint types. A suitable notion of
disjointness between two types $A$ and $B$ in the presence of $\top$
would be to require that the only common supertype of $A$ and $B$ is $\top$.
Finally we would like to study the
addition of union types. This will also require changes in our
notion of disjointness, since with union types there always exists
a type $A \union B$, which is the common supertype of two
types $A$ and $B$.

% Some immediate topics for
% further improvement of the results in this paper are discussed next.
%
% \paragraph{Union types.}
%
% If a type system ever contains union types (the counterpart of intersection
% types), with the following standard subtyping rules,
% \begin{mathpar}
%   \inferrule* [right=Sub\_Union\_1]
%     { }
%     {A \subtype A \union B}
%
%   \inferrule* [right=Sub\_Union\_2]
%     { }
%     {B \subtype A \union B}
% \end{mathpar}
% then no two types $A$ and $B$ can ever be disjoint, since there always exists
% the type $A \union B$, which is their common supertype. So it is reasonable to
% conjecture that such system cannot be coherent.
% \bruno{I wouldn't say this is a motivation: it sounds like we caould
%   not support union types, when I think this is not true. For example
% we could say something like: there does not exist an \emph{atomic} C ...}
%
%
% \paragraph{Implementation.}
%
% We implemented the core functionalities of the \name as part of a JVM-based
% compiler. Based on the type system of \name, we built an ML-like
% source language compiler that offers interoperability with Java (such as object
% creation and method calls). The source language is loosely based on the more
% general System $F_{\omega}$ and supports a
% number of other features, including records, mutually recursive
% \code{let} bindings, type aliases, algebraic data types, pattern matching, and
% first-class modules that are encoded using \code{letrec} and records.
%
% Relevant to this paper are the three phases in the compiler, which
% collectively turn source programs into System $F$:
%
% \begin{enumerate}
% \item A \emph{typechecking} phase that checks the usage of \name features and
%   other source language features against an abstract syntax tree that follows
%   the source syntax.
%
% \item A \emph{desugaring} phase that translates well-typed source terms into
%   \name terms. Source-level features such as multi-field records, type aliases
%   are removed at this phase. The resulting program is just an \name term
%   extended with some other constructs necessary for code generation.
%
% \item A \emph{translation} phase that turns well-typed \name terms into System
%   $F$ ones.
% \end{enumerate}
%
% Phase 3 is what we have formalized in this paper.
%
%
% \paragraph{Reduce the number of coercions.}
%
% Our translation inserts a coercion (many of them are identity functions)
% whenever subtyping occurs during a function application, which could mean
% notable run-time overhead. In the current implementation, we introduced a
% partial evaluator with three simple rewriting rules to eliminate the redundant
% identity functions as another compiler phase after the translation. In another
% version of our implementation, partial evaluation is weaved into the process of
% translation so that the unwanted identity functions are not introduced during
% the translation. Besides, since the order of the two types in a binary
% intersection does not matter, we may normalize them to avoid unnecessary
% coercions.


\acks

Acknowledgments, if needed.

\appendix

\input{sections/proof.tex}

% We recommend abbrvnat bibliography style.

\bibliographystyle{abbrvnat}

% The bibliography should be embedded for final submission.

\bibliography{references}

\begin{thebibliography}{}
\softraggedright

\bibitem[Smith et~al.(2009)Smith, Jones]{smith02}
P. Q. Smith, and X. Y. Jones. ...reference text...

\end{thebibliography}

\end{document}
