\documentclass[preprint]{sigplanconf}

% For pdflatex, replaced by fontspec:
\usepackage[T1]{fontenc}
\usepackage[utf8]{inputenc}


%% For general writing
\usepackage{fixltx2e}
\usepackage[usenames,dvipsnames,svgnames,table]{xcolor}
\usepackage{url}
\usepackage{fancyvrb}
\usepackage{mdwlist}  % Miscellaneous list-related commands
\usepackage{xspace}   % Smart spacing
\usepackage{ucs}

% https://www.nesono.com/?q=book/export/html/347
% Package for inserting TODO statements in nice colorful boxes - so that you
% won’t forget to fix/remove them. To add a todo statement, use something like
% \todo{Find better wording here}.
\usepackage{todonotes}


%% Math & theoretical computer science
\usepackage{amsmath}
\usepackage{amssymb}
\usepackage{amsthm}
\usepackage{bm}         % Bold symbols in maths mode
\usepackage{dsfont}
\usepackage{stmaryrd}
\usepackage{mathtools}  % For "::=" ( \Coloneqq )
% http://tex.stackexchange.com/questions/114151/how-do-i-reference-in-appendix-a-theorem-given-in-the-body
\usepackage{thmtools, thm-restate}

\theoremstyle{definition}
\newtheorem{definition}{Definition}

\theoremstyle{plain}
\newtheorem{theorem}{Theorem}
\newtheorem{lemma}{Lemma}


%% Code listings
\usepackage{listings}

\lstdefinestyle{f2j}{
    basicstyle=\ttfamily\small,
    keywordstyle=\sffamily\bfseries,
    tabsize=2,
    keepspaces=true,
    showstringspaces=false,
    escapeinside={(*}{*)},
    morekeywords={let,in}
}

\lstset{style=f2j}


%% Font
\usepackage[euler-digits,euler-hat-accent]{eulervm}

%% Typesetting inference rules
% \usepackage{styles/bcprules}    % by Benjamin C. Pierce
% \usepackage{styles/cmll}
\usepackage{styles/mathpartir}  % by Didier Rémy (http://gallium.inria.fr/~remy/latex/mathpartir.html


% Copied from the FCore paper:
\usepackage[colorlinks=true,allcolors=black,breaklinks,draft=false]{hyperref}   % hyperlinks, including DOIs and URLs in bibliography
% known bug: http://tex.stackexchange.com/questions/1522/pdfendlink-ended-up-in-different-nesting-level-than-pdfstartlink

\newcommand{\hast}{\!:\!}

% Relations
\newcommand{\subtype}   {<:}

\definecolor{facebook}{HTML}{3B5998}
\newcommand{\yields}[1]{\textcolor{facebook}{\; \hookrightarrow {#1}}}

% Helpers
\newcommand{\ftv}[1]{\textit{ftv}({#1})}

% Spacing
\newcommand{\binderSpacing}{\,}
\newcommand{\appSpacing}{\;}

% Types
% \newcommand{\top}{\{\}}
\newcommand{\andOp}{\with}

% Expressions
\newcommand{\lam}[3]{\lambda (#1 \hast #2).\binderSpacing #3}
\newcommand{\mergeOp}{,,}
\newcommand{\restrictOp}{\setminus}
\newcommand{\recordUpdate}[3]{#1 \; \mathbf{with} \; \{#2 = #3\}}


\newcommand{\Int}{\code{Int}}
\newcommand{\String}{\code{String}}
\newcommand{\Bool}{\code{Bool}}
\newcommand{\I}{\code{i}}
\newcommand{\J}{\code{j}}


% Rules

% Couleurs
\colorlet{subColor}{OliveGreen}
\colorlet{targetColor}{BrickRed}

% Subtyping labels
\newcommand{\ruleLabelSub}{\bm{\textcolor{subColor}{sub}}}
\newcommand{\ruleLabelSubVar}{\ruleLabelSub\text{var}}
\newcommand{\ruleLabelSubTop}{\ruleLabelSub\text{top}}
\newcommand{\ruleLabelSubFun}{\ruleLabelSub\text{fun}}
\newcommand{\ruleLabelSubForall}{\ruleLabelSub\text{forall}}
\newcommand{\ruleLabelSubAnd}{\ruleLabelSub\text{and}}
\newcommand{\ruleLabelSubAndLeft}{\ruleLabelSub{\text{and}_1}}
\newcommand{\ruleLabelSubAndRight}{\ruleLabelSub{\text{and}_2}}
\newcommand{\ruleLabelSubRec}{\ruleLabelSub\text{rec}}

% Source/elaboration and labels
\newcommand{\judgeSourceWF}[2]{#1 \; \textcolor{sourceColor}{\turns} \; #2}
\newcommand{\ruleLabelSourceRecUpd}{\ruleLabelSource\text{rec-upd}}


% Presentation
\definecolor{lightyellow}{HTML}{FFFFE0}


% To be retired
\newcommand{\turnsGet}{\turns_{\textrm{get}}}
\newcommand{\turnsPut}{\turns_{\textrm{put}}}
\newcommand{\turnsrec}{\turns_{\textrm{rec}}}
\newcommand{\rulename}[1]{(\textrm{#1})}



\begin{mathpar}
\framebox{$ A \orthog A $}

\ruleOrthogVar

\ruleOrthogFun

\ruleOrthogForall

\ruleOrthogAndLeft

\ruleOrthogAndRight

% \ruleOrthogRecLab

% \ruleOrthogRec

\end{mathpar}

\newcommand{\rulelabelSub}{\text{Sub\_}}
\newcommand{\rulelabelsubvar}{\rulelabelSub\text{Var}}
\newcommand{\rulelabelSubTop}{\rulelabelSub\text{Top}}
\newcommand{\rulelabelsubfun}{\rulelabelSub\text{Fun}}
\newcommand{\rulelabelsubforall}{\rulelabelSub\text{Forall}}
\newcommand{\rulelabelsubinter}{\rulelabelSub\text{And}}
\newcommand{\rulelabelsubinterl}{\rulelabelSub{\text{And}_1}}
\newcommand{\rulelabelsubinterr}{\rulelabelSub{\text{And}_2}}

\newcommand{\rulesubvar}{
\inferrule* [right=$\rulelabelsubvar$]
  { }
  {\alpha \subtype \alpha \yields {\lamty x {\im \alpha} x}}
}

\newcommand{\rulesubfun}{
\inferrule* [right=$\rulelabelsubfun$]
  {A_3 \subtype A_1 \yields {C_1} \\ A_2 \subtype A_4 \yields {C_2}}
  {A_1 \to A_2 \subtype A_3 \to A_4
  \yields
      {\lamty f {\im {A_1 \to A_2}}
      {\lamty x {\im {A_3}}
          {\app {C_2} {(\app f {(\app {C_1} x)})}}}}}
}

\newcommand{\rulesubforall}{
\inferrule* [right=$\rulelabelsubforall$]
  {A_1 \subtype \subst {\alpha_1} {\alpha_2} A_2 \yields C}
  {\for {\alpha_1} A_1 \subtype \for {\alpha_2} A_2
    \yields
      {\lamty f {\im {\for {\alpha_1} A_1}}
        {\blam \alpha {\app C {(\app f \alpha)}}}}}
}

\newcommand{\rulesubinter}{
\inferrule* [right=$\rulelabelsubinter$]
  {A_1 \subtype A_2 \yields {C_1} \\ A_1 \subtype A_3 \yields {C_2}}
  {A_1 \subtype A_2 \inter A_3
    \yields
      {\lamty x {\im {A_1}}
        {\pair {\app {C_1} x} {\app {C_2} x}}}}
}

\newcommand{\rulesubinterl}{
\inferrule* [right=$\rulelabelsubinterl$]
  {A_1 \subtype A_3 \yields C}
  {A_1 \inter A_2 \subtype A_3
    \yields
      {\lamty x {\im {A_1 \inter A_2}}
        {\app C {(\proj 1 x)}}}}
}

\newcommand{\rulesubinterr}{
\inferrule* [right=$\rulelabelsubinterr$]
  {A_2 \subtype A_3 \yields C}
  {A_1 \inter A_2 \subtype A_3
    \yields
      {\lamty x {\im {A_1 \inter A_2}}
        {\app C {(\proj 2 x)}}}}
}

\newcommand{\rulelabel}{\text{Ty}}
\newcommand{\rulelabelSelect}{\text{Sel}}
\newcommand{\rulelabelRestrict}{\text{Res}}

% Var
\newcommand{\rulelabelVar}{\rulelabel\text{Var}}
\newcommand{\ruleVar} {
\inferrule* [right=$\rulelabelVar$]
  {x \hast A \in \Gamma}
  {\hastype \Gamma x A \yields x}
}

% Top
\newcommand{\rulelabelTop}{\rulelabel\text{Top}}
\newcommand{\ruleTop} {
\inferrule* [right=$\rulelabelTop$]
  { }
  {\hastype \Gamma \top \top \yields {()}}
}

% Lam
\newcommand{\rulelabelLam}{\rulelabel\text{Lam}}
\newcommand{\ruleLam} {
\inferrule* [right=$\rulelabelLam$]
  {\istype \Gamma A \\ \hastype {\Gamma, x \hast A} e B \yields E}
  {\hastype \Gamma {\lam x A e} {A \to B} \yields {\lam x {\im A} E}}
}

% App
\newcommand{\rulelabelApp}{\rulelabel\text{App}}
\newcommand{\ruleApp}{
\inferrule* [right=$\rulelabelApp$]
  {\hastype \Gamma {e_1} {A_1 \to A_2} \yields {E_1} \\
   \hastype \Gamma {e_2} {A_3} \yields {E_2} \\
   A_3 \subtype A_1 \yields C}
  {\hastype \Gamma {\app {e_1} {e_2}} {A_2} \yields {\app {E_1} {(\app C E_2)}}}
}

% BLam
\newcommand{\rulelabelBLam}{\rulelabel\text{BLam}}
\newcommand{\ruleBLam}{
\inferrule* [right=$\rulelabelBLam$]
  {\hastype {\Gamma, \alpha} e A \yields E}
  {\hastype \Gamma {\blam \alpha e} {\for \alpha A} \yields {\blam \alpha E}}
}

% TApp
\newcommand{\rulelabelTApp}{\rulelabel\text{TApp}}
\newcommand{\ruleTApp}{
\inferrule* [right=$\rulelabelTApp$]
  {\hastype \Gamma e {\for \alpha B} \yields E \\ \istype \Gamma A}
  {\hastype \Gamma {\tapp e A} {\subst A \alpha B} \yields {\tapp E {\im A}}}
}

% Merge
\newcommand{\rulelabelMerge}{\rulelabel\text{Merge}}
\newcommand{\ruleMerge}{
\inferrule* [right=$\rulelabelMerge$]
  {\hastype \Gamma {e_1} A \yields {E_1} \\
   \hastype \Gamma {e_2} B \yields {E_2} \\
   % A \bot B}
   \isdisjoint \Gamma A B}
  {\hastype \Gamma {e_1 \mergeOp e_2} {A \intersect B} \yields {\pair {E_1} {E_2}}}
}

% ConstraintIntro
\newcommand{\rulelabelConstraintIntro}{\rulelabel\text{ConstraintIntro}}
\newcommand{\ruleConstraintIntro}{
  \inferrule* [right=$\rulelabelConstraintIntro$]
    {\istype \Gamma {A_1} \\ \istype \Gamma {A_2} \\
     \hastype {\Gamma, A_1 \disjoint A_2} e B \yields E}
    {\hastype \Gamma {\assume {(A_1 \disjoint A_1)} e} {\constraints {A_1   \disjoint A_2} B} \yields E}
}

% ConstraintElim
\newcommand{\rulelabelConstraintElim}{\rulelabel\text{ConstraintElim}}
\newcommand{\ruleConstraintElim}{
\inferrule* [right=$\rulelabelConstraintElim$]
  {\hastype \Gamma e {\constraints {A_1 \disjoint A_2} B} \yields E \\
  \isdisjoint \Gamma {A_1} {A_2}}
  {\hastype \Gamma {\app e {\_}} B \yields E}
}

% rec-con
\newcommand{\rulelabelRecConstruct}{\rulelabel\text{rec-construct}}
\newcommand{\rulerecordConstruct}{
\inferrule* [right=$\rulelabelRecConstruct$]
  {\hastype \Gamma e A \yields E}
  {\hastype \Gamma {\recordCon l e} {\recordType l A} \yields E}
}

% rec-select
\newcommand{\rulelabelRecSelect}{\rulelabel\text{rec-select}}
\newcommand{\ruleRecSelect}{
\inferrule* [right=$\rulelabelRecSelect$]
  {\hastype \Gamma e A \yields E \\
   \judgeSelect A l {A_1} \yields C}
  {\hastype \Gamma {e.l} {A_1} \yields {\app C E}}
}

% rec-restrict
\newcommand{\rulelabelRecRestrict}{\rulelabel\text{rec-restrict}}
\newcommand{\ruleRecRestrict}{
\inferrule* [right=$\rulelabelRecRestrict$]
  {\hastype \Gamma e A \yields E \\
   \judgeRestrict A l {A_1} \yields C}
  {\hastype \Gamma {e \restrictOp l} {A_1} \yields {\app C E}}
}

\newcommand{\judgeSelect}[3]{#1 \bullet #2 = #3}

% select
\newcommand{\ruleGet}{
  \inferrule* [right=$\rulelabelSelect$]
  { }
  {\judgeSelect {\recordType l A} l A \yields {\lam x {\im {\recordType l A}} x}}
}

% select1
\newcommand{\rulelabelSelectLeft}{{\rulelabelSelect}_1}
\newcommand{\ruleGetLeft}{
  \inferrule* [right=$\rulelabelSelectLeft$]
  {\judgeSelect {A_1} l A \yields C}
  {\judgeSelect {A_1 \intersect A_2} l A \yields {\lam x {\im {A_1
          \intersect A_2}} {\app C {(\proj 1 x)}}}}
}

% select2
\newcommand{\rulelabelSelectRight}{{\rulelabelSelect}_2}
\newcommand{\ruleGetRight}{
  \inferrule* [right=$\rulelabelSelectRight$]
  {\judgeSelect {A_2} l A \yields C}
  {\judgeSelect {A_1 \intersect A_2} l A \yields {\lam x {\im {A_1
          \intersect A_2}} {\app C {(\proj 2 x)}}}}
}

\newcommand{\judgeRestrict}[3]{#1 \bm{\restrictOp} #2 = #3}

% restrict
\newcommand{\ruleRestrict}{
  \inferrule* [right=$\rulelabelRestrict$]
  { }
  {\judgeRestrict {\recordType l A} l \top \yields {\lam x {\im {\recordType l A}} {()}}}
}

% restrict1
\newcommand{\rulelabelRestrictleft}{{\rulelabelRestrict}_1}
\newcommand{\ruleRestrictLeft}{
  \inferrule* [right=$\rulelabelRestrictleft$]
  {\judgeRestrict {A_1} l A \yields C}
  {\judgeRestrict {A_1 \intersect A_2} l {A \intersect A_2} \yields {\lam x {\im {A_1
          \intersect A_2}} {\pair {\app C {(\proj 1 x)}} {\proj 2 x}}}}
}

% restrict2
\newcommand{\rulelabelRestrictRight}{{\rulelabelRestrict}_2}
\newcommand{\ruleRestrictRight}{
  \inferrule* [right=$\rulelabelRestrictRight$]
  {\judgeRestrict {A_2} l A \yields C}
  {\judgeRestrict {A_1 \intersect A_2} l {A_1 \intersect A} \yields {\lam x {\im {A_1
          \intersect A_2}} {\pair {\proj 1 x} {\app C {(\proj 2 x)}}}}}
}

% \newcommand{\judgeTargetWF}[2]{#1 \; \textcolor{targetColor}{\turns} \; #2}
\newcommand{\judgeTarget}[3]{#1 \; \textcolor{targetColor}{\turns} \; #2 \; \textcolor{targetColor}{:} \; #3}
\newcommand{\ruleLabelTarget}{\bm{\textcolor{targetColor}{T}}}

\newcommand{\ruleLabelTargetvar}{\ruleLabelTarget\text{var}}
\newcommand{\ruleTargetVar} {
\inferrule* [right=$\ruleLabelTargetvar$]
  {(x,T) \in \Gamma}
  {\judgeTarget \Gamma x T}
}

\newcommand{\ruleLabelTargetUnit}{\ruleLabelTarget\text{unit}}
\newcommand{\ruleTargetUnit} {
\inferrule* [right=$\ruleLabelTargetUnit$]
  { }
  {\judgeTarget \Gamma {()} {()}}
}

\newcommand{\ruleLabelTargetlam}{\ruleLabelTarget\text{lam}}
\newcommand{\ruleTargetLam} {
\inferrule* [right=$\ruleLabelTargetlam$]
  {\judgeTarget {\Gamma, x \hast T} E {T_1} \andalso \judgeTargetWF \Gamma T}
  {\judgeTarget \Gamma {\lam x T E} {T \to T_1}}
}

\newcommand{\ruleLabelTargetApp}{\ruleLabelTarget\text{app}}
\newcommand{\ruleTargetApp}{
\inferrule* [right=$\ruleLabelTargetApp$]
  {\judgeTarget \Gamma {E_1} {T_1 \to T_2} \andalso \judgeTarget \Gamma {E_2} {T_1}}
  {\judgeTarget \Gamma {\app {E_1} {E_2}} {T_2}}
}

\newcommand{\ruleLabelTargetBLam}{\ruleLabelTarget\text{blam}}
\newcommand{\ruleTargetBLam}{
\inferrule* [right=$\ruleLabelTargetBLam$]
  {\judgeSource {\Gamma, \alpha} E T}
  {\judgeSource \Gamma {\blam \alpha E} {\for \alpha T}}
}

\newcommand{\ruleLabelTargetTApp}{\ruleLabelTarget\text{tapp}}
\newcommand{\ruleTargetTApp}{
\inferrule* [right=$\ruleLabelTargetTApp$]
  {\judgeTarget \Gamma E {\for \alpha {T_1}} \andalso \judgeTargetWF \Gamma T}
  {\judgeTarget \Gamma {\tapp E T} {\subst T \alpha T_1}}
}

\newcommand{\ruleLabelTargetPair}{\ruleLabelTarget\text{pair}}
\newcommand{\ruleTargetPair}{
\inferrule* [right=$\ruleLabelTargetPair$]
  {\judgeTarget \Gamma {E_1} {T_1} \andalso \judgeTarget \Gamma {E_2} {T_2}}
  {\judgeTarget \Gamma {\pair {E_1} {E_2}} {\pair {T_1} {T_2}}}
}

\newcommand{\ruleLabelTargetProjLeft}{\ruleLabelTarget\text{proj}_1}
\newcommand{\ruleTargetProjLeft}{
\inferrule* [right=$\ruleLabelTargetProjLeft$]
  {\judgeTarget \Gamma E {\pair {T_1} {T_2}}}
  {\judgeTarget \Gamma {\proj 1 E} {T_1}}
}

\newcommand{\ruleLabelTargetProjRight}{\ruleLabelTarget\text{proj}_2}
\newcommand{\ruleTargetProjRight}{
\inferrule* [right=$\ruleLabelTargetProjRight$]
  {\judgeTarget \Gamma E {\pair {T_1} {T_2}}}
  {\judgeTarget \Gamma {\proj 2 E} {T_2}}
}
\colorlet{wfcolor}{BrickRed}


\newcommand{\rulelabelWF}{\bm{\textcolor{wfcolor}{wf}}}

\newcommand{\rulelabelWFVar}{\rulelabelWF\text{var}}
\newcommand{\rulelabelWFTop}{\rulelabelWF\text{top}}
\newcommand{\rulelabelWFFun}{\rulelabelWF\text{fun}}
\newcommand{\rulelabelWFForall}{\rulelabelWF\text{forall}}
\newcommand{\rulelabelWFAnd}{\rulelabelWF{\text{and}}}
\newcommand{\rulelabelWFRec}{\rulelabelWF\text{rec}}

\newcommand{\ruleWF}{
\inferrule* [right=$\rulelabelWF$]
  {\ftv \A \in \gamma}
  {\istype \gamma \A}
}

\newcommand{\ruleLabelTargetWF}{\ruleLabelTarget\text{wf}}
\newcommand{\ruleTargetWF}{
\inferrule* [right=$\ruleLabelTargetWF$]
  {\ftv T \in \Gamma}
  {\judgeTargetWF \Gamma T}
}

% Expanded form of well-formedness

\newcommand{\ruleWFVar}{
\inferrule* [right=$\rulelabelWFVar$]
  {\alpha \in \gamma}
  {\istype \gamma \alpha}
}

\newcommand{\ruleWFTop}{
\inferrule* [right=$\rulelabelWFTop$]
  { }
  {\istype \gamma \top}
}

\newcommand{\ruleWFFun}{
\inferrule* [right=$\rulelabelWFFun$]
  {\istype \gamma {\A_1} \\ \istype \Gamma {\A_2}}
  {\istype \gamma {\A_1 \to \A_2}}
}

\newcommand{\ruleWFForall}{
\inferrule* [right=$\rulelabelWFForall$]
  {\istype {\gamma, \alpha} \A}
  {\istype \gamma {\for \alpha \A}}
}

\newcommand{\ruleWFAnd}{
\inferrule* [right=$\rulelabelWFAnd$]
  {\istype \gamma {\A_1} \\ \istype \Gamma {\A_2}}
  {\istype \gamma {\A_1 \intersect \A_2}}
}

\newcommand{\ruleWFRec}{
\inferrule* [right=$\rulelabelWFRec$]
  {\istype \gamma \A}
  {\istype \gamma {\recordType l \A}}
}


\newcommand{\name}{{\bf $F_{\&}$}\xspace}

\newcommand{\target}{{\bf f}\xspace}
\newcommand{\Target}{{\bf f}\xspace}

\begin{document}

\special{papersize=8.5in,11in}
\setlength{\pdfpageheight}{\paperheight}
\setlength{\pdfpagewidth}{\paperwidth}

\title{\name}

% Coherence for well-typed terms.

\begin{figure}
  \[
  \begin{array}{l}
    \begin{array}{llrll}
      \text{Types}
      & A, B, C, D & \Coloneqq & \alpha     & \text{Type variable} \\
      &         & \mid & \top            & \text{Top type} \\
      &         & \mid & A \to B         & \text{Function type} \\
      &         & \mid & \for \alpha A   & \text{Universal quantification} \\
      &         & \mid & A \intersect B  & \text{Intersection type} \\
      &         & \mid & \constraints {A \disjoint B} C & \text{Disjoint constraint} \\
      % &         & \mid & \recordType l A & \text{Record type} \\

      \\
      \text{Expressions}
      & e & \Coloneqq & x            & \text{Variable} \\
      &   & \mid & \top              & \text{Top} \\
      &   & \mid & \lam x A e        & \text{Lambda} \\
      &   & \mid & \app {e_1} {e_2}  & \text{Application} \\
      &   & \mid & \blam \alpha e    & \text{Big lambda} \\
      &   & \mid & \tapp e A         & \text{Type application} \\
      &   & \mid &  e_1 \mergeOp e_2 & \text{Merge} \\
      % &   & \mid & {\_}                 & \text{Disjointness evidence}
      &   & \mid & \assume {(A \disjoint B)} e & \text{Constraint intro} \\
      &   & \mid & \app e {\_}                 & \text{Constraint elim} \\
      % &   & \mid & \recordCon l e    & \text{Record} \\
      % &   & \mid & e.l               & \text{Record selection} \\
      % &   & \mid & e \restrictOp l   & \text{Record restriction} \\

      \\
      \text{Contexts}
      & \Gamma & \Coloneqq & \epsilon         & \\
      &        & \mid & \Gamma, \alpha        & \\
      &        & \mid & \Gamma, x \hast A     & \\
      &        & \mid & \Gamma, A \disjoint B & \\

      % \\
      % \text{Labels} & l
    \end{array}
  \end{array}
  \]
  \caption{Syntax.}
\end{figure}

% \begin{figure*}
%   \caption{Disjointness between types.}
% \end{figure*}

% \begin{figure*}
%   \colorlet{wfcolor}{BrickRed}


\newcommand{\rulelabelWF}{\bm{\textcolor{wfcolor}{wf}}}

\newcommand{\rulelabelWFVar}{\rulelabelWF\text{var}}
\newcommand{\rulelabelWFTop}{\rulelabelWF\text{top}}
\newcommand{\rulelabelWFFun}{\rulelabelWF\text{fun}}
\newcommand{\rulelabelWFForall}{\rulelabelWF\text{forall}}
\newcommand{\rulelabelWFAnd}{\rulelabelWF{\text{and}}}
\newcommand{\rulelabelWFRec}{\rulelabelWF\text{rec}}

\newcommand{\ruleWF}{
\inferrule* [right=$\rulelabelWF$]
  {\ftv \A \in \gamma}
  {\istype \gamma \A}
}

\newcommand{\ruleLabelTargetWF}{\ruleLabelTarget\text{wf}}
\newcommand{\ruleTargetWF}{
\inferrule* [right=$\ruleLabelTargetWF$]
  {\ftv T \in \Gamma}
  {\judgeTargetWF \Gamma T}
}

% Expanded form of well-formedness

\newcommand{\ruleWFVar}{
\inferrule* [right=$\rulelabelWFVar$]
  {\alpha \in \gamma}
  {\istype \gamma \alpha}
}

\newcommand{\ruleWFTop}{
\inferrule* [right=$\rulelabelWFTop$]
  { }
  {\istype \gamma \top}
}

\newcommand{\ruleWFFun}{
\inferrule* [right=$\rulelabelWFFun$]
  {\istype \gamma {\A_1} \\ \istype \Gamma {\A_2}}
  {\istype \gamma {\A_1 \to \A_2}}
}

\newcommand{\ruleWFForall}{
\inferrule* [right=$\rulelabelWFForall$]
  {\istype {\gamma, \alpha} \A}
  {\istype \gamma {\for \alpha \A}}
}

\newcommand{\ruleWFAnd}{
\inferrule* [right=$\rulelabelWFAnd$]
  {\istype \gamma {\A_1} \\ \istype \Gamma {\A_2}}
  {\istype \gamma {\A_1 \intersect \A_2}}
}

\newcommand{\ruleWFRec}{
\inferrule* [right=$\rulelabelWFRec$]
  {\istype \gamma \A}
  {\istype \gamma {\recordType l \A}}
}

%   \caption{Well-formedness of types.}
% \end{figure*}

% \begin{figure*}
% \begin{mathpar}
% \begin{array}{l}
%   \begin{array}{llrl}
%     \text{Values} & v & \Coloneqq & \top \mid \lam x \tau e \mid \blam \alpha e \mid v_1 \mergeOp v_2 \mid \recordCon l e
%   \end{array}
% \end{array}
% \end{mathpar}
%
%   \caption{Values.}
% \end{figure*}

% \begin{figure*}

%   \begin{mathpar}
%     \begin{array}{lcl}
%       \fields {v_1 \mergeOp v_2} &=& \fields {v_1} \concatOp \fields {v_2} \\
%       \fields {\recordCon l e}   &=& [(l, e)] \\
%       \fields v                  &=& []
%     \end{array}
%   \end{mathpar}
%   \caption{\code{fields}.}
% \end{figure*}

% \begin{figure*}
%   \begin{mathpar}
%     \begin{array}{lcl}
%       \remove {\recordCon l e} l &=& \top \\
%       \remove {\recordCon l e \mergeOp v_2} l &=& v_2 \\
%       \remove {\recordCon l e \mergeOp v_2} {l'} &=& \recordCon l e \mergeOp \remove {v_2} {l'} \quad \quad (l \neq l') \\
%       \remove {v_1 \mergeOp \recordCon l e} l &=& v_1 \\
%       \remove {v_1 \mergeOp \recordCon l e} {l'} &=& \remove {v_1} {l'} \mergeOp \recordCon l e \quad \quad (l \neq l') \\

%       \remove v l                  &=& v
%     \end{array}

%   \end{mathpar}

%   \caption{\code{remove}.}
% \end{figure*}

% \begin{figure*}
%   \begin{mathpar}
%     \inferrule* [right=Cast/UpCast]
%       {\tau_1 \subtype \tau}
%       {\cast \tau {\withType v {\tau_1}} \hookrightarrow v}
%
%     \inferrule* [right=Cast/TakeLeft]
%       {\cast \tau {\withType {v_1} {\tau_1}} \hookrightarrow v}
%       {\cast \tau {\withType {v_1 \mergeOp v_2} {\tau_1 \intersect \tau_2}} \hookrightarrow v}
%
%     \inferrule* [right=Cast/TakeRight]
%       {\cast \tau {\withType {v_2} {\tau_2}} \hookrightarrow v}
%       {\cast \tau {\withType {v_1 \mergeOp v_2} {\tau_1 \intersect \tau_2}} \hookrightarrow v}
%   \end{mathpar}
%
%   \caption{Casts.}
% \end{figure*}

% \begin{figure*}
%   \begin{mathpar}
%     \inferrule* [right=Dyn/Val]
%       { }
%       {v \Downarrow v}
%
%     \inferrule* [right=Dyn/App]
%       {e_1 \Downarrow \lam x \tau e \\
%        e_2 \Downarrow v_2 \\
%        \cast \tau {\withType {v_2} {\tau_2}} \hookrightarrow v_3 \\
%        \subst {v_3} x e \Downarrow v}
%       {\app {e_1} {\withType {e_2} {\tau_2}} \Downarrow v}
%
%     \inferrule* [right=Dyn/TApp]
%       {e_1 \Downarrow \for \alpha e \\
%        \subst \tau \alpha e \Downarrow v}
%       {\tapp {e_1} \tau \Downarrow v}
%
%     \inferrule* [right=Dyn/Merge]
%       {e_1 \Downarrow v_1 \\ e_2 \Downarrow v_2}
%       {e_1 \mergeOp e_2 \Downarrow v_1 \mergeOp v_2}
%
%     % \inferrule* [right=Dyn/RecSelect]
%     %   {e \Downarrow v \\
%     %    (l, e_1) \; \code{`uniqueElem`} \; \fields v \\
%     %    e_1 \Downarrow v_1}
%     %   {e.l \Downarrow v_1}
%
%     % \inferrule* [right=Dyn/RecRestrict]
%     %   {e \Downarrow v \\
%     %    (l, e_1) \; \code{`uniqueElem`} \; \fields v}
%     %   {e \restrictOp l \Downarrow v \; \code{`remove`} \; l}
%   \end{mathpar}
%
%   \caption{Dynamic semantics.}
% \end{figure*}
%
% \begin{figure*}
%   \framebox{$\im \tau = T$}

\begin{align*}
  \im \alpha                    &= \alpha \\
  \im \top                      &= () \\
  \im {\tau_1} \to \im {\tau_2} &= \im {\tau_1} \to \im {\tau_2} \\
  \im {\for \alpha \tau}        &= \for \alpha \im \tau \\
  \im {\tau_1 \intersect \tau_2} &= \pair {\im {\tau_1}} {\im {\tau_2}} \\
\end{align*}

%   \caption{Type translation.}
% \end{figure*}

\begin{figure}
  \begin{mathpar}
  \framebox{$ A \subtype B \yields F $} \\
  \subVar \\
  \subTop \\
  \subFun \\
  \subForall \\
  \subAnd \\
  \subAndleft \\
  \subAndright \\
  \subConstraint
  \end{mathpar}
  \caption{Subtyping.}
\end{figure}

\begin{figure}
  \begin{mathpar}
    \framebox{$\isdisjoint \Gamma A B$} \\

    \inferrule* [right=DisjointRefl]
      {\isdisjoint \Gamma A B}
      {\isdisjoint \Gamma B A}

    \inferrule* [right=DisjointSub]
      {\isdisjoint \Gamma A B \\ A' \subtype A \\ B' \subtype B}
      {\isdisjoint \Gamma {A'} {B'}}

  \end{mathpar}
  \caption{Disjointness.}
\end{figure}

\begin{figure}
  \begin{mathpar}
    \framebox{$ \hastype \Gamma e A \yields E $} \\
    \ruleVar \and
    \ruleTop \and
    \ruleLam \and
    \ruleApp \and
    \ruleBLam \and
    \ruleTApp \and
    \ruleMerge \and
    \ruleConstraintIntro \and
    \ruleConstraintElim
    % \rulerecordConstruct \and
    % \ruleRecSelect \and \ruleRecRestrict
  \end{mathpar}

  % % Selection
  % \begin{mathpar}
  %   \framebox{$\judgeSelect {\tau_1} l \tau_2 \yields C$} \and
  %   \ruleGet \and \ruleGetLeft \and \ruleGetRight
  % \end{mathpar}
  %
  % % Restriction
  % \begin{mathpar}
  %   \framebox{$\judgeRestrict {\tau_1} l \tau_2 \yields C$} \and
  %   \ruleRestrict \and \ruleRestrictLeft \and \ruleRestrictRight
  % \end{mathpar}

  \caption{Typing.}
\end{figure}

\begin{mathpar}
  \inferrule
  {}
  {\hastype \epsilon {1 \mergeOp 2} {\constraints {\integer \disjoint \integer} \integer \intersect \integer}}
\end{mathpar}

\begin{definition}{(Disjointness)}
Two sets $S$ and $T$ are \emph{disjoint} if there does not exist an element $x$, such that $x \in S$ and $x \in T$.
\end{definition}

\begin{definition}{(Disjointness)}
Two types $A$ and $B$ are \emph{disjoint} if there does not exist an expression $e$, which is not a merge, such that $\hastype \epsilon e A'$, $\hastype \epsilon e B'$, $A' \subtype A$, and $B' \subtype B$.
\end{definition}

\begin{definition}{(Disjointness)}
$A \bot B = \not \exists C. A <: C \wedge B <: C$ \\

Two types $A$ and $B$ are \emph{disjoint} if their least common supertype is $\top$.
\end{definition}

% \begin{figure}
%   % Typing
%   \begin{mathpar}
%     \framebox{$ \hastype \Gamma e A \yields E $} \\
%     \ruleVar \and
%     \ruleTop \and
%     \ruleLam \and
%     \ruleApp \and
%     \ruleBLam \and
%     \ruleTApp \and
%     \ruleMerge \and
%     \ruleDisjointAssume \and
%     \ruleDisjointCheck
%     % \rulerecordConstruct \and
%     % \ruleRecSelect \and \ruleRecRestrict
%   \end{mathpar}
%
%   % % Selection
%   % \begin{mathpar}
%   %   \framebox{$\judgeSelect {\tau_1} l \tau_2 \yields C$} \and
%   %   \ruleGet \and \ruleGetLeft \and \ruleGetRight
%   % \end{mathpar}
%   %
%   % % Restriction
%   % \begin{mathpar}
%   %   \framebox{$\judgeRestrict {\tau_1} l \tau_2 \yields C$} \and
%   %   \ruleRestrict \and \ruleRestrictLeft \and \ruleRestrictRight
%   % \end{mathpar}
%
%   \caption{Disjointness.}
% \end{figure}

\end{document}
