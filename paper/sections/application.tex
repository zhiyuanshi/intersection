\section{Application}

Algebra -> P1,2
Visitor -> P2

Yanlin
Mixin

This section shows that the features in System FI are enough to encode
extensible designs, and even improve on those designs. In particular,
System FI has two main advantages over existing languages:

\begin{itemize}
\item It supports dynamic composition of intersecting values.
\item It supports contravariant parameter types in the subtyping relation.
\end{itemize}

These two features can be used to improve existing designs
of modular programs.

\subsection{Object Algebras}

% Introduce the expression problem

The expression problem refers to the difficulty of adding a new operations and a
new data variant without changing or duplicating existing code in statically
typed functional languages.

There has been recently a lightweight solution to the expression problem that
takes advantage of covariant return types in Java. We show that FI is able to
solve the expression problem in the same spirit. The
A)

% OA

Object algebras allow one to modularly extend the base

We first define the interface capable of evaluation $ IEVAL $.

\lstinputlisting[linerange=10-14]{../src/Algebra.sf} % APPLY:linerange=EXPALG
\lstinputlisting[linerange=22-28]{../src/Algebra.sf} % APPLY:linerange=SUBEXPALG
\lstinputlisting[linerange=44-44]{../src/Algebra.sf} % APPLY:linerange=NEWALG
\lstinputlisting[linerange=37-40]{../src/Algebra.sf} % APPLY:linerange=MERGE

The merge operator $ ,, $ is used in the definition of $ merge $.

\lstinputlisting[linerange=48-48]{../src/Algebra.sf} % APPLY:linerange=PRINT

\subsection{From Algebras Back to Visitors}

Contravariant param type helpers programmers to write more intuitive programs.

\subsection{Yanlin}

% Mixin goes last

\subsection{Mixins}

In Haskell, one is able to write programs in mixin style using records. However,
this approach has a serious drawback: it is not possible to refine the mixin by
adding more fields to the records. This means that the type of the family of the
mixins has to be determined upfront.

\subsection{Composing Mixins and Object Algebras}

Combining mixins and OAs
