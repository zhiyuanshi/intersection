\section{Proofs}

\begin{proof}
By structural induction on the types and the corresponding inference rule. \\

\texttt{(SubVar)}

\texttt{(SubFun)}

\texttt{(SubForall)}

\texttt{(SubAnd1)}

\texttt{(SubAnd2)}

\texttt{(SubAnd3)}

\texttt{(SubRcd)}

\end{proof}

\begin{lemma}
  If $$ \Gamma \turnsget t ; l = C ; t_1 $$
  then $$ \image \Gamma \turns C : \image t \to \image {t_1} $$
\end{lemma}

\begin{proof}
By structural induction on the type and the corresponding inference rule. \\

\texttt{(Get-Base)} $ \Gamma \turnsget \{ l : t \} ; l = \lambda (x : \image {\{ l : t \}}). x ; t $ \\

By the induction hypothesis
$$ \image \Gamma \turns \lambda (x : \image {\{ l : t \}}). x : \image {\{ l : t \}} \to \image t $$

\texttt{(Get-Left)} \\
\texttt{(Get-Right)} \\

\end{proof}

\begin{lemma}
  If $$ \Gamma \turnsput t ; l ; E = C ; t_1 $$
  then $$ \image \Gamma \turns C : \image t \to \image t $$
\end{lemma}

\begin{proof}
By structural induction on the type and the corresponding inference rule. \\

\texttt{(Put-Base)} \\
\texttt{(Put-Left)} \\
\texttt{(Put-Right)} \\
\end{proof}

\begin{lemma} \label{preserve-wf}
  If   $$ \Gamma \turns t $$
  then $$ \image \Gamma \turns \image t $$
\end{lemma}

\begin{proof}
Since $$ \Gamma \turns t $$
It follows from \texttt{(FI-WF)} that
  $$ \ftv t  \subseteq \ftv{\Gamma} $$
And hence
  $$ \ftv {\image t} \subseteq \ftv{\image \Gamma} $$
By \texttt{(F-WF)} we have
  $$ \Gamma \turns t $$
\end{proof}

\begin{theorem}[Type preserving translation]
  If   $$ \Gamma \turns e : t \yields E  $$
  then $$ \image \Gamma \turns E : \image t $$
\end{theorem}

\begin{proof}
By structural induction on the expression and the corresponding inference rule. \\

\texttt{(TrVar)} $ \Gamma \turns x : t \yields x $ \\

It follows from \texttt{(TrVar)} that
  $$ (x : t) \in \Gamma $$
Based on the definition of $ \image \cdot $,
  $$ (x : \image t) \in \image \Gamma $$
Thus we have by \texttt{(F-Var)} that
  $$ \image \Gamma \turns x : \image t $$

\texttt{(TrAbs)} $ \Gamma \turns \lambda (x : t_1). e : t_1 \to t_2 \yields {\lambda x : \image {t_1}. E} $ \\

It follows from \texttt{(TrAbs)} that
  $$ \Gamma, x : t_1 \turns e : t_2 \yields E $$
And by the induction hypothesis that
  $$ \image \Gamma, x : \image {t_1} \turns E : \image {t_2} $$
By \texttt{(TrAbs)} we also have
  $$ \Gamma \turns t_1 $$
It follows from Lemma \ref{preserve-wf} that
  $$ \image \Gamma \turns \image {t_1} $$
Hence by \texttt{(F-Abs)} and the definition of $ \image \cdot $ we have
  $$ \image \Gamma \turns \lambda x : \image {t_1}. E : \image {t_1 \to t_2} $$

\texttt{(TrApp)} $ \Gamma \turns e_1 e_2 : t_2 \yields{E_1 (\app C {E_2})} $ \\

From \texttt{(TrApp)} we have
  $$ \Gamma \turns t_3 <: t_1 \yields{C} $$
Applying Lemma \ref{type-coerce} to the above we have
  $$ \image \Gamma \turns C : \image {t_3} \to \image {t_1} $$
Also from \texttt{(TrApp)} and the induction hypothesis
  $$ \image \Gamma \turns E_1 : \image {t_1} \to \image {t_2} $$
Also from \texttt{(TrApp)} and the induction hypothesis
  $$ \image \Gamma \turns E_2 : \image {t_3} $$
Assembling those parts using \texttt{(F-App)} we come to
  $$ \image \Gamma \turns E_1 (\app C {E_2}) : \image {t_2} $$
\end{proof}

\texttt{(TrTAbs)} $ \Gamma \turns \Lambda \alpha. e : \forall \alpha. t \yields {\forall \alpha. E} $ \\

From \texttt{(TrTAbs)} we have
  $$ \Gamma \turns e : t \yields E $$
By the induction hypothesis we have
  $$ \image \Gamma \turns E : \image t $$
Thus by \texttt{(F-TAbs)} and the definition of $ \image \cdot $
  $$ \Gamma \turns \Lambda \alpha. E : \image {\forall \alpha. t} $$


\texttt{(TrTApp)} $ \Gamma \turns e \; t  : [\alpha := t] t_1 \yields {E \; \image t} $ \\

From \texttt{(TrTApp)} we have
  $$ \Gamma \turns e : \forall \alpha. t_1 \yields E $$
And by the induction hypothesis that
  $$ \image \Gamma \turns E : \forall \alpha. \image {t_1} $$
Also from \texttt{(TrTApp)} and Lemma \ref{preserve-wf} we have
  $$ \image \Gamma \turns \image t $$
Then by \texttt{(F-TApp)} that
  $$ \image \Gamma \turns E \; \image t : [\alpha := \image t ] \image {t_1} $$
Therefore
  $$ \image \Gamma \turns E \; \image t : \image {[\alpha := t ] \image {t_1}} $$

% \texttt{(TrMerge)} $ \Gamma \turns e_1 \merge e_2 : t_1 \& t_2 \yields{\langle E1, E2  \rangle}$ \\

From \texttt{(TrMerge)} and the induction hypothesis we have
  $$ \image \Gamma \turns E_1 : \image {t_1} $$
and
  $$ \image \Gamma \turns E_2 : \image {t_2} $$
Hence by \texttt{(F-Pair)}
  $$ \image \Gamma \turns \langle E_1, E_2 \rangle : \langle \image {t_1}, \image {t_2} \rangle $$
Hence by the definition of $ \image \cdot $
  $$ \image \Gamma \turns \langle E_1, E_2 \rangle : \image {t_1 \& t_2} $$

\texttt{(TrRcdIntro)} $ \Gamma \turns \{ l = e \} : \{ l : t \} \yields E $ \\

From \texttt{(TrRcdIntro)} we have
  $$ \Gamma \turns e : t \yields E $$
And by the induction hypothesis that
  $$ \image \Gamma \turns E : \image t $$
Thus by the definition of $ \image \cdot $
  $$ \image \Gamma \turns E : \image {\{ l : t \}} $$

\texttt{(TrRcdElim)} $ \Gamma \turns e.l : t_1 \yields {C E} $ \\

From \texttt{(TrRcdElim)}
  $$ \Gamma \turns e : t \yields E $$
And by the induction hypothesis that
  $$ \image \Gamma \turns E : \image t $$
Also from \texttt{(TrRcdEim)}
  $$ \Gamma \turnsget e ; l = C ; t_1 $$
Applying Lemma \ref{type-get} to the above we have
  $$ \image \Gamma \turns C : \image t \to \image {t_1}  $$
Hence by \texttt{(F-App)} we have
  $$ \image \Gamma \turns \app C E : \image {t_1} $$

% \texttt{(TrRcdUpd)} $ \Gamma \turns \rcdupd{e}{l}{e_1} : t \yields {\app C E} $ \\

From \texttt{(TrRcdUpd)}
  $$ \Gamma \turns e : t \yields E $$
And by the induction hypothesis that
  $$ \image \Gamma \turns E : \image t $$
Also from \texttt{(TrRcdUpd)}
  $$ \Gamma \turnsput t ; l; E = C ; t_1 $$
Applying Lemma \ref{type-put} to the above we have
  $$ \image \Gamma \turns C : \image t \to \image t  $$
Hence by \texttt{(F-App)} we have
  $$ \image \Gamma \turns \app C E : \image t $$
